% !TEX root = ProjetoPedagogico.tex

\chapter{Engenharia de Computação}
O curso ora proposto de Engenharia de Computação obedecerá ao regime de créditos, oferecendo 50 vagas anuais, repartidas igualmente em dois semestres letivos. O aluno interessado em cursar a graduação em Engenharia de Computação fará tal opção diretamente a partir da sua inscrição no vestibular.

\section{Concepção}

A estrutura curricular do curso de Engenharia de Computação do Departamento de Engenharia de Sistemas e Computação da Faculdade de Engenharia da UERJ orientar-se-á pelas \textit{Diretrizes Curriculares Nacionais para os cursos de graduação na área da Computação,} do Ministério da Educação (anexo \ref{cne11}), pelos \textit{Referenciais de Formação para os cursos de Graduação em Computação}, da Sociedade Brasileira de Computação, e pela regulamentação do exercício da profissão de Engenheiro, estabelecida pelo Sistema CREA/CONFEA (Resolução 1.010 CONFEA, anexo \ref{res1010}), em vigor atualmente.

A grade curricular totaliza 4350 aulas de 50 minutos cada (doravante chamadas de horas-aula), distribuídas em 58 disciplinas (55 obrigatórias e mais 3 eletivas restritas), dentre as quais estão incluídas práticas laboratoriais em complementação à base teórica. Estão incluídos também Estágio Supervisionado (\EstSupCH\,horas-aula) e Projeto de Graduação (trabalho de conclusão de curso) como atividade de síntese e integração do conhecimento científico, tecnológico e instrumental. Ainda, como atividades acadêmicas complementares facultativas incluem-se Estágio Interno, Monitoria e Iniciação Científica, Cursos, Eventos, Palestras e Visitas Técnicas ocasionais como atividades direcionadas a proporcionar uma melhor percepção do que é a Engenharia, como se encontra o setor no Brasil e quais são as áreas de atuação e as atividades desenvolvidas pelos Engenheiros de Computação. As 4350 horas-aulas correspondem a 3625 horas no total, atendendo o mínimo exigido pela legislação.

\section{Objetivos Gerais}

Formar engenheiros aptos para a inserção em setores profissionais e para a participação no desenvolvimento da sociedade brasileira, habilitando-os para o exercício pleno de todas as funções nas diversas atividades de sua área de atuação e colaborando para a sua formação contínua.

\section{Objetivos Específicos}
Preparar engenheiros capazes de desenvolver e implementar soluções nas diferentes áreas de aplicação da tecnologia computacional, incluindo: análise e otimização de sistemas; sistemas de produção; sistemas digitais; sistemas operacionais; sistemas de comunicação de dados; sistemas de banco de dados; engenharia de software e sistemas de informação.

\section{Finalidades Gerais}
Formar engenheiros em condições de atuar em todos os setores da economia (comércio, indústria e serviços), em particular nas empresas fabricantes de equipamentos computacionais ou que prestam serviços de assistência técnica a esses tipos de produtos.

\section{Finalidades Específicas}
Formar profissionais habilitados a diagnosticar problemas, avaliar alternativas, propor soluções e conduzir projetos nas áreas de Inteligência Computacional, Banco de Dados, Engenharia de Software, Arquitetura de Sistemas Computacionais, Redes de Computadores, Processamento Distribuído e de Alto Desempenho, Automação, Processamento Gráfico, Linguagens Formais, Compiladores e Análise de Algoritmos Computacionais.

\section{Nível de Formação e Título Acadêmico Concedido}
O curso é de graduação plena em Engenharia com Habilitação em Computação e a titulação concedida é:

\textbf{Título:} Engenheiro de Computação.

\section{Perfil do Egresso (competência, habilidades e atitudes pretendidas)}
O curso de Engenharia de Computação tem como perfil do egresso o engenheiro, com formação técnico-científica sólida, generalista, humanista, crítica e reflexiva, capacitado a absorver e desenvolver novas tecnologias, estimulando a sua atuação crítica e criativa na identificação e resolução de problemas, considerando seus aspectos políticos, econômicos, sociais, ambientais e culturais, com visão ética e humanística, em atendimento às demandas da sociedade. Faz parte do perfil do egresso a postura de permanente busca da atualização profissional, além das seguintes habilidades:
\begin{enumerate} [I -]
	\item Possuir sólidos conhecimentos em teorias e princípio da Ciência da Computação, Matemática, Ciências e Engenharia; Ser capaz de aplicar estas teorias e princípios para resolver problemas técnicos de sistemas computacionais e sistemas de aplicação específica.
	\item Ter capacidade de planejar, implementar e manter soluções computacionais eficientes para diversos tipos de problemas, envolvendo hardware, software e processos. Saibam explorar o espaço de projeto considerando restrições e fazer análise de custo-benefício; e ser apto a criar e integrar componentes de hardware, de software e sua interface.
	\item Demonstrar autonomia e análise crítica. Gerenciar projetos, serviços e experimentos de engenharia na área de computação, de forma colaborativa em equipes multidisciplinares e em grupos sociais complexos e heterogêneos, integrando o desenvolvimento humano, profissional e organizacional. Ser capaz de se expressar verbalmente e na forma escrita; e de avaliar corretamente seus resultados e de terceiros. Saber transferir conhecimento e se manter atualizado.
	\item Ter habilidades de criatividade e inovação. Produzir ferramentas, técnicas e conhecimentos científicos e/ou tecnológicos inovadores na área.
	\item Ser capaz de empreender na área de engenharia de computação, reconhecendo oportunidades e resolvendo problemas de forma transformadora, agregando valor à sociedade.
	\item Entender a importância e a responsabilidade da sua prática profissional, agindo de forma ética, sustentável e socialmente responsável, respeitando aspectos legais e normas envolvidas. Observem direitos e propriedades intelectuais inerentes à produção e à utilização de sistemas de computação.
\end{enumerate}

O egresso da Engenharia de Computação, no processo de sua formação, deverá desenvolver as seguintes competências:
\begin{enumerate} [I -]
	\item antever as implicações humanísticas, sociais, ambientais, éticas, profissionais, legais (inclusive relacionadas à propriedade intelectual) e políticas dos sistemas computacionais;
	\item identificar demandas socioeconômicas e ambientais relevantes, planejar, especificar e projetar sistemas de computação, seguindo teorias, princípios, métodos e procedimentos interdisciplinares;
	\item construir, testar, verificar e validar sistemas de computação, seguindo métodos, técnicas e procedimentos interdisciplinares;
	\item perceber as necessidades de atualização decorrentes da evolução tecnológica e social;
	\item relacionar problemas do mundo real com suas soluções, considerando aspectos de computabilidade e de escalabilidade;
	\item analisar, desenvolver, avaliar e aperfeiçoar software e hardware em arquiteturas de computadores;
	\item analisar, desenvolver, avaliar e aperfeiçoar sistemas de automação e sistemas inteligentes;
	\item analisar, desenvolver, avaliar e aperfeiçoar sistemas de informação computacionais;
	\item analisar, desenvolver, avaliar e aperfeiçoar circuitos eletroeletrônicos;
	\item gerenciar pessoas e infraestrutura de Sistemas de Computação;
	\item perceber as necessidades de inovação e inserção internacional com atitudes criativas e empreendedoras.
\end{enumerate}

O curso de Engenharia de Computação tem, predominantemente, o ensino da computação como atividade fim, visando à formação de recursos humanos para o desenvolvimento científico e tecnológico da computação. Assim sendo, o curso deve capacitar indivíduos para desenvolver software e hardware, com uma forte base matemática e física.

Os egressos do curso de Engenharia de Computação estarão situados no estado da arte da ciência e da tecnologia da computação, de tal forma que possam continuar suas atividades na pesquisa, promovendo o desenvolvimento científico, ou aplicando os conhecimentos científicos, propiciando o desenvolvimento tecnológico. Para tal, é dada uma forte ênfase no uso de laboratórios para capacitar os egressos no projeto e construção tanto de software quanto de hardware.
\section{Estrutura Curricular}
O currículo do curso de Engenharia de Computação é constituído por disciplinas obrigatórias e eletivas, estágio supervisionado, trabalho de conclusão de curso e atividades complementares. O curso é organizado em 10 semestres, podendo o aluno cumpri-lo em um máximo de 18 semestres.

Para uma eficaz orientação pedagógica, é proposto o aconselhamento curricular apresentado nas tabelas \ref{tab1p} a \ref{tab10p}. Os pré-requisitos das disciplinas podem ser observados no fluxograma do curso (anexo \ref{fluxograma}).

O aluno deverá cursar no mínimo três das disciplinas eletivas restritas oferecidas (ver tabela \ref{tabeletivas}). Deve ser
ressaltado que estas disciplinas são oferecidas de acordo com o interesse dos corpos
docente e discente, não sendo necessariamente disponibilizadas todos os semestres.

\rowcolors{1}{gray!5}{white}
\setlength{\tabcolsep}{5pt}
\renewcommand{\arraystretch}{1.5}
\begin{table}[ht]
	\centering
	\caption{1\textordmasculine Período}
	\label{tab1p}
	\begin{spreadtab}{{tabularx}{\textwidth}{ | X|c|c| }}
		\hline
		@ {\textbf{Disciplina}} & @ {\textbf{CH}} & @ {\textbf{Créditos}} \\
		\hline
		@ \AlgComp	& \AlgCompCH	& \AlgCompCred	\\
		@ \IntComp  & \IntCompCH	& \IntCompCred	\\
		@ \CalcI	& \CalcICH		& \CalcICred	\\
		@ \QuiT 	& \QuiTCH 		& \QuiTCred		\\
		@ \QuiE 	& \QuiECH 		& \QuiECred		\\
		@ \DesBas	& \DesBasCH		& \DesBasCred	\\
		\hline
		@ Total 	& sum(b2:b7) 	& sum(c2:c7)	\\
		\hline
	\end{spreadtab}
\end{table}

\rowcolors{1}{gray!5}{white}
\begin{table}
	\centering
	\caption{2\textordmasculine Período}
	\label{tab2p}
	\begin{spreadtab}{{tabularx}{\textwidth}{|X|c|c|}}
		\hline
		@ {\textbf{Disciplina}} & @ {\textbf{CH}} & @ {\textbf{Créditos}} \\
		\hline
		@ \EstrInf	& \EstrInfCH	& \EstrInfCred 	\\
		@ \EngComput& \EngComputCH	& \EngComputCred\\
		@ \CalcII	& \CalcIICH		& \CalcIICred	\\
		@ \LogProg	& \LogProgCH	& \LogProgCred	\\
		@\AlgLin	& \AlgLinCH		& \AlgLinCred	\\
		@ \FisI		& \FisICH		& \FisICred		\\
		@ \FisEI	& \FisEICH		& \FisEICred	\\
		\hline
		@ Total 	& sum(b2:b8) 	& sum(c2:c8)	\\
		\hline
	\end{spreadtab}
\end{table}
	
\rowcolors{1}{gray!5}{white}
\begin{table}
	\centering
	\caption{3\textordmasculine Período}
	\label{tab3p}
	\begin{spreadtab}{{tabularx}{\textwidth}{|X|c|c|}}
		\hline
		@ {\textbf{Disciplina}} & @ {\textbf{CH}} & @ {\textbf{Créditos}} \\
		\hline
		@ \AnAlg	& \AnAlgCH		& \AnAlgCred	\\
		@ \IntAmb	& \IntAmbCH		& \IntAmbCred	\\
		@ \CalcIII	& \CalcIIICH 	& \CalcIIICred	\\
		@ \ProbEst	& \ProbEstCH	& \ProbEstCred	\\
		@ \MecTec	& \MecTecCH		& \MecTecCred	\\
		@ \FisII	& \FisIICH		& \FisIICred	\\
		@ \FisEII	& \FisEICH		& \FisEICred	\\
		\hline
		@ Total 	& sum(b2:b8) 	& sum(c2:c8)	\\
		\hline
	\end{spreadtab}
\end{table}

\rowcolors{1}{gray!5}{white}
\begin{table}
	\centering
	\caption{4\textordmasculine Período}
	\label{tab4p}
	\begin{spreadtab}{{tabularx}{\textwidth}{|X|c|c|}}
		\hline
		@ {\textbf{Disciplina}} & @ {\textbf{CH}} & @ {\textbf{Créditos}} \\
		\hline
		@ \LabProgA	& \LabProgACH	& \LabProgACred	\\
		@ \FundIComp	& \FundICompCH	& \FundICompCred	\\
		@ \LabProgB	& \LabProgBCH	& \LabProgBCred	\\
		@ \FenTran	& \FenTranCH	& \FenTranCred	\\
		@ \ResMat	& \ResMatCH		& \ResMatCred	\\
		@ \FisIII	& \FisIIICH		& \FisIIICred	\\
		@ \FisEIII	& \FisEIIICH	& \FisEIIICred	\\
		\hline
		@ Total 	& sum(b2:b8) 	& sum(c2:c8)	\\
		\hline
	\end{spreadtab}
\end{table}

\rowcolors{1}{gray!5}{white}
\begin{table}
	\centering
	\caption{5\textordmasculine Período}
	\label{tab5p}
	\begin{spreadtab}{{tabularx}{\textwidth}{|X|c|c|}}
		\hline
		@ {\textbf{Disciplina}} & @ {\textbf{CH}} & @ {\textbf{Créditos}} \\
		\hline
		@ \Grafos	& \GrafosCH		& \GrafosCred	\\
		@ \FundComp	& \FundCompCH	& \FundCompCred	\\
		@ \MatEle 	& \MatEleCH		& \MatEleCred	\\
		@ \CEV		& \CEVCH		& \CEVCred		\\
		@ \ModMat	& \ModMatCH		& \ModMatCred	\\
		@ \FisIV	& \FisIVCH		& \FisIVCred	\\
		@ \FisEIV	& \FisEIVCH		& \FisEIVCred	\\
		\hline
		@ Total 	& sum(b2:b8) 	& sum(c2:c8)	\\
		\hline
	\end{spreadtab}
\end{table}

\rowcolors{1}{gray!5}{white}
\begin{table}
	\centering
	\caption{6\textordmasculine Período}
	\label{tab6p}
	\begin{spreadtab}{{tabularx}{\textwidth}{|X|c|c|}}
		\hline
		@ {\textbf{Disciplina}} & @ {\textbf{CH}} & @ {\textbf{Créditos}} \\
		\hline
		@ \EngSistA & \EngSistACH	& \EngSistACred	\\
		@ \IC		& \ICCH			& \ICCred		\\
		@ \ArqComp	& \ArqCompCH	& \ArqCompCred	\\
		@ \ICII 	& \ICIICH		& \ICIICred		\\
		@ \ProcImag & \ProcImagCH	& \ProcImagCred	\\
		@ \CEVI		& \CEVICH 		& \CEVICred		\\
		@ \EletI	& \EletICH		& \EletICred	\\
		\hline
		@ Total 	& sum(b2:b8) 	& sum(c2:c8)	\\
		\hline
	\end{spreadtab}
\end{table}

\rowcolors{1}{gray!5}{white}
\begin{table}
	\centering
	\caption{7\textordmasculine Período}
	\label{tab7p}
	\begin{spreadtab}{{tabularx}{\textwidth}{|X|c|c|}}
		\hline
		@ {\textbf{Disciplina}} & @ {\textbf{CH}} & @ {\textbf{Créditos}} \\
		\hline
		@ \ProjBD		& \ProjBDCH		& \ProjBDCred		\\
		@ \TeoComp		& \TeoCompCH	& \TeoCompCred		\\
		@ \MineraDados 	& \MineraDadosCH & \MineraDadosCred	\\
		@ \EngCompSoc 	& \EngCompSocCH & \EngCompSocCred	\\
		@ \Telep 		& \TelepCH		& \TelepCred		\\
		@ \ProjSO		& \ProjSOCH		& \ProjSOCred		\\
		@ \SegHig		& \SegHigCH		& \SegHigCred		\\
		\hline
		@ Total			& sum(b2:b8)	& sum(c2:c8)		\\
		\hline
	\end{spreadtab}
\end{table}

\rowcolors{1}{gray!5}{white}
\begin{table}
	\centering
	\caption{8\textordmasculine Período}
	\label{tab8p}
	\begin{spreadtab}{{tabularx}{\textwidth}{|X|c|c|}}
		\hline
		@ {\textbf{Disciplina}} & @ {\textbf{CH}} & @ {\textbf{Créditos}} \\
		\hline
		@ \EngSistC 	& \EngSistCCH		& \EngSistCCred		\\
		@ \Sredes 		& \SredesCH			& \SredesCred		\\
		@ \SistEmb		& \SistEmbCH		& \SistEmbCred		\\
		@ \Control		& \ControlCH		& \ControlCred		\\
		@ \MetQuant 	& \MetQuantCH		& \MetQuantCred		\\
		@ \CompParal	& \CompParalCH		& \CompParalCred	\\
		@ \Empre 		& \EmpreCH			& \EmpreCred		\\
		\hline
		@ Total				& sum(b2:b8)			& sum(c2:c8)			\\
		\hline
	\end{spreadtab}
\end{table}

\rowcolors{1}{gray!5}{white}
\begin{table}
	\centering
	\caption{9\textordmasculine Período}
	\label{tab9p}
	\begin{spreadtab}{{tabularx}{\textwidth}{|X|c|c|}}
		\hline
		@ {\textbf{Disciplina}} & @ {\textbf{CH}} & @ {\textbf{Créditos}} \\
		\hline
		@ \ProjA		& \ProjACH		& \ProjACred	\\
		@ \Instala 		& \InstalaCH	& \InstalaCred	\\
		@ \EstSup		& \EstSupCH		& \EstSupCred	\\
		@ \EletA		& \EletACH		& \EletACred	\\
		@ \IntEco		& \IntEcoCH		& \IntEcoCred	\\
		\hline
		@ Total			& sum(b2:b6)	& sum(c2:c6)	\\
		\hline
	\end{spreadtab}
\end{table}

\begin{table}
	\centering
	\caption{10\textordmasculine Período}
	\label{tab10p}
	\begin{spreadtab}{{tabularx}{\textwidth}{|X|c|c|}}
		\hline
		@ {\textbf{Disciplina}} & @ {\textbf{CH}} & @ {\textbf{Créditos}} \\
		\hline
		@ \ProjB	& \ProjBCH	& \ProjBCred	\\
		@ \EletB	& \EletBCH	& \EletBCred	\\
		@ \EletC	& \EletCCH	& \EletCCred	\\
		@ \Adm		& \AdmCH	& \AdmCred		\\
		\hline
		@ Total		& sum(b2:b5)& sum(c2:c5)	\\
		\hline
	\end{spreadtab}
\end{table}

\begin{table}
	\centering
	\caption{Disciplinas Eletivas Restritas}
	\label{tabeletivas}
	\begin{spreadtab}{{tabularx}{\textwidth}{|X|c|c|}}
		\hline
		@ {\textbf{Disciplina}} & @ {\textbf{CH}} & @ {\textbf{Créditos}} \\
		\hline
		@ \EletArq	& \EletArqCH	& \EletArqCred	\\
		@ \EletGeo	& \EletGeoCH	& \EletGeoCred	\\
		@ \EletPadroes	& \EletPadroesCH	& \EletPadroesCred	\\
		@ \EletRec	& \EletRecCH	& \EletRecCred	\\
		@ \EletRedes	& \EletRedesCH& \EletRedesCred	\\
		@ \EletMov	& \EletMovCH	& \EletMovCred	\\
		\hline
	\end{spreadtab}
\end{table}

%%%% CRIS: Begin

\section{Coordenação de Áreas}

As disciplinas do curso de Engenharia de Computação estão divididas em quatro grandes áreas de conhecimento: (1) Sistemas de Informação; (2) Arquitetura de Sistemas de Computação; (3) Algoritmos e Linguagens de Programação; (4) Lógica e Inteligência Computacional.

A integração das disciplinas em áreas de conhecimento permite o compartilhamento de informações sobre interesses e objetivos comuns. Favorece a atuação conjunta de alunos e professores em temas globais e impulsiona a criação de linhas de pesquisa.

Cada área de conhecimento deverá possuir um Professor Coordenador. O Coordenador de área será responsável pelas disciplinas de sua área, cabendo a ele(a): orientar os alunos em questões referentes às disciplinas, analisar os requerimentos de quebras de pré-requisitos e conflitos de horário, tratar questões relativas aos conteúdos programáticos das disciplinas, promover a integração dos professores da mesma área, e  incentivar a pesquisa na área.

A tabela \ref{tab:areas} mostra a distribuição das disciplinas por área de conhecimento.

\begin{table}[ht]
	\centering
	\caption{Tabela de divisão de disciplinas por área de conhecimento}
	\label{tab:areas}
	\begin{tabularx}{\textwidth}{ X  l }
		\hiderowcolors
		\hline
		{\bf Área de Conhecimento}                              & {\bf Disciplinas} \\
		\hline
		\multirow{5}{*}{Algoritmos e Linguagens de Programação} & \AlgComp          \\
																& \AnAlg            \\
		                                                        & \EstrInf          \\
		                                                        & \LabProgA         \\
		                                                        & \LabProgB         \\
		                                                        & \TeoComp          \\ 
																& \Grafos			\\ \hline
		\multirow{8}{*}{Arquitetura de Sistemas de Computação}  & \ArqComp          \\ % Arqutetura de Computadores
																& \CompParal        \\ % Computação Paralela
		                                                        & \Control          \\ % Controle de Processos
		                                                        & \FundIComp         \\ % FUndamentos de Computadores I
																& \FundComp         \\ % FUndamentos de Computadores II
																& \Instala			\\ % Instalações de Ambientes Computacionais
		                                                        & \ProjSO           \\ % Projeto de Sistemas 
																& \Telep            \\ % Redes de Computadores
		                                                        & \Sredes         	\\ % Segurança em Redes
		                                                        & \SistEmb          \\ % Sistema Embutidos
		                                                        \hline	
		\multirow{3}{*}{Lógica e Inteligência Computacional}    & \IC               \\ % Inteligência Computacional
		                                                        & \ICII              \\ % Inteligência Computacional II
		                                                        & \LogProg          \\ % Lógica de Programação
																& \MineraDados      \\ % Mineração de Dados
		                                                        & \ProcImag         \\ % Processamento de Imagens
		                                                        \hline													
		\multirow{4}{*}{Sistemas de Informação}                 & \EngSistC         \\ % Análise de Projeto de Sistemas
																& \EngSistA		 \\  % Engenharia de Sistemas
		                                                        & \ProjBD           \\
		                                                        & \EngCompSoc       \\
		                                                        \hline
	\end{tabularx}
\end{table}


%%%% CRIS: End


\section{Equivalência com o Curso Anterior}
O curso de Engenharia de Computação ora proposto substituirá o curso de Engenharia Elétrica com ênfase em Sistemas e Computação e, na hipótese de algum aluno desejar migrar do curso antigo para este novo, será possível dispensar disciplinas do novo currículo iguais ou equivalentes às disciplinas do curso antigo.

A tabela \ref{DiscIguais} apresenta a relação das disciplinas que são iguais nos dois cursos, logo, a dispensa é direta, e a tabela \ref{equivalencias} mostra a equivalência entre as disciplinas dos currículos antigo e novo. Finalmente, a tabela \ref{DiscSemEqui} exibe as disciplinas do novo curso sem equivalência com alguma(s) disciplina(s) do curso anterior.

\rowcolors{1}{gray!5}{white}
\begin{table}[ht]
	\caption{Disciplinas Iguais em Ambos os Currículos}
	\label{DiscIguais}
	\centering
	\renewcommand{\arraystretch}{1.5}
	\begin{tabularx}{\textwidth}{|X|l|}
		\showrowcolors
		\hline
		{\textbf{Disciplina}} & \textbf{Código} \\
		\hline
		\Adm                  & \AdmCod         \\
		\AlgLin               & \AlgLinCod      \\
		\AnaFis               & \AnaFisCod      \\
		\AnaVet               & \AnaVetCod      \\
		\CEV                  & \CEVCod         \\
		\CServMec             & \CServMecCod    \\
		\DesBas               & \DesBasCod      \\
		\EletI                & \EletICod       \\
		\EletIIA              & \EletIIACod     \\
		\FenTran              & \FenTranCod     \\
		\FisI                 & \FisICod        \\
		\FisII                & \FisIICod       \\
		\FisIII               & \FisIIICod      \\
		\FisIV                & \FisIVCod       \\
		\GeoAna               & \GeoAnaCod      \\
		\IntEco               & \IntEcoCod      \\
		\IntAmb               & \IntAmbCod      \\
		\MatEle               & \MatEleCod      \\
		\MecTec               & \MecTecCod      \\
		\ModMat               & \ModMatCod      \\
		\PrincTelec           & \PrincTelecCod  \\
		\ProbEst              & \ProbEstCod     \\
		\ProjA                & \ProjACod       \\
		\ProjB                & \ProjBCod       \\
		\QuiT                 & \QuiTCod        \\
		\QuiE                 & \QuiECod        \\
		\ResMat               & \ResMatCod      \\
		\hline
	\end{tabularx}
\end{table}

\rowcolors{1}{gray!5}{white}
\begin{table}
	\centering
	\renewcommand{\arraystretch}{1.5}
	\caption{Equivalências no novo currículo}
	\label{equivalencias}
	\begin{tabularx}{\textwidth}{|X|l||l|}
		\hline
		{\textbf{Disciplinas do Currículo Antigo}}                & \textbf{Código}             & \textbf{Equivalente no Currículo Novo} \\
		\hline
		Algoritmos Computacionais                                 & FEN06-03559                 & \AlgComp                               \\
		Análise de Algoritmos                                     & FEN06-03713                 & \AnAlg                                 \\
		Arquitetura de Computadores I                             & FEN06-04119                 & \ArqComp                               \\
		Arquitetura de Sistemas Operacionais                      & FEN06-04664                 & \ProjSO                                \\
		Cálculo Diferencial e Integral I                          & IME01-00508                 & \CalcI                                 \\
		Cálculo Diferencial e Integral II                         & IME01-00854                 & \CalcII                                \\
		Cálculo Diferencial e Integral III                        & IME01-03646                 & \CalcIII                               \\
		Cálculo Numérico IV                                       & IME04-04541                 & \EngComput                             \\
		Carac. das Linguagens de Prog. I                          & FEN06-03980                 & \LabProgB                              \\
		Circuitos Elétricos IV                                    & FEN04-05222                 & \CEVI                                  \\
		Controle de Processos por Comp.                           & FEN06-05080                 & \Control                               \\
		Engenharia de Sistemas A                                  & FEN06-04243                 & \EngSistA                              \\
		Engenharia de Sistemas B                                  & FEN06-04314                 & \ProjBD                                \\
		Estruturas de Informação I                                & FEN06-03648                 & \EstrInf                               \\
		Fundamentos de Comp. Digitais I e Técnicas Digitais II    & \parbox[t]{2cm}{FEN06-03787                                          \\FEN05-04561} 				&\FundComp\\
		Laboratório de Programação I                              & FEN06-04049                 & \LabProgA                              \\
		Segurança e Higiene do Trabalho                           & FEN07-02722                 & \SegHig                                \\
		Teleproc. e Redes de Computadores                         & FEN06-04718                 & \Telep                                 \\
		%Teoria de Compiladores                                    & FEN06-04516                 & \TeoComp                               \\
		Tóp. Especiais em Eng. de Sistemas e Computação A, B ou C & \parbox[t]{2cm}{FEN06-04889                                          \\FEN06-04939\\FEN06-04990}  & Eletivas Restritas\\
		\hline
	\end{tabularx}
\end{table}

\begin{table}
	\centering
	\renewcommand{\arraystretch}{1.5}
	\caption{Disciplinas sem Equivalências}
	\label{DiscSemEqui}
	\begin{tabularx}{\textwidth}{|X|}
		\hline
		{\textbf{Disciplinas do Novo Currículo sem Equivalência}} \\
		\hline
		\MetQuant                                                 \\
		\LogProg                                                  \\
		\IC                                                       \\
		\EngCompSoc                                               \\
		\MineraDados                                              \\
		\SistEmb                                                  \\
		\ProcImag                                                 \\
		\CompParal                                                \\
		\EstSup                                                   \\
		\hline
	\end{tabularx}
\end{table}

\section{Ementário das Disciplinas}
As ementas das disciplinas obrigatórias e eletivas são apresentadas no anexo \ref{ementas}. As ementas das disciplinas já existentes foram obtidas no site do próprio DEP, Departamento de Orientação e Supervisão Pedagógica. Essas disciplinas são apresentadas no formulário antigo e não foram feitas correções ou alterações no texto original.

\section{Normas Gerais de Ensino de Graduação da UERJ}
O curso de Engenharia de Computação obedecerá ao regime de créditos e as aulas serão oferecidas nos turnos manhã e tarde, com aulas predominantemente pela manhã, para os aprovados classificados no primeiro semestre; e tarde e noite, com aulas predominantemente pela tarde, para os aprovados classificados no segundo semestre. O turno da manhã transcorre no horário das 07:00h às 12:20h; o da tarde das 12:30h às 17:50h e o da noite das 18:00h às 22:40h. As aulas têm duração de 50 minutos nos turnos da manhã e tarde e de 45 minutos no turno da noite.

As Normas Gerais de Ensino de Graduação da UERJ são definidas pela deliberação n\textordmasculine~33/95 da UERJ (anexo \ref{delib3395}), sendo seus aspectos principais apresentados a seguir:

\subsection{Relação entre crédito e carga horária}
\textit{
	\textbf{Art. 57} -- O número mínimo de créditos necessários para integralizar o currículo será estabelecido com base na carga horária total do curso.}

\textit{
	\textbf{Parágrafo Único} - A unidade de crédito corresponde a:
	\begin{enumerate}[a)]
		\item 15 (quinze) horas de aula teórica, ou
		\item 30 (trinta) horas de aula prática, laboratório ou estágio curricular.
	\end{enumerate}}

\subsection{Aproveitamento escolar}
\begin{itquotation}
	\setcounter{artigo}{94}
	\artigo A aprovação do aluno em disciplinas do Curso de Graduação desta Universidade terá por base notas e frequência. São condições para aprovação: obtenção de nota final mínima 5,0 (cinco vírgula zero), constituída pela média aritmética da média semestral e nota da prova final, frequência mínima de 75\% (setenta e cinco por cento) do total de horas/aula determinado para a disciplina.

	\begin{paragrafos}
		\paragrafo Para cada disciplina haverá, pelo menos, duas avaliações por turma, por período letivo, sendo uma delas necessariamente individual e escrita. A média dos resultados dessas avaliações constitui a média semestral do aluno na disciplina.\\
		\paragrafo O aluno que obtiver média semestral igual ou superior a 4,0 (quatro vírgula zero) terá direito à prova final.\\
		\paragrafo O aluno que obtiver média semestral igual ou superior a 7,0 (sete vírgula zero) estará dispensado de prestar prova final.\\
		\ldots

		\setcounter{paragrafo}{6}
		\paragrafo  O aluno que obtiver nota final menor que 5,0 (cinco vírgula zero) ou média semestral inferior a 4,0 (quatro vírgula zero) será reprovado.\\
		\paragrafo O aluno que não obtiver frequência mínima de 75\% (setenta e cinco por cento) do total de horas/aula determinadas pela disciplina será reprovado, sem direito à prova final e independente de alcançar nota final superior a 7,0 (sete vírgula zero).\\
	\end{paragrafos}

\end{itquotation}
\subsection{Período de integralização do curso}
\setcounter{artigo}{98}
\begin{itquotation}
	\artigo Somente receberá o diploma o aluno que cumprir a Integralização Curricular.
\end{itquotation}

O período mínimo de integralização curricular dos cursos de engenharia é de 10 (dez) semestres, exceto para os casos de isenção de disciplinas, em que é possível um tempo mínimo menor. Já o prazo máximo para essa integralização é de 18 (dezoito) semestres.


\section{Identificação das Condições Técnico-Ambientais}

\subsection{Edificações e Instalações}
A Faculdade de Engenharia está situada no quinto andar do pavilhão João Lyra Filho e possui, no bloco F, 22 salas de aula com capacidade média para 40 alunos.

\subsection{Biblioteca}
Os recursos bibliográficos postos à disposição dos alunos estão sob a guarda da biblioteca central e das bibliotecas setoriais. São mais de vinte mil (20.000) títulos com cerca de trinta mil exemplares (30.000), cerca de mil e duzentos títulos de periódicos sobre os mais diversos assuntos de todas as áreas.

A Biblioteca setorial do curso está situada no quinto andar do pavilhão João Lyra Filho e reúne o acervo básico, oferecendo área de estudos específica para os discentes e docentes.

Associado a esses recursos, os alunos, por meio do uso de computadores e da Internet, têm acesso ao sistema automático de busca bibliográfica.

Em relação aos mecanismos de atualização, a biblioteca conta com doações e verbas próprias da UERJ.

\subsection{Laboratórios}
A Faculdade de Engenharia possui laboratórios que atendem tanto os cursos de graduação como também à pós-graduação. O Curso de Engenharia de Computação utilizará, para as aulas práticas das disciplinas do núcleo de conteúdos básicos, os laboratórios vinculados às Ciências Básicas: Física, Química e de Informática. Para as disciplinas do núcleo profissional, serão utilizados o Laboratório de Engenharia Elétrica e o Laboratório de Computação.

O Laboratório de Engenharia Elétrica (LEE) apoia as atividades de ensino e pesquisa em Eletricidade, Eletrônica, Máquinas Elétricas, Sistemas de Controle, Acionamentos Elétricos, Eletrônica Industrial, Conversão Eletromecânica de Energia, Sistemas Digitais e Telecomunicações.

O Laboratório de Computação (LabComp) apoia as atividades de ensino e pesquisa em Arquitetura de Computadores, Desenvolvimento de Sistemas, Linguagens de Programação, Análise de Algoritmos e Sistemas Embutidos.

\subsection{Perfil do Corpo Docente}
O corpo docente do DESC que ministrará as disciplinas de responsabilidade do DESC do curso de Engenharia de Computação é composto por 11 doutores e 1 graduado, conforme é apresentado na tabela \ref{CorpoDocente}, com suas respectivas cargas horárias e cargos na UERJ.

%%%% CRIS: Begin

É importante ressaltar que o corpo docente atual é suficiente para atender a demanda de todas as disciplinas oferecidas no curso de Engenharia de Computação. Desta forma, a implementação do novo currículo não irá requerer a contratação de nenhum outro professor.

%%%% CRIS: End
\rowcolors{1}{gray!5}{white}
\begin{table}
	\centering
	\caption{Corpo Docente}
	\label{CorpoDocente}
	\begin{tabular}{|l|c|l|l|}
		\hline
		{\textbf{Docente}}                   & \textbf{CH} & \textbf{Titulação}                         & \textbf{Cargo}  \\
		\hline
		%Arnaldo Vieira da Rocha Filho        & 40h         & Doutorado em Sistemas de Informação        & Prof. Adjunto   \\
		Cristiana Barbosa Bentes             & 40h         & Doutorado em Eng. de Sistemas e Computação & Prof. Associado \\
		Felipe Cassemiro Ulrichsen	         & 40h         & Mestrado em Ciências Computacionais      	& Prof. Assistente\\
		Gabriel Cardoso de Carvalho          & 40h         & Mestrado em Computação                     & Prof. Assistente\\
		Giomar Oliver Sequeiros Olivera      & 40h         & Doutorado em Computação                    & Prof. Adjunto   \\
		João Araujo Ribeiro                  & 40h         & Doutorado em Computação                    & Prof. Associado \\
		%Jorge Duarte Pires Valério          & 40h         & Doutorado em Engenharia Elétrica           & Prof. Associado \\
		Luigi Maciel Ribeiro 	             & 40h         & Mestrado em Engenharia Eletrônica          & Prof. Assistente \\
		Luiza de Macedo Mourelle             & 40h         & Doutorado em Computação                    & Prof. Associado \\
		Margareth Gonçalves Simões           & 20h         & Doutorado em Geografia                     & Prof. Associado \\
		Rafaela Correia Brum		         & 40h         & Doutorado em Computação                    & Prof. Adjunto   \\	
		Robert Mota Oliveira		         & 40h         & Doutorado em Engenharia Elétrica           & Prof. Adjunto   \\	
		Simone Ingrid Monteiro Gama		     & 40h         & Mestrado em Informática             		& Prof. Assistente\\	
		Thiago Medeiros Carvalho		     & 40h         & Mestrado em Engenharia Elétrica            & Prof. Assistente\\	
		%Nival Nunes de Almeida               & 40h         & Doutorado em Engenharia Elétrica           & Prof. Associado \\
		%Orlando Bernardo Filho               & 40h         & Doutorado em Engenharia Elétrica           & Prof. Associado \\
		%Oscar Luiz Monteiro de Farias        & 40h         & Doutorado em Informática                   & Prof. Associado \\
		%Pedro Paulo Thompson de Vasconcellos & 20h         & Graduado em Engenharia                     & Prof. Auxiliar  \\
		%Raul Queiroz Feitosa                 & 40h         & Doutorado em Ciência da Computação         & Prof. Adjunto   \\
		%Sheila Regina Murgel Veloso          & 40h         & Doutorado em Eng. de Sistemas e Computação & Prof. Titular   \\
		\hline
	\end{tabular}
\end{table}

