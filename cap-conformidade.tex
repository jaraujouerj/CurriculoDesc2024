\chapter{Conformidade com as Diretrizes Nacionais}
\thispagestyle{plain}

Este capítulo tem como objetivo apresentar uma análise da relação entre as \textit{Diretrizes Curriculares Nacionais para Cursos de Graduação em Engenharia de Computação} e a estrutura curricular do curso proposto.

A Engenharia de Computação é uma área dinâmica e em constante evolução, exigindo uma formação sólida que contemple tanto os fundamentos teóricos quanto as habilidades práticas necessárias para o desenvolvimento de soluções inovadoras que integrem hardware e software. Os referenciais da SBC (Anexo \ref{sbc2017}) surgem como um guia para a elaboração de Projetos Pedagógicos de Curso  alinhados com as D\textit{iretrizes Curriculares Nacionais} (DCNs) e com as demandas da sociedade e do mercado de trabalho.

Primeiramente, é importante notar que os \textit{Referenciais de Formação da SBC} foram elaborados com base nas DCNs, especificamente a \textit{Resolução \ordm{n} 5 de 16 de novembro de 2016} (Anexo \ref{cne2016}), para servir como referência na elaboração de Projetos Pedagógicos para cursos de Bacharelado em Engenharia de Computação. Portanto, há uma forte correlação entre esses dois documentos.

As DCNs permitem que a formação em Engenharia de Computação siga tanto as próprias \textbf{Diretrizes da área da Computação} quanto as Diretrizes gerais para os cursos de Engenharia. Independentemente da linha adotada, espera-se que o projeto pedagógico defina o perfil do egresso, competências e habilidades, conteúdos curriculares, organização curricular, estágio, trabalho de curso (TCC), e atividades complementares. A carga horária mínima para os cursos de bacharelado na área de Computação, incluindo Engenharia de Computação, é de 3.200 horas.

Buscaremos, neste texto, demonstrar como as disciplinas ofertadas no curso dialogam com os eixos de formação, competências e conteúdos preconizados pela SBC, visando a formação de engenheiros de computação qualificados, críticos, criativos e éticos.

\section{Eixos de Formação}
Os Referenciais de \textbf{Formação da SBC para Engenharia de Computação }estão estruturados em cinco eixos principais, cada um com suas competências gerais e derivadas, além de um conjunto de conteúdos essenciais. A seguir, analisaremos cada um desses eixos e como as disciplinas do curso, conforme suas ementas (Anexo \ref{ementas}), contribuem para o desenvolvimento das competências propostas.

\section{Eixo 1: Fundamentos de Sistemas de Computação}
Este eixo visa desenvolver a competência geral de ``\textit{Lembrar e entender teorias e princípios da computação, matemática e ciências; aplicando estas teorias e princípios para resolver problemas técnicos de sistemas computacionais, incluindo sistemas de aplicação específica}''. Ele abrange uma base sólida em programação, teoria da computação, arquitetura de computadores, sistemas operacionais, matemática e física.

\rowcolors{2}{gray!10}{white}
\begin{small}
    \begin{longtable}{p{10cm} L{4.8cm}}
        \caption{Relação entre as competências do Eixo 1 da SBC e as disciplinas do curso} \label{eixo1}                                                                                                                                                                                                                                                            \\
        \toprule
        \textbf{Competência - Conteúdo (SBC)}                                                                                                                                                                                                                                                    & \textbf{Disciplinas}                                             \\
        \midrule
        \endfirsthead

        \multicolumn{2}{c}%
        {{\bfseries \tablename\ \thetable{} -- Continuação da página anterior}}                                                                                                                                                                                                                                                                                     \\
        \toprule
        \textbf{Competência - Conteúdo (SBC)}                                                                                                                                                                                                                                                    & \textbf{Disciplinas}                                             \\
        \midrule
        \endhead

        \midrule \multicolumn{2}{r}{{Continua na próxima página}}                                                                                                                                                                                                                                                                                                   \\
        \endfoot

        \bottomrule
        \endlastfoot

        C.1.1 - Aplicar os conceitos de programação imperativa e dominar o uso de abstrações de controle e dados, analisando o problema em questão para determinar trade\textit{off}s de memória e processamento ao aplicar diferentes estruturas de controle e de dados.                        & \AlgComp                                                         \\
        \addlinespace
        C.1.2 - Dominar noções básicas de teoria da computação, como lógica básica, complexidade de algoritmos, e linguagens formais e autômatos.                                                                                                                                                & \AnAlg, \LogProg, \TeoComp                                       \\
        \addlinespace
        C.1.3 - Elaborar sistemas considerando o mapeamento de programas para arquiteturas de computadores convencionais: representação de código e de dados, entrada e saída, geração de programas e sua execução. Analisar programas e avaliar o custo de aplicação de diferentes construções. & \FundComp, \ArqComp                                              \\
        \addlinespace
        C.1.4 - Criticar e escolher sistemas operacionais para contextos específicos, considerando como funcionam os principais componentes de cada sistema e os requisitos do contexto de aplicação.                                                                                            & \ProjSO                                                          \\
        \addlinespace
        C.1.5 - Avaliar o desempenho de sistemas computacionais usando técnicas teóricas e práticas de forma complementar.                                                                                                                                                                       & \AnAlg, \ArqComp                                                 \\
        \addlinespace
        C.1.6 - Dominar o ferramental matemático básico, da Engenharia compreendendo noções de cálculo e mapeá-lo para técnicas de cálculo numérico e métodos de matemática aplicada.                                                                                                            & \CalcI, II, III, \CalcNum                                        \\
        \addlinespace
        C.1.7 - Dominar conceitos de probabilidade e estatística e aplicá-los em diferentes contextos, como análise de desempenho ou estudo de capacidade.                                                                                                                                       & \ProbEst                                                         \\
        \addlinespace
        C.1.8 - Aplicar conceitos de matemática, como indução, combinatória e teoria de grafos, em diferentes situações e problemas.                                                                                                                                                             & \Grafos                                                          \\
        \addlinespace
        C.1.9 - Dominar conceitos básicos da física relacionados a eletricidade e magnetismo e transmissão de ondas.                                                                                                                                                                             & \FisI, II e IV, \FisEI, II e IV, \FisIII~e Experimental, \MatEle \\
        \addlinespace
        C.1.10 - Analisar e projetar circuitos eletrônicos simples, entendendo requisitos e \textit{tradeoffs}. Avaliar circuitos digitais usados em sistemas computacionais. Analisar os efeitos de características e estilos de projeto sobre temporização, desempenho e energia.              & \CCC, \CCA, \CircEletI, \TecDig                                  \\
        \addlinespace
        C.1.11 - Aplicar e integrar os conhecimentos teóricos aprendidos nas diferentes disciplinas na resolução de problemas práticos. Criar soluções para novos problemas e analisar os tradeoffs associados a soluções alternativas.                                                          & \LabProgA. \LabProgPOO, \Ext, \EstSup, \ProjB                    \\
    \end{longtable}
\end{small}
\section{Eixo 2: Desenvolvimento de Sistemas Computacionais}
A competência geral deste eixo é ``\textit{Criar, implementar e manter soluções computacionais eficientes para diversos tipos de problemas, envolvendo hardware, software e processos, analisando o espaço de projeto considerando restrições e custo-benefício; e criar e integrar componentes de hardware, de software e sua interface}''. Este eixo é o cerne da atividade prática do engenheiro de computação.

\rowcolors{2}{gray!10}{white}
\begin{small}
    \begin{longtable}{p{10cm} L{4.8cm}}
        \caption{Relação entre as competências do Eixo 2 da SBC e as disciplinas do curso} \label{eixo1}    \\
        \toprule
        \textbf{Competência - Conteúdo (SBC)}           & \textbf{Disciplinas}                              \\
        \midrule
        \endfirsthead

        \multicolumn{2}{c}%
        {{\bfseries \tablename\ \thetable{} -- Continuação da página anterior}}                             \\
        \toprule
        \textbf{Competência - Conteúdo (SBC)}           & \textbf{Disciplinas}                              \\
        \midrule
        \endhead

        \midrule \multicolumn{2}{r}{{Continua na próxima página}}                                           \\
        \endfoot

        \bottomrule
        \endlastfoot

        C.2.1 - Determinar os requisitos de desempenho e
        confiabilidade, projeto, implementação e teste de
        componentes eletrônicos e sistemas em hardware. & \ArqComp, \TecDig, \CCC, \CCA                     \\
        \addlinespace
        C.2.2 - Especificar e validar os requisitos, projetar,
        Especificar e validar os requisitos, projetar,
        soluções de software baseadas no conhecimento
        apropriado de teorias, modelos e técnicas.      & \EstrInf, \EngSistA, \AnaProjSist, \ProjBD        \\
        \addlinespace
        C.2.3 - Conhecer técnicas, arquiteturas e
        ferramentas para a seleção e integração otimizada
        de recursos de hardware e software. Para
        construção desta capacidade, são necessários
        conhecimentos em: sistemas operacionais,
        sistemas paralelos e distribuídos, programação de
        periféricos, sistemas em tempo real e sistemas
        embarcados.                                     & \ArqComp, \ProjSO, \CompParal, \SistEmb, \Control \\
        \addlinespace
        C.2.4 - Realizar o projeto de sistemas integrados de
        hardware e software para diversas áreas da
        indústria eletro-eletrônica. Esta capacitação
        envolve o conhecimento de áreas relacionadas a
        telecomunicações, redes de computadores,
        tratamento digital de sinais (para aplicações de
        tratamento de imagens, vídeo e áudio), e projeto de
        Controle e Automação de processos.              & \AnaProjSist, \SistEmb, \Control, \Telep          \\
    \end{longtable}
\end{small}

\section*{Eixo 3: Gerenciamento de Sistemas Computacionais}

Este eixo tem como objetivo desenvolver no estudante a capacidade de compreender, planejar e gerenciar projetos, serviços e experimentos de engenharia no contexto da Computação. As competências associadas incluem a elaboração e acompanhamento de projetos de software e hardware, o estudo de viabilidade técnica e econômica, o gerenciamento de recursos e equipes, bem como o entendimento das estruturas organizacionais e da governança corporativa.
\rowcolors{2}{gray!10}{white}
\begin{small}
    \begin{longtable}{p{10cm} L{4.8cm}}
        \caption{Relação entre as competências do Eixo 3 da SBC e as disciplinas do curso} \label{eixo1} \\
        \toprule
        \textbf{Competência - Conteúdo (SBC)}                        & \textbf{Disciplinas}              \\
        \midrule
        \endfirsthead

        \multicolumn{2}{c}%
        {{\bfseries \tablename\ \thetable{} -- Continuação da página anterior}}                          \\
        \toprule
        \textbf{Competência - Conteúdo (SBC)}                        & \textbf{Disciplinas}              \\
        \midrule
        \endhead

        \midrule \multicolumn{2}{r}{{Continua na próxima página}}                                        \\
        \endfoot

        \bottomrule
        \endlastfoot
        C.3.1 - Compreender conceitos relevantes sobre
        projetos, serviços e experimentos de engenharia na
        área de computação.                                          & \EngSistA, \AnaProjSist           \\
        \addlinespace
        C.3.2 - Compreender as estruturas organizacionais
        e os papéis relacionados ao desenvolvimento de
        projetos, serviços e experimentos de Engenharia de
        Computação.                                                  & \Adm, \MacroEco                   \\
        \addlinespace
        C.3.3 - Identificar normas e documentações
        técnicas necessárias em projetos, serviços e
        experimentos de Engenharia de Computação.                    & \EstSup, \ProjA, \ProjB           \\
        \addlinespace
        C.3.4 - Aplicar metodologias de gestão de projetos,
        serviços e experimentos de engenharia na área de computação. & \Adm, \Empre                      \\
    \end{longtable}
\end{small}

\section*{Eixo  4: Inovação e Empreendedorismo}
Com a competência geral de ``\textit{Criar ferramentas, técnicas e conhecimentos científicos e/ou tecnológicos inovadores na área, empreendendo na área de engenharia de computação, reconhecendo oportunidades e resolvendo problemas de forma a agregar valor à sociedade}'', este eixo é vital para a formação de um profissional transformador.

\rowcolors{2}{gray!10}{white}
\begin{small}
    \begin{longtable}{p{10cm} L{4.8cm}}
        \caption{Relação entre as competências do Eixo 4 da SBC e as disciplinas do curso} \label{eixo1} \\
        \toprule
        \textbf{Competência - Conteúdo (SBC)}              & \textbf{Disciplinas}                        \\
        \midrule
        \endfirsthead

        \multicolumn{2}{c}%
        {{\bfseries \tablename\ \thetable{} -- Continuação da página anterior}}                          \\
        \toprule
        \textbf{Competência - Conteúdo (SBC)}              & \textbf{Disciplinas}                        \\
        \midrule
        \endhead

        \midrule \multicolumn{2}{r}{{Continua na próxima página}}                                        \\
        \endfoot

        \bottomrule
        \endlastfoot
        C.4.1 - Entender a relação entre teoria e prática. & \Ext, \ProjA                                \\
        \addlinespace
        C.4.2 - Entender processos e questões
        relativos ao desenvolvimento de produto
        e sua manufatura.                                  & \Empre, \EstSup                             \\
        \addlinespace
        C.4.3 - Aplicar os fundamentos da economia na análise e no
        desenvolvimento de projetos de Engenharia de Computação, realizando
        estudos de viabilidade técnico-econômica, considerando o contexto
        social.                                            & \Adm, \Empre, \MacroEco                     \\
        \addlinespace
        C.4.4 - Integrar conceitos de áreas
        diferentes em um sistema completo para
        prover uma solução.                                & \ProjB, \Ext                                \\
        \addlinespace
        C.4.5 - Aplicar
        fundamentos
        administração na análise e
        desenvolvimento de projetos
        Engenharia de Computação.                          & \Adm                                        \\
        \addlinespace
        C.4.6  - Empreender e exercer liderança
        na sua área de atuação profissional.               & \Empre                                      \\
    \end{longtable}
\end{small}

\section*{Eixo  5: Desenvolvimento Pessoal e Profissional}
Este eixo busca ``\textit{Compreender a importância e responsabilidade da prática profissional, agindo de forma ética, sustentável e socialmente responsável, respeitando aspectos legais e normas envolvidas e observando direitos e propriedades intelectuais inerentes à produção e à utilização de sistemas de computação}".

\rowcolors{2}{gray!10}{white}
\begin{small}
    \begin{longtable}{p{10cm} L{4.8cm}}
        \caption{Relação entre as competências do Eixo 5 da SBC e as disciplinas do curso} \label{eixo1} \\
        \toprule
        \textbf{Competência - Conteúdo (SBC)} & \textbf{Disciplinas}                                     \\
        \midrule
        \endfirsthead

        \multicolumn{2}{c}%
        {{\bfseries \tablename\ \thetable{} -- Continuação da página anterior}}                          \\
        \toprule
        \textbf{Competência - Conteúdo (SBC)} & \textbf{Disciplinas}                                     \\
        \midrule
        \endhead

        \midrule \multicolumn{2}{r}{{Continua na próxima página}}                                        \\
        \endfoot

        \bottomrule
        \endlastfoot
        C.5.1 - Conhecer os direitos e propriedades
        intelectuais inerentes à produção e à
        utilização de sistemas de computação  & \EngCompSoc                                              \\
        \addlinespace
        C.5.2 - Compreender a importância da
        conduta ética e cidadã no exercício da
        Engenharia de Computação              & \EngCompSoc, Atividades Curriculares de Extensão         \\
        \addlinespace
        C.5.3 - Compreender o impacto que as
        soluções de sistemas de computação
        podem causar na sociedade e no meio
        ambiente.                             & \IntAmb                                                  \\
    \end{longtable}
\end{small}


