\chapter{Conformidade com as Diretrizes Nacionais e Referenciais da SBC}

Este capítulo detalha a conformidade do currículo de graduação em Engenharia de Computação da UERJ com as Diretrizes Curriculares Nacionais para os cursos de Computação e com os  Referenciais de Formação da Sociedade Brasileira de Computação.


\section{Carga Horária Total}

A carga horária total do curso de Engenharia de Computação da UERJ é de \tHorasCurso horas, atendendo ao valor mínimo estabelecido pelo Art. 11 das Diretrizes Curriculares Nacionais para os cursos de graduação na área da Computação, que determina uma carga horária mínima de 3200 horas.


\section{Eixos de Formação de Engenharia da Computação}

O currículo do curso de graduação em Engenharia da Computação da UERJ contempla plenamente os conteúdos básicos, específicos e profissionais mencionados nas Diretrizes Curriculares Nacionais dos cursos de Engenharia de Computação.

Os referenciais de formação para cursos de Engenharia de Computação são organizados em eixos de formação que visam garantir uma base sólida de conhecimentos técnicos, científicos e humanísticos. Esses eixos integram competências e conteúdos essenciais para o desenvolvimento de profissionais capazes de atuar em um mercado dinâmico e inovador.

Na Tabela \ref{tab:eixos} são apresentadas as disciplinas de conteúdos específicos e profissionais do curso distribuídas nos eixos de formação, competências e conteúdos que compõem os referenciais de formação para cursos de Engenharia de Computação.



\rowcolors{1}{gray!10}{white}
\begin{table}[ht]
  \centering
  \caption{Tabela das disciplinas de conteúdos específicos e profissionais de Engenharia da Computação por Eixos de formação.}
  \label{tab:eixos}
  \begin{tabular}{l l}
    %\hiderowcolors
    \hline
    {\bf Eixos de formação} & {\bf Disciplinas Específicas e Profissionais}          \\
    \hline
    \textbf{Fundamentos de Sistemas de Computação}
                            & - Algoritmos Computacionais I                          \\
                            & - Análise de Algoritmos I                              \\
                            & - Estruturas de Informação A                           \\
                            & - Lógica em Programação                                \\
                            & - Laboratório de Programação                           \\
                            & - Laboratório de Programação Orientada a Objetos       \\
                            & - Teoria dos Grafos e Aplicações                       \\
                            & - Teoria de Compiladores I                             \\ \hline
    \textbf{Desenvolvimento de Sistemas Computacionais}
                            & - Arquitetura de Computadores A                        \\
                            & - Computação Paralela e Distribuída                    \\
                            & - Engenharia de Sistemas                               \\
                            & - Lógica e Circuitos Digitais                          \\
                            & - Fundamentos de Computadores I                        \\
                            & - Instalação de Ambientes Computacionais               \\
                            & - Projeto de Sistemas Operacionais                     \\
                            & - Projeto de Banco de Dados                            \\
                            & - Redes de Computadores e Sistemas Distribuídos        \\
                            & - Segurança de Redes                                   \\
                            & - Sistemas Embutidos                                   \\
    \hline
    \textbf{Gerenciamento de sistemas computacionais}
                            & - Análise e Projeto de Sistemas                        \\

                            & - Inteligência Computacional I e II                    \\

                            & - Mineração de Dados                                   \\

                            & - Processamento de Sinais e Imagens                    \\

    \hline

    \textbf{Inovação e Empreendedorismo }
                            & - Empreendedorismo                                     \\

                            & - Macroeconomia aplicada à Engenharia                  \\

                            & - Administração Financeira de Projeto                  \\

    \hline

    \textbf{Desenvolvimento Pessoal e Profissional}
                            & - Engenharia de Computação e Sociedade                 \\

                            & - Introdução à Engenharia Ambiental                    \\

                            & - Estágio Supervisionado para Engenharia de Computação \\

    \hline
  \end{tabular}
\end{table}


\section{Projeto Final de Curso}
O Projeto Final de Curso, tal como descrito no Capítulo 4, Seção 4.6.7 deste documento, é
desenvolvido ao longo das disciplinas obrigatórias Metodologia Científica para Computação e Projeto de Graduação XI. A proposta descrita para a atividade no Capítulo 4 atende plenamente as orientações das DCNs de Engenharia da computação.

