\chapter{Conformidade com as Diretrizes Nacionais e os Referenciais da Sociedade Brasileira de Computação}

Este capítulo apresenta a análise da conformidade do currículo do curso de Engenharia de Computação da Universidade do Estado do Rio de Janeiro (UERJ) com as Diretrizes Curriculares Nacionais (DCNs) para os cursos da área de Computação, bem como com os Referenciais de Formação da Sociedade Brasileira de Computação (SBC).

Cabe destacar que os Referenciais de Formação da SBC foram desenvolvidos com base nas DCNs, em especial na Resolução CNE/CES \ordm{n} 5, de 16 de novembro de 2016, com o objetivo de subsidiar a formulação de Projetos Pedagógicos de cursos de Bacharelado em Engenharia de Computação. Dessa forma, observa-se uma forte correlação entre ambos os documentos.

Os Referenciais da SBC complementam as DCNs ao descrever de forma mais detalhada o perfil do egresso, apresentando um conjunto de competências alinhadas às diretrizes nacionais. Além disso, estruturam o conhecimento em Eixos de Formação, aos quais são associadas competências específicas e conteúdos recomendados.

Neste capítulo, realiza-se uma análise da aderência do currículo do curso de Engenharia de Computação da UERJ a esses eixos e competências, por meio do mapeamento das disciplinas previstas na matriz curricular às diretrizes estabelecidas nos documentos orientadores supracitados.

\section{Eixo de Formação: Fundamentos de Sistemas de Computação}

\rowcolors{1}{gray!10}{white}
\begin{table}
  \centering
  \begin{tabular}{p{5cm}p{9cm}}
    \hline
    \textbf{Competência - Conteúdo}                                                                                         & \textbf{Disciplinas}                                 \\
    \hline
    \textbf{C.1.1 - Técnicas de programação, Estruturas de dados, Paradigmas e padrões de
    programação }                                                                                                           & \AlgComp, \EstrInf, \LabProgA, \LabProgPOO, \LogProg \\
    \textbf{C.1.2 - Custo computacional e
    complexidade de algoritmos, Linguagens formais e autômatos, Lógica básica, Indução }                                    & \AnAlg, \LogProg                                     \\
    \textbf{C.1.3 - Representação de código e dados, Custo computacional, Compilação, ligação, carga,
    interpretação}                                                                                                          &                                                      \\
    \textbf{C.1.4 - Sistemas operacionais, Requisitos de sistemas}                                                          &                                                      \\
    \textbf{C.1.5 - Modelos de análise de desempenho, Simulação}                                                            &                                                      \\
    \textbf{C.1.6 - Cálculo numérico, Métodos de matemática aplicada, Provas matemáticas }                                  &                                                      \\
    \textbf{C.1.7 - Probabilidade e estatística }                                                                           &                                                      \\
    \textbf{C.1.8 - Matemática discreta}                                                                                    &                                                      \\
    \textbf{C.1.9 - Eletricidade e magnetismo, Transferência de calor}                                                      &                                                      \\
    \textbf{C.1.10 - Análise de circuitos elétricos, Eletrônica digital, Eletrônica geral}                                  &                                                      \\
    \textbf{C.1.11 - Laboratório de programação, Oficina de integração, Estágio integrado, Trabalho de conclusão de curso } &                                                      \\
  \end{tabular}
\end{table}

Este eixo tem como objetivo capacitar os estudantes a lembrar, compreender e aplicar teorias e princípios da Computação, Matemática e Ciências em geral para a solução de problemas técnicos envolvendo sistemas computacionais. As competências aqui desenvolvidas estão diretamente relacionadas à formação sólida e analítica exigida pelas Diretrizes Curriculares Nacionais (DCNs) e pelos Referenciais de Formação da Sociedade Brasileira de Computação (SBC).

As disciplinas do curso de Engenharia de Computação da UERJ que se relacionam com este eixo são as seguintes:

\begin{itemize}
  \item \textbf{\AlgComp:} Introduz os conceitos de algoritmos e linguagens de programação, abordando estruturas de dados compostas, como vetores e matrizes, e promovendo o desenvolvimento do pensamento lógico. Está diretamente alinhada com os conteúdos de \textit{Algoritmos e Estrutura de Dados} e \textit{Lógica básica} dos referenciais da SBC, bem como com a competência das DCNs de identificar problemas solucionáveis algoritmicamente.

  \item \textbf{\LogProg:} Apresenta noções de lógica de primeira ordem e formalismos para representação e raciocínio em Computação. Corresponde aos conteúdos de \textit{Lógica básica} e \textit{Lógica para computação} nos referenciais da SBC.

  \item \textbf{\LogCircDig:} Aborda fundamentos de lógica matemática, lógica digital e a construção de circuitos combinacionais e sequenciais. Está relacionada aos conteúdos de \textit{Lógica básica} e \textit{Eletrônica digital}, conforme os referenciais da SBC.

  \item \textbf{\FundComp:} Estuda os conceitos e características lógicas e físicas de um processador. Relaciona-se com os conteúdos de \textit{Mapeamento de programas para arquiteturas convencionais}, \textit{Eletrônica digital}, e com a competência das DCNs referente ao conhecimento do funcionamento e das especificações técnicas de hardware.

  \item \textbf{\ArqComp:} Dá continuidade à disciplina \FundComp, aprofundando o estudo das características lógicas e físicas da arquitetura de computadores digitais. Relaciona-se aos conteúdos de \textit{Mapeamento de programas para arquiteturas}, \textit{Eletrônica digital}, e à competência das DCNs para análise e avaliação de arquiteturas computacionais.

  \item \textbf{\CCC~e \CCA:} Disciplinas que abordam a análise de circuitos elétricos em diferentes regimes. A primeira requer Física Teórica III como pré-requisito, enquanto a segunda exige conhecimento prévio em Circuitos em Corrente Contínua e Sinais e Sistemas. Ambas correspondem aos conteúdos de \textit{Eletricidade e magnetismo} e \textit{Análise de circuitos elétricos}, essenciais para a formação em Eletrônica conforme previsto nas DCNs.

  \item \textbf{\TeoComp:} Apresenta as bases formais da Ciência da Computação e as principais técnicas de construção de compiladores. Está alinhada aos conteúdos de \textit{Linguagens formais e autômatos} e à competência das DCNs de compreensão de conceitos e teorias fundamentais da área de Computação.
\end{itemize}

\section{Eixo de Formação: Desenvolvimento de Sistemas Computacionais}

Este eixo contempla a criação, implementação e manutenção de soluções computacionais eficientes, integrando hardware, software e processos. Os conteúdos associados abrangem tópicos fundamentais de programação, estruturas de dados, sistemas operacionais, redes de computadores, segurança, sistemas embarcados e aplicações avançadas, como inteligência artificial e robótica. Está alinhado aos conteúdos indicados pela SBC, como \textit{Algoritmos e estruturas de dados}, \textit{Programação orientada a objetos}, \textit{Sistemas operacionais}, \textit{Redes de computadores}, \textit{Sistemas paralelos e distribuídos}, \textit{Sistemas embarcados} e \textit{Sistemas em tempo real}, bem como às competências das DCNs, que incluem projetar, implementar, administrar e manter sistemas computacionais robustos e seguros.



\begin{itemize}
  \item \textbf{Algoritmos Computacionais I:} Introduz os fundamentos da programação e a implementação de algoritmos em linguagens computacionais. Alinha-se aos conteúdos de \textit{Algoritmos e estruturas de dados} e \textit{Laboratório de programação}, contribuindo para a competência das DCNs de resolver problemas computacionais utilizando ambientes de desenvolvimento.

  \item \textbf{Laboratório de Programação} e \textbf{Laboratório de Programação Orientada a Objetos:} Disciplinas com forte ênfase prática, promovendo o domínio da implementação de sistemas por meio de técnicas de programação estruturada e orientada a objetos, conforme previsto na SBC e nas competências das DCNs voltadas ao desenvolvimento de soluções computacionais.

  \item \textbf{Estruturas de Informação A:} Aborda a modelagem, organização e manipulação de dados, além de conceitos de bancos de dados relacionais e SQL. Relaciona-se com os tópicos de \textit{Gerência, organização e recuperação da informação} e \textit{Algoritmos e estruturas de dados} da SBC, atendendo à competência das DCNs de gerenciar o armazenamento e o acesso eficiente à informação.

  \item \textbf{Projeto de Sistemas Operacionais:} Introduz o projeto e a implementação de sistemas operacionais. Relaciona-se diretamente ao conteúdo de \textit{Sistemas operacionais} e fornece base para o desenvolvimento em \textit{Sistemas paralelos e distribuídos}, conforme a SBC.

  \item \textbf{Redes de Computadores e Sistemas Distribuídos:} Estuda redes de comunicação, protocolos, internet e conceitos de sistemas distribuídos. Alinha-se aos conteúdos de \textit{Redes de computadores}, \textit{Sistemas distribuídos} e \textit{Programação distribuída}, e às competências das DCNs de projetar, implantar e gerenciar redes e soluções de comunicação.

  \item \textbf{Segurança de Redes de Computadores:} Trata de fundamentos de segurança em redes, como criptografia, controle de acesso e proteção contra ameaças. Relaciona-se com \textit{Confiabilidade e segurança de software} nos referenciais da SBC.

  \item \textbf{Sistemas Embarcados:} Explora a especificação, modelagem e implementação de sistemas embarcados. Relaciona-se aos conteúdos de \textit{Sistemas embarcados} e \textit{Sistemas em tempo real}, e às competências das DCNs de projetar sistemas baseados em microcontroladores e plataformas específicas.

  \item \textbf{Computação Paralela e Distribuída:} Introduz técnicas e paradigmas de programação paralela e distribuída. Alinha-se a \textit{Sistemas paralelos e distribuídos} e \textit{Programação paralela e distribuída}, e contribui para a competência das DCNs de desenvolver e otimizar software para plataformas com múltiplos núcleos ou distribuídas.

  \item \textbf{Teoria dos Grafos e Aplicações:} Estuda representações e algoritmos relacionados a grafos e suas aplicações computacionais. Relaciona-se com \textit{Algoritmos e estruturas de dados} e \textit{Matemática discreta}, ampliando a capacidade analítica e de resolução algorítmica de problemas.

  \item \textbf{Disciplinas Eletivas Avançadas:} As disciplinas \textit{Inteligência Computacional I e II}, \textit{Mineração de Dados}, \textit{Aprendizado por Reforço}, \textit{Aprendizado Profundo para Processamento de Linguagem Natural}, \textit{Aprendizado Profundo para Visão Computacional}, \textit{Automação de Processos Robóticos}, \textit{Robótica} e \textit{Geomática} tratam de temas avançados como redes neurais, aprendizado de máquina, sistemas multiagentes, RPA, sistemas georreferenciados e robótica. Essas disciplinas ampliam a formação nas áreas de \textit{Inteligência artificial}, \textit{Sistemas inteligentes}, \textit{Computação evolutiva}, \textit{Sistemas fuzzy} e \textit{Sistemas de informação}, conforme os referenciais da SBC. Contribuem diretamente para as competências das DCNs de aplicar a computação a domínios diversos, projetando soluções inovadoras e interdisciplinares.
\end{itemize}

\section*{Eixo de Formação: Gerenciamento e Liderança}

Este eixo tem como objetivo desenvolver no estudante a capacidade de compreender, planejar e gerenciar projetos, serviços e experimentos de engenharia no contexto da Computação. As competências associadas incluem a elaboração e acompanhamento de projetos de software e hardware, o estudo de viabilidade técnica e econômica, o gerenciamento de recursos e equipes, bem como o entendimento das estruturas organizacionais e da governança corporativa.

As disciplinas que contribuem para esse eixo incluem:

\begin{itemize}
  \item \textbf{Análise e Projeto de Sistemas:} Aborda processos de engenharia de software, incluindo levantamento e especificação de requisitos, modelagem de sistemas, e princípios de projeto. Relaciona-se diretamente com os conteúdos descritos nos referenciais da SBC, tais como \textit{Técnicas para Especificação de Requisitos}, \textit{Ciclo de vida de produtos de software e hardware} e \textit{Ciclo de gerenciamento de projetos}. Contribui para a competência estabelecida nas DCNs de planejar, especificar e projetar sistemas de computação.

  \item \textbf{Engenharia de Sistemas:} Contempla temas como teoria geral de sistemas, estudo de viabilidade de projetos, técnicas de levantamento de dados, integração de sistemas, qualidade da informação, governança corporativa e gestão de projetos. Requer a disciplina Laboratório de Programação Orientada a Objetos como pré-requisito. Está alinhada aos conteúdos dos referenciais da SBC, especialmente \textit{Gerenciamento do andamento de projetos}, \textit{Estudo de viabilidade técnico-econômica} e \textit{Estrutura organizacional e papéis}. Também reforça competências previstas nas DCNs relativas à gestão e coordenação de equipes e processos.

  \item \textbf{Metodologia Científica para Computação:} Embora com foco na pesquisa científica, a disciplina inclui planejamento e elaboração de projetos, aplicação de normas técnicas e questões éticas. Relaciona-se ao conteúdo dos referenciais da SBC sobre \textit{Elaboração de documentação técnica} e contribui para a formação gerencial e organizacional esperada no perfil do egresso.
\end{itemize}

\section*{Eixo de Formação: Inovação e Empreendedorismo}

Este eixo tem como finalidade fomentar no estudante a capacidade de reconhecer oportunidades, propor soluções inovadoras e agregar valor a produtos, processos e serviços. São desenvolvidas habilidades ligadas à criatividade, iniciativa, visão estratégica e atuação empreendedora, em consonância com as demandas da sociedade e do setor produtivo.

As disciplinas que sustentam este eixo incluem:

\begin{itemize}
  \item \textbf{Engenharia de Computação e Sociedade:} Trata de temas como inovação, empreendedorismo e o papel do engenheiro na sociedade. Relaciona-se diretamente com os conteúdos dos referenciais da SBC sobre \textit{Conceitos de empreendedorismo}, \textit{Inovação}, e a atuação profissional com responsabilidade social. Contribui para as competências das DCNs de empreender e reconhecer o caráter fundamental da criatividade e da inovação.

  \item \textbf{Engenharia de Sistemas:} Além de sua ênfase em planejamento e integração de sistemas, contempla o estudo de viabilidade técnica e econômica de projetos e a aplicação de técnicas como \textit{benchmarking} e \textit{brainstorming}, que estimulam a inovação. Alinha-se aos conteúdos de \textit{Estudo de viabilidade técnico-econômica} e \textit{Conceitos de empreendedorismo} dos referenciais da SBC, além de reforçar as competências ligadas ao planejamento e liderança de iniciativas inovadoras.

  \item \textbf{Metodologia Científica para Computação:} Ao propor o desenvolvimento de um projeto de Engenharia desde a etapa inicial, estimula a aplicação criativa de conhecimentos, incentivando a investigação de soluções novas e a integração de saberes. Relaciona-se com a competência de aplicar o conhecimento científico e tecnológico para gerar inovação.

  \item \textbf{Estágio Supervisionado e Projeto de Graduação XI:} O Estágio Supervisionado proporciona ao aluno vivência prática em ambientes profissionais, favorecendo a identificação de oportunidades de inovação. O Projeto de Graduação XI, que tem como pré-requisito a disciplina de Metodologia Científica para Computação, constitui uma atividade de síntese, integração e aplicação dos conhecimentos adquiridos, incentivando a proposição de soluções criativas e empreendedoras. Ambas as atividades contribuem para a competência de empreender, exercer liderança e aplicar inovação de forma responsável e estratégica.
\end{itemize}

\section*{Eixo de Formação: Desenvolvimento Pessoal e Profissional}

Este eixo contempla o desenvolvimento da postura ética, crítica e responsável do futuro profissional de Engenharia de Computação, destacando a importância da atuação comprometida com os aspectos legais, sociais, ambientais e culturais da prática profissional. Visa promover a compreensão do impacto das tecnologias na sociedade e o respeito aos direitos e deveres associados ao exercício da profissão.

As disciplinas relacionadas a esse eixo são:

\begin{itemize}
  \item \textbf{Engenharia de Computação e Sociedade:} Discute a função social do engenheiro, a ética na prática profissional e os impactos da engenharia na sociedade e no meio ambiente. Abrange temas como \textit{Ética e cidadania}, \textit{Impacto da computação na sociedade}, e \textit{Legislação aplicada à informática}, conforme os referenciais da SBC. Contribui diretamente para as competências previstas nas DCNs de compreender os aspectos sociais, legais, éticos, ambientais e políticos da computação, bem como de atuar com responsabilidade profissional e social. A disciplina também aborda propriedade intelectual, conectando-se à competência de conhecer os direitos e propriedades intelectuais relacionados ao desenvolvimento e uso de sistemas computacionais.

  \item \textbf{Metodologia Científica para Computação:} Introduz e reforça princípios éticos no contexto da pesquisa científica, o que fortalece a formação ética e a responsabilidade na produção e aplicação do conhecimento técnico e científico.

  \item \textbf{Estágio Supervisionado:} Proporciona a vivência no ambiente profissional real, permitindo ao aluno refletir sobre sua atuação na área de Engenharia de Computação e avaliar criticamente os impactos de suas atividades na sociedade. Reforça o compromisso com a ética, a sustentabilidade e a responsabilidade social, conforme as competências estabelecidas pelas DCNs.
\end{itemize}

\section*{Considerações Finais}

Além da organização por eixos de formação, destaca-se que tanto as Diretrizes Curriculares Nacionais (DCNs) quanto os Referenciais de Formação da Sociedade Brasileira de Computação (SBC) enfatizam a importância do Estágio Curricular Supervisionado, o qual está devidamente contemplado na estrutura curricular apresentada.

De maneira semelhante, o Trabalho de Conclusão de Curso (TCC), representado no conjunto de ementas pela disciplina \textit{Projeto de Graduação XI} (cujo pré-requisito é \textit{Metodologia Científica para Computação}), constitui uma atividade de síntese e integração de conhecimentos. Trata-se de um componente curricular fundamental, mencionado explicitamente nos referenciais da SBC como parte do eixo \textit{Fundamentos}.

As \textit{Atividades Complementares}, igualmente previstas no currículo, contribuem para a ampliação da vivência acadêmica e para o desenvolvimento de competências adicionais, conforme recomendado pelas DCNs e pelos referenciais da SBC.

Ressalta-se, ainda, que a estrutura das disciplinas, conforme apresentada nas ementas, contempla cargas horárias distribuídas entre atividades teóricas e práticas/laboratoriais, em consonância com a orientação dos referenciais da SBC no sentido de integrar teoria e prática por meio de metodologias ativas de ensino-aprendizagem.

Em síntese, o conjunto de disciplinas descrito no documento de ementas atende aos conteúdos básicos e tecnológicos requeridos para a formação em Engenharia de Computação, conforme estabelecido pelas Diretrizes Curriculares Nacionais e detalhado pelos Referenciais de Formação da SBC. A organização curricular abrange de maneira sistemática os eixos de \textit{Fundamentos}, \textit{Desenvolvimento}, \textit{Gerenciamento e Liderança}, \textit{Inovação e Empreendedorismo} e \textit{Desenvolvimento Pessoal e Profissional}, além de incorporar componentes formativos essenciais como o Estágio Curricular Supervisionado, o TCC e as Atividades Complementares.

\begin{table}[ht]
  \centering
  \caption{Resumo dos Eixos de Formação do Curso}
  \begin{tabularx}{\textwidth}{>{\bfseries}l X X}
    \toprule
    \textbf{Eixo}                                                                                                                                      & \textbf{Finalidade} & \textbf{Disciplinas Relacionadas} \\
    \midrule
    Fundamentos                                                                                                                                        &
    Consolidar a base teórica em ciência e engenharia da computação, incluindo raciocínio lógico, matemático e conhecimentos fundamentais.             &
    Matemática Discreta, Lógica para Computação, Fundamentos de Física, Cálculo, Álgebra Linear, Estruturas de Dados, Arquitetura de Computadores                                                                \\
    \addlinespace

    Desenvolvimento                                                                                                                                    &
    Desenvolver competências práticas para o projeto e construção de soluções computacionais, com foco em técnicas, linguagens e plataformas.          &
    Programação, Laboratórios de Programação, Estruturas de Dados, Sistemas Operacionais, Computação Gráfica, Sistemas Embarcados, Redes de Computadores                                                         \\
    \addlinespace

    Gerenciamento e Liderança                                                                                                                          &
    Capacitar o estudante para planejar, coordenar e gerenciar projetos de engenharia, com foco em organização, requisitos e viabilidade.              &
    Análise e Projeto de Sistemas, Engenharia de Sistemas, Metodologia Científica para Computação                                                                                                                \\
    \addlinespace

    Inovação e Empreendedorismo                                                                                                                        &
    Estimular a criatividade, a inovação e o reconhecimento de oportunidades, com foco em agregar valor e propor soluções originais.                   &
    Engenharia de Computação e Sociedade, Engenharia de Sistemas, Metodologia Científica, Estágio Supervisionado, Projeto de Graduação XI                                                                        \\
    \addlinespace

    Desenvolvimento Pessoal e Profissional                                                                                                             &
    Promover a consciência ética, legal, social e ambiental da prática profissional, desenvolvendo o compromisso com a cidadania e a sustentabilidade. &
    Engenharia de Computação e Sociedade, Metodologia Científica para Computação, Estágio Supervisionado                                                                                                         \\
    \bottomrule
  \end{tabularx}
\end{table}

\section{Carga Horária Total}

A carga horária total do curso de Engenharia de Computação da UERJ é de \tHorasCurso horas, atendendo ao valor mínimo estabelecido pelo Art. 11 das Diretrizes Curriculares Nacionais para os cursos de graduação na área da Computação, que determina uma carga horária mínima de 3200 horas.


\section{Eixos de Formação de Engenharia da Computação}

O currículo do curso de graduação em Engenharia da Computação da UERJ contempla plenamente os conteúdos básicos, específicos e profissionais mencionados nas Diretrizes Curriculares Nacionais dos cursos de Engenharia de Computação.

Os referenciais de formação para cursos de Engenharia de Computação são organizados em eixos de formação que visam garantir uma base sólida de conhecimentos técnicos, científicos e humanísticos. Esses eixos integram competências e conteúdos essenciais para o desenvolvimento de profissionais capazes de atuar em um mercado dinâmico e inovador.

Na Tabela \ref{tab:eixos} são apresentadas as disciplinas de conteúdos específicos e profissionais do curso distribuídas nos eixos de formação, competências e conteúdos que compõem os referenciais de formação para cursos de Engenharia de Computação.



\rowcolors{1}{gray!10}{white}
\begin{table}[ht]
  \centering
  \caption{Tabela das disciplinas de conteúdos específicos e profissionais de Engenharia da Computação por Eixos de formação.}
  \label{tab:eixos}
  \begin{tabular}{l l}
    %\hiderowcolors
    \hline
    {\bf Eixos de formação} & {\bf Disciplinas Específicas e Profissionais}          \\
    \hline
    \textbf{Fundamentos de Sistemas de Computação}
                            & - Algoritmos Computacionais I                          \\
                            & - Análise de Algoritmos I                              \\
                            & - Estruturas de Informação A                           \\
                            & - Lógica em Programação                                \\
                            & - Laboratório de Programação                           \\
                            & - Laboratório de Programação Orientada a Objetos       \\
                            & - Teoria dos Grafos e Aplicações                       \\
                            & - Teoria de Compiladores I                             \\ \hline
    \textbf{Desenvolvimento de Sistemas Computacionais}
                            & - Arquitetura de Computadores A                        \\
                            & - Computação Paralela e Distribuída                    \\
                            & - Engenharia de Sistemas                               \\
                            & - Lógica e Circuitos Digitais                          \\
                            & - Fundamentos de Computadores I                        \\
                            & - Instalação de Ambientes Computacionais               \\
                            & - Projeto de Sistemas Operacionais                     \\
                            & - Projeto de Banco de Dados                            \\
                            & - Redes de Computadores e Sistemas Distribuídos        \\
                            & - Segurança de Redes                                   \\
                            & - Sistemas Embutidos                                   \\
    \hline
    \textbf{Gerenciamento de sistemas computacionais}
                            & - Análise e Projeto de Sistemas                        \\

                            & - Inteligência Computacional I e II                    \\

                            & - Mineração de Dados                                   \\

                            & - Processamento de Sinais e Imagens                    \\

    \hline

    \textbf{Inovação e Empreendedorismo }
                            & - Empreendedorismo                                     \\

                            & - Macroeconomia aplicada à Engenharia                  \\

                            & - Administração Financeira de Projeto                  \\

    \hline

    \textbf{Desenvolvimento Pessoal e Profissional}
                            & - Engenharia de Computação e Sociedade                 \\

                            & - Introdução à Engenharia Ambiental                    \\

                            & - Estágio Supervisionado para Engenharia de Computação \\

    \hline
  \end{tabular}
\end{table}


\section{Projeto Final de Curso}
O Projeto Final de Curso, tal como descrito no Capítulo 4, Seção 4.6.7 deste documento, é
desenvolvido ao longo das disciplinas obrigatórias Metodologia Científica para Computação e Projeto de Graduação XI. A proposta descrita para a atividade no Capítulo 4 atende plenamente as orientações das DCNs de Engenharia da computação.

