\chapter{Conformidade com as Diretrizes Nacionais}
\thispagestyle{plain}

Este capítulo tem como objetivo apresentar uma análise da relação entre os R\textit{eferenciais de Formação para os Cursos de Graduação em Engenharia de Computação}, propostos pela \textbf{Sociedade Brasileira de Computação (SBC)} em 2017, e a estrutura curricular do curso proposto.

A Engenharia de Computação é uma área dinâmica e em constante evolução, exigindo uma formação sólida que contemple tanto os fundamentos teóricos quanto as habilidades práticas necessárias para o desenvolvimento de soluções inovadoras que integrem hardware e software. Os referenciais da SBC surgem como um guia para a elaboração de Projetos Pedagógicos de Curso (PPCs) alinhados com as D\textit{iretrizes Curriculares Nacionais} (DCNs) e com as demandas da sociedade e do mercado de trabalho.

Buscaremos, neste texto, demonstrar como as disciplinas ofertadas no curso dialogam com os eixos de formação, competências e conteúdos preconizados pela SBC, visando a formação de engenheiros de computação qualificados, críticos, criativos e éticos.

\section{Eixos de Formação}
Os Referenciais de \textbf{Formação da SBC para Engenharia de Computação }estão estruturados em cinco eixos principais, cada um com suas competências gerais e derivadas, além de um conjunto de conteúdos essenciais. A seguir, analisaremos cada um desses eixos e como as disciplinas do curso, conforme suas ementas (Anexo \ref{ementas}), contribuem para o desenvolvimento das competências propostas.

\section{Eixo 1: Fundamentos de Sistemas de Computação}
Este eixo tem como competência geral esperada ``\textit{Lembrar e entender teorias e princípios da computação, matemática e ciências; aplicando estas teorias e princípios para resolver problemas técnicos de sistemas computacionais, incluindo sistemas de aplicação específica}''. Ele abrange uma base sólida em programação, teoria da computação, arquitetura de computadores, sistemas operacionais, matemática e física.

A disciplina \textbf{\AlgComp}~ contribui diretamente para a competência \textbf{C.1.1} (``\textit{Aplicar os conceitos de programação imperativa e dominar o uso de abstrações de controle e dados, analisando o problema em questão para determinar tradeoffs de memória e processamento ao aplicar diferentes estruturas de controle e de dados}''). Sua ementa fornece os alicerces para o desenvolvimento do pensamento lógico e da capacidade de traduzir problemas em soluções algorítmicas. A ênfase em técnicas de estruturação de programas também é fundamental.

A competência \textbf{C.1.2} ("\textit{Dominar noções básicas de teoria da computação, como lógica básica, complexidade de algoritmos, e linguagens formais e autômatos}") é abordada pela disciplina \textbf{\AnAlg}. A ementa desta disciplina é crucial para que o estudante compreenda a eficiência e os limites da computação. A disciplina \textbf{\LogProg}, mencionada como pré-requisito para \textbf{\AnAlg}, também é fundamental para o desenvolvimento da lógica básica exigida.

No que tange à \textbf{C.1.3} ("\textit{Elaborar sistemas considerando o mapeamento de programas para arquiteturas de computadores convencionais: representação de código e de dados, entrada e saída, geração de programas e sua execução. Analisar programas e avaliar o custo de aplicação de diferentes construções}") e \textbf{C.1.4} ("\textit{Criticar e escolher sistemas operacionais para contextos específicos, considerando como funcionam os principais componentes de cada sistema e os requisitos do contexto de aplicação}"), a disciplina \textbf{\ArqComp} desempenha um papel central. Sua ementa permite ao aluno entender a interação entre software e hardware.  Os conceitos de arquitetura são precursores indispensáveis para a compreensão de sistemas operacionais, tema abordado na disciplina \textbf{\ProjSO}, que complementa a competência \textbf{C.1.4}.

A avaliação de desempenho (\textbf{C.1.5}) é um tema transversal, tocado em \textbf{\ArqComp} e aprofundado em \textbf{\AnAlg}, onde se aprende a "\textit{avaliar e comparar a eficiência computacional de algoritmos}".

O ferramental matemático (\textbf{C.1.6}, \textbf{C.1.7}, \textbf{C.1.8}) é construído por meio de disciplinas como \textbf{\CalcI}, \textbf{\CalcII}, \textbf{\CalcIII} e \textbf{\AlgLin}. A ementa de \textbf{\AlgLin} fornece a base para diversas aplicações em computação, como processamento de imagens e aprendizado de máquina. O Cálculo, por sua vez, é fundamental para a modelagem e análise de sistemas contínuos e discretos. Completa o conjunto a disciplina \ProbEst~ essencial para cumprir integralmente estas competências.

Os conceitos de física (\textbf{C.1.9}) são fundamentais para a Engenharia de Computação. Nesse grupo diversas disciplinas cobrem o conteúdo, promovendo uma forte base teórica e prática: \textbf{\FisI}, \textbf{\FisEI}, \textbf{\FisII}, \textbf{\FisEII}, \textbf{\FisIII}, \textbf{\FisEIII}, \textbf{\FisIV}, \textbf{\FisEIV}, \textbf{\MatEle}. A análise e projeto de circuitos (\textbf{C.1.10}), incluindo eletrônica digital, assim como o conhecimento em circuitos elétricos é crucial e é coberta em disciplinas específicas: \textbf{\CircEle}, \textbf{\TecDig}, \textbf{\CCC~} e \textbf{\CCA}.

A competência integradora \textbf{C.1.11} ("\textit{Aplicar e integrar os conhecimentos teóricos aprendidos nas diferentes disciplinas na resolução de problemas práticos. Criar soluções para novos problemas e analisar os tradeoffs associados a soluções alternativas}") é desenvolvida progressivamente ao longo do curso, mas particularmente nas disciplinas \LabProgA e LabProgB, além de seu uso prático nos Projetos de Extensão, \EstSup e no Trabalho de Conclusão de Curso, conforme preconizado pela SBC.

\section{Eixo 2: Desenvolvimento de Sistemas Computacionais}
A competência geral deste eixo é "\textit{Criar, implementar e manter soluções computacionais eficientes para diversos tipos de problemas, envolvendo hardware, software e processos, analisando o espaço de projeto considerando restrições e custo-benefício; e criar e integrar componentes de hardware, de software e sua interface}". Este eixo é o cerne da atividade prática do engenheiro de computação.

A competência \textbf{C.2.1}, que trata de "\textit{Determinar os requisitos de desempenho e confiabilidade, projeto, implementação e teste de componentes eletrônicos e sistemas em hardware}", encontra respaldo em disciplinas como \textbf{\ArqComp}. Embora as ementas fornecidas não detalhem disciplinas específicas sobre projeto de circuitos ou microprocessadores de forma aprofundada, a disciplina Fundamentos de Computadores I, como pré-requisito, certamente introduz os elementos básicos para o entendimento de hardware, que são aprofundados em Arquitetura. A formação completa nesta competência exigiria um aprofundamento em eletrônica analógica, digital e microeletrônica, que são componentes curriculares esperados em um curso de Engenharia de Computação.

Para a competência C.2.2 ("Especificar e validar os requisitos, projetar, implementar, verificar, implantar e documentar soluções de software baseadas no conhecimento apropriado de teorias, modelos e técnicas"), disciplinas como Algoritmos Computacionais I, Análise de Algoritmos I, e, de forma central, Análise e Projeto de Sistemas são fundamentais. A ementa de Análise e Projeto de Sistemas, que inclui "Projeto de sistemas: processos, engenharia de requisitos, modelos, princípios do projeto", aborda diretamente as fases iniciais e cruciais do ciclo de vida do desenvolvimento de software. A implementação é praticada desde Algoritmos Computacionais I, e a análise de eficiência em Análise de Algoritmos I. A documentação e verificação são aspectos que devem permear todas as disciplinas de projeto.

A competência C.2.3 ("Conhecer técnicas, arquiteturas e ferramentas para a seleção e integração otimizada de recursos de hardware e software") é desenvolvida com o suporte de Arquitetura de Computadores A, que permite entender a organização interna dos computadores. A integração otimizada de hardware e software é um tema complexo que também se beneficia do entendimento de sistemas operacionais, compiladores e sistemas embarcados, conteúdos que, embora não explicitamente detalhados nas ementas fornecidas, são essenciais e podem estar contemplados em disciplinas subsequentes ou eletivas.

Finalmente, a C.2.4 ("Realizar o projeto de sistemas integrados de hardware e software para diversas áreas da indústria eletro-eletrônica") representa a aplicação integrada dos conhecimentos. Análise e Projeto de Sistemas contribui para a parte de software e integração. A capacidade de projetar sistemas integrados é um dos diferenciais do engenheiro de computação e é consolidada através de projetos práticos, laboratórios e, idealmente, disciplinas que abordem sistemas embarcados, controle e automação, e telecomunicações, conforme as vocações do curso e as demandas regionais.


\section*{Eixo 3: Gerenciamento de Sistemas Computacionai}

Este eixo tem como objetivo desenvolver no estudante a capacidade de compreender, planejar e gerenciar projetos, serviços e experimentos de engenharia no contexto da Computação. As competências associadas incluem a elaboração e acompanhamento de projetos de software e hardware, o estudo de viabilidade técnica e econômica, o gerenciamento de recursos e equipes, bem como o entendimento das estruturas organizacionais e da governança corporativa.

Este eixo visa "Gerenciar projetos, serviços e experimentos de engenharia na área de computação, de forma colaborativa em equipes multidisciplinares e em grupos sociais". A disciplina Administração Financeira de Projeto contribui significativamente para este eixo, especialmente no que tange à viabilidade e ao financiamento de projetos, aspectos cruciais do gerenciamento.

A competência C.3.1 ("Compreender conceitos relevantes sobre projetos, serviços e experimentos de engenharia na área de computação") e C.3.4 ("Aplicar metodologias de gestão de projetos, serviços e experimentos de engenharia na área de computação") são parcialmente abordadas pela disciplina mencionada. A ementa de Administração Financeira de Projeto ("CONCEITOS BÁSICOS DE MATEMÁTICA FINANCEIRA... INTRODUÇÃO A AVALIAÇÃO DE INVESTIMENTOS... VIABILIDADE DE UM PROJETO... FINANCIAMENTO DE PROJETOS") foca nos aspectos econômico-financeiros da gestão. Para uma cobertura completa das competências de gerenciamento de projetos, incluindo ciclo de vida, escopo, tempo, riscos e qualidade, outras disciplinas ou módulos específicos em gerenciamento de projetos de software e hardware seriam complementares e desejáveis, alinhando-se com os conteúdos de "Ciclo de gerenciamento de projetos" e "Ferramentas para gestão de projetos" listados pela SBC.

As competências C.3.2 ("Compreender as estruturas organizacionais e os papéis relacionados ao desenvolvimento de projetos") e C.3.3 ("Identificar normas e documentações técnicas necessárias em projetos") são desenvolvidas não apenas em disciplinas teóricas, mas também através da vivência em projetos práticos ao longo do curso e, fundamentalmente, no estágio supervisionado e no Trabalho de Conclusão de Curso.

Eixo 4: Inovação e Empreendedorismo
Com a competência geral de "Criar ferramentas, técnicas e conhecimentos científicos e/ou tecnológicos inovadores na área, empreendendo na área de engenharia de computação, reconhecendo oportunidades e resolvendo problemas de forma a agregar valor à sociedade", este eixo é vital para a formação de um profissional transformador.

A disciplina Administração Financeira de Projeto também se relaciona com este eixo, particularmente com a competência C.4.3 ("Aplicar os fundamentos da economia na análise e no desenvolvimento de projetos de Engenharia de Computação, realizando estudos de viabilidade técnico-econômica") e, indiretamente, com C.4.5 ("Aplicar fundamentos da administração na análise e desenvolvimento de projetos de Engenharia de Computação"). O estudo de viabilidade é um passo essencial no processo de inovação e empreendedorismo.

As demais competências, como C.4.1 ("Entender a relação entre teoria e prática"), C.4.2 ("Entender processos e questões relativos ao desenvolvimento de produto e sua manufatura"), C.4.4 ("Integrar conceitos de áreas diferentes em um sistema completo para prover uma solução") e C.4.6 ("Empreender e exercer liderança na sua área de atuação profissional"), são fomentadas por meio de metodologias ativas de ensino, projetos integradores, participação em atividades de pesquisa e extensão, e, de forma mais direcionada, por disciplinas eletivas focadas em empreendedorismo, desenvolvimento de novos produtos e gestão da inovação, que complementariam a formação básica oferecida.

Eixo 5: Desenvolvimento Pessoal e Profissional
Este eixo busca "Compreender a importância e responsabilidade da prática profissional, agindo de forma ética, sustentável e socialmente responsável, respeitando aspectos legais e normas envolvidas e observando direitos e propriedades intelectuais inerentes à produção e à utilização de sistemas de computação".

As ementas fornecidas não especificam disciplinas dedicadas exclusivamente a Ética, Legislação em Informática ou Impactos Sociais e Ambientais da Computação. No entanto, estes são temas transversais de extrema importância e que devem ser abordados ao longo de diversas disciplinas do curso. A SBC preconiza conteúdos como "Legislação Aplicada à Informática", "Ética e Cidadania" e "Engenharia Ambiental / Tecnologia e Meio Ambiente". É fundamental que o PPC do curso contemple esses conteúdos, seja através de disciplinas específicas ou de forma integrada, para assegurar o desenvolvimento das competências C.5.1 ("Conhecer os direitos e propriedades intelectuais"), C.5.2 ("Compreender a importância da conduta ética e cidadã") e C.5.3 ("Compreender o impacto que as soluções de sistemas de computação podem causar na sociedade e no meio ambiente").

Relação com as Diretrizes Curriculares Nacionais (DCNs)
Os referenciais da SBC são elaborados em consonância com as Diretrizes Curriculares Nacionais para os cursos da área de Computação. Ao alinhar sua estrutura curricular com os eixos e competências propostos pela SBC, o curso de Engenharia de Computação naturalmente busca atender às competências e habilidades gerais e específicas definidas pelas DCNs. Por exemplo, a ênfase em fundamentos (Eixo 1), desenvolvimento de sistemas (Eixo 2) e gerenciamento (Eixo 3) contribui diretamente para competências das DCNs como "Identificar problemas que tenham solução algorítmica", "Resolver problemas usando ambientes de programação", "Planejar, especificar, projetar, implementar, testar, verificar e validar sistemas de computação", e "Gerenciar projetos e manter sistemas de computação". Similarmente, os eixos de Inovação e Empreendedorismo (Eixo 4) e Desenvolvimento Pessoal e Profissional (Eixo 5) dialogam com as DCNs no que tange a "Tomar decisões e inovar", "Empreender e exercer liderança", e agir com "consciência dos aspectos éticos, legais e dos impactos ambientais".