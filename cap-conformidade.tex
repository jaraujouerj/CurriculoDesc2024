\chapter{Conformidade do Currículo de Graduação em Engenharia de Computação
com as DCNs de Engenharia e Computação - \textcolor{red}{Simone}}


O curso de graduação em Engenharia da Computação da UERJ está totalmente em
consonância com as Diretrizes Curriculares Nacionais dos Cursos de Engenharia (DCNs
de Engenharia) publicadas na Resolução CNE/CES No 2 de 24 de abril de 2019 do
Conselho Nacional de Educação, posteriormente modificadas na Resolução CNE/CES No
1 de 26 de março de 2021. 

Além disso, a Câmara de Educação Superior (CES) do Conselho Nacional de Educação (CNE) aprovou as Diretrizes Curriculares Nacionais (DCNs) para Cursos 
de Graduação em Computação por meio do Parecer CNE/CSE 136/2012 de 09 de março de 2012, homologadas pela Portaria Nº 05 de 16/11/2016.


Dentro desses contextos, este capítulo detalha como as orientações a
respeito do currículo estão contempladas na estrutura da DCN.

\section{Carga Horária Total}

A carga horária total do curso é de 3735 horas, acima do valor estabelecido como 
mínimo (3600 horas) na Resolução CNE/CES Nº 2 de 18 de junho de 2007.

\section{Conteúdos Básicos Obrigatórios}

O currículo do curso de graduação em Engenharia da Computação da UERJ contempla plenamente os conteúdos obrigatórios mencionados nas Diretrizes Curriculares Nacionais dos cursos de Engenharia (DCNs de Engenharia), conforme Parágrafo primeiro do Art. 9º. Abaixo são apresentadas as disciplinas de conteúdos obrigatórios do curso de Engenharia de Computação.

\begin{enumerate}

    \item \textbf{Administração}: Empreendedorismo

    \item \textbf{Ciência e Tecnologia dos Materiais}: Materiais Elétricos e Magnéticos I

    \item \textbf{Ciências do Ambiente}: Introdução à Engenharia Ambiental

    \item \textbf{Comunicação e Expressão}: Engenharia de Computação e Sociedade

    \item \textbf{Economia}: Macroeconomia aplicada à Engenharia

    \item \textbf{Eletricidade Aplicada}  
    \begin{itemize}
        \item Circuitos em Corrente Contínua
        \item Circuitos em Corrente Alternada
    \end{itemize}

    \item \textbf{Física}  
    \begin{itemize}
        \item Física Teórica I
        \item Física Experimental I
        \item Física Teórica II
        \item Física Experimental II
        \item Física Teórica III
        \item Física Experimental III
        \item Física Teórica IV
        \item Física Experimental IV
    \end{itemize}

    \item \textbf{Humanidades, Ciências Sociais e Cidadania}: Engenharia de Computação e Sociedade

    \item \textbf{Informática}  
    \begin{itemize}
        \item Processamento de Sinais e Imagens
        \item Lógica e Circuitos Digitais
        \item Sinais e Sistemas
    \end{itemize}

    \item \textbf{Matemática}  
    \begin{itemize}
        \item Cálculo Diferencial e Integral I
        \item Cálculo Diferencial e Integral II
        \item Cálculo Diferencial e Integral III
        \item Álgebra Linear
        \item Cálculo Numérico
        \item Probabilidade e Estatística
    \end{itemize}

    \item \textbf{Metodologia Científica e Tecnológica}: Metodologia Científica para Computação   

    \item \textbf{Química}: Materiais Elétricos e Magnéticos I
\end{enumerate}


\section{Conteúdos Específicos e Profissionais}


O currículo do curso de graduação em Engenharia da Computação da UERJ contempla plenamente os conteúdos específicos e profissionais mencionados nas Diretrizes Curriculares Nacionais dos cursos de Engenharia (DCNs de Engenharia), conforme Parágrafo primeiro do Art. 9º. Na Tabela \ref{tab:areas} são apresentadas as disciplinas de conteúdos específicos e profissionais do curso distribuídas nos eixos de formação, competências e conteúdos que compõem os referenciais de formação para cursos de Engenharia de Computação.




\begin{table}[ht]

\centering

\caption{Tabela das disciplinas de conteúdos específicos e profissionais de Engenharia da Computação por Eixos de formação.}

\label{tab:areas}

\begin{tabularx}{\textwidth}{ p{4cm} p{8cm} }
    %\hiderowcolors
    
    \hline
    
    {\bf Eixos de formação} & {\bf Disciplinas Específicas e Profissionais} \\
    
    \hline
    
    \multirow{5}{=}{Fundamentos de Sistemas de Computação} 
    & - Algoritmos Computacionais I \\ 
    
    & - Análise de Algoritmos I \\ 
    
    & - Estruturas de Informação A \\ 
    
    & - Lógica em Programação \\ 
    
    & - Laboratório de Programação \\ 
    
    & - Laboratório de Programação Orientada a Objetos \\ 
    
    & - Teoria dos Grafos e Aplicações \\
    
    & - Teoria de Compiladores I \\ \hline
    
    \multirow{8}{=}{Desenvolvimento de Sistemas Computacionais} 
    & - Arquitetura de Computadores A \\ 
    
    & - Computação Paralela e Distribuída \\ 
    
    & - Engenharia de Sistemas \\ 
    
    & - Lógica e Circuitos Digitais \\ 
    
    & - Fundamentos de Computadores I \\ 
    
    & - Instalação de Ambientes Computacionais \\ 
    
    & - Projeto de Sistemas Operacionais \\ 
    
    & - Projeto de Banco de Dados \\ 
    
    & - Redes de Computadores e Sistemas Distribuídos \\ 
    
    & - Segurança de Redes \\ 
    
    & - Sistemas Embutidos \\ 
    
    \hline
    
    \multirow{3}{=}{Gerenciamento de sistemas computacionais} 
    & - Análise e Projeto de Sistemas \\
    
    & - Inteligência Computacional I e II \\ 
    
    & - Mineração de Dados \\ 
    
    & - Processamento de Sinais e Imagens\\
    
    \hline
    
    \multirow{2}{=}{Inovação e Empreendedorismo } 
    & - Empreendedorismo \\ 
    
    & - Macroeconomia aplicada à Engenharia \\ 
    
    & - Administração Financeira de Projeto \\
    
    \hline
    
    \multirow{4}{=}{Desenvolvimento Pessoal e Profissional}
    & - Engenharia de Computação e Sociedade \\ 
    
    & - Introdução à Engenharia Ambiental \\ 
    
    & - Estágio Supervisionado para Engenharia de Computação \\
    
    \hline
    
\end{tabularx}
\end{table} 

\section{Atividades Práticas e Laboratoriais}
O Parágrafo 3º do Art. 9º das DCNs de Engenharia preconiza a presença de 
atividades práticas e de laboratório. O DESC apresenta uma estrutura completa de laboratórios, como pode ser constatado no Item 7.3 e 
que visam atender práticas laboratoriais para complementar a base teórica das disciplinas de conteúdo obrigatórios e específicos.

\section{Projeto Final de Curso}
O Projeto Final de Curso, tal como descrito no Capítulo 4, Seção 4.6.7 deste documento, é 
desenvolvido ao longo das disciplinas obrigatórias \textbf{Projeto de Graduação A} e \textbf{Projeto 
de Graduação B}. A proposta descrita para a atividade no Capítulo 4 atende 
plenamente as orientações das DCNs de Engenharia relativas ao seu Art. 12. 

\section{Estágio Curricular Obrigatório}
O Estágio Curricular Obrigatório está completamemte descrito no Capítulo 4, Seção 4.6.6 deste documento e 
é desenvolvido ao longo da disciplina eletiva obrigatória \textbf{Estágio Supervisionado para Engenharia de Computação}, 
com carga horária de 165 horas e 11 créditos, em concordância com a obrigatoriedade e carga horária mínima (160 horas) estipuladas pelo Art. 11 das DCNs de Engenharia para as atividades de estágio.