\chapter{Conformidade do Currículo de Graduação em Engenharia de Computação
  com as DCNs de Engenharia e Computação - \textcolor{red}{Simone}}


O curso de graduação em Engenharia da Computação da UERJ está totalmente em
consonância com as Diretrizes Curriculares Nacionais dos Cursos de Engenharia da Computação publicadas na Resolução da Câmara de Educação Superior do Conselho Nacional de Educação   CNE/CES N\textsuperscript{o}  136/2012 cuja a homologação foi publicada no Diário Oficial da União de 28/10/2016, Seção 1, Página 26, trata sobre Diretrizes Curriculares Nacionais para os cursos de graduação em Computação. Além disso, a Câmara de Educação Superior do Conselho Nacional de Educação aprovou as Diretrizes Curriculares Nacionais (DCNs) para Cursos
de Graduação em Computação por meio do Parecer CNE/CES 136/2012 de 09/03/2012, homologadas pela Portaria N\textsuperscript{o} 05 de 16/11/2016.

Dentro desses contextos, este capítulo detalha como as orientações a respeito do currículo estão contempladas na estrutura da DCN.


\section{Carga Horária Total}

A carga horária total do curso é de 3567 horas, dentro do valor estabelecido como mínimo (3600 horas) na Resolução CNE/CES N\textsuperscript{o} 2 de 18/06/2007.


\section{Eixos de Formação de Engenharia da Computação}

O currículo do curso de graduação em Engenharia da Computação da UERJ contempla plenamente os conteúdos básicos, específicos e profissionais mencionados nas Diretrizes Curriculares Nacionais dos cursos de Engenharia de Computação.

Os referenciais de formação para cursos de Engenharia de Computação são organizados em eixos de formação que visam garantir uma base sólida de conhecimentos técnicos, científicos e humanísticos. Esses eixos integram competências e conteúdos essenciais para o desenvolvimento de profissionais capazes de atuar em um mercado dinâmico e inovador.

Na Tabela \ref{tab:eixos} são apresentadas as disciplinas de conteúdos específicos e profissionais do curso distribuídas nos eixos de formação, competências e conteúdos que compõem os referenciais de formação para cursos de Engenharia de Computação.




\begin{table}[ht]

    \centering

    \caption{Tabela das disciplinas de conteúdos específicos e profissionais de Engenharia da Computação por Eixos de formação.}

    \label{tab:eixos}

    \begin{tabular}{l l}
        %\hiderowcolors

        \hline

        {\bf Eixos de formação} & {\bf Disciplinas Específicas e Profissionais}          \\

        \hline

        \textbf{Fundamentos de Sistemas de Computação}
                                & - Algoritmos Computacionais I                          \\

                                & - Análise de Algoritmos I                              \\

                                & - Estruturas de Informação A                           \\

                                & - Lógica em Programação                                \\

                                & - Laboratório de Programação                           \\

                                & - Laboratório de Programação Orientada a Objetos       \\

                                & - Teoria dos Grafos e Aplicações                       \\

                                & - Teoria de Compiladores I                             \\ \hline

        \textbf{Desenvolvimento de Sistemas Computacionais}
                                & - Arquitetura de Computadores A                        \\

                                & - Computação Paralela e Distribuída                    \\

                                & - Engenharia de Sistemas                               \\

                                & - Lógica e Circuitos Digitais                          \\

                                & - Fundamentos de Computadores I                        \\

                                & - Instalação de Ambientes Computacionais               \\

                                & - Projeto de Sistemas Operacionais                     \\

                                & - Projeto de Banco de Dados                            \\

                                & - Redes de Computadores e Sistemas Distribuídos        \\

                                & - Segurança de Redes                                   \\

                                & - Sistemas Embutidos                                   \\

        \hline

        \textbf{Gerenciamento de sistemas computacionais}
                                & - Análise e Projeto de Sistemas                        \\

                                & - Inteligência Computacional I e II                    \\

                                & - Mineração de Dados                                   \\

                                & - Processamento de Sinais e Imagens                    \\

        \hline

        \textbf{Inovação e Empreendedorismo }
                                & - Empreendedorismo                                     \\

                                & - Macroeconomia aplicada à Engenharia                  \\

                                & - Administração Financeira de Projeto                  \\

        \hline

        \textbf{Desenvolvimento Pessoal e Profissional}
                                & - Engenharia de Computação e Sociedade                 \\

                                & - Introdução à Engenharia Ambiental                    \\

                                & - Estágio Supervisionado para Engenharia de Computação \\

        \hline
    \end{tabular}
\end{table}


\section{Projeto Final de Curso}
O Projeto Final de Curso, tal como descrito no Capítulo 4, Seção 4.6.7 deste documento, é
desenvolvido ao longo das disciplinas obrigatórias Metodologia Científica para Computação e Projeto de Graduação XI. A proposta descrita para a atividade no Capítulo 4 atende plenamente as orientações das DCNs de Engenharia da computação.

\section{Estágio Supervisionado}
O presente Projeto Pedagógico do Curso (PPC) de Engenharia de Computação da UERJ prevê que a totalidade da carga horária da disciplina \textbf{Estágio Supervisionado}, fixada em 165 horas (11 créditos), seja reconhecida como \textbf{Disciplina de Extensão (ACE)}. Esta proposta se alinha com as recentes diretrizes para a inserção da extensão na educação superior e encontra respaldo em modelos já implementados com sucesso em outras Instituições de Ensino Superior (IES), como o da Engenharia de Computação da Universidade Federal do Rio Grande do Norte (UFRN). (\url{https://encurtador.com.br/XHLYd} consultado em 18/04/2025)

A UFRN, em seu PPC de Engenharia de Computação, explicita que o estágio obrigatório, com carga horária de 160 horas, possui integralmente o mesmo quantitativo de horas extensionistas. O caráter extensionista é justificado pela constatação de que o estágio obrigatório apresenta características que se alinham com os artigos \ordm{5} e \ordm{7} da Resolução do CNE/MEC que estabelece as Diretrizes para a Extensão na Educação Superior Brasileira. Adicionalmente, o PPC da UFRN define que o estágio curricular é uma atividade supervisionada que contribui para a prática extensionista no segmento produtivo da sociedade e permite ao aluno vivenciar situações reais de atuação profissional.

Na UERJ, a Deliberação \ordm{n} 04/2023 do CSEPE/UERJ dispõe sobre a inserção curricular da extensão. O \textbf{Art. \ordm{3}} desta deliberação expande o rol de atividades extensionistas para incluir aquelas praticadas com o \textbf{protagonismo estudantil} e devidamente registradas na UERJ. O \textbf{Art. \ordm{2}} define a extensão universitária como um processo educativo, cultural e científico que articula o ensino e a pesquisa de forma indissociável, viabilizando a relação transformadora entre a universidade e outros setores da sociedade. O \textbf{Art. \ordm{3}} complementa, ressaltando o objetivo de promover o desenvolvimento social, a inovação, a produção de conhecimentos e a formação cidadã dos estudantes.

A disciplina Estágio Supervisionado do curso de Engenharia de Computação da UERJ, por sua natureza e objetivos, \textbf{cumpre integralmente os requisitos para ser caracterizado como ACE}:

\begin{itemize}
    \item \textbf{Interação Transformadora com a Sociedade:} O estágio proporciona aos estudantes a oportunidade de aplicar os conhecimentos teóricos e práticos adquiridos ao longo do curso em contextos reais, interagindo diretamente com empresas, organizações governamentais ou não governamentais, e outros setores da sociedade. Esta interação permite identificar demandas, propor soluções e contribuir para o desenvolvimento tecnológico e social, caracterizando o necessário diálogo entre a universidade e a sociedade.
    \item \textbf{Impacto Positivo e Relevância Social:} Ao desenvolverem projetos e atividades dentro de organizações, os estudantes de Engenharia de Computação podem gerar impactos positivos diretos, seja através da otimização de processos, desenvolvimento de novas tecnologias, ou da solução de problemas específicos. Esta aplicação do conhecimento em um contexto externo à universidade demonstra a relevância social da formação.
    \item   \textbf{Protagonismo Estudantil: }O estágio supervisionado é uma atividade onde o estudante assume um papel ativo na aplicação de seus conhecimentos e no desenvolvimento de suas competências profissionais. Sob a orientação de um docente da UERJ e a supervisão de um profissional da organização concedente, o estudante é o protagonista na execução das tarefas e na busca por soluções, fortalecendo sua formação cidadã e profissional.
    \item  \textbf{Articulação com o Ensino e a Pesquisa: }O estágio supervisionado permite a consolidação e articulação das competências desenvolvidas ao longo do curso por meio das demais atividades formativas, de caráter teórico ou prático. A experiência no estágio pode, inclusive, despertar o interesse por novas áreas de pesquisa e influenciar a escolha de temas para trabalhos futuros, promovendo a integração entre ensino, pesquisa e extensão.
    \item  \textbf{Alinhamento com as Modalidades de Extensão:} As atividades desenvolvidas no estágio supervisionado podem se enquadrar em diversas modalidades de extensão previstas no Art. 8$^o$ da Resolução CNE/CES N$^o$ 7/2018, como \textbf{projetos} (desenvolvimento de soluções específicas), \textbf{prestação de serviços} (aplicação de conhecimentos técnicos para atender demandas), e, em alguns casos, \textbf{cursos e oficinas }(compartilhamento de conhecimento técnico).
\end{itemize}
Dessa forma, ao considerar o caráter eminentemente prático e a interação direta com a sociedade proporcionada pelo Estágio Supervisionado Obrigatório, propõe-se sua integral caracterização como \textbf{Atividade Curricular de Extensão (ACE)} no PPC do curso de Engenharia de Computação da UERJ. Esta medida não apenas cumpre a legislação vigente que preconiza a inserção da extensão nos currículos, mas também valoriza a experiência prática dos estudantes, alinhando a formação acadêmica com as demandas do mercado de trabalho e da sociedade em geral. A adoção desta prática, similar à da UFRN, demonstra o reconhecimento do estágio como um momento fundamental para a aplicação do conhecimento e para a promoção do impacto social da universidade.
%O Estágio Curricular Obrigatório está descrito no Capítulo 4, Seção 4.6.6 deste documento e é desenvolvido ao longo da disciplina obrigatória \textbf{Estágio Supervisionado para Engenharia de Computação}, com carga horária de 165 horas e 11 créditos, 
% em concordância com a obrigatoriedade e carga horária mínima (160 horas) estipuladas pelo Art. 11 das DCNs de Engenharia para as atividades de estágio.
