\documentclass[oneside,envcountsame,envcountchap,openany]{svmono}
\usepackage{fancyhdr} % Required to customize headers
\pagestyle{fancy}
\usepackage[portuguese]{babel}
\usepackage[utf8]{inputenc}
\usepackage[T1]{fontenc}
\usepackage{textcomp}
\usepackage{xcolor} % Pacote para cores
\usepackage{xspace} % Adiciona suporte para espaçamento inteligente
\usepackage{graphicx} % Para incluir imagens
\usepackage{pdfpages} % Para incluir arquivos PDF
\usepackage[top=1in,left=2cm,bottom=1in,right=2.5cm]{geometry}
%\usepackage{lipsum} % Para gerar texto de exemplo
%\renewcommand{\headrulewidth}{0pt}
% Defina o caminho para o logotipo da universidade
\newcommand{\universitylogo}{imagens/logo_uerj_cor.jpg}
\fancyheadoffset[L]{2mm}
\renewcommand{\headrulewidth}{0pt} % Remove a linha horizontal no cabeçalho
% Pacotes de hiperlinks
\usepackage{hyperref}
\hypersetup{
	colorlinks,
	citecolor=blue,
	filecolor=magenta,
	linkcolor=blue,
	urlcolor=cyan
}
\newenvironment{itquotation}
{\begin{quotation}
  \itshape % itálico
  }
{\end{quotation}}

% Pacotes de tabelas e cores
\usepackage[table]{xcolor}  % para tabelas coloridas
\usepackage{tabularx}		% para tabelas com largura automática
\usepackage{longtable}		% para tabelas longas
\usepackage{makecell}     % para tabelas com células mescladas
\usepackage{spreadtab}	    % para tabelas com cálculos
\usepackage{array}  % para tabelas com colunas de largura variável
\usepackage{booktabs}		% para tabelas com linhas horizontais
\usepackage{multirow}		 % para tabelas com multiplas linhas
\usepackage{caption}
\usepackage{chngcntr}
\counterwithout{table}{section}  % ou {chapter}, dependendo da classe


% Define uma nova coluna para alinhamento superior e quebra automática
\newcolumntype{L}[1]{>{\raggedright\arraybackslash}p{#1}}
\captionsetup{labelfont=bf, font=small, textfont=small}  % Força o negrito somente no "Tabela x:"
\renewcommand{\arraystretch}{1.5}

% Configurar o cabeçalho apenas para a primeira página
\fancypagestyle{firstpage}{
  \fancyhf{} % Limpar os estilos padrão

  % logo e Nome da universidade
  \lhead{
    \begin{minipage}{0.15\textwidth}
      \includegraphics[width=2.5cm]{\universitylogo}
    \end{minipage}
    \hfill
    \begin{minipage}{0.85\textwidth} % Ajuste o tamanho conforme necessário
      \raggedright % Alinha o texto à esquerda
      UNIVERSIDADE DO ESTADO DO RIO DE JANEIRO \\
      CENTRO DE TECNOLOGIA E CIÊNCIAS \\
      FACULDADE DE ENGENHARIA \\
      DEPARTAMENTO DE ENGENHARIA DE SISTEMAS E COMPUTAÇÃO \\
    \end{minipage}
  }
}
% reduzir tamanho das letras nas seções
\makeatletter
\renewcommand{\section}{\@startsection{section}{1}{\z@}%
      {-3.5ex \@plus -1ex \@minus -.2ex}%
      {2.3ex \@plus.2ex}%
      {\normalfont\normalsize\bfseries}}
% Remover cabeçalho das outras páginas
%\pagestyle{plain} % Define o estilo das páginas subsequentes como básico (sem cabeçalho)
% Nomes das Disciplinas
\usepackage{./disciplinasDB} % Ensure the file disciplinasDB.sty is in the same directory as this .tex file
% NAO EDITE MANUALMENTE
% Este arquivo foi gerado automaticamente em 2025-5-6
% MACROS CRIADAS
% 1. tHorasCurso: Total de horas do curso: 
% 2. hTotaisDisc: Total de horas de disciplinas (regulares+eletivas+extensao): 
% 3. hTotaisDiscObrigComDiscExt: Total de horas de disciplinas obrigatórias (regulares+extensao)
% 4. hDiscObrigSemExtensao: Total de horas de disciplinas obrigatórias sem extensão
% 5. tHorasSemExtensao: Total de horas de Disc obrig sem extensao
% 6. hDiscExtensao: Total de horas de disciplinas de extensão
% 7. hExtensao: Total de horas de extensão
% 8. hACE Total de horas de atividades de extensão (sem disciplinas)
% 9. nEletivas: Número de disciplinas eletivas
% 10. credEletivas: Total de créditos de disciplinas eletivas
% 11. tCredCurso: Total de créditos do curso
% 12. credObrigSemExtensao: Total de créditos de disciplinas obrigatórias sem extensão
% 13. hEletivas: Total de horas de disciplinas eletivas
% 14. credDiscExtensao: Total de créditos disciplinas de extensao
% 15. nDisciplinas: Número total de disciplinas
% 16. nDiscObrigatorias: Número de disciplinas obrigatorias
\newcommand {\tHorasCurso}{3567\xspace }
\newcommand {\hTotaisDisc}{3405\xspace }
\newcommand {\hTotaisDiscObrigComDiscExt}{3225\xspace }
\newcommand {\tHorasSemExtensao}{3210\xspace }
\newcommand {\hDiscObrigSemExtensao}{3030\xspace }
\newcommand {\hDiscExtensao}{195\xspace }
\newcommand {\hExtensao}{357\xspace }
\newcommand {\hACE}{162\xspace }
\newcommand {\nEletivas}{3\xspace }
\newcommand {\credEletivas}{12\xspace }
\newcommand {\hEletivas}{180\xspace }
\newcommand {\tCredCurso}{227\xspace }
\newcommand {\credObrigSemExtensao}{202\xspace }
\newcommand {\credDiscExtensao}{13\xspace }
\newcommand {\nDisciplinas}{56\xspace }
\newcommand {\nDiscObrigatorias}{53\xspace }

 % Arquivo com os dados do curso

\begin{document}
\thispagestyle{firstpage} % Aplica o cabeçalho na primeira página
\headsep = 20pt
%\setlength{\parindent}{0cm} % Remove paragraph indentation
\setlength{\tabcolsep}{5pt} % Espaço horizontal
\vspace*{2.0cm}

% Retira cabeçalho das páginas seguintes
\pagestyle{plain} % Define o estilo das páginas subsequentes como básico (sem cabeçalho)

\begin{center}
  \textbf{\LARGE Plano de Periodização}
\end{center}
\vspace*{0.5cm}

O currículo do curso de Engenharia de Computação é constituído por disciplinas obrigatórias e eletivas, estágio supervisionado, trabalho de conclusão de curso e atividades de extensão e complementares. O curso é organizado em 10 semestres, podendo o aluno cumprí-lo em um máximo de 18 semestres.

Para uma eficaz orientação pedagógica, é proposto o plano de periodização apresentado na Tabela \ref{tab:desc-periodos-long}.

O aluno deverá cursar no mínimo duas das disciplinas eletivas restritas oferecidas (ver Tabela \ref{tabeletivas}). Deve ser
ressaltado que estas disciplinas são oferecidas de acordo com o interesse dos corpos
docente e discente, não sendo necessariamente disponibilizadas todos os semestres.

\renewcommand{\arraystretch}{1.5}
\rowcolors{2}{white}{gray!10}
\begin{small}
  \setcounter{table}{0} % Garante que a próxima tabela será numerada como 1
  \begin{longtable}{ >{\raggedright\arraybackslash\hspace{4pt}}p{9cm} c c }
    \caption{Plano de Periodização das Disciplinas}
    \label{tab:desc-periodos-long}                                                                                                      \\
    \hline
    \rowcolor{gray!30}
    \multicolumn{3}{>{\raggedright\arraybackslash\hspace{4pt}}l}{\textbf{1\textordmasculine~Período}}                                   \\
    \hline
    \endfirsthead

    \hline
    \rowcolor{gray!20}
    \textbf{Disciplina}                                                                 & \textbf{CH}           & \textbf{Créditos}     \\
    \hline
    \endhead

    \multicolumn{3}{ r}{\small\itshape Continuação na próxima página}
    \endfoot

    \bottomrule
    \endlastfoot
    % Período 1
    \rowcolor{gray!20}
    \textbf{Disciplina}                                                                 & \textbf{CH}           & \textbf{Créditos}     \\
    \AlgComp                                                                            & \AlgCompCH            & \AlgCompCred          \\
    \AlgLin                                                                             & \AlgLinCH             & \AlgLinCred           \\
    \CalcI                                                                              & \CalcICH              & \CalcICred            \\
    \EngCompSoc                                                                         & \EngCompSocCH         & \EngCompSocCred       \\
    \hline \hline
    \rowcolor{gray!20}\multicolumn{1}{r}{\textbf{Subtotal 1\textordmasculine~Período}}  & \hPerUm               & \credPerUm            \\
    \hline \hline
    % Período 2
    \rowcolor{gray!30}
    \multicolumn{3}{>{\raggedright\arraybackslash\hspace{4pt}}l}{\textbf{2\textordmasculine~Período}}                                   \\
    \hline
    \CalcII                                                                             & \CalcIICH             & \CalcIICred           \\
    \CalcNum                                                                            & \CalcNumCH            & \CalcNumCred          \\
    \EstrInf                                                                            & \EstrInfCH            & \EstrInfCred          \\
    \FisI                                                                               & \FisICH               & \FisICred             \\
    \FisEI                                                                              & \FisEICH              & \FisEICred            \\
    \LogProg                                                                            & \LogProgCH            & \LogProgCred          \\
    \hline
    \rowcolor{gray!20}\multicolumn{1}{r}{\textbf{Subtotal 2\textordmasculine~Período}}  & \hPerDois             & \credPerDois          \\
    \hline
    % Período 3
    \hline
    \rowcolor{gray!30}
    \multicolumn{3}{>{\raggedright\arraybackslash\hspace{4pt}}l}{\textbf{3\textordmasculine~Período}}                                   \\
    \hline
    \AnAlg                                                                              & \AnAlgCH              & \AnAlgCred            \\
    \CalcIII                                                                            & \CalcIIICH            & \CalcIIICred          \\
    \CircEletI                                                                          & \CircEletICH          & \CircEletICred        \\
    \FisII                                                                              & \FisIICH              & \FisIICred            \\
    \FisEII                                                                             & \FisEICH              & \FisEICred            \\
    \ProbEst                                                                            & \ProbEstCH            & \ProbEstCred          \\
    \Ext                                                                                & \ExtCH                & \ExtCred              \\
    \hline
    \rowcolor{gray!20}\multicolumn{1}{r}{\textbf{Subtotal 3\textordmasculine~Período}}  & \hPerTres             & \credPerTres          \\
    \hline
    % Período 4
    \hline
    \rowcolor{gray!30}
    \multicolumn{3}{>{\raggedright\arraybackslash\hspace{4pt}}l}{\textbf{4\textordmasculine~Período}}                                   \\
    \hline
    \FisIII                                                                             & \FisIIICH             & \FisIIICred           \\ % Física III
    \FisEIII                                                                            & \FisEIIICH            & \FisEIIICred          \\ % Física Experimental III
    \LabProgA                                                                           & \LabProgACH           & \LabProgACred         \\ % Laboratório de Programação A
    \LabProgPOO                                                                         & \LabProgPOOCH         & \LabProgPOOCred       \\ % Laboratório de Programação OO        
    \ProcImag                                                                           & \ProcImagCH           & \ProcImagCred         \\ % Processamento de Sinais e Imagens
    \TecDig                                                                             & \TecDigCH             & \TecDigCred           \\ % Técnicas Digitais I
    \hline
    \rowcolor{gray!20}\multicolumn{1}{r}{\textbf{Subtotal 4\textordmasculine~Período}}  & \hPerQuatro           & \credPerQuatro        \\
    \hline
    % Período 5
    \hline \rowcolor{gray!30}
    \multicolumn{3}{>{\raggedright\arraybackslash\hspace{4pt}}l}{\textbf{5\textordmasculine~Período}}                                   \\
    \hline
    \CCC                                                                                & \CCCCH                & \CCCCred              \\ % Circuitos em Corrente Contínua      
    \FisIV                                                                              & \FisIVCH              & \FisIVCred            \\ % Física IV
    \FisEIV                                                                             & \FisEIVCH             & \FisEIVCred           \\ % Física Experimental IV
    \FundComp                                                                           & \FundCompCH           & \FundCompCred         \\ % Fundamentos de Computadores I  
    \IC                                                                                 & \ICCH                 & \ICCred               \\ % Inteligência Computacional I
    \Grafos                                                                             & \GrafosCH             & \GrafosCred           \\ % Teoria dos Grafos e Aplicações
    \hline
    \rowcolor{gray!20}\multicolumn{1}{r}{\textbf{Subtotal 5\textordmasculine~Período}}  & \hPerCinco            & \credPerCinco         \\
    \hline
    % Período 6
    \hline \rowcolor{gray!30}
    \multicolumn{3}{>{\raggedright\arraybackslash\hspace{4pt}}l}{\textbf{6\textordmasculine~Período}}                                   \\
    \hline
    \ArqComp                                                                            & \ArqCompCH            & \ArqCompCred          \\ % Arquitetura de Computadores A
    \CCA                                                                                & \CCACH                & \CCACred              \\ % Circuitos em Corrente Alternada
    \EngSistA                                                                           & \EngSistACH           & \EngSistACred         \\ % Engenharia de Sistemas
    \ICII                                                                               & \ICIICH               & \ICIICred             \\ % Inteligência Computacional II        
    \MatEle                                                                             & \MatEleCH             & \MatEleCred           \\ % Materiais Elétricos e Eletrônicos
    \MineraDados                                                                        & \MineraDadosCH        & \MineraDadosCred      \\ % Mineração de Dados
    \hline
    \rowcolor{gray!20}\multicolumn{1}{r}{\textbf{Subtotal 6\textordmasculine~Período}}  & \hPerSeis             & \credPerSeis          \\
    \hline
    % Período 7
    \hline \rowcolor{gray!30}
    \multicolumn{3}{>{\raggedright\arraybackslash\hspace{4pt}}l}{\textbf{7\textordmasculine~Período}}                                   \\
    \hline
    \Instala                                                                            & \InstalaCH            & \InstalaCred          \\ % Instalações de Ambientes Computacionais 
    \MacroEco                                                                           & \MacroEcoCH           & \MacroEcoCred         \\ % Macroeconomia 
    \ProjBD                                                                             & \ProjBDCH             & \ProjBDCred           \\ % Projeto de Banco de Dados
    \ProjSO                                                                             & \ProjSOCH             & \ProjSOCred           \\ % Projeto de Sistemas Operacionais
    \Telep                                                                              & \TelepCH              & \TelepCred            \\ % Redes de Computadores
    \TeoComp                                                                            & \TeoCompCH            & \TeoCompCred          \\ % Teoria da Compiladores
    \hline
    \rowcolor{gray!20}\multicolumn{1}{r}{\textbf{Subtotal 7\textordmasculine~Período}}  & \hPerSete             & \credPerSete          \\
    \hline
    % Período 8
    \hline \rowcolor{gray!30}
    \multicolumn{3}{>{\raggedright\arraybackslash\hspace{4pt}}l}{\textbf{8\textordmasculine~Período}}                                   \\
    \hline
    \AnaProjSist                                                                        & \AnaProjSistCH        & \AnaProjSistCred      \\ % Análise e Projeto de Sistemas
    \CompParal                                                                          & \CompParalCH          & \CompParalCred        \\ % Computação Paralela
    \Control                                                                            & \ControlCH            & \ControlCred          \\ % Controle de Processos
    \Empre                                                                              & \EmpreCH              & \EmpreCred            \\ % Empreendedorismo    
    \Sredes                                                                             & \SredesCH             & \SredesCred           \\ % Segurança em Redes
    \SistEmb                                                                            & \SistEmbCH            & \SistEmbCred          \\ % Sistemas Embutidos
    \hline
    \rowcolor{gray!20}\multicolumn{1}{r}{\textbf{Subtotal 8\textordmasculine~Período}}  & \hPerOito             & \credPerOito          \\
    \hline
    % Período 9
    \hline \rowcolor{gray!30}
    \multicolumn{3}{>{\raggedright\arraybackslash\hspace{4pt}}l}{\textbf{9\textordmasculine~Período}}                                   \\
    \hline
    \EstSup                                                                             & \EstSupCH             & \EstSupCred           \\ % Estágio Supervisionado
    \EletA                                                                              & \EletACH              & \EletACred            \\ % Disciplina Eletiva A       
    \ProjA                                                                              & \ProjACH              & \ProjACred            \\ % Metodologia Científica
    \hline
    \rowcolor{gray!20}\multicolumn{1}{r}{\textbf{Subtotal 9\textordmasculine~Período}}  & \hPerNove             & \credPerNove          \\
    \hline
    % Período 10
    \hline \rowcolor{gray!30}
    \multicolumn{3}{>{\raggedright\arraybackslash\hspace{4pt}}l}{\textbf{10\textordmasculine~Período}}                                  \\
    \hline
    \Adm                                                                                & \AdmCH                & \AdmCred              \\ % Administração
    \EletB                                                                              & \EletBCH              & \EletBCred            \\ % Disciplina Eletiva B
    \ProjB                                                                              & \ProjBCH              & \ProjBCred            \\ % Projeto de Graduação XI
    \hline
    \rowcolor{gray!20}\multicolumn{1}{r}{\textbf{Subtotal 10\textordmasculine~Período}} & \hPerDez              & \credPerDez           \\
    \hline

    \hline
    \rowcolor{gray!20}\textbf{Total Geral}                                              & \textbf{\hTotaisDisc} & \textbf{\tCredCurso } \\
  \end{longtable}
\end{small}


\rowcolors{1}{gray!10}{white}
\begin{table}[!ht]
  \centering
  \caption{Disciplinas Eletivas Restritas}
  \label{tabeletivas}
  \begin{tabularx}{0.75\textwidth}{Xcc}
    \hline \rowcolor{gray!20}
    \textbf{Disciplina} & \textbf{CH}      & \textbf{Créditos}  \\
    \hline
    \EletReforco        & \EletReforcoCH   & \EletReforcoCred   \\ % Aprendizado por Reforço
    \EletVisao          & \EletVisaoCH     & \EletVisaoCred     \\ % Aprendizado Profundo para Visão Computacional
    \AprendProfPLN      & \AprendProfPLNCH & \AprendProfPLNCred \\ % Aprendizado Prof. p/ Proc.de Ling. Natural
    \EletArq            & \EletArqCH       & \EletArqCred       \\ % Arquiteturas Avançadas de Computadores 
    \AutomProcRob       & \AutomProcRobCH  & \AutomProcRobCred  \\ % Automação de Processos Robóticos
    \EletGeo            & \EletGeoCH       & \EletGeoCred       \\ % Geomática
    \EletRedes          & \EletRedesCH     & \EletRedesCred     \\ % Redes de Interconexão			
    \SistOpRobInt       & \SistOpRobIntCH  & \SistOpRobIntCred  \\ % Sistemas Operacionais p/ Robótica Inteligente
    \TecProgOtim        & \TecProgOtimCH   & \TecProgOtimCred   \\ % Técnicas de Programação em Otimização Combinatória
    \TopEspVisComp      & \TopEspVisCompCH & \TopEspVisCompCred \\ % Tópicos Especiais em Visão Computacional
    \hline
  \end{tabularx}
\end{table}

\end{document}