\section{Extensão}

O currículo do curso de Engenharia de Sistemas e Computação da UERJ tem como uma de suas vertentes combater a evasão, propiciando maior motivação e engajamento dos alunos oferecendo-lhes disciplinas profissionalizantes desde o primeiro período do curso. 

Nesta mesma direção, o curso busca proporcionar aos alunos uma visão realística e analítica sobre o papel da tecnologia na sociedade através da oferta de novas disciplinas e de atividades de Extensão Universitária, suscitando o desenvolvimento de uma perspectiva disruptiva e contextualizada através da qual os alunos se sintam encorajados a se envolver com iniciativas voltadas para o desenvolvimento social. Desta forma, as atividades de Extensão Universitária são vistas como um processo de aprofundamento educativo e cultural, propiciando o exercício da interdisciplinaridade e estimulando o pensamento reflexivo, analítico e crítico nos estudantes. A partir do conjunto coordenado de projetos, programas, eventos e disciplinas, promovem interações transformadoras entre a universidade e os diversos setores da sociedade.

Para alcançar esses objetivos, as seguintes mudanças foram implementadas: cada aluno do curso deve completar pelo menos \hextensao horas em atividades de extensão complementares, conforme exigido pela Lei 10.172 que aprova o Plano Nacional de Educação. Isso corresponde a pelo menos 10\% do total da carga horária do curso.

A participação discente e o cumprimento das horas pode ser obtido através das seguintes formas de atividades de extensão:
\begin{enumerate}[I -]
    \item Participação em programas e projetos de extensão coordenados por professores ou técnicos da carreira de nível superior na Universidade do Estado do Rio de Janeiro, com ou sem o recebimento de bolsa;
    \item Promoção de cursos de extensão, incluindo a organização, preparação e apresentação de aulas, videoaulas e reuniões com a comunidade;
    \item Participação em disciplinas e atividades relacionadas à extensão fornecidas pela universidade, visando a compreensão, aprimoramento e melhor desempenho do discente na realização de tarefas de extensão;
    \item Desenvolvimento conjunto de soluções tecnológicas visando atender as demandas dos atores da sociedade civil;
    \item Participação em eventos, tanto na organização quanto na realização.
\end{enumerate}

A Universidade do Estado do Rio de Janeiro é uma instituição compromissada com a formação da cidadania e a inclusão social. Neste contexto, o Curso de Engenharia de Sistemas e Computação da UERJ, pretende colaborar com a inclusão digital dos cidadãos fluminenses.

A extensão universitária permite o estreitamento dos laços entre a academia e a sociedade, inseridas em diversas realidades socioeconômicas no âmbito de comunidades rurais, periurbanas e urbanas. Ao aproximar os conhecimentos obtidos em sala de aula à realidade, contribui para a sustentabilidade de setores socioeconômicos que desempenham um papel essencial para o futuro das gerações, mas, que enfrentam no dia a dia constantes desafios, notadamente a exclusão digital. Além disso, oferece aos alunos a chance de se envolverem em atividades nas quais poderão exercitar a cidadania ao prestar um serviço relevante a segmentos da sociedade frequentemente esquecidos e carentes de assistência.