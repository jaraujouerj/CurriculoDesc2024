% !TEX root = ProjetoPedagogico.tex

\chapter{Organização Didático Pedagógica}

A nova versão do curso de Engenharia de Computação obedecerá ao regime de créditos. O aluno interessado em cursar a graduação em Engenharia de Computação fará tal opção diretamente a partir da sua inscrição no vestibular.


\section{Justificativa das Necessidades Sociais do Curso -\textcolor{red}{Gabriel}}

As demandas sociais e tecnológicas da sociedade brasileira são crescentes, especialmente em um cenário global marcado pela revolução digital, pela inovação tecnológica e pela recente explosão do uso de inteligência artificial e aprendizado de máquina na resolução de demandas tecnológicas. O desenvolvimento de competências técnicas e humanas no campo da computação tornou-se essencial para atender os desafios impostos pela modernização e pela globalização.

No contexto brasileiro, existe a necessidade de formar profissionais qualificados e preparados para atuar em sistemas informatizados nos mais variados setores da sociedade. O uso da computação é fundamental em diversas áreas estratégicas, como saúde, educação, administração, indústria, segurança, empreendedorismo, entre outras. Dessa forma, a demanda por profissionais aptos a desenvolver, implementar e gerenciar soluções computacionais inovadoras, que contribuam para a resolução de problemas complexos e promovam o desenvolvimento econômico e social do país se torna urgente e essencial.

Neste contexto, a UERJ, em especial o DESC, responsável pela mais antiga ênfase em engenharia da computação do país, desempenha um papel fundamental na formação de engenheiros preparados para os desafios do mercado atual. Com essa visão, surgiu o curso de Engenharia da Computação.

\section{Finalidades e Objetivos do Curso - \textcolor{red}{Simone}}

\subsection{Concepção}

A estrutura curricular do curso de Engenharia de Computação do \desc da Faculdade de Engenharia da UERJ orientar-se-á pelas \textit{Diretrizes Curriculares Nacionais para os cursos de graduação na área da Computação}, do Ministério da Educação (anexo \ref{cne2016}), pelos \textit{Referenciais de Formação para os cursos de Graduação em Computação}, da Sociedade Brasileira de Computação (anexo \ref{sbc}), e pela regulamentação do exercício da profissão de Engenheiro, estabelecida pelo Sistema CREA/CONFEA (Resolução 1.010 CONFEA, anexo \ref{res1010}), em vigor atualmente. A inserção da extensão no curriculo terá como base a Deliberação n\textordmasculine{} 4/2023 do Conselho Superior de Ensino, Pesquisa e Extensão (CSEPE/UERJ, anexo \ref{del4}).

A grade curricular totaliza \totalhoras horas, sendo \hobrigatorias horas em disciplinas e \hextensao horas em atividades de extensão. As \hobrigatorias horas de disciplinas estão distribuídas em \ndisciplinas disciplinas, sendo \nobrigatorias  obrigatórias e \neletivas eletivas. A carga horária das disciplinas está distribuída entre teoria, prática, laboratório, estágio e extensão, de acordo com as especificidades do respectivo conteúdo programático. O currículo também contempla Estágio Supervisionado e Projeto de Graduação (Trabalho de Conclusão de Curso) como atividades de síntese e integração do conhecimento científico, tecnológico e instrumental. Como atividades acadêmicas complementares facultativas, os alunos podem optar por Estágio Interno, Monitoria, Iniciação Científica, Cursos, Eventos, Palestras e Visitas Técnicas, que visam proporcionar uma melhor compreensão da Engenharia, do setor no Brasil e das áreas de atuação e atividades dos Engenheiros de Computação.

\subsection{Finalidades}

A finalidade do curso de Engenharia de Computação é formar profissionais capacitados a atuar na concepção, desenvolvimento, implementação e gestão de sistemas computacionais, promovendo a inovação tecnológica e atendendo às demandas da sociedade, da indústria e do mercado global. O curso busca aliar conhecimento técnico-científico a valores éticos e responsabilidade social, contribuindo para o progresso tecnológico e sustentável.

\subsection{Objetivos}

\begin{enumerate}
	\item Prover conhecimentos técnicos e científicos necessários para o desenvolvimento de hardware e software, capacitando o estudante a criar soluções tecnológicas inovadoras e funcionais.

	\item Desenvolver habilidades práticas que possibilitem a aplicação do conhecimento em projetos de sistemas computacionais e eletrônicos, considerando critérios de eficiência, segurança e sustentabilidade.

	\item Fomentar a capacidade de análise crítica e resolução de problemas, preparando os estudantes para enfrentar desafios complexos em diversos setores, como automação, inteligência artificial, telecomunicações e computação em nuvem.

	\item Promover a formação ética e cidadã, incentivando o uso da tecnologia para o bem-estar social e ambiental, com respeito às normas legais e regulatórias.

	\item Estimular a pesquisa e a inovação tecnológica, incentivando o envolvimento dos alunos em projetos acadêmicos e profissionais que ampliem as fronteiras do conhecimento na área da computação.

	\item Preparar para o mercado de trabalho global, oferecendo uma base sólida de conhecimentos e habilidades em engenharia da computação, alinhada às tendências e demandas internacionais.

\end{enumerate}

\section{Nível de Formação e Título Acadêmico}

O curso é de graduação plena e a titulação concedida e habilitação são:

\begin{itemize}
	\item{Título: Engenheiro}
	\item{Habilitação: Engenharia de Computação}
\end{itemize}

\section{Perfil do Egresso (competência, habilidades e atitudes pretendidas) - \textcolor{red}{Gabriel}}

O curso de Engenharia de Computação tem como perfil do egresso o engenheiro, com formação técnico-científica sólida, generalista, humanista, crítica e reflexiva, capacitado a absorver e desenvolver novas tecnologias, estimulando a sua atuação crítica e criativa na identificação e resolução de problemas, considerando seus aspectos políticos, econômicos, sociais, ambientais e culturais, com visão ética e humanística, em atendimento às demandas da sociedade. Faz parte do perfil do egresso a postura de permanente busca da atualização profissional, além das seguintes habilidades:
\begin{enumerate} [I -]
	\item possuir conhecimento das questões humanísticas, sociais, ambientais, éticas, profissionais, legais e políticas;
	\item possuir compreensão do impacto da Engenharia de Computação e suas tecnologias no que concerne ao atendimento e à antecipação estratégica das necessidades da sociedade;
	\item possuir atitude crítica, interdisciplinar e criativa na identificação e resolução de problemas;
	\item possuir compreensão das necessidades de contínua atualização e aprimoramento de suas competências e habilidades;
	\item possuir uma sólida formação em Computação, Física, Matemática, Eletrônica, Automação e Telecomunicações.
	\item conhecer a estrutura dos sistemas de computação e os processos envolvidos na sua análise e construção;
	\item considerar os aspectos ambientais, econômicos, financeiros, de gestão e de qualidade, associados a novos produtos e organizações;
	\item considerar fundamental a inovação, a criatividade, a atitude empreendedora e a inserção internacional.
\end{enumerate}

O egresso da Engenharia de Computação, no processo de sua formação, deverá desenvolver as seguintes competências:
\begin{enumerate} [I -]
	\item antever as implicações humanísticas, sociais, ambientais, éticas, profissionais, legais (inclusive relacionadas à propriedade intelectual) e políticas dos sistemas computacionais;
	\item identificar demandas socioeconômicas e ambientais relevantes, planejar, especificar e projetar sistemas de computação, seguindo teorias, princípios, métodos e procedimentos interdisciplinares;
	\item construir, testar, verificar e validar sistemas de computação, seguindo métodos, técnicas e procedimentos interdisciplinares;
	\item perceber as necessidades de atualização decorrentes da evolução tecnológica e social;
	\item relacionar problemas do mundo real com suas soluções, considerando aspectos de computabilidade e de escalabilidade;
	\item analisar, desenvolver, avaliar e aperfeiçoar software e hardware em arquiteturas de computadores;
	\item analisar, desenvolver, avaliar e aperfeiçoar sistemas de automação e sistemas inteligentes;
	\item analisar, desenvolver, avaliar e aperfeiçoar sistemas de informação computacionais;
	\item analisar, desenvolver, avaliar e aperfeiçoar circuitos eletroeletrônicos;
	\item gerenciar pessoas e infraestrutura de Sistemas de Computação;
	\item perceber as necessidades de inovação e inserção internacional com atitudes criativas e empreendedoras.
\end{enumerate}

O curso de Engenharia de Computação tem, predominantemente, o ensino da Computação como atividade fim, visando à formação de recursos humanos para o desenvolvimento científico e tecnológico da computação. Assim sendo, o curso deve capacitar indivíduos para desenvolver software e hardware, com uma forte base matemática e física.

Os egressos do curso de Engenharia de Computação estarão situados no estado da arte da ciência e da tecnologia da computação, de tal forma que possam continuar suas atividades na pesquisa, promovendo o desenvolvimento científico, ou aplicando os conhecimentos científicos, propiciando o desenvolvimento tecnológico. Para tal, é dada uma forte ênfase no uso de laboratórios para capacitar os egressos no projeto e construção tanto de software quanto de hardware.

\section{Administração Acadêmica do Curso - \textcolor{red}{Rafaela}}

Caberá ao DESC exercer a administração do curso de Engenharia da Computação.
Como a reformulação do curso propõe sua reestruturação como uma habilitação específica em Engenharia de Computação, a administração do curso deverá ser reestruturada também.

\subsection{Chefia e Vice-Chefia do Departamento}

A administração do Departamento de Engenharia de Sistemas e Computação (DESC) é conduzida por um Chefe de Departamento e um Vice-Chefe. Esses cargos são ocupados por docentes efetivos do departamento. A eleição do chefe e vice-chefe se dá por meio do voto dos membros do corpo deliberativo do DESC com um mandato de até dois anos. O corpo deliberativo do DESC é formado por todos os professores efetivos do departamento e é convocado pela chefia, a fim de deliberar sobre as questões pertinentes ao curso, tanto acadêmicas quanto administrativas.

A chefia do departamento tem a responsabilidade de coordenar as atividades acadêmicas e administrativas do departamento, incluindo:

\begin{itemize}
	\item Gestão do corpo docente, distribuição de disciplinas e planejamento acadêmico;
	\item Alocação de salas e os planos individuais dos docentes (PLANINDs) a cada semestre;
	\item Representação do departamento junto à direção da Faculdade de Engenharia e demais instâncias da UERJ;
	\item Organizações de reuniões com o corpo deliberativo do DESC, registrando em ata o que foi decidido e os professores que compareceram;
\end{itemize}

\subsection{Conselho Departamental}

O Conselho Departamental da Faculdade de Engenharia é o órgão deliberativo da unidade, presidio pela Direção da Faculdade de Engenharia e composto pelas chefias dos departamentos, representante do técnicos-administrativos e representantes dos alunos. O DESC tem um representante no Conselho Departamental da Faculdade de Engenharia, o representante do DESC no Conselho Departamental é o chefe de departamento ou o seu vice.


\subsection{Coordenação de Graduação}

A Coordenação de Graduação é responsável pelo gerenciamento acadêmico do curso, garantindo o cumprimento das diretrizes curriculares e auxiliando os estudantes em sua trajetória acadêmica. Suas funções incluem:

\begin{itemize}
	\item Gestão Acadêmica: Supervisiona a execução do projeto pedagógico do curso, garantindo o cumprimento das diretrizes curriculares e regulamentos institucionais;
	\item Apoio aos Estudantes: Acompanha o desempenho dos alunos, orientando-os em questões acadêmicas, como matrícula, integralização curricular e aproveitamento de disciplinas;
	\item Articulação com Professores: Coordena e apoia o corpo docente, promovendo a atualização e melhoria das práticas de ensino.

\end{itemize}

\subsection{Coordenação das Atividades Curriculares de Extensão (CACE)}

A Coordenação das Atividades Curriculares de Extensão (CACE) orienta e verifica o cumprimento das ações de extensão realizadas pelos estudantes e valida a carga horária a ser contabilizada semestralmente. A CACE deve garantir que cada aluno cumpra as 357 horas obrigatórias de extensão.

De acordo com a Deliberação n\textordmasculine~04/2023, as atribuições da CACE incluem:

\begin{itemize}
	\item Propor ao Conselho Departamental a indicação de atividades de extensão adicionais;
	\item Receber, analisar e validar a documentação comprobatória da realização das atividades de extensão;
	\item Fixar e divulgar a data-limite para o recebimento da documentação mencionada no item anterior;
	\item Consultar o Departamento de Extensão da PR-3 (DEPEXT/PR-3) a respeito das atividades de extensão ativas e vinculadas a ações devidamente cadastradas e/ou validadas na PR-3, assim como as demais Pró-reitorias;
	\item Encaminhar à Secretaria do curso de graduação ou ao setor devidamente designado pela Unidade Acadêmica a relação de alunos que desenvolveram atividades de extensão no semestre, bem como a carga horária total da atividade, tipo e título da atividade em período consoante o calendário acadêmico da UERJ, para fins de registro acadêmico;
	\item Computar a carga horária atribuída pela instituição de origem;
	\item Validar o aproveitamento das cargas horárias das atividades de extensão certificadas/declaradas por outras IES no Brasil ou no exterior;
	\item Analisar e emitir parecer sobre o aproveitamento integral da carga horária nas atividades curriculares de extensão executadas anteriormente, nos casos de transferência externa, transferência interna, ingressos por aproveitamento de estudos ou por Vestibular;
	\item Apreciar pedidos de recurso formulados pelos alunos em relação ao indeferimento do cômputo de carga horária relativa às atividades de extensão.

\end{itemize}

Compete à Direção da Faculdade de Engenharia a indicação de docente responsável pela CACE para o curso de Engenharia de Computação.

\subsection{Coordenação de Áreas}

O curso também conta com Coordenações de Área, que são responsáveis por organizar e orientar o ensino dentro de diferentes campos do conhecimento da Engenharia de Computação.
A integração das disciplinas em áreas de conhecimento permite o compartilhamento de informações sobre interesses e objetivos comuns. Além disso, favorece a atuação conjunta de alunos e professores em temas globais e impulsiona a criação de linhas de pesquisa.


As coordenações de área têm como função:

\begin{itemize}
	\item Orientar os alunos em questões referentes às disciplinas;
	\item Analisar os requerimentos de quebras de pré-requisitos e conflitos de horário;
	\item Tratar questões relativas aos conteúdos programáticos das disciplinas;
	\item Promover a integração dos professores da mesma área, e incentivar a pesquisa na área.
\end{itemize}

As disciplinas do curso de Engenharia de Computação estão divididas em quatro grandes áreas de conhecimento: (1) Sistemas de Informação; (2) Arquitetura de Sistemas de Computação; (3) Algoritmos e Linguagens de Programação; (4) Lógica e Semântica de Programas. Cada área de conhecimento deverá possuir um Professor Coordenador. A Tabela \ref{tab:areas} mostra a distribuição das disciplinas por área de conhecimento.

\begin{table}[ht]

	\centering

	\caption{Tabela de divisão de disciplinas por área de conhecimento}

	\label{tab:areas}

	\begin{tabularx}{\textwidth}{ X l }

		\hiderowcolors

		\hline

		{\bf Área de Conhecimento}                              & {\bf Disciplinas} \\

		\hline

		\multirow{5}{*}{Algoritmos e Linguagens de Programação} & \AlgComp          \\ % Algoritmos Computacionais

		                                                        & \AnAlg            \\ % Análise de Algoritmos

		                                                        & \EstrInf          \\ % Estruturas de Informação

		                                                        & \LabProgA         \\ % Laboratório de Programação A

		                                                        & \LabProgB         \\ % Laboratório de POO

		                                                        & \TeoComp          \\ % Teoria da Compiladores

		                                                        & \Grafos           \\ \hline

		\multirow{8}{*}{Arquitetura de Sistemas de Computação}  & \ArqComp          \\ % Arqutetura de Computadores

		                                                        & \CompParal        \\ % Computação Paralela

		                                                        & \Control          \\ % Controle de Processos

		                                                        & \FundIComp        \\ % FUndamentos de Computadores I

		                                                        & \FundComp         \\ % FUndamentos de Computadores II

		                                                        & \Instala          \\ % Instalações de Ambientes Computacionais

		                                                        & \ProjSO           \\ % Projeto de Sistemas

		                                                        & \Telep            \\ % Redes de Computadores

		                                                        & \Sredes           \\ % Segurança em Redes

		                                                        & \SistEmb          \\ % Sistema Embutidos

		\hline

		\multirow{3}{*}{Lógica e Semântica de Programas}        & \IC               \\ % Inteligência Computacional

		                                                        & \ICII             \\ % Inteligência Computacional II

		                                                        & \LogProg          \\ % Lógica de Programação

		                                                        & \MineraDados      \\ % Mineração de Dados

		                                                        & \ProcImag         \\ % Processamento de Imagens

		\hline

		\multirow{4}{*}{Sistemas de Informação}                 & \EngSistC         \\ % Análise de Projeto de Sistemas

		                                                        & \EngSistA         \\ % Engenharia de Sistemas

		                                                        & \ProjBD           \\

		                                                        & \EngCompSoc       \\

		\hline
	\end{tabularx}

\end{table}


\section{Currículo Pleno e Estrutura Curricular - \textcolor{red}{Robert}}

\subsection{Organização do Currículo}

O currículo do curso de Engenharia de Computação é constituído por disciplinas obrigatórias e eletivas, estágio supervisionado, trabalho de conclusão de curso e atividades de extensão. O curso é organizado em 10 semestres, podendo o aluno cumprí-lo em um máximo de 18 semestres.

Para uma eficaz orientação pedagógica, é proposto o aconselhamento curricular apresentado nas tabelas \ref{tab1p} a \ref{tab10p}. Os pré-requisitos das disciplinas podem ser observados no fluxograma do curso (Anexo \ref{fluxograma}).

O aluno deverá cursar no mínimo três das disciplinas eletivas restritas oferecidas (ver Tabela \ref{tabeletivas}). Deve ser
ressaltado que estas disciplinas são oferecidas de acordo com o interesse dos corpos
docente e discente, não sendo necessariamente disponibilizadas todos os semestres.

\rowcolors{1}{gray!5}{white}
\setlength{\tabcolsep}{5pt}
\renewcommand{\arraystretch}{1.5}
\begin{table}[!ht]
	\centering
	\caption{1\textordmasculine~Período}
	\label{tab1p}
	\begin{spreadtab}{{tabularx}{\textwidth}{ | X|c|c| }}
		\hline
		@ {\textbf{Disciplina}} & @ {\textbf{CH}} & @ {\textbf{Créditos}} \\
		\hline
		@ \AlgComp              & \AlgCompCH      & \AlgCompCred          \\ % Algoritmos Computacionais
		@ \EngCompSoc           & \EngCompSocCH   & \EngCompSocCred       \\ % Engenharia e Computação Sociedade
		@\AlgLin                & \AlgLinCH       & \AlgLinCred           \\ % Álgebra Linear
		@ \CalcI                & \CalcICH        & \CalcICred            \\ % Cálculo I
		@ \IntAmb               & \IntAmbCH       & \IntAmbCred           \\ % Introdução à Engenharia Ambiental
		\hline
		@ Total                 & sum(b2:b6)      & sum(c2:c6)            \\
		\hline
	\end{spreadtab}
\end{table}


\rowcolors{1}{gray!5}{white}
\begin{table}[!ht]
	\centering
	\caption{2\textordmasculine~Período}
	\label{tab2p}
	\begin{spreadtab}{{tabularx}{\textwidth}{|X|c|c|}}
		\hline
		@ {\textbf{Disciplina}} & @ {\textbf{CH}} & @ {\textbf{Créditos}} \\
		\hline
		@ \EstrInf              & \EstrInfCH      & \EstrInfCred          \\ % Estruturas de Informação
		@ \LogProg              & \LogProgCH      & \LogProgCred          \\ % Lógica de Programação
		@ \CalcII               & \CalcIICH       & \CalcIICred           \\ % Cálculo II
		@ \EngComput            & \EngComputCH    & \EngComputCred        \\ % Calculo Numérico
		@ \FisI                 & \FisICH         & \FisICred             \\ % Física I
		@ \FisEI                & \FisEICH        & \FisEICred            \\ % Física Experimental I
		\hline
		@ Total                 & sum(b2:b7)      & sum(c2:c7)            \\
		\hline
	\end{spreadtab}
\end{table}

\rowcolors{1}{gray!5}{white}
\begin{table}[!ht]
	\centering
	\caption{3\textordmasculine~Período}
	\label{tab3p}
	\begin{spreadtab}{{tabularx}{\textwidth}{|X|c|c|}}
		\hline
		@ {\textbf{Disciplina}} & @ {\textbf{CH}} & @ {\textbf{Créditos}} \\
		\hline
		@ \AnAlg                & \AnAlgCH        & \AnAlgCred            \\ % Análise de Algoritmos
		@ \CalcIII              & \CalcIIICH      & \CalcIIICred          \\ % Cálculo III
		@ \FisII                & \FisIICH        & \FisIICred            \\ % Física II
		@ \FisEII               & \FisEICH        & \FisEICred            \\ % Física Experimental II
		@ \ProbEst              & \ProbEstCH      & \ProbEstCred          \\ % Probabilidade e Estatística
		\hline
		@ Total                 & sum(b2:b6)      & sum(c2:c6)            \\
		\hline
	\end{spreadtab}
\end{table}

\rowcolors{1}{gray!5}{white}
\begin{table}[!ht]
	\centering
	\caption{4\textordmasculine~Período}
	\label{tab4p}
	\begin{spreadtab}{{tabularx}{\textwidth}{|X|c|c|}}
		\hline
		@ {\textbf{Disciplina}} & @ {\textbf{CH}} & @ {\textbf{Créditos}} \\
		\hline
		@ \LabProgA             & \LabProgACH     & \LabProgACred         \\ % Laboratório de Programação A
		@ \LabProgB             & \LabProgBCH     & \LabProgBCred         \\ % Laboratório de Programação B
		@ \FisIII               & \FisIIICH       & \FisIIICred           \\ % Física III
		@ \FisEIII              & \FisEIIICH      & \FisEIIICred          \\ % Física Experimental III
		@ \ProcImag             & \ProcImagCH     & \ProcImagCred         \\ % Processamento de Sinais e Imagens
		@ \FundIComp            & \FundICompCH    & \FundICompCred        \\ %Técnicas Digitais I
		\hline
		@ Total                 & sum(b2:b7)      & sum(c2:c7)            \\
		\hline
	\end{spreadtab}
\end{table}

\rowcolors{1}{gray!5}{white}
\begin{table}[!ht]
	\centering
	\caption{5\textordmasculine~Período}
	\label{tab5p}
	\begin{spreadtab}{{tabularx}{\textwidth}{|X|c|c|}}
		\hline
		@ {\textbf{Disciplina}} & @ {\textbf{CH}} & @ {\textbf{Créditos}} \\
		\hline
		@ \Grafos               & \GrafosCH       & \GrafosCred           \\ % Teoria dos Grafos e Aplicações
		@ \FundComp             & \FundCompCH     & \FundCompCred         \\ % Fundamentos de Computadores I
		@\CEV                   & \CEVCH          & \CEVCred              \\ % Circuitos em Corrente 
		@ \FisIV                & \FisIVCH        & \FisIVCred            \\ % Física IV
		@ \FisEIV               & \FisEIVCH       & \FisEIVCred           \\ % Física Experimental IV
		@ \MatEle               & \MatEleCH       & \MatEleCred           \\ % Materiais Elétricos e Magnéticos 
		@ \ModMat               & \ModMatCH       & \ModMatCred           \\ % Sinais e Sistemas
		\hline
		@ Total                 & sum(b2:b8)      & sum(c2:c8)            \\
		\hline
	\end{spreadtab}
\end{table}

\rowcolors{1}{gray!5}{white}
\begin{table}[!ht]
	\centering
	\caption{6\textordmasculine~Período}
	\label{tab6p}
	\begin{spreadtab}{{tabularx}{\textwidth}{|X|c|c|}}
		\hline
		@ {\textbf{Disciplina}} & @ {\textbf{CH}} & @ {\textbf{Créditos}} \\
		\hline
		@ \ArqComp              & \ArqCompCH      & \ArqCompCred          \\ % Arquitetura de Computadores A
		@ \EngSistA             & \EngSistACH     & \EngSistACred         \\ % Engenharia de Sistemas
		@ \IC                   & \ICCH           & \ICCred               \\ % Inteligência Computacional I
		@ \ICII                 & \ICIICH         & \ICIICred             \\ % Inteligência Computacional II
		@ \CEVI                 & \CEVICH         & \CEVICred             \\ % Circuitos em Corrente Alternada
		@ \EletI                & \EletICH        & \EletICred            \\ % Eletrônica I
		\hline
		@ Total                 & sum(b2:b7)      & sum(c2:c7)            \\
		\hline
	\end{spreadtab}
\end{table}

\rowcolors{1}{gray!5}{white}
\begin{table}[!ht]
	\centering
	\caption{7\textordmasculine~Período}
	\label{tab7p}
	\begin{spreadtab}{{tabularx}{\textwidth}{|X|c|c|}}
		\hline
		@ {\textbf{Disciplina}} & @ {\textbf{CH}} & @ {\textbf{Créditos}} \\
		\hline
		@ \MineraDados          & \MineraDadosCH  & \MineraDadosCred      \\ % Mineração de Dados
		@ \ProjBD               & \ProjBDCH       & \ProjBDCred           \\ % Projeto de Banco de 
		@ \ProjSO               & \ProjSOCH       & \ProjSOCred           \\ % Projeto de Sistemas Operacionais
		@ \Telep                & \TelepCH        & \TelepCred            \\ % Redes de Computadores
		@ \TeoComp              & \TeoCompCH      & \TeoCompCred          \\ % Teoria da Compiladores
		@ \IntEco               & \IntEcoCH       & \IntEcoCred           \\ % Macroeconomia 
		\hline
		@ Total                 & sum(b2:b7)      & sum(c2:c7)            \\
		\hline
	\end{spreadtab}
\end{table}

\rowcolors{1}{gray!5}{white}
\begin{table}[!ht]
	\centering
	\caption{8\textordmasculine~Período}
	\label{tab8p}
	\begin{spreadtab}{{tabularx}{\textwidth}{|X|c|c|}}
		\hline
		@ {\textbf{Disciplina}} & @ {\textbf{CH}} & @ {\textbf{Créditos}} \\
		\hline
		@ \EngSistC             & \EngSistCCH     & \EngSistCCred         \\ % Análise e Projeto de Sistemas
		@ \Control              & \ControlCH      & \ControlCred          \\ % Controle de Processos
		@ \CompParal            & \CompParalCH    & \CompParalCred        \\ % Computação Paralela
		@ \Sredes               & \SredesCH       & \SredesCred           \\ % Segurança em Redes
		@ \SistEmb              & \SistEmbCH      & \SistEmbCred          \\ % Sistemas Embutidos
		@ \Empre                & \EmpreCH        & \EmpreCred            \\ % Empreendedorismo
		\hline
		@ Total                 & sum(b2:b7)      & sum(c2:c7)            \\
		\hline
	\end{spreadtab}
\end{table}

\rowcolors{1}{gray!5}{white}
\begin{table}[!ht]
	\centering
	\caption{9\textordmasculine~Período}
	\label{tab9p}
	\begin{spreadtab}{{tabularx}{\textwidth}{|X|c|c|}}
		\hline
		@ {\textbf{Disciplina}} & @ {\textbf{CH}} & @ {\textbf{Créditos}} \\
		\hline
		@ \EletA                & \EletACH        & \EletACred            \\ % Disciplina Eletiva A
		@ \EstSup               & \EstSupCH       & \EstSupCred           \\ % Estágio Supervisionado
		@ \ProjA                & \ProjACH        & \ProjACred            \\ % Metodologia Ciêntífica
		@ \Instala              & \InstalaCH      & \InstalaCred          \\ % Instalações de Ambientes Computacionais
		\hline
		@ Total                 & sum(b2:b5)      & sum(c2:c5)            \\
		\hline
	\end{spreadtab}
\end{table}

\begin{table}[!ht]
	\centering
	\caption{10\textordmasculine~Período}
	\label{tab10p}
	\begin{spreadtab}{{tabularx}{\textwidth}{|X|c|c|}}
		\hline
		@ {\textbf{Disciplina}} & @ {\textbf{CH}} & @ {\textbf{Créditos}} \\
		\hline
		@ \EletB                & \EletBCH        & \EletBCred            \\ % Disciplina Eletiva B
		@ \EletC                & \EletCCH        & \EletCCred            \\ % Disciplina Eletiva C
		@ \ProjB                & \ProjBCH        & \ProjBCred            \\ % Projeto de Graduação XI
		@ \Adm                  & \AdmCH          & \AdmCred              \\ % Administração
		\hline
		@ Total                 & sum(b2:b5)      & sum(c2:c5)            \\
		\hline
	\end{spreadtab}
\end{table}

\FloatBarrier % Evitando que exista texto entre as 10 tabelas.

\begin{table}[!ht]
	\centering
	\caption{Disciplinas Eletivas Restritas}
	\label{tabeletivas}
	\begin{spreadtab}{{tabularx}{\textwidth}{|X|c|c|}}
		\hline
		@ {\textbf{Disciplina}} & @ {\textbf{CH}}  & @ {\textbf{Créditos}} \\
		\hline
		@ \EletArq              & \EletArqCH       & \EletArqCred          \\ % Arquiteturas Avançadas de Computadores 
		@ \EletReforco          & \EletReforcoCH   & \EletReforcoCred      \\ % Aprendizado por Reforço
		@ \EletVisao            & \EletVisaoCH     & \EletVisaoCred        \\ % Aprendizado Profundo para Visão Computacional
		@ \AprendProfPLN        & \AprendProfPLNCH & \AprendProfPLNCred    \\ % Aprendizado Prof. p/ Proc.de Ling. Natural
		@ \AutomProcRob         & \AutomProcRobCH  & \AutomProcRobCred     \\ % Automação de Processos Robóticos
		@ \EletGeo              & \EletGeoCH       & \EletGeoCred          \\ % Geomática
		@ \EletRedes            & \EletRedesCH     & \EletRedesCred        \\ % Redes de Interconexão			
		@ \SistOpRobInt         & \SistOpRobIntCH  & \SistOpRobIntCred     \\ % Sistemas Operacionais p/ Robótica Inteligente
		@ \TecProgOtim          & \TecProgOtimCH   & \TecProgOtimCred      \\ %  Técnicas de Programação em Otimização Combinatória
		@ \TopEspVisComp        & \TopEspVisCompCH & \TopEspVisCompCred    \\ % Tópicos Especiais em Visão Computacional
		\hline
	\end{spreadtab}
\end{table}

\subsection{Normas Gerais de Ensino de Graduação da UERJ}

O curso de Engenharia de Computação obedecerá ao regime de créditos e as aulas serão oferecidas nos turnos manhã e tarde, com aulas predominantemente pela manhã, para os aprovados classificados no primeiro semestre; e com aulas predominantemente pela tarde, para os aprovados classificados no segundo semestre.

As Normas Gerais de Ensino de Graduação da UERJ são definidas pela deliberação n\textordmasculine~33/95 da UERJ (anexo \ref{delib3395}), sendo seus aspectos principais apresentados a seguir:

\subsection{Relação entre crédito e carga horária}
A deliberação n\textordmasculine~59/2019 (anexo \ref{delib592019}) da UERJ alterou a antiga deliberação, passando o artigo 57 a contar com a redação a seguir:

\textit{
	\textbf{Art. 57} -- O número mínimo de créditos necessários para integralizar o currículo será estabelecido com base na carga horária total do curso.}

\textit{
	\textbf{\S 1\textsuperscript{o}}- Nos cursos de regime de crédito, a unidade padrão de crédito
	corresponde a 15 (quinze) horas, e as atividades de que trata o caput do
	presente artigo são:
	\begin{enumerate}[a)]
		\item Aula teórica;
		\item Trabalho de campo;
		\item Laboratório/aula prática;
		\item Estágio curricular;
		\item Prática como componente curricular.
	\end{enumerate}}

\subsection{Aproveitamento escolar}
\begin{itquotation}
	\setcounter{artigo}{94}
	\artigo A aprovação do aluno em disciplinas do Curso de Graduação desta Universidade terá por base notas e frequência. São condições para aprovação: obtenção de nota final mínima 5,0 (cinco vírgula zero), constituída pela média aritmética da média semestral e nota da prova final, frequência mínima de 75\% (setenta e cinco por cento) do total de horas/aula determinado para a disciplina.

	\begin{description}
		\item[]	\S 1\textsuperscript{o} Para cada disciplina haverá, pelo menos, duas avaliações por turma, por período letivo, sendo uma delas necessariamente individual e escrita. A média dos resultados dessas avaliações constitui a média semestral do aluno na disciplina.\\
		\item[]	\S 2\textsuperscript{o} O aluno que obtiver média semestral igual ou superior a 4,0 (quatro vírgula zero) terá direito à prova final.\\
		\item[]	\S 3\textsuperscript{o} O aluno que obtiver média semestral igual ou superior a 7,0 (sete vírgula zero) estará dispensado de prestar prova final.\\
		\item[]	\ldots

		      %\setcounter{paragrafo}{6}
		\item[]	\S 7\textsuperscript{o}  O aluno que obtiver nota final menor que 5,0 (cinco vírgula zero) ou média semestral inferior a 4,0 (quatro vírgula zero) será reprovado.\\
		\item[]	\S 8\textsuperscript{o} O aluno que não obtiver frequência mínima de 75\% (setenta e cinco por cento) do total de horas/aula determinadas pela disciplina será reprovado, sem direito à prova final e independente de alcançar nota final superior a 7,0 (sete vírgula zero).\\
	\end{description}

\end{itquotation}
\subsection{Período de integralização do curso}
\setcounter{artigo}{98}
\begin{itquotation}
	\artigo Somente receberá o diploma o aluno que cumprir a Integralização Curricular.
\end{itquotation}

O período mínimo de integralização curricular dos cursos de engenharia é de 10 (dez) semestres, exceto para os casos de isenção de disciplinas, em que é possível um tempo mínimo menor. Já o prazo máximo para essa integralização é de 18 (dezoito) semestres.


\subsection{Estágio Supervisionado}

A atividade de Estágio Supervisionado é um elemento curricular obrigatório.
A disciplina de \textbf{Estágio Supervisionado para Engenharia de Computação} é considerada a primeira experiência profissional do aluno como futuro Engenheiro de Computação, e, por isso, espera-se que o trabalho seja realizado em um ambiente corporativo. Durante o estágio, o aluno contará com a supervisão de um profissional da empresa e com o suporte do professor da disciplina, que poderá orientar e sugerir melhorias no desenvolvimento das atividades.

A inscrição nesta disciplina será permitida apenas aos alunos que tiverem concluído pelo menos 150 créditos, requisito estabelecido para garantir o nível de maturidade necessário ao desempenho dessa atividade. Independentemente do tipo de trabalho realizado, é essencial que ele combine conhecimentos, habilidades e competências adquiridos ao longo do curso.

O estágio deve oferecer aos alunos a oportunidade de refletir, analisar e propor soluções para problemas reais, promovendo a integração entre teoria e prática. As atividades poderão abranger qualquer área da Engenharia de Computação e devem envolver a identificação e abordagem de um problema com enfoque científico e tecnológico, a análise de viabilidade de possíveis soluções e a proposição de um projeto de engenharia. Além disso, o trabalho deve considerar aspectos econômicos e os possíveis impactos ambientais e sociais. Espera-se ainda que o aluno demonstre familiaridade com os avanços da ciência e da tecnologia, bem como com os desafios relacionados à inovação.

Ao final da disciplina, o aluno deverá elaborar um Relatório Final detalhando as atividades desenvolvidas e evidenciando os conteúdos do curso de Engenharia de Computação aplicados durante o estágio. Esse relatório será avaliado em conjunto com uma apresentação oral perante o professor da disciplina. Na avaliação do Relatório Final, serão considerados a qualidade da redação, o uso adequado da língua portuguesa, a relevância do trabalho e as contribuições proporcionadas à formação do aluno. Na apresentação oral, serão avaliadas a clareza na exposição das ideias e a capacidade de responder às questões formuladas pelo professor.

\subsection{Projeto de Graduação}

A conclusão do curso ocorre por meio das disciplinas \textbf{Metodologia Científica para Computação} e \textbf{Projeto de Graduação XI}, que correspondem ao desenvolvimento do Projeto Final de Curso. Nessas disciplinas, o aluno elabora um projeto aplicado de engenharia de computação, sob a orientação de docentes.

Para garantir que o aluno inicie o Projeto Final de Curso somente após cursar um conjunto significativo de disciplinas que lhe permita enfrentar essa etapa de forma adequada, exige-se a conclusão de, no mínimo, 150 créditos (trava de créditos) como pré-requisito para a inscrição na primeira disciplina, \textbf{Metodologia Científica para Computação}.

Durante essas disciplinas, os alunos poderão trabalhar individualmente ou em duplas, e, sob orientação efetiva dos professores, conceberão e desenvolverão soluções criativas e viáveis para o tema selecionado. O trabalho envolverá a aplicação integrada de todo o arcabouço técnico adquirido ao longo do curso, promovendo uma síntese dos conhecimentos e habilidades desenvolvidos.

Dessa forma, essas disciplinas expõem os alunos ao desafio de conduzir um projeto de grande porte ao longo de dois períodos letivos, exigindo a aplicação prática e a integração dos conceitos aprendidos durante a graduação. As etapas de desenvolvimento do Projeto Final de Curso ao longo dessas disciplinas incluem:

\begin{itemize}
	\item \textbf{Metodologia Científica para Computação}: Levantamento de informações, definição do tema, pesquisa bibliográfica e desenvolvimento da parte teórica de fundamentação do projeto, de acordo com as normas técnicas.
	\item \textbf{Projeto de Graduação XI}: Realização de ensaios em bancada, prototipagem e/ou simulação computacional, finalização do documento e apresentação do trabalho final perante uma banca examinadora composta, no mínimo, por dois membros convidados pelo professor da disciplina, além do orientador.
\end{itemize}

Os documentos gerados ao longo do projeto deverão seguir o formato sugerido pela biblioteca da instituição.

Na defesa, serão avaliados a qualidade do documento final, a relevância e a coerência dos resultados obtidos, bem como o domínio do tema demonstrado pelo aluno durante a arguição.



\section{Equivalência com o Curso Anterior - \textcolor{red}{Thiago}}

% O curso de Engenharia de Computação será uma reformulação do curso de Engenharia Elétrica com ênfase em Sistemas e Computação. Os estudantes matriculados no curso anterior não poderão realizar a migração direta para a nova versão do curso, sendo necessário seguir os meios de ingresso descritos na Seção \ref{sec:forma-ingresso}.

Como o curso de Engenharia de Computação é uma reformulação do curso de Engenharia Elétrica com ênfase em Sistemas e Computação, os estudantes matriculados no curso anterior poderão realizar a migração para a nova versão do curso, seguindo a equivalência existente entre as disciplinas.%, sendo necessário seguir os meios de ingresso descritos na Seção \ref{sec:forma-ingresso}.

A transição entre as duas versões do curso exige atenção às equivalências curriculares, que estão detalhadas nas tabelas apresentadas a seguir. A Tabela \ref{tab:desc-equi} mapeia as disciplinas equivalentes entre a nova versão do curso de Engenharia de Computação e o curso anterior de Engenharia Elétrica com ênfase em Sistemas e Computação. Além disso, a Tabela \ref{tab:desc-sem-equi} apresenta as novas disciplinas que foram introduzidas no currículo e, por isso, não possuem correspondência direta com as disciplinas do curso anterior, refletindo as atualizações e a modernização da proposta pedagógica.

% \rowcolors{1}{gray!5}{white}
% \begin{table}[ht]
% 	\caption{Disciplinas ofertadas por outros departamentos}
% 	\label{tab:desc-iguais}
% 	\centering
% 	\renewcommand{\arraystretch}{1.5}
% 	\begin{tabularx}{\textwidth}{|X|l|}
% 		\showrowcolors
% 		\hline
% 		{\textbf{Disciplina}} & \textbf{Código} \\
% 		\hline
% 		\Adm                  & \AdmCod         \\ % Administração
% 		\AlgLin               & \AlgLinCod      \\ % Álgebra Linear
% 		\CalcI 				  & \CalcICod		\\ % Cálculo 1
% 		\CalcII 			  & \CalcIICod		\\ % Cálculo 2
% 		\CalcIII 			  & \CalcIIICod		\\ % Cálculo 3
% 		\CEV                  & \CEVCod         \\ % Circuitos em Corrente Contínua
% 		\EletI                & \EletICod       \\ % Eletrônica I
% 		\Empre
% 		% \FisI                 & \FisICod        \\ % Física I
% 		% \FisII                & \FisIICod       \\ % Física II
% 		% \FisIII               & \FisIIICod      \\ % Física III
% 		% \FisIV                & \FisIVCod       \\ % Física IV
% 		\IntEco               & \IntEcoCod      \\ % Macroeconomia
% 		\IntAmb               & \IntAmbCod      \\ % Introdução à Engenharia Ambiental
% 		\MatEle               & \MatEleCod      \\ % Materiais Elétricos e Magnéticos
% 		\ModMat               & \ModMatCod      \\ % Sinais e Sistemas
% 		\ProbEst              & \ProbEstCod     \\ % Probabilidade e Estatística
% 		\ProjA                & \ProjACod       \\ % Metodologia Científica
% 		\ProjB                & \ProjBCod       \\ % Projeto de Graduação XI
% 		\hline
% 	\end{tabularx}
% \end{table}

\rowcolors{1}{gray!5}{white}
\begin{table}[!ht]
	\centering
	\renewcommand{\arraystretch}{1.5}
	\caption{Equivalências no novo currículo}
	\label{tab:desc-equi}
	\resizebox{0.95\textwidth}{!}{%
		\begin{tabularx}{\textwidth}{|X||X|l|}
			\hline
			{\textbf{Currículo Novo}} & \textbf{Equivalente no Currículo Antigo}                  & \textbf{Código}               \\
			\hline
			\AlgComp                  & Algoritmos Computacionais                                 & FEN06-03559                   \\
			\EngCompSoc               & \parbox[t]{4cm}{Engenharia na Sociedade I                                                 \\ Engenharia na Sociedade II} & \parbox[t]{2cm}{FEN02-01064 \\ FEN02-01390}\\
			\CalcI                    & Cálculo Diferencial e Integral I                          & IME01-00508                   \\
			\IntAmb                   & Introdução à Engenharia Ambiental                         & FEN02-01064                   \\


			\AlgLin                   & Álgebra Linear III                                        & IME02-01388                   \\
			\EstrInf                  & Estruturas de Informação I                                & FEN06-03648                   \\
			\CalcII                   & Cálculo Diferencial e Integral II                         & IME01-00854                   \\

			\EngComput                & Cálculo Numérico IV                                       & IME04-04541                   \\
			\parbox[t]{4cm}{\FisI                                                                                                 \\ \FisEI} & Física Teórica e Experimental I & FIS01-05095 \\


			\AnAlg                    & Análise de Algoritmos                                     & FEN06-03713                   \\
			\CalcIII                  & Cálculo Diferencial e Integral III                        & IME01-03646                   \\
			\ProbEst                  & Probabilidade e Estatística                               & IME05-05316                   \\
			\parbox[t]{4cm}{\FisII                                                                                                \\ \FisEII} & Física Teórica e Experimental II & FIS02-05143 \\

			\LabProgA                 & Laboratório de Programação I                              & FEN06-04049                   \\

			\LabProgBSName            & Carac. das Linguagens de Prog. I                          & FEN06-03980                   \\
			\parbox[t]{4cm}{\FisIII                                                                                               \\ \FisEIII} & Física Teórica e Experimental III & FIS03-05185 \\


			\FundComp                 & Fundamentos de Comp. Digitais I                           & FEN06-03787                   \\
			\MatEle                   & Materiais Elétricos e Magnéticos I                        & FEN04-05197                   \\
			\ModMat                   & Modelos Mat. Aplicados a Eng. Elétrica III                & FEN05-04923                   \\
			\CEV                      & Circuitos Elétricos I                                     & FEN04-00944                   \\
			\parbox[t]{4cm}{\FisIV                                                                                                \\ \FisEIV} & Física Teórica e Experimental IV & FIS04-05212 \\
			\EngSistA                 & Engenharia de Sistemas A                                  & FEN06-04243                   \\
			\ArqComp                  & Arquitetura de Computadores I                             & FEN06-04119                   \\
			\CEVI                     & Circuitos Elétricos IV                                    & FEN04-05222                   \\
			\EletI                    & Eletrônica I                                              & FEN05-01620                   \\

			\ProjBD                   & Engenharia de Sistemas B                                  & FEN06-04314                   \\
			\TeoComp                  & Teoria de Compiladores                                    & FEN06-04516                   \\
			\TelepSName               & Teleproc. e Redes de Computadores                         & FEN06-04718                   \\
			\ProjSO                   & Arquitetura de Sistemas Operacionais                      & FEN06-04664                   \\

			\EngSistC                 & Engenharia de Sistemas C                                  & FEN06-04386                   \\
			\Control                  & Controle de Processos por Comp.                           & FEN06-05080                   \\

			\CompParal                & Microcomp. e Microprocessadores                           & FEN06-04192                   \\

			\ProjA                    & Projeto de Graduação XI-A                                 & FEN06-04578                   \\
			\Instala                  & Instalações de Ambiente Computacional                     & FEN06-03860                   \\
			\EstSupSName              & Estágio Supervisionado XI-A                               & FEN06-05038                   \\
			\IntEco                   & Introdução à Economia III                                 & FCE02-04657                   \\


			\ProjB                    & Projeto de Graduação XI-B                                 & FEN06-04635                   \\
			\Adm                      & Administração Aplicada à Engenharia III                   & FAF03-04439                   \\

			Eletivas Restritas        & Tóp. Especiais em Eng. de Sistemas e Computação A, B ou C & \parbox[t]{2.5cm}{FEN06-04889 \\FEN06-04939\\FEN06-04990}  \\
			\hline
		\end{tabularx}}
\end{table}

\begin{table}[!ht]
	\centering
	\renewcommand{\arraystretch}{1.5}
	\caption{Disciplinas sem Equivalências}
	\label{tab:desc-sem-equi}
	\begin{tabularx}{0.6\textwidth}{|X|}
		\hline
		{\textbf{Disciplinas do Novo Currículo sem Equivalência}} \\
		\hline
		\LogProg                                                  \\
		\ProcImag                                                 \\
		\Grafos                                                   \\
		\IC                                                       \\
		\ICII                                                     \\
		\MineraDados                                              \\
		\Sredes                                                   \\
		\SistEmb                                                  \\
		\Empre                                                    \\
		% \EngCompSoc                                               \\
		% \EstSup													  \\
		\hline
	\end{tabularx}
\end{table}

\section{Ementário das Disciplinas - \textcolor{red}{Thiago}}

As ementas das disciplinas obrigatórias e eletivas são apresentadas no anexo \ref{ementas}.

\section{Atividades Acadêmicas da Graduação articuladas ao ensino de Pós-Graduação (Aperfeiçoamento, Mestrado, Doutorado) \textcolor{red}{Luiza} }

A maioria das disciplinas do curso de Engenharia de Computação envolve tanto atividades de teoria quanto de laboratório. Neste sentido, há um enfoque em aplicar os conhecimentos teóricos na resolução de problemas, tanto em Engenharia quanto em áreas afins, como Ciência da Computação. Além das atividades de laboratório vinculadas às disciplinas da graduação, os docentes desenvolvem atividades de pesquisa como parte da carga horária semanal. Isto oferece a oportunidade dos alunos participarem desses projetos através da Iniciação Científica (IC), contando com o apoio do Programa Institucional de Bolsa de Iniciação Científica da UERJ (PIBIC/UERJ), ou através do Trabalho de Conclusão de Curso (TCC). O PIBIC não somente oferece bolsa de IC como também permite vincular o aluno sem bolsa ao projeto de pesquisa através do Estágio Voluntário. A inclusão dos alunos nos projetos de pesquisa permite despertar neles o interesse e a motivação pela investigação científica e novas tecnologias, promovendo a inserção na pós-graduação. Há docentes no departamento participando de Programas de Pós-Graduação da UERJ, contribuindo para a consolidação e crescimento dos mesmos, em termos de disciplinas oferecidas, orientações, produção científica, participação e organização de reuniões científicas (congressos, seminários, simpósios). Dessa forma, há um constante estímulo para que os alunos da graduação venham a participar dos cursos de pós-graduação como forma de agregar conhecimento sobre novas tecnologias e metodologias, promovendo o crescimento acadêmico e melhorando o perfil do egresso.

\section{Atividades de Extensão \textcolor{red}{Giomar}}

\subsection{Filosofia}
Entende-se a extensão universitária como uma relação transformadora entre a Universidade e a sociedade. As atividades de extensão enriquecem e ampliam o perfil dos estudantes, possibilitando que compartilhem habilidades, conhecimentos, competências e atitudes com a comunidade. Dessa forma, a extensão permite à Universidade levar o conhecimento além de seus limites, promovendo a democratização do saber.

A extensão universitária viabiliza o fortalecimento dos vínculos entre a academia e a sociedade, considerando as diversas realidades socioeconômicas de comunidades rurais, periurbanas e urbanas. Ao integrar os conhecimentos adquiridos em sala de aula com as demandas práticas da realidade, essas iniciativas contribuem para a sustentabilidade de setores socioeconômicos que desempenham um papel crucial no futuro das gerações, mas que enfrentam desafios constantes, como a exclusão digital.

Adicionalmente, a extensão oferece aos estudantes a oportunidade de participar de atividades que promovem o exercício da cidadania, ao prestar serviços relevantes a segmentos da sociedade frequentemente marginalizados e desprovidos de assistência. Por meio dessas ações, a universidade não apenas amplia seu impacto social, mas também fomenta uma formação mais ética e solidária entre seus discentes.


\subsection{Regulamentação}

O Conselho Nacional de Educação (CNE), por meio da Câmara de Ensino Superior (CES), estabeleceu pela Resolução CNE/CES N\textsuperscript{o}~7, de 18 de dezembro de 2018, que todos os cursos de graduação no Brasil devem incluir, no mínimo, 10\% de sua carga horária em atividades de extensão.

Na Universidade do Estado do Rio de Janeiro (UERJ), a regulamentação sobre a curricularização das atividades extensionistas foi definida pela Deliberação N\textsuperscript{o}~04/2023, que mantém o limite mínimo estipulado pelo CNE. Além disso, a norma determina que essas atividades podem ser desenvolvidas em três modalidades (conforme o Art. 6\textordmasculine~da referida Deliberação):

\begin{itemize}
	\item Disciplinas com carga horária parcialmente dedicada à extensão;
	\item Disciplinas integralmente dedicadas à extensão;
	\item Atividade Curricular de Extensão (ACE).
\end{itemize}

\subsection{Proposta pedagógica}

As atividades de extensão no curso de Engenharia de Computação da UERJ oferecem aos estudantes a oportunidade de aplicar os conhecimentos adquiridos ao longo da graduação em iniciativas que promovem interação direta com a sociedade. Um dos principais objetivos dessas ações é combater a evasão acadêmica, incentivando a motivação e o engajamento dos discentes por meio da oferta de disciplinas profissionalizantes desde o início do curso.

Nesta mesma direção, o curso busca proporcionar aos alunos uma visão realística e analítica sobre o papel da tecnologia na sociedade através da oferta de novas disciplinas e de atividades de Extensão Universitária, suscitando o desenvolvimento de uma perspectiva disruptiva e contextualizada através da qual os alunos se sintam encorajados a se envolver com iniciativas voltadas para o desenvolvimento social. Desta forma, as atividades de Extensão Universitária são vistas como um processo de aprofundamento educativo e cultural, propiciando o exercício da interdisciplinaridade e estimulando o pensamento reflexivo, analítico e crítico nos estudantes. A partir do conjunto coordenado de projetos, programas, eventos e disciplinas, promovem interações transformadoras entre a universidade e os diversos setores da sociedade.

Tais iniciativas possuem caráter formativo e buscam alinhar teoria e prática, indo além do simples cumprimento de horas acadêmicas. Em essência, a extensão universitária deve gerar impacto educacional significativo, promovendo uma troca mútua: enquanto a sociedade se beneficia das ações realizadas, os estudantes enriquecem sua formação acadêmica e cultural, ampliando sua visão sobre as demandas sociais.

\subsection{Atividades extensionistas}

No currículo do curso de Engenharia de Computação da UERJ, as atividades extensionistas totalizam 357 horas, representando 10,04\% da carga horária total do curso, que é de 3.567 horas, em conformidade com a determinação do CNE.

Essas horas de extensão estão distribuídas entre duas modalidades: \textbf{Disciplinas integralmente dedicadas à extensão} e \textbf{Atividade Curricular de Extensão}, conforme detalhado a seguir.

\textbf{Disciplinas integralmente dedicadas à extensão}

Propõe-se uma disciplina obrigatória denominada “Introdução a Projetos de Extensão” oferecida pelo Departamento de Engenharia de Sistemas e Computação que contempla 30h de carga horária voltada à extensão. Para facilitar o planejamento e a execução das atividades extensionistas pelos alunos, recomenda-se que a disciplina  extensionista seja realizada no terceiro ou quarto período pelos alunos. Essa recomendação é fundamentada na estrutura curricular, que prevê uma menor carga horária de disciplinas obrigatórias nesses períodos, permitindo a realização das atividades extensionistas sem sobrecarregar os estudantes. Ao mesmo tempo, este primeiro contato com a Extensão possibilita ao estudante planejar como suas atividades extensionistas serão realizadas ao longo do curso.

A disciplina proposta, contempla a definição de um tema específico pelo docente a cada semestre. Os estudantes matriculados desenvolverão atividades relacionadas ao tema ao longo do período letivo, envolvendo-se em ações como a elaboração de conteúdos, produção de materiais, estratégias de divulgação, suporte acadêmico e práticas didáticas. No âmbito do curso de Engenharia de Computação, exemplos de atividades de extensão incluem o desenvolvimento de um site utilizando plataformas gratuitas para disseminação de informações relacionadas ao tema; a criação de um canal no YouTube voltado à divulgação de tutoriais, aulas e outros conteúdos técnicos; e a realização de oficinas práticas que abordem temas como introdução à programação, segurança digital, e o uso de ferramentas tecnológicas para resolução de problemas sociais. Além disso, outras ações podem incluir a realização de \textit{hackathons} com foco em soluções comunitárias ou a implementação de sistemas de gerenciamento voltados a organizações locais ou escolas, promovendo um impacto direto e positivo na sociedade.

\textbf{Atividade Curricular de Extensão (ACE)}

Para integralizar o curso de graduação em Engenharia de Computação da UERJ, o aluno deve cumprir, no mínimo, 327 horas de Atividade Curricular de Extensão (ACE).

Conforme estabelecido na Deliberação n\textordmasculine~04/2023, as ACEs devem ser realizadas ao longo do curso e concluídas antes do término da graduação. Nessas atividades, é imprescindível que o aluno atue de forma ativa e protagonista.

Em consonância com a referida deliberação e alinhado aos objetivos pedagógicos da formação em Engenharia de Computação, as ACEs válidas para a integralização do curso incluem:


\begin{itemize}
	\item \textbf{Participação em programas e projetos de extensão} (como bolsista ou voluntário), coordenados por professores ou técnicos da carreira de nível superior na Universidade do Estado do Rio de Janeiro, com validação da carga horária por meio de Declaração ou Certificado emitido pelo coordenador do projeto;
	\item \textbf{Ações extensionistas realizadas em programas institucionais das Pró-Reitorias}, com validação da carga horária por meio de Declaração ou Certificado do coordenador do programa;
	\item \textbf{Estágios não obrigatórios caracterizados como ação extensionista}, com validação da carga horária por meio de declaração ou Certificado da instituição responsável;
	\item \textbf{Desenvolvimento conjunto de soluções tecnológicas} visando atender as demandas dos atores da sociedade civil, com validação da carga horária por meio de Declaração ou Certificado da instituição responsável;
	\item \textbf{Participação em eventos}, tanto na organização quanto na realização, com validação da carga horária por meio de Declaração ou Certificado da instituição responsável;
	\item \textbf{Publicações relacionadas à extensão}, nas seguintes categorias:
	      \begin{itemize}
		      \item Artigo em revista indexada ou capítulo de livro com ISBN: 15 horas por publicação, validadas com informações catalográficas, ISSN ou ficha catalográfica e primeira página do artigo ou capítulo;
		      \item Livro com ISBN: 30 horas por publicação, validadas com capa, contracapa e ficha catalográfica;
		      \item Resumos e resumos expandidos publicados em anais de eventos: 5 horas por publicação, validadas com informações catalográficas e primeira página do resumo.
	      \end{itemize}

\end{itemize}

Em todas as situações citadas acima, as declarações ou certificados emitidos pelas entidades responsáveis deverão conter a carga horária e a descrição das atividades realizadas pelo aluno. Compete à Coordenação das Atividades Curriculares de Extensão (CACE) do curso de Engenharia de Computação receber, analisar e validar a documentação comprobatória da realização das ACE.

Para que possam ser computadas para a integralização curricular do curso, as atividades devem ser preferencialmente realizadas no âmbito das unidades pertencentes ao Centro de Tecnologia e Ciências (CTC/UERJ). No caso de alunos que tenham cursado Atividades Curriculares de Extensão (ACEs) em outras instituições, a carga horária poderá ser aproveitada mediante análise realizada pela CACE do curso de Engenharia de Computação. Essa análise avaliará a pertinência e a compatibilidade da atividade com os objetivos do curso.

Após a conclusão da ACE, o estudante deverá encaminhar à CACE do curso de Engenharia de Computação a documentação comprobatória contendo a carga horária total, o tipo e o título da atividade. Com base nesta documentação, o responsável pela coordenação deliberará sobre a aprovação ou não da atividade realizada. Detalhes adicionais sobre o processo de homologação podem ser consultados na Deliberação n\textordmasculine~04/2023, recomendando-se a leitura deste documento para obtenção de informações complementares.

\textbf{Observação:} Em caso de dúvidas, os alunos são orientados a consultar antecipadamente a CACE para verificar a pertinência das atividades planejadas, evitando problemas de interpretação.
