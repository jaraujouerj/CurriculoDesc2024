% !TEX root = ProjetoPedagogico.tex

\chapter{Organização Didático Pedagógica}

O curso ora proposto de Engenharia de Computação obedecerá ao regime de créditos, oferecendo \vagas vagas anuais, repartidas igualmente em dois semestres letivos. O aluno interessado em cursar a graduação em Engenharia de Computação fará tal opção diretamente a partir da sua inscrição no vestibular.




\section{Justificativa das Necessidades Sociais do Curso -\textcolor{red}{Gabriel}}

Olhar PP da Engenharia Eletrica pagina 376

\section{Finalidades e Objetivos do Curso - \textcolor{red}{Simone}}

\subsection{Histórico}

\subsection{Concepção}

A estrutura curricular do curso de Engenharia de Computação do \desc da Faculdade de Engenharia da UERJ orientar-se-á pelas \textit{Diretrizes Curriculares Nacionais para os cursos de graduação na área da Computação}, do Ministério da Educação (anexo \ref{cne}), pelos \textit{Referenciais de Formação para os cursos de Graduação em Computação}, da Sociedade Brasileira de Computação (anexo \ref{sbc}), e pela regulamentação do exercício da profissão de Engenheiro, estabelecida pelo Sistema CREA/CONFEA (Resolução 1.010 CONFEA, anexo \ref{res1010}), em vigor atualmente. A inserção da extensão no curriculo terá como base a Deliberação n\textordmasculine{} 4/2023 do Conselho Superior de Ensino, Pesquisa e Extensão (CSEPE/UERJ, anexo \ref{del4}).

A grade curricular totaliza \totalhoras horas, sendo \hobrigatorias horas em disciplinas e \hextensao horas em atividades de extensão. As \hobrigatorias horas de disciplinas estão distribuídas em \ndisciplinas disciplinas, sendo \nobrigatorias  obrigatórias e \neletivas eletivas restritas. Além das disciplinas teóricas, o curso inclui práticas laboratoriais para complementar a base teórica. O currículo também contempla Estágio Supervisionado e Projeto de Graduação (trabalho de conclusão de curso) como atividades de síntese e integração do conhecimento científico, tecnológico e instrumental. Como atividades acadêmicas complementares facultativas, os alunos podem optar por Estágio Interno, Monitoria, Iniciação Científica, Cursos, Eventos, Palestras e Visitas Técnicas, que visam proporcionar uma melhor compreensão da Engenharia, do setor no Brasil e das áreas de atuação e atividades dos Engenheiros de Computação.

\subsection{Finalidades}

\subsection{Objetivos}

\section{Nível de Formação e Título Acadêmico}

O curso é de graduação plena e a titulação concedida e habilitação são:

\begin{itemize}
\item{Título: Engenheiro}
\item{Habilitação: Engenharia de Computação}
\end{itemize}

\section{Perfil do Egresso (competência, habilidades e atitudes pretendidas) - \textcolor{red}{Gabriel}}

O curso de Engenharia de Computação tem como perfil do egresso o engenheiro, com formação técnico-científica sólida, generalista, humanista, crítica e reflexiva, capacitado a absorver e desenvolver novas tecnologias, estimulando a sua atuação crítica e criativa na identificação e resolução de problemas, considerando seus aspectos políticos, econômicos, sociais, ambientais e culturais, com visão ética e humanística, em atendimento às demandas da sociedade. Faz parte do perfil do egresso a postura de permanente busca da atualização profissional, além das seguintes habilidades:
\begin{enumerate} [I -]
	\item possuir conhecimento das questões humanísticas, sociais, ambientais, éticas, profissionais, legais e políticas;
	\item possuir compreensão do impacto da Engenharia de Computação e suas tecnologias no que concerne ao atendimento e à antecipação estratégica das necessidades da sociedade;
	\item possuir atitude crítica, interdisciplinar e criativa na identificação e resolução de problemas;
	\item possuir compreensão das necessidades de contínua atualização e aprimoramento de suas competências e habilidades;
	\item possuir uma sólida formação em Computação, Física, Matemática, Eletrônica, Automação e Telecomunicações.
	\item conhecer a estrutura dos sistemas de computação e os processos envolvidos na sua análise e construção;
	\item considerar os aspectos ambientais, econômicos, financeiros, de gestão e de qualidade, associados a novos produtos e organizações;
	\item considerar fundamental a inovação, a criatividade, a atitude empreendedora e a inserção internacional.
\end{enumerate}

O egresso da Engenharia de Computação, no processo de sua formação, deverá desenvolver as seguintes competências:
\begin{enumerate} [I -]
	\item antever as implicações humanísticas, sociais, ambientais, éticas, profissionais, legais (inclusive relacionadas à propriedade intelectual) e políticas dos sistemas computacionais;
	\item identificar demandas socioeconômicas e ambientais relevantes, planejar, especificar e projetar sistemas de computação, seguindo teorias, princípios, métodos e procedimentos interdisciplinares;
	\item construir, testar, verificar e validar sistemas de computação, seguindo métodos, técnicas e procedimentos interdisciplinares;
	\item perceber as necessidades de atualização decorrentes da evolução tecnológica e social;
	\item relacionar problemas do mundo real com suas soluções, considerando aspectos de computabilidade e de escalabilidade;
	\item analisar, desenvolver, avaliar e aperfeiçoar software e hardware em arquiteturas de computadores;
	\item analisar, desenvolver, avaliar e aperfeiçoar sistemas de automação e sistemas inteligentes;
	\item analisar, desenvolver, avaliar e aperfeiçoar sistemas de informação computacionais;
	\item analisar, desenvolver, avaliar e aperfeiçoar circuitos eletroeletrônicos;
	\item gerenciar pessoas e infraestrutura de Sistemas de Computação;
	\item perceber as necessidades de inovação e inserção internacional com atitudes criativas e empreendedoras.
\end{enumerate}

O curso de Engenharia de Computação tem, predominantemente, o ensino da computação como atividade fim, visando à formação de recursos humanos para o desenvolvimento científico e tecnológico da computação. Assim sendo, o curso deve capacitar indivíduos para desenvolver software e hardware, com uma forte base matemática e física.

Os egressos do curso de Engenharia de Computação estarão situados no estado da arte da ciência e da tecnologia da computação, de tal forma que possam continuar suas atividades na pesquisa, promovendo o desenvolvimento científico, ou aplicando os conhecimentos científicos, propiciando o desenvolvimento tecnológico. Para tal, é dada uma forte ênfase no uso de laboratórios para capacitar os egressos no projeto e construção tanto de software quanto de hardware.

\section{Administração Acadêmica do Curso - \textcolor{red}{Rafaela}}

\subsection{Coordenação de Áreas}

As disciplinas do curso de Engenharia de Computação estão divididas em quatro grandes áreas de conhecimento: (1) Sistemas de Informação; (2) Arquitetura de Sistemas de Computação; (3) Algoritmos e Linguagens de Programação; (4) Lógica e Inteligência Computacional.

A integração das disciplinas em áreas de conhecimento permite o compartilhamento de informações sobre interesses e objetivos comuns. Favorece a atuação conjunta de alunos e professores em temas globais e impulsiona a criação de linhas de pesquisa.

Cada área de conhecimento deverá possuir um Professor Coordenador. O Coordenador de área será responsável pelas disciplinas de sua área, cabendo a ele(a): orientar os alunos em questões referentes às disciplinas, analisar os requerimentos de quebras de pré-requisitos e conflitos de horário, tratar questões relativas aos conteúdos programáticos das disciplinas, promover a integração dos professores da mesma área, e  incentivar a pesquisa na área.

A tabela \ref{tab:areas} mostra a distribuição das disciplinas por área de conhecimento.

\begin{table}[ht]
	\centering
	\caption{Tabela de divisão de disciplinas por área de conhecimento}
	\label{tab:areas}
	\begin{tabularx}{\textwidth}{ X  l }
		\hiderowcolors
		\hline
		{\bf Área de Conhecimento}                              & {\bf Disciplinas} \\
		\hline
		\multirow{5}{*}{Algoritmos e Linguagens de Programação} & \AlgComp          \\ % Algoritmos Computacionais
																& \AnAlg            \\ % Análise de Algoritmos
		                                                        & \EstrInf          \\ % Estruturas de Informação
		                                                        & \LabProgA         \\ % Laboratório de Programação A
		                                                        & \LabProgB         \\ % Laboratório de POO
		                                                        & \TeoComp          \\ % Teoria da Compiladores
																& \Grafos			\\ \hline
		\multirow{8}{*}{Arquitetura de Sistemas de Computação}  & \ArqComp          \\ % Arqutetura de Computadores
																& \CompParal        \\ % Computação Paralela
		                                                        & \Control          \\ % Controle de Processos
		                                                        & \FundIComp         \\ % FUndamentos de Computadores I
																& \FundComp         \\ % FUndamentos de Computadores II
																& \Instala			\\ % Instalações de Ambientes Computacionais
		                                                        & \ProjSO           \\ % Projeto de Sistemas 
																& \Telep            \\ % Redes de Computadores
		                                                        & \Sredes         	\\ % Segurança em Redes
		                                                        & \SistEmb          \\ % Sistema Embutidos
		                                                        \hline	
		\multirow{3}{*}{Lógica e Inteligência Computacional}    & \IC               \\ % Inteligência Computacional
		                                                        & \ICII              \\ % Inteligência Computacional II
		                                                        & \LogProg          \\ % Lógica de Programação
																& \MineraDados      \\ % Mineração de Dados
		                                                        & \ProcImag         \\ % Processamento de Imagens
		                                                        \hline													
		\multirow{4}{*}{Sistemas de Informação}                 & \EngSistC         \\ % Análise de Projeto de Sistemas
																& \EngSistA		 \\  % Engenharia de Sistemas
		                                                        & \ProjBD           \\
		                                                        & \EngCompSoc       \\
		                                                        \hline
	\end{tabularx}
\end{table}

\subsection{Conselho Departamental}

\subsection{Chefia de Departamento}

\section{Currículo Pleno e Estrutura Curricular - \textcolor{red}{Robert}}

\subsection{Organização do Currículo}

O currículo do curso de Engenharia de Computação é constituído por disciplinas obrigatórias e eletivas, estágio supervisionado, trabalho de conclusão de curso e atividades de extensão. O curso é organizado em 10 semestres, podendo o aluno cumpri-lo em um máximo de 18 semestres.

Para uma eficaz orientação pedagógica, é proposto o aconselhamento curricular apresentado nas tabelas \ref{tab1p} a \ref{tab10p}. Os pré-requisitos das disciplinas podem ser observados no fluxograma do curso (anexo \ref{fluxograma}).

O aluno deverá cursar no mínimo três das disciplinas eletivas restritas oferecidas (ver tabela \ref{tabeletivas}). Deve ser
ressaltado que estas disciplinas são oferecidas de acordo com o interesse dos corpos
docente e discente, não sendo necessariamente disponibilizadas todos os semestres.

\rowcolors{1}{gray!5}{white}
\setlength{\tabcolsep}{5pt}
\renewcommand{\arraystretch}{1.5}
\begin{table}[ht]
	\centering
	\caption{1\textordmasculine~Período}
	\label{tab1p}
	\begin{spreadtab}{{tabularx}{\textwidth}{ | X|c|c| }}
		\hline
		@ {\textbf{Disciplina}} & @ {\textbf{CH}} & @ {\textbf{Créditos}} \\
		\hline
		@ \AlgComp	& \AlgCompCH	& \AlgCompCred	\\ % Algoritmos Computacionais
		@ \EngCompSoc 	& \EngCompSocCH & \EngCompSocCred	\\ % Engenharia e Computação Sociedade
		@\AlgLin	& \AlgLinCH		& \AlgLinCred	\\ % Álgebra Linear
		@ \CalcI	& \CalcICH		& \CalcICred	\\ % Cálculo I
		@ \IntAmb	& \IntAmbCH		& \IntAmbCred	\\ % Introdução à Engenharia Ambiental
		\hline
		@ Total 	& sum(b2:b6) 	& sum(c2:c6)	\\
		\hline
	\end{spreadtab}
\end{table}

\rowcolors{1}{gray!5}{white}
\begin{table}
	\centering
	\caption{2\textordmasculine~Período}
	\label{tab2p}
	\begin{spreadtab}{{tabularx}{\textwidth}{|X|c|c|}}
		\hline
		@ {\textbf{Disciplina}} & @ {\textbf{CH}} & @ {\textbf{Créditos}} \\
		\hline
		@ \EstrInf	& \EstrInfCH	& \EstrInfCred 	\\ % Estruturas de Informação
		@ \LogProg	& \LogProgCH	& \LogProgCred	\\ % Lógica de Programação
		@ \CalcII	& \CalcIICH		& \CalcIICred	\\ % Cálculo II
		@ \EngComput& \EngComputCH	& \EngComputCred\\ % Calculo Numérico
		@ \FisI		& \FisICH		& \FisICred		\\ % Física I
		@ \FisEI	& \FisEICH		& \FisEICred	\\ % Física Experimental I
		\hline
		@ Total 	& sum(b2:b7) 	& sum(c2:c7)	\\
		\hline
	\end{spreadtab}
\end{table}
	
\rowcolors{1}{gray!5}{white}
\begin{table}
	\centering
	\caption{3\textordmasculine~Período}
	\label{tab3p}
	\begin{spreadtab}{{tabularx}{\textwidth}{|X|c|c|}}
		\hline
		@ {\textbf{Disciplina}} & @ {\textbf{CH}} & @ {\textbf{Créditos}} \\
		\hline
		@ \AnAlg	& \AnAlgCH		& \AnAlgCred	\\ % Análise de Algoritmos
		@ \CalcIII	& \CalcIIICH 	& \CalcIIICred	\\ % Cálculo III
		@ \FisII	& \FisIICH		& \FisIICred	\\ % Física II
		@ \FisEII	& \FisEICH		& \FisEICred	\\ % Física Experimental II
		@ \ProbEst	& \ProbEstCH	& \ProbEstCred	\\ % Probabilidade e Estatística
		\hline
		@ Total 	& sum(b2:b6) 	& sum(c2:c6)	\\
		\hline
	\end{spreadtab}
\end{table}

\rowcolors{1}{gray!5}{white}
\begin{table}
	\centering
	\caption{4\textordmasculine~Período}
	\label{tab4p}
	\begin{spreadtab}{{tabularx}{\textwidth}{|X|c|c|}}
		\hline
		@ {\textbf{Disciplina}} & @ {\textbf{CH}} & @ {\textbf{Créditos}} \\
		\hline
		@ \LabProgA	& \LabProgACH	& \LabProgACred		\\ % Laboratório de Programação A
		@ \LabProgB	& \LabProgBCH	& \LabProgBCred		\\ % Laboratório de Programação B
		@ \FisIII	& \FisIIICH		& \FisIIICred		\\ % Física III
		@ \FisEIII	& \FisEIIICH	& \FisEIIICred		\\ % Física Experimental III
		@ \ProcImag 	& \ProcImagCH	& \ProcImagCred	\\ % Processamento de Sinais e Imagens
		@ \FundIComp	& \FundICompCH	& \FundICompCred\\ %Técnicas Digitais I
		\hline
		@ Total 	& sum(b2:b7) 	& sum(c2:c7)	\\
		\hline
	\end{spreadtab}
\end{table}

\rowcolors{1}{gray!5}{white}
\begin{table}
	\centering
	\caption{5\textordmasculine~Período}
	\label{tab5p}
	\begin{spreadtab}{{tabularx}{\textwidth}{|X|c|c|}}
		\hline
		@ {\textbf{Disciplina}} & @ {\textbf{CH}} & @ {\textbf{Créditos}} \\
		\hline
		@ \Grafos	& \GrafosCH		& \GrafosCred	\\ % Teoria dos Grafos e Aplicações
		@ \FundComp	& \FundCompCH	& \FundCompCred	\\ % Fundamentos de Computadores I
		@\CEV		& \CEVCH		& \CEVCred		\\ % Circuitos em Corrente 
		@ \FisIV	& \FisIVCH		& \FisIVCred	\\ % Física IV
		@ \FisEIV	& \FisEIVCH		& \FisEIVCred	\\ % Física Experimental IV
		@ \MatEle 	& \MatEleCH		& \MatEleCred	\\ % Materiais Elétricos e Magnéticos 
		@ \ModMat	& \ModMatCH		& \ModMatCred	\\ % Sinais e Sistemas
		\hline
		@ Total 	& sum(b2:b8) 	& sum(c2:c8)	\\
		\hline
	\end{spreadtab}
\end{table}

\rowcolors{1}{gray!5}{white}
\begin{table}
	\centering
	\caption{6\textordmasculine~Período}
	\label{tab6p}
	\begin{spreadtab}{{tabularx}{\textwidth}{|X|c|c|}}
		\hline
		@ {\textbf{Disciplina}} & @ {\textbf{CH}} & @ {\textbf{Créditos}} \\
		\hline
		@ \ArqComp	& \ArqCompCH	& \ArqCompCred	\\ % Arquitetura de Computadores A
		@ \EngSistA & \EngSistACH	& \EngSistACred	\\ % Engenharia de Sistemas
		@ \IC		& \ICCH			& \ICCred		\\ % Inteligência Computacional I
		@ \ICII 	& \ICIICH		& \ICIICred		\\ % Inteligência Computacional II
		@ \CEVI		& \CEVICH 		& \CEVICred		\\ % Circuitos em Corrente Alternada
		@ \EletI	& \EletICH		& \EletICred	\\ % Eletrônica I
		\hline
		@ Total 	& sum(b2:b7) 	& sum(c2:c7)	\\
		\hline
	\end{spreadtab}
\end{table}

\rowcolors{1}{gray!5}{white}
\begin{table}
	\centering
	\caption{7\textordmasculine~Período}
	\label{tab7p}
	\begin{spreadtab}{{tabularx}{\textwidth}{|X|c|c|}}
		\hline
		@ {\textbf{Disciplina}} & @ {\textbf{CH}} & @ {\textbf{Créditos}} \\
		\hline
		@ \MineraDados	& \MineraDadosCH	& \MineraDadosCred	\\ % Mineração de Dados
		@ \ProjBD		& \ProjBDCH		& \ProjBDCred		\\ % Projeto de Banco de 
		@ \ProjSO		& \ProjSOCH		& \ProjSOCred		\\ % Projeto de Sistemas Operacionais
		@ \Telep 		& \TelepCH		& \TelepCred		\\ % Redes de Computadores
		@ \TeoComp		& \TeoCompCH	& \TeoCompCred		\\ % Teoria da Compiladores
		@ \IntEco		& \IntEcoCH		& \IntEcoCred	\\ % Macroeconomia 
		\hline
		@ Total			& sum(b2:b7)	& sum(c2:c7)		\\
		\hline
	\end{spreadtab}
\end{table}

\rowcolors{1}{gray!5}{white}
\begin{table}
	\centering
	\caption{8\textordmasculine~Período}
	\label{tab8p}
	\begin{spreadtab}{{tabularx}{\textwidth}{|X|c|c|}}
		\hline
		@ {\textbf{Disciplina}} & @ {\textbf{CH}} & @ {\textbf{Créditos}} \\
		\hline
		@ \EngSistC 	& \EngSistCCH		& \EngSistCCred		\\ % Análise e Projeto de Sistemas
		@ \Control		& \ControlCH		& \ControlCred		\\ % Controle de Processos
		@ \CompParal	& \CompParalCH		& \CompParalCred	\\ % Computação Paralela
		@ \Sredes 		& \SredesCH			& \SredesCred		\\ % Segurança em Redes
		@ \SistEmb		& \SistEmbCH		& \SistEmbCred		\\ % Sistemas Embutidos
		@ \Empre 		& \EmpreCH			& \EmpreCred		\\ % Empreendedorismo
		\hline
		@ Total				& sum(b2:b7)			& sum(c2:c7)			\\
		\hline
	\end{spreadtab}
\end{table}

\rowcolors{1}{gray!5}{white}
\begin{table}
	\centering
	\caption{9\textordmasculine~Período}
	\label{tab9p}
	\begin{spreadtab}{{tabularx}{\textwidth}{|X|c|c|}}
		\hline
		@ {\textbf{Disciplina}} & @ {\textbf{CH}} & @ {\textbf{Créditos}} \\
		\hline
		@ \EletA		& \EletACH		& \EletACred	\\ % Disciplina Eletiva A
		@ \EstSup		& \EstSupCH		& \EstSupCred	\\ % Estágio Supervisionado
		@ \ProjA		& \ProjACH		& \ProjACred	\\ % Metodologia Ciêntífica
		@ \Instala 		& \InstalaCH	& \InstalaCred	\\ % Instalações de Ambientes Computacionais
		\hline
		@ Total			& sum(b2:b5)	& sum(c2:c5)	\\
		\hline
	\end{spreadtab}
\end{table}

\begin{table}
	\centering
	\caption{10\textordmasculine~Período}
	\label{tab10p}
	\begin{spreadtab}{{tabularx}{\textwidth}{|X|c|c|}}
		\hline
		@ {\textbf{Disciplina}} & @ {\textbf{CH}} & @ {\textbf{Créditos}} \\
		\hline
		@ \EletB	& \EletBCH	& \EletBCred	\\ % Disciplina Eletiva B
		@ \EletC	& \EletCCH	& \EletCCred	\\ % Disciplina Eletiva C
		@ \ProjB	& \ProjBCH	& \ProjBCred	\\ % Projeto de Graduação XI
		@ \Adm		& \AdmCH	& \AdmCred		\\ % Administração
		\hline
		@ Total		& sum(b2:b5)& sum(c2:c5)	\\
		\hline
	\end{spreadtab}
\end{table}

\begin{table}
	\centering
	\caption{Disciplinas Eletivas Restritas}
	\label{tabeletivas}
	\begin{spreadtab}{{tabularx}{\textwidth}{|X|c|c|}}
		\hline
		@ {\textbf{Disciplina}} & @ {\textbf{CH}} & @ {\textbf{Créditos}} \\
		\hline
		@ \EletArq	& \EletArqCH	& \EletArqCred	\\
		@ \EletGeo	& \EletGeoCH	& \EletGeoCred	\\
		@ \EletPadroes	& \EletPadroesCH	& \EletPadroesCred	\\
		@ \EletRec	& \EletRecCH	& \EletRecCred	\\
		@ \EletRedes	& \EletRedesCH& \EletRedesCred	\\
		@ \EletMov	& \EletMovCH	& \EletMovCred	\\
		\hline
	\end{spreadtab}
\end{table}

\subsection{Normas Gerais de Ensino de Graduação da UERJ}

O curso de Engenharia de Computação obedecerá ao regime de créditos e as aulas serão oferecidas nos turnos manhã e tarde, com aulas predominantemente pela manhã, para os aprovados classificados no primeiro semestre; e tarde e noite, com aulas predominantemente pela tarde, para os aprovados classificados no segundo semestre. O turno da manhã transcorre no horário das 07:00h às 12:20h; o da tarde das 12:30h às 17:50h e o da noite das 18:00h às 22:40h. As aulas têm duração de 50 minutos nos turnos da manhã e tarde e de 45 minutos no turno da noite.

As Normas Gerais de Ensino de Graduação da UERJ são definidas pela deliberação n\textordmasculine~33/95 da UERJ (anexo \ref{delib3395}), sendo seus aspectos principais apresentados a seguir:

\subsection{Relação entre crédito e carga horária}
\textit{
	\textbf{Art. 57} -- O número mínimo de créditos necessários para integralizar o currículo será estabelecido com base na carga horária total do curso.}

\textit{
	\textbf{Parágrafo Único} - A unidade de crédito corresponde a:
	\begin{enumerate}[a)]
		\item 15 (quinze) horas de aula teórica, ou
		\item 30 (trinta) horas de aula prática, laboratório ou estágio curricular.
	\end{enumerate}}

\subsection{Aproveitamento escolar}
\begin{itquotation}
	\setcounter{artigo}{94}
	\artigo A aprovação do aluno em disciplinas do Curso de Graduação desta Universidade terá por base notas e frequência. São condições para aprovação: obtenção de nota final mínima 5,0 (cinco vírgula zero), constituída pela média aritmética da média semestral e nota da prova final, frequência mínima de 75\% (setenta e cinco por cento) do total de horas/aula determinado para a disciplina.

	\begin{paragrafos}
		\paragrafo Para cada disciplina haverá, pelo menos, duas avaliações por turma, por período letivo, sendo uma delas necessariamente individual e escrita. A média dos resultados dessas avaliações constitui a média semestral do aluno na disciplina.\\
		\paragrafo O aluno que obtiver média semestral igual ou superior a 4,0 (quatro vírgula zero) terá direito à prova final.\\
		\paragrafo O aluno que obtiver média semestral igual ou superior a 7,0 (sete vírgula zero) estará dispensado de prestar prova final.\\
		\ldots

		\setcounter{paragrafo}{6}
		\paragrafo  O aluno que obtiver nota final menor que 5,0 (cinco vírgula zero) ou média semestral inferior a 4,0 (quatro vírgula zero) será reprovado.\\
		\paragrafo O aluno que não obtiver frequência mínima de 75\% (setenta e cinco por cento) do total de horas/aula determinadas pela disciplina será reprovado, sem direito à prova final e independente de alcançar nota final superior a 7,0 (sete vírgula zero).\\
	\end{paragrafos}

\end{itquotation}
\subsection{Período de integralização do curso}
\setcounter{artigo}{98}
\begin{itquotation}
	\artigo Somente receberá o diploma o aluno que cumprir a Integralização Curricular.
\end{itquotation}

O período mínimo de integralização curricular dos cursos de engenharia é de 10 (dez) semestres, exceto para os casos de isenção de disciplinas, em que é possível um tempo mínimo menor. Já o prazo máximo para essa integralização é de 18 (dezoito) semestres.


\subsection{Estágio Supervisionado}

A atividade de Estágio Supervisionado é um elemento curricular obrigatório. De forma a possibilitar que docentes de diferentes departamentos possam participar das atividades de orientação, esta atividade é oferecida como uma disciplina eletiva restrita, na qual o discente deve selecionar uma das alternativas oferecidas de acordo com sua ênfase.
Disciplinas obrigatórias do curso de Engenharia Elétrica da UERJ relativas à atividade de Estágio Supervisionado
Todas as disciplinas possuem a mesma carga horária e número de créditos, respectivamente, 165 horas e 11 créditos, atendendo assim às Diretrizes Curriculares Nacionais dos cursos de Engenharia. O conjunto das ementas destas disciplinas estão contidos no anexo do ementário. A cada período o Departamento de Engenharia Elétrica irá abrir uma turma de forma a atender a demanda discente.
Como a disciplina de Estágio Supervisionado é encarada como sendo o primeiro trabalho profissional do aluno no papel de futuro Engenheiro Eletricista, espera-se que o trabalho a ser desenvolvido seja realizado em um ambiente corporativo. Além da supervisão de um profissional da empresa, o aluno contará com o apoio do professor da disciplina que poderá dar sugestões e contribuir com o desenvolvimento do trabalho. O aluno só poderá se inscrever nesta disciplina tendo cursado um número mínimo de 180 créditos, valor compatível com o desejado amadurecimento do estudante para a realização desta atividade (trava de créditos).
Independentemente da sua natureza, esse trabalho deve unir conhecimentos, competências e habilidades que foram adquiridos durante o curso. O Estágio deverá dar aos estudantes a oportunidade de refletir, analisar e propor soluções para problemas reais em desenvolvimento, através da articulação da teoria e da prática. A realização desse trabalho deverá abordar qualquer área do conhecimento da Engenharia Elétrica e contemplar a identificação e abordagem de um problema com foco
científico/tecnológico, a análise da viabilidade de possíveis soluções, a proposição de desenvolvimento de um projeto específico de engenharia, além de considerar a questão econômica e os impactos ambientais e sociais que possam estar envolvidos. Também se espera uma atualização em relação aos avanços da ciência e da tecnologia, bem como aos desafios da inovação. Ao término da disciplina, o aluno deverá redigir um Relatório que descreva as atividades realizadas, explicitando claramente os conteúdos do curso de Engenharia Elétrica que foram aplicados durante o período de Estágio. Este Relatório será avaliado concomitantemente a uma apresentação oral perante ao Professor da Disciplina. No texto do Relatório Final, serão avaliadas a redação, o uso correto da língua portuguesa, a qualidade do trabalho e as contribuições conferidas à formação do estudante. Na apresentação oral será computada a exposição do trabalho e a arguição realizada pelo professor da Disciplina.

\subsection{Projeto de Graduação}

A conclusão do curso se dá com as disciplinas de Projeto de Graduação A-II e Projeto de Graduação B-II para Engenharia Elétrica com ênfase em Sistemas de Potência, e com as disciplinas de Projeto de Graduação A-I e Projeto de Graduação B-I para Engenharia Elétrica com ênfase em Sistemas Elétricos e de Automação Industrial.
Estas disciplinas correspondem à elaboração do Projeto Final de Curso em suas respectivas ênfases, e envolvem o desenvolvimento de um projeto aplicado de engenharia elétrica, sob a orientação de docentes. Para garantir que o aluno só inicie o Projeto Final de Curso tendo já realizado um número de disciplinas compatível com o processo de finalização do seu curso, há um limite mínimo de 180 créditos (trava de créditos) a serem cursados para inscrição na primeira das duas disciplinas (Projeto de Graduação A).
Nas disciplinas, os alunos se organizarão individualmente ou em duplas de trabalho e, sob efetiva orientação docente, conceberão e/ou projetarão soluções criativas e viáveis no contexto do desenvolvimento do tema selecionado.
Trata-se de um trabalho que irá congregar todo o arcabouço técnico abordado ao longo do curso, com ênfase na integração de conceitos e conhecimentos adquiridos. Desta forma, estas duas disciplinas expõem cada aluno a uma tarefa de grande porte a ser realizada em dois períodos, ao longo dos quais será necessário aplicar todos os conhecimentos adquiridos ao longo do curso de forma integrada.
As etapas do desenvolvimento desta atividade ao longo das duas disciplinas contemplam:

\begin{itemize}
\item Projeto de Graduação A: o levantamento das informações, elaboração do tema, pesquisa bibliográfica e desenvolvimento da parte teórica de embasamento do projeto final de curso conforme as normas ABNT.
\item Projeto de Graduação B: Ensaios em bancada, prototipagem e/ou simulação computacional, finalização do documento e defesa com banca composta, no mínimo, por dois membros convidados pelo professor da disciplina, mais o orientador.
Neste contexto, os documentos gerados irão seguir a formatação sugerida pela Biblioteca.
\end{itemize}

Na defesa serão avaliados o texto, a qualidade do trabalho, a coerência dos resultados e o domínio do assunto durante a arguição.

\section{Equivalência com o Curso Anterior - \textcolor{red}{Thiago}}

O curso de Engenharia de Computação ora proposto substituirá o curso de Engenharia Elétrica com ênfase em Sistemas e Computação e, na hipótese de algum aluno desejar migrar do curso antigo para este novo, será possível dispensar disciplinas do novo currículo iguais ou equivalentes às disciplinas do curso antigo.

A tabela \ref{DiscIguais} mostra as disciplinas que são equivalentes entre o novo curso de Engenharia de Computação e o antigo curso de Engenharia Elétrica com ênfase em Sistemas e Computação. Por outro lado, a tabela \ref{DiscSemEqui} lista as disciplinas do novo currículo que não possuem equivalência direta com as disciplinas do curso anterior.

\rowcolors{1}{gray!5}{white}
\begin{table}[ht]
	\caption{Disciplinas Equivalentes}
	\label{DiscIguais}
	\centering
	\renewcommand{\arraystretch}{1.5}
	\begin{tabularx}{\textwidth}{|X|l|}
		\showrowcolors
		\hline
		{\textbf{Disciplina}} & \textbf{Código} \\
		\hline
		\Adm                  & \AdmCod         \\ % Administração
		\AlgLin               & \AlgLinCod      \\ % Álgebra Linear
		\CEV                  & \CEVCod         \\ % Circuitos em Corrente Contínua
		\EletI                & \EletICod       \\ % Eletrônica I
		\FisI                 & \FisICod        \\ % Física I
		\FisII                & \FisIICod       \\ % Física II
		\FisIII               & \FisIIICod      \\ % Física III
		\FisIV                & \FisIVCod       \\ % Física IV
		\IntEco               & \IntEcoCod      \\ % Macroeconomia
		\IntAmb               & \IntAmbCod      \\ % Introdução à Engenharia Ambiental
		\MatEle               & \MatEleCod      \\ % Materiais Elétricos e Magnéticos
		\ModMat               & \ModMatCod      \\ % Sinais e Sistemas
		\ProbEst              & \ProbEstCod     \\ % Probabilidade e Estatística
		\ProjA                & \ProjACod       \\ % Metodologia Científica
		\ProjB                & \ProjBCod       \\ % Projeto de Graduação XI
		\hline
	\end{tabularx}
\end{table}

\rowcolors{1}{gray!5}{white}
\begin{table}
	\centering
	\renewcommand{\arraystretch}{1.5}
	\caption{Equivalências no novo currículo}
	\label{equivalencias}
	\begin{tabularx}{\textwidth}{|X||X|l|}
		\hline
		{\textbf{Currículo Novo}}	& \textbf{Equivalente no Currículo Antigo} 	& \textbf{Código}\\
		\hline
		\AlgComp	& Algoritmos Computacionais				    	& FEN06-03559       \\
		\AnAlg      & Análise de Algoritmos                       & FEN06-03713       \\
		\ArqComp    & Arquitetura de Computadores I               & FEN06-04119       \\
		\ProjSO     & Arquitetura de Sistemas Operacionais        & FEN06-04664       \\
		\CalcI      & Cálculo Diferencial e Integral I            & IME01-00508       \\
		\CalcII     & Cálculo Diferencial e Integral II           & IME01-00854       \\
		\CalcIII    & Cálculo Diferencial e Integral III          & IME01-03646       \\
		\EngComput  & Cálculo Numérico IV                         & IME04-04541       \\
		\LabProgB   & Carac. das Linguagens de Prog. I            & FEN06-03980       \\
		\CEVI       & Circuitos Elétricos IV                      & FEN04-05222       \\
		\Control    & Controle de Processos por Comp.             & FEN06-05080       \\
		\EngSistA   & Engenharia de Sistemas A					& FEN06-04243       \\
		\ProjBD     & Engenharia de Sistemas B                    & FEN06-04314       \\
		\EstrInf    & Estruturas de Informação I                  & FEN06-03648       \\
		\FundComp   & Fundamentos de Comp. Digitais I    & FEN06-03787  				\\
		\LabProgA   & Laboratório de Programação I        & FEN06-04049                 \\
		\Telep      & Teleproc. e Redes de Computadores   & FEN06-04718                 \\
		\TeoComp                               & Teoria de Compiladores            & FEN06-04516                 \\
		Eletivas Restritas & Tóp. Especiais em Eng. de Sistemas e Computação A, B ou C & \parbox[t]{2cm}{FEN06-04889                                          \\FEN06-04939\\FEN06-04990}  \\
		\hline
	\end{tabularx}
\end{table}

\begin{table}
	\centering
	\renewcommand{\arraystretch}{1.5}
	\caption{Disciplinas sem Equivalências}
	\label{DiscSemEqui}
	\begin{tabularx}{\textwidth}{|X|}
		\hline
		{\textbf{Disciplinas do Novo Currículo sem Equivalência}} \\
		\hline
		\LogProg                                                  \\
		\IC                                                       \\
		\EngCompSoc                                               \\
		\MineraDados                                              \\
		\SistEmb                                                  \\
		\ProcImag                                                 \\
		\CompParal                                                \\
		\EstSup                                                   \\
		\hline
	\end{tabularx}
\end{table}

\section{Ementário das Disciplinas - \textcolor{red}{Thiago}}

As ementas das disciplinas obrigatórias e eletivas são apresentadas no anexo \ref{ementas}. As ementas das disciplinas já existentes foram obtidas no site do próprio DEP, Departamento de Orientação e Supervisão Pedagógica. Essas disciplinas são apresentadas no formulário antigo e não foram feitas correções ou alterações no texto original.

\section{Atividades Acadêmicas da Graduação articuladas ao ensino de Pós-Graduação (Aperfeiçoamento, Mestrado, Doutorado) \textcolor{red}{Luiza} }

\section{Atividades de Extensão \textcolor{red}{Giomar}}

O currículo do curso de Engenharia de Sistemas e Computação da UERJ tem como uma de suas vertentes combater a evasão, propiciando maior motivação e engajamento dos alunos oferecendo-lhes disciplinas profissionalizantes desde o primeiro período do curso. 

Nesta mesma direção, o curso busca proporcionar aos alunos uma visão realística e analítica sobre o papel da tecnologia na sociedade através da oferta de novas disciplinas e de atividades de Extensão Universitária, suscitando o desenvolvimento de uma perspectiva disruptiva e contextualizada através da qual os alunos se sintam encorajados a se envolver com iniciativas voltadas para o desenvolvimento social. Desta forma, as atividades de Extensão Universitária são vistas como um processo de aprofundamento educativo e cultural, propiciando o exercício da interdisciplinaridade e estimulando o pensamento reflexivo, analítico e crítico nos estudantes. A partir do conjunto coordenado de projetos, programas, eventos e disciplinas, promovem interações transformadoras entre a universidade e os diversos setores da sociedade.

Para alcançar esses objetivos, as seguintes mudanças foram implementadas: cada aluno do curso deve completar pelo menos \hextensao horas em atividades de extensão complementares, conforme exigido pela Lei 10.172 que aprova o Plano Nacional de Educação. Isso corresponde a pelo menos 10\% do total da carga horária do curso.

A participação discente e o cumprimento das horas pode ser obtido através das seguintes formas de atividades de extensão:
\begin{enumerate}[I -]
    \item Participação em programas e projetos de extensão coordenados por professores ou técnicos da carreira de nível superior na Universidade do Estado do Rio de Janeiro, com ou sem o recebimento de bolsa;
    \item Promoção de cursos de extensão, incluindo a organização, preparação e apresentação de aulas, videoaulas e reuniões com a comunidade;
    \item Participação em disciplinas e atividades relacionadas à extensão fornecidas pela universidade, visando a compreensão, aprimoramento e melhor desempenho do discente na realização de tarefas de extensão;
    \item Desenvolvimento conjunto de soluções tecnológicas visando atender as demandas dos atores da sociedade civil;
    \item Participação em eventos, tanto na organização quanto na realização.
\end{enumerate}

A Universidade do Estado do Rio de Janeiro é uma instituição compromissada com a formação da cidadania e a inclusão social. Neste contexto, o Curso de Engenharia de Sistemas e Computação da UERJ, pretende colaborar com a inclusão digital dos cidadãos fluminenses.

A extensão universitária permite o estreitamento dos laços entre a academia e a sociedade, inseridas em diversas realidades socioeconômicas no âmbito de comunidades rurais, periurbanas e urbanas. Ao aproximar os conhecimentos obtidos em sala de aula à realidade, contribui para a sustentabilidade de setores socioeconômicos que desempenham um papel essencial para o futuro das gerações, mas, que enfrentam no dia a dia constantes desafios, notadamente a exclusão digital. Além disso, oferece aos alunos a chance de se envolverem em atividades nas quais poderão exercitar a cidadania ao prestar um serviço relevante a segmentos da sociedade frequentemente esquecidos e carentes de assistência.

