% !TEX root = ProjetoPedagogico.tex
\chapter{Organização Didático-Pedagógica}
\thispagestyle{plain}

\section{Justificativa das Necessidades Sociais do Curso}

As demandas sociais e tecnológicas da sociedade brasileira têm se intensificado, especialmente em um cenário global caracterizado pela transformação digital, pela constante inovação tecnológica e pela crescente aplicação de inteligência artificial e aprendizado de máquina na resolução de desafios complexos. Nesse contexto, a formação de profissionais com competências técnicas e humanas no campo da computação torna-se imprescindível para enfrentar os desafios impostos pela modernização e pela globalização.

No Brasil, observa-se uma necessidade crescente de profissionais qualificados para atuar com sistemas informatizados em distintos setores da sociedade. A computação desempenha um papel estratégico em áreas como saúde, educação, administração pública, indústria, segurança, inovação e empreendedorismo. Assim, torna-se fundamental a formação de engenheiros capazes de conceber, desenvolver, implementar e gerenciar soluções computacionais inovadoras, que contribuam de maneira efetiva para o desenvolvimento econômico e social do país.

Diante desse cenário, o Departamento de Engenharia de Sistemas e Computação (DESC) propõe a transformação do curso de Engenharia Elétrica com Ênfase em Sistemas e Computação no curso de Engenharia de Computação. Essa reformulação busca atender à crescente demanda por profissionais altamente qualificados, capazes de responder aos desafios tecnológicos contemporâneos e de contribuir de forma significativa para as necessidades do mercado e da sociedade.

\section{Concepção}

A estrutura curricular do curso de Engenharia de Computação do \desc, da Faculdade de Engenharia da UERJ, orienta-se pelas \textit{Diretrizes Curriculares Nacionais para os cursos de graduação na área da Computação}, estabelecidas pelo Ministério da Educação (Anexo~\ref{cne2016}); pelos \textit{Referenciais de Formação para os cursos de Graduação em Computação}, elaborados pela Sociedade Brasileira de Computação (Anexo~\ref{sbc2017}); e pela regulamentação do exercício profissional do Engenheiro, estabelecida pelo Sistema CONFEA/CREA (Anexo~\ref{confea1993}), atualmente em vigor.

A matriz curricular totaliza \tHorasCurso horas, sendo \hTotaisDiscObrigComDiscExt horas destinadas a disciplinas obrigatórias (das quais \hDiscExtensao horas em disciplinas extensionistas), \hEletivas horas em disciplinas eletivas e \hACE horas voltadas às Atividades Curriculares de Extensão (ACE). As \hTotaisDisc horas de disciplinas estão distribuídas em \nDisciplinas disciplinas curriculares, sendo \nDiscObrigatorias obrigatórias e duas  eletivas. A carga horária das disciplinas contempla atividades teóricas, práticas, laboratoriais, de estágio e de extensão, conforme a natureza dos conteúdos abordados. A tabela \ref{tab:cargahoraria} apresenta a distribuição da carga horária do curso.

Destaca-se a redução substancial da carga horária total do curso, que passou de 4.260 horas na versão anterior -- o curso de Engenharia Elétrica com Ênfase em Sistemas e Computação -- para \tHorasCurso horas no novo curso de Engenharia de Computação, representando uma diminuição de \inteval{4260 - \thethorasCursoCounter} horas. Essa adequação foi viabilizada pelas novas Diretrizes Curriculares Nacionais específicas para os cursos da área de Computação, que estabelecem um mínimo de 3.200 horas, em contraste com as 3.600 horas exigidas pelas diretrizes gerais das Engenharias. A redução permitiu a eliminação de conteúdos mais relevantes para outras engenharias, mas menos pertinentes à formação em Computação, mantendo a qualidade e a abrangência da formação profissional exigida para o engenheiro de computação.

\rowcolors{1}{gray!10}{white}
\renewcommand{\arraystretch}{1.5}
\begin{table}[h!]
    \centering
    \caption{Distribuição da carga horária do curso de Engenharia de Computação}
    \label{tab:cargahoraria}
    \begin{tabularx}{0.7\textwidth}{Xrr}
        \hline
        \textbf{Componente}                 & \textbf{Carga Horária (h)}  & \textbf{Extensão (h)} \\
        \hline
        Disciplinas obrigatórias            & \hTotaisDiscObrigComDiscExt & \hDiscExtensao        \\
        Disciplinas eletivas                & \hEletivas                  & 0                     \\
        Atividades Curriculares de extensão & \hACE                       & \hACE                 \\
        \hline
        \textbf{Total do curso}             & \textbf{\tHorasCurso}       & \textbf{\hExtensao}   \\
        \hline
    \end{tabularx}
\end{table}

O currículo inclui, ainda, Estágio Supervisionado e Projeto de Graduação (Trabalho de Conclusão de Curso), componentes que visam à integração e à aplicação dos conhecimentos científicos, tecnológicos e instrumentais adquiridos ao longo da formação. Como atividades acadêmicas complementares de caráter facultativo, os discentes poderão participar de Estágio Interno, Monitoria, Iniciação Científica, Cursos, Eventos, Palestras e Visitas Técnicas, com vistas a ampliar sua compreensão sobre a Engenharia de Computação, o setor produtivo nacional e as possíveis áreas de atuação do egresso.

\section{Objetivos Gerais}

O curso de Engenharia de Computação tem como objetivo formar profissionais altamente qualificados, aptos a atuar de forma crítica, ética e inovadora na concepção, desenvolvimento, implementação e gestão de sistemas computacionais. Visa contribuir com o avanço tecnológico e atender às necessidades da sociedade, da indústria e do mercado global, promovendo soluções sustentáveis e socialmente responsáveis.

\section{Objetivos Específicos}

\begin{enumerate}
    \item Proporcionar uma formação técnico-científica sólida, capacitando o discente a desenvolver soluções em hardware e software de forma criativa, eficiente e inovadora;

    \item Desenvolver competências práticas que permitam a atuação em projetos de sistemas computacionais e eletrônicos, observando critérios de desempenho, segurança e sustentabilidade;

    \item Estimular a capacidade de análise crítica e resolução de problemas complexos, preparando o egresso para enfrentar desafios nos diversos setores de aplicação da Engenharia de Computação, tais como automação, inteligência artificial, telecomunicações e computação em nuvem;

    \item Fomentar a formação ética e cidadã, promovendo o uso da tecnologia em benefício da sociedade e do meio ambiente, em consonância com a legislação vigente e os princípios de responsabilidade social;

    \item Incentivar a pesquisa e o desenvolvimento tecnológico, por meio da participação dos discentes em projetos acadêmicos e profissionais voltados à expansão do conhecimento na área da computação;

    \item Preparar os discentes para a inserção e atuação no mercado de trabalho nacional e internacional, com base em competências alinhadas às tendências tecnológicas globais.
\end{enumerate}

\section{Nível de Formação e Título Acadêmico}

O curso é de graduação plena e a titulação concedida é de Engenheiro de Computação.

\section{Perfil do Egresso (competência, habilidades e atitudes pretendidas)}

O curso de Engenharia de Computação tem como perfil do egresso o engenheiro, com formação técnico-científica sólida, generalista, humanista, crítica e reflexiva, capacitado a absorver e desenvolver novas tecnologias, estimulando a sua atuação crítica e criativa na identificação e resolução de problemas, considerando seus aspectos políticos, econômicos, sociais, ambientais e culturais, com visão ética e humanística, em atendimento às demandas da sociedade. Faz parte do perfil do egresso a postura de permanente busca da atualização profissional, além das seguintes habilidades:
\begin{enumerate} [I -]
    \item possuir conhecimento das questões humanísticas, sociais, ambientais, éticas, profissionais, legais e políticas;
    \item possuir compreensão do impacto da Engenharia de Computação e suas tecnologias no que concerne ao atendimento e à antecipação estratégica das necessidades da sociedade;
    \item possuir atitude crítica, interdisciplinar e criativa na identificação e resolução de problemas;
    \item possuir compreensão das necessidades de contínua atualização e aprimoramento de suas competências e habilidades;
    \item possuir uma sólida formação em Computação, Física, Matemática, Eletrônica, Automação e Telecomunicações.
    \item conhecer a estrutura dos sistemas de computação e os processos envolvidos na sua análise e construção;
    \item considerar os aspectos ambientais, econômicos, financeiros, de gestão e de qualidade, associados a novos produtos e organizações;
    \item considerar fundamental a inovação, a criatividade, a atitude empreendedora e a inserção internacional.
\end{enumerate}

O egresso da Engenharia de Computação, no processo de sua formação, deverá desenvolver as seguintes competências:
\begin{enumerate} [I -]
    \item antever as implicações humanísticas, sociais, ambientais, éticas, profissionais, legais (inclusive relacionadas à propriedade intelectual) e políticas dos sistemas computacionais;
    \item identificar demandas socioeconômicas e ambientais relevantes, planejar, especificar e projetar sistemas de computação, seguindo teorias, princípios, métodos e procedimentos interdisciplinares;
    \item construir, testar, verificar e validar sistemas de computação, seguindo métodos, técnicas e procedimentos interdisciplinares;
    \item perceber as necessidades de atualização decorrentes da evolução tecnológica e social;
    \item relacionar problemas do mundo real com suas soluções, considerando aspectos de computabilidade e de escalabilidade;
    \item analisar, desenvolver, avaliar e aperfeiçoar software e hardware em arquiteturas de computadores;
    \item analisar, desenvolver, avaliar e aperfeiçoar sistemas de automação e sistemas inteligentes;
    \item analisar, desenvolver, avaliar e aperfeiçoar sistemas de informação computacionais;
    \item analisar, desenvolver, avaliar e aperfeiçoar circuitos eletroeletrônicos;
    \item gerenciar pessoas e infraestrutura de Sistemas de Computação;
    \item perceber as necessidades de inovação e inserção internacional com atitudes criativas e empreendedoras.
\end{enumerate}

O curso de Engenharia de Computação tem, predominantemente, o ensino da Computação como atividade fim, visando à formação de recursos humanos para o desenvolvimento científico e tecnológico da computação. Assim sendo, o curso deve capacitar indivíduos para desenvolver software e hardware, com uma forte base matemática e física.

Os egressos do curso de Engenharia de Computação estarão situados no estado da arte da ciência e da tecnologia da computação, de tal forma que possam continuar suas atividades na pesquisa, promovendo o desenvolvimento científico, ou aplicando os conhecimentos científicos, propiciando o desenvolvimento tecnológico. Para tal, é dada uma forte ênfase no uso de laboratórios para capacitar os egressos no projeto e construção tanto de software quanto de hardware.

\section{Organização do Currículo}

O currículo do curso de Engenharia de Computação é constituído por disciplinas obrigatórias e eletivas, estágio supervisionado, trabalho de conclusão de curso e atividades de extensão. O curso é organizado em 10 semestres, podendo o aluno cumprí-lo em um máximo de 18 semestres.

Para uma eficaz orientação pedagógica, é proposto o aconselhamento curricular apresentado nas tabelas \ref{tab1p} a \ref{tab10p}. Os pré-requisitos das disciplinas podem ser observados no fluxograma do curso (Anexo \ref{fluxograma}).

O aluno deverá cursar no mínimo duas das disciplinas eletivas restritas oferecidas (ver Tabela \ref{tabeletivas}). Deve ser
ressaltado que estas disciplinas são oferecidas de acordo com o interesse dos corpos
docente e discente, não sendo necessariamente disponibilizadas todos os semestres.

\rowcolors{1}{gray!10}{white}
\setlength{\tabcolsep}{5pt}
\renewcommand{\arraystretch}{1.5}
\begin{table}[!ht]
    \centering
    \caption{1\textordmasculine~Período}
    \label{tab1p}
    \begin{spreadtab}{{tabularx}{0.8\textwidth}{ Xcc }}
        \hline
        \rowcolor{gray!20}
        @ {\textbf{Disciplina}} & @ {\textbf{CH}} & @ {\textbf{Créditos}} \\
        \hline
        @ \AlgComp              & \AlgCompCH      & \AlgCompCred          \\ % Algoritmos Computacionais
        @\AlgLin                & \AlgLinCH       & \AlgLinCred           \\ % Álgebra Linear
        @ \CalcI                & \CalcICH        & \CalcICred            \\ % Cálculo I
        @ \EngCompSoc           & \EngCompSocCH   & \EngCompSocCred       \\ % Engenharia e Computação Sociedade        
        @ \IntAmb               & \IntAmbCH       & \IntAmbCred           \\ % Introdução à Engenharia Ambiental
        \hline
        @ \textbf{Total}        & sum(b2:b6)      & sum(c2:c6)            \\
        \hline
    \end{spreadtab}
\end{table}


\rowcolors{1}{gray!10}{white}
\begin{table}[!ht]
    \centering
    \caption{2\textordmasculine~Período}
    \label{tab2p}
    \begin{spreadtab}{{tabularx}{0.8\textwidth}{Xcc}}
        \hline
        \rowcolor{gray!20}
        @ {\textbf{Disciplina}} & @ {\textbf{CH}} & @ {\textbf{Créditos}} \\
        \hline
        @ \CalcII               & \CalcIICH       & \CalcIICred           \\ % Cálculo II
        @ \CalcNum              & \CalcNumCH      & \CalcNumCred          \\ % Calculo Numérico
        @ \EstrInf              & \EstrInfCH      & \EstrInfCred          \\ % Estruturas de Informação
        @ \FisI                 & \FisICH         & \FisICred             \\ % Física I
        @ \FisEI                & \FisEICH        & \FisEICred            \\ % Física Experimental I
        @ \LogProg              & \LogProgCH      & \LogProgCred          \\ % Lógica de Programação        
        \hline
        @ \textbf{Total}        & sum(b2:b7)      & sum(c2:c7)            \\
        \hline
    \end{spreadtab}
\end{table}

\rowcolors{1}{gray!10}{white}
\begin{table}[!ht]
    \centering
    \caption{3\textordmasculine~Período}
    \label{tab3p}
    \begin{spreadtab}{{tabularx}{0.8\textwidth}{Xcc}}
        \hline \rowcolor{gray!20}
        @ {\textbf{Disciplina}} & @ {\textbf{CH}} & @ {\textbf{Créditos}} \\
        \hline
        @ \AnAlg                & \AnAlgCH        & \AnAlgCred            \\ % Análise de Algoritmos
        @ \CalcIII              & \CalcIIICH      & \CalcIIICred          \\ % Cálculo III
        @ \CircEletI            & \CircEletICH    & \CircEletICred        \\ % Circuitos Elétricos I
        @ \FisII                & \FisIICH        & \FisIICred            \\ % Física II
        @ \FisEII               & \FisEICH        & \FisEICred            \\ % Física Experimental II
        @ \ProbEst              & \ProbEstCH      & \ProbEstCred          \\ % Probabilidade e Estatística
        @ \Ext                  & \ExtCH          & \ExtCred              \\ % Projeto de Extensão
        \hline
        @ \textbf{Total}        & sum(b2:b8)      & sum(c2:c8)            \\
        \hline
    \end{spreadtab}
\end{table}

\rowcolors{1}{gray!10}{white}
\begin{table}[!ht]
    \centering
    \caption{4\textordmasculine~Período}
    \label{tab4p}
    \begin{spreadtab}{{tabularx}{0.8\textwidth}{Xcc}}
        \hline \rowcolor{gray!20}
        @ {\textbf{Disciplina}} & @ {\textbf{CH}} & @ {\textbf{Créditos}} \\
        \hline
        @ \FisIII               & \FisIIICH       & \FisIIICred           \\ % Física III
        @ \FisEIII              & \FisEIIICH      & \FisEIIICred          \\ % Física Experimental III
        @ \LabProgA             & \LabProgACH     & \LabProgACred         \\ % Laboratório de Programação A
        @ \LabProgPOO           & \LabProgPOOCH   & \LabProgPOOCred       \\ % Laboratório de Programação OO        
        @ \ProcImag             & \ProcImagCH     & \ProcImagCred         \\ % Processamento de Sinais e Imagens
        @ \TecDig               & \TecDigCH       & \TecDigCred           \\ % Técnicas Digitais I
        \hline
        @ \textbf{Total}        & sum(b2:b7)      & sum(c2:c7)            \\
        \hline
    \end{spreadtab}
\end{table}

\rowcolors{1}{gray!10}{white}
\begin{table}[!ht]
    \centering
    \caption{5\textordmasculine~Período}
    \label{tab5p}
    \begin{spreadtab}{{tabularx}{0.8\textwidth}{Xcc}}
        \hline \rowcolor{gray!20}
        @ {\textbf{Disciplina}} & @ {\textbf{CH}} & @ {\textbf{Créditos}} \\
        \hline
        @\CCC                   & \CCCCH          & \CCCCred              \\ % Circuitos em Corrente       
        @ \FisIV                & \FisIVCH        & \FisIVCred            \\ % Física IV
        @ \FisEIV               & \FisEIVCH       & \FisEIVCred           \\ % Física Experimental IV
        @ \FundComp             & \FundCompCH     & \FundCompCred         \\ % Fundamentos de Computadores I  
        @ \MatEle               & \MatEleCH       & \MatEleCred           \\ % Materiais Elétricos e Eletrônicos 
        @ \SinaisESist          & \SinaisESistCH  & \SinaisESistCred      \\ % Sinais e Sistemas
        @ \Grafos               & \GrafosCH       & \GrafosCred           \\ % Teoria dos Grafos e Aplicações
        \hline
        @ \textbf{Total}        & sum(b2:b8)      & sum(c2:c8)            \\
        \hline
    \end{spreadtab}
\end{table}

\rowcolors{1}{gray!10}{white}
\begin{table}[!ht]
    \centering
    \caption{6\textordmasculine~Período}
    \label{tab6p}
    \begin{spreadtab}{{tabularx}{0.8\textwidth}{Xcc}}
        \hline \rowcolor{gray!20}
        @ {\textbf{Disciplina}} & @ {\textbf{CH}} & @ {\textbf{Créditos}} \\
        \hline
        @ \ArqComp              & \ArqCompCH      & \ArqCompCred          \\ % Arquitetura de Computadores A
        @ \CCA                  & \CCACH          & \CCACred              \\ % Circuitos em Corrente Alternada
        @ \EngSistA             & \EngSistACH     & \EngSistACred         \\ % Engenharia de Sistemas
        @ \IC                   & \ICCH           & \ICCred               \\ % Inteligência Computacional I
        @ \ICII                 & \ICIICH         & \ICIICred             \\ % Inteligência Computacional II        
        @ \MatEle               & \MatEleCH       & \MatEleCred           \\ % Materiais Elétricos e Eletrônicos
        \hline
        @ \textbf{Total}        & sum(b2:b7)      & sum(c2:c7)            \\
        \hline
    \end{spreadtab}
\end{table}

\rowcolors{1}{gray!10}{white}
\begin{table}[!ht]
    \centering
    \caption{7\textordmasculine~Período}
    \label{tab7p}
    \begin{spreadtab}{{tabularx}{0.8\textwidth}{Xcc}}
        \hline
        @ {\textbf{Disciplina}} & @ {\textbf{CH}} & @ {\textbf{Créditos}} \\
        \hline \rowcolor{gray!20}
        @ \MacroEco             & \MacroEcoCH     & \MacroEcoCred         \\ % Macroeconomia 
        @ \MineraDados          & \MineraDadosCH  & \MineraDadosCred      \\ % Mineração de Dados
        @ \ProjBD               & \ProjBDCH       & \ProjBDCred           \\ % Projeto de Banco de Dados
        @ \ProjSO               & \ProjSOCH       & \ProjSOCred           \\ % Projeto de Sistemas Operacionais
        @ \Telep                & \TelepCH        & \TelepCred            \\ % Redes de Computadores
        @ \TeoComp              & \TeoCompCH      & \TeoCompCred          \\ % Teoria da Compiladores
        \hline
        @ \textbf{Total}        & sum(b2:b7)      & sum(c2:c7)            \\
        \hline
    \end{spreadtab}
\end{table}

\rowcolors{1}{gray!10}{white}
\begin{table}[!ht]
    \centering
    \caption{8\textordmasculine~Período}
    \label{tab8p}
    \begin{spreadtab}{{tabularx}{0.8\textwidth}{Xcc}}
        \hline \rowcolor{gray!20}
        @ {\textbf{Disciplina}} & @ {\textbf{CH}} & @ {\textbf{Créditos}} \\
        \hline
        @ \AnaProjSist          & \AnaProjSistCH  & \AnaProjSistCred      \\ % Análise e Projeto de Sistemas
        @ \CompParal            & \CompParalCH    & \CompParalCred        \\ % Computação Paralela
        @ \Control              & \ControlCH      & \ControlCred          \\ % Controle de Processos
        @ \Empre                & \EmpreCH        & \EmpreCred            \\ % Empreendedorismo    
        @ \Sredes               & \SredesCH       & \SredesCred           \\ % Segurança em Redes
        @ \SistEmb              & \SistEmbCH      & \SistEmbCred          \\ % Sistemas Embutidos
        \hline
        @ \textbf{Total}        & sum(b2:b7)      & sum(c2:c7)            \\
        \hline
    \end{spreadtab}
\end{table}

\rowcolors{1}{gray!10}{white}
\begin{table}[!ht]
    \centering
    \caption{9\textordmasculine~Período}
    \label{tab9p}
    \begin{spreadtab}{{tabularx}{0.8\textwidth}{Xcc}}
        \hline \rowcolor{gray!20}
        @ {\textbf{Disciplina}} & @ {\textbf{CH}} & @ {\textbf{Créditos}} \\
        \hline
        @ \EstSup               & \EstSupCH       & \EstSupCred           \\ % Estágio Supervisionado
        @ \EletA                & \EletACH        & \EletACred            \\ % Disciplina Eletiva A 
        @ \Instala              & \InstalaCH      & \InstalaCred          \\ % Instalações de Ambientes Computacionais       
        @ \ProjA                & \ProjACH        & \ProjACred            \\ % Metodologia Científica
        \hline
        @ \textbf{Total}        & sum(b2:b5)      & sum(c2:c5)            \\
        \hline
    \end{spreadtab}
\end{table}

\rowcolors{1}{gray!10}{white}
\begin{table}[!ht]
    \centering
    \caption{10\textordmasculine~Período}
    \label{tab10p}
    \begin{spreadtab}{{tabularx}{0.8\textwidth}{Xcc}}
        \hline \rowcolor{gray!20}
        @ {\textbf{Disciplina}} & @ {\textbf{CH}} & @ {\textbf{Créditos}} \\
        \hline
        @ \Adm                  & \AdmCH          & \AdmCred              \\ % Administração
        @ \EletB                & \EletBCH        & \EletBCred            \\ % Disciplina Eletiva B
        @ \ProjB                & \ProjBCH        & \ProjBCred            \\ % Projeto de Graduação XI
        \hline
        @ \textbf{Total }       & sum(b2:b4)      & sum(c2:c4)            \\
        \hline
    \end{spreadtab}
\end{table}

\rowcolors{1}{gray!10}{white}
\begin{table}[!ht]
    \centering
    \caption{Disciplinas Eletivas Restritas}
    \label{tabeletivas}
    \begin{spreadtab}{{tabularx}{0.85\textwidth}{Xcc}}
        \hline \rowcolor{gray!20}
        @ {\textbf{Disciplina}} & @ {\textbf{CH}}  & @ {\textbf{Créditos}} \\
        \hline
        @ \EletArq              & \EletArqCH       & \EletArqCred          \\ % Arquiteturas Avançadas de Computadores 
        @ \EletReforco          & \EletReforcoCH   & \EletReforcoCred      \\ % Aprendizado por Reforço
        @ \EletVisao            & \EletVisaoCH     & \EletVisaoCred        \\ % Aprendizado Profundo para Visão Computacional
        @ \AprendProfPLN        & \AprendProfPLNCH & \AprendProfPLNCred    \\ % Aprendizado Prof. p/ Proc.de Ling. Natural
        @ \AutomProcRob         & \AutomProcRobCH  & \AutomProcRobCred     \\ % Automação de Processos Robóticos
        @ \EletGeo              & \EletGeoCH       & \EletGeoCred          \\ % Geomática
        @ \EletRedes            & \EletRedesCH     & \EletRedesCred        \\ % Redes de Interconexão			
        @ \SistOpRobInt         & \SistOpRobIntCH  & \SistOpRobIntCred     \\ % Sistemas Operacionais p/ Robótica Inteligente
        @ \TecProgOtim          & \TecProgOtimCH   & \TecProgOtimCred      \\ % Técnicas de Programação em Otimização Combinatória
        @ \TopEspVisComp        & \TopEspVisCompCH & \TopEspVisCompCred    \\ % Tópicos Especiais em Visão Computacional
        \hline
    \end{spreadtab}
\end{table}

\FloatBarrier % Evitando que exista texto entre as 10 tabelas.



\section{Estágio Supervisionado}

A disciplina de \textbf{\EstSup} representa a primeira vivência profissional do aluno como futuro Engenheiro de Computação. Por isso, espera-se que as atividades sejam desenvolvidas em um ambiente corporativo ou outro contexto de atuação profissional. O estágio curricular, supervisionado, contribui para a prática extensionista no setor produtivo da sociedade e permite ao estudante enfrentar situações reais do exercício da profissão. Essa caracterização do estágio como prática extensionista e profissional está detalhada na Seção~\ref{sec:estagio-supervisionado-extensionista}.

Durante o estágio, o aluno será acompanhado por um supervisor da empresa e contará com o apoio de um professor da disciplina, que poderá orientar o desenvolvimento das atividades e sugerir aprimoramentos.

A matrícula nesta disciplina está condicionada à conclusão de um número mínimo de créditos, conforme especificado na ementa, garantindo que o aluno possua a maturidade necessária para desempenhar bem essa função. Independentemente do local ou do tipo de trabalho, é fundamental que o estágio envolva a aplicação integrada de conhecimentos, habilidades e competências adquiridas ao longo do curso.

O estágio deve permitir que o aluno reflita sobre problemas reais, proponha soluções e aprofunde a articulação entre teoria e prática. As atividades podem ser desenvolvidas em qualquer área da Engenharia de Computação, desde que envolvam a identificação e análise de um problema com base científica e tecnológica, a avaliação de soluções possíveis e a proposta de um projeto de engenharia. Também é necessário considerar os aspectos econômicos envolvidos e os impactos sociais e ambientais. Espera-se que o aluno demonstre domínio de avanços científicos e tecnológicos e sensibilidade aos desafios da inovação.

Ao final da disciplina, o aluno deverá elaborar um Relatório Final, no qual descreva detalhadamente as atividades realizadas e a aplicação dos conhecimentos adquiridos no curso. Esse relatório será avaliado em conjunto com uma apresentação oral feita ao professor da disciplina. A avaliação do relatório considerará a clareza da redação, o uso correto da língua portuguesa, a relevância do trabalho e sua contribuição para a formação do aluno. Na apresentação, serão avaliadas a organização das ideias e a capacidade de responder às perguntas do professor.

\subsection{Exercício Profissional}
Considerando a importância de reconhecer a experiência profissional como parte integrante da formação acadêmica, o curso de Engenharia de Computação da UERJ permitirá que alunos com vínculo empregatício formal possam solicitar o reconhecimento de seu exercício profissional como equivalente ao estágio supervisionado obrigatório, desde que a atividade se relacione diretamente com as áreas de formação ou de exercício profissional do Engenheiro de Computação.

Para o reconhecimento, será seguido procedimento análogo ao da realização regular do estágio supervisionado, incluindo:

\begin{itemize}
    \item Cumprimento dos pré-requisitos curriculares exigidos para matrícula no estágio supervisionado;
    \item Definição de um período de supervisão da atividade profissional, com carga horária igual ou superior a \EstSupCH~ horas;
    \item Apresentação e aprovação de um relatório técnico ao final do período supervisionado, nos mesmos moldes exigidos para o estágio supervisionado tradicional. \end{itemize}

O aluno deverá apresentar, previamente ao início da contagem de horas, documentação comprobatória do vínculo empregatício, como Carteira de Trabalho e Previdência Social (CTPS) assinada, contrato de trabalho ou outro documento oficial equivalente.



\subsection{Projeto de Graduação}

A conclusão do curso se dá por meio das disciplinas \textbf{\ProjA} e \textbf{\ProjB}, que compõem o Projeto Final de Curso. Nessas disciplinas, o aluno desenvolve um projeto aplicado em Engenharia de Computação, com orientação de professores do curso.

Para garantir que essa etapa ocorra com a devida maturidade acadêmica, é exigida a conclusão de um número mínimo de créditos, conforme especificado como pré-requisito para matrícula na disciplina \textbf{\ProjA}.

Durante o desenvolvimento do projeto, os alunos podem trabalhar individualmente ou em duplas. Com o apoio de seus orientadores, deverão conceber e implementar soluções criativas e viáveis para o tema escolhido, integrando os conhecimentos e competências adquiridos ao longo da graduação.

Essas disciplinas colocam o aluno diante do desafio de conduzir um projeto de maior complexidade ao longo de dois períodos letivos, promovendo a síntese entre teoria e prática. As etapas do Projeto Final de Curso incluem:

\begin{itemize}
    \item \textbf{\ProjA}: Levantamento de informações, definição do tema, pesquisa bibliográfica e elaboração da fundamentação teórica do projeto, seguindo as normas técnicas vigentes.
    \item \textbf{\ProjB}: Desenvolvimento prático do projeto por meio de ensaios em bancada, prototipagem e/ou simulação computacional, finalização do relatório e apresentação para uma banca examinadora composta por, no mínimo, dois membros convidados, além do orientador.
\end{itemize}

Os documentos produzidos devem seguir o formato estabelecido pela biblioteca da UERJ.

Na defesa, serão avaliadas a qualidade técnica e textual do relatório final, a relevância e a consistência dos resultados obtidos, bem como o domínio do tema demonstrado pelo aluno durante a apresentação e arguição.

\section{Equivalência com o Curso Anterior}

O curso de Engenharia de Computação resulta da reformulação do curso de Engenharia Elétrica com Ênfase em Sistemas e Computação. Dessa forma, os estudantes que ingressaram na versão anterior poderão migrar para o novo currículo, desde que observadas as equivalências entre as disciplinas.

Essa transição requer atenção às correspondências curriculares, que estão organizadas na Tabela~\ref{tab:desc-equi-long}. A tabela apresenta o mapeamento das disciplinas do curso anterior para as equivalentes na nova estrutura curricular, respeitando a ordem dos períodos.

\renewcommand{\arraystretch}{1.5}
\rowcolors{2}{gray!10}{white}
\begin{small}
    \begin{longtable}{p{5.1cm}l|p{5.1cm}l}
        \caption{Equivalências no novo currículo.}
        \label{tab:desc-equi-long}                                                                                                                    \\
        \hline
        \rowcolor{gray!20}
        \textbf{Currículo Novo} & \textbf{Código} & \textbf{Currículo Antigo}                                          & \textbf{Código}              \\
        \hline
        \endfirsthead

        \hline
        \rowcolor{gray!20}
        \textbf{Currículo Novo} & \textbf{Código} & \textbf{Currículo Antigo}                                          & \textbf{Código}              \\
        \hline
        \endhead

        \multicolumn{4}{r}{\small\itshape Continuação na próxima página}
        \endfoot

        \bottomrule
        \endlastfoot
        \AlgLin                 & \AlgLinCod      & Álgebra Linear III                                                 & IME 02-01388                 \\
        \AlgComp                & \AlgCompCod     & Algoritmos Computacionais                                          & FEN 06-03559                 \\
        \CalcI                  & \CalcICod       & Cálculo Diferencial e Integral I                                   & IME 01-00508                 \\
        \EngCompSoc             & \EngCompSocCod  & Engenharia na Sociedade I                                          & FEN 02-01064                 \\
        \IntAmb                 & \IntAmbCod      & Introdução à Engenharia Ambiental                                  & FEN 02-01064                 \\
        \hline
        \CalcII                 & \CalcIICod      & Cálculo Diferencial e Integral II                                  & IME 01-00854                 \\
        \CalcNum                & \CalcNumCod     & Cálculo Numérico IV                                                & IME 04-04541                 \\
        \parbox[t]{4cm}{\FisI~e                                                                                                                       \\ \FisEI} & \parbox[t]{2cm}{\FisICod \\ \FisEICod} & Física Teórica e Experimental I & FIS 01-05095 \\
        \EstrInf                & \EstrInfCod     & Estruturas de Informação I                                         & FEN 06-03648                 \\
        \LogProg                & \LogProgCod     & Sem equivalência                                                   & --                           \\
        \hline
        \AnAlg                  & \AnAlgCod       & Análise de Algoritmos                                              & FEN 06-03713                 \\
        \CalcIII                & \CalcIIICod     & Cálculo Diferencial e Integral III                                 & IME 01-03646                 \\
        \CircEletI              & \CircEletICod   & Eletrônica I                                                       & FEN 05-01620                 \\
        \parbox[t]{4cm}{\FisII~e                                                                                                                      \\ \FisEII} & \parbox[t]{2cm}{\FisIICod \\ \FisEIICod} & Física Teórica e Experimental II & FIS 02-05143 \\
        \ProbEst                & \ProbEstCod     & Probabilidade e Estatística                                        & IME 05-05316                 \\
        \Ext                    & \ExtCod         & Sem equivalência                                                   & --                           \\
        \hline
        \parbox[t]{4cm}{\FisIII~e                                                                                                                     \\ \FisEIII} & \parbox[t]{2cm}{\FisIIICod \\ \\ \FisEIIICod} & Física Teórica e Experimental III & FIS 03-05185 \\
        \LabProgA               & \LabProgACod    & Laboratório de Programação I                                       & FEN 06-04049                 \\
        \LabProgPOO             & \LabProgPOOCod  & Características das Linguagens de Programação I                    & FEN 06-03980                 \\
        \ProcImag               & \ProcImagCod    & Sem equivalência                                                   & --                           \\
        \TecDig                 & \TecDigCod      & Técnicas Digitais I                                                & FEN 05-054498                \\
        \hline
        \CCC                    & \CCCCod         & Circuitos Elétricos I                                              & FEN 04-00944                 \\
        \FundComp               & \FundCompCod    & Fundamentos de Comp. Digitais I                                    & FEN 06-03787                 \\
        \SinaisESist            & \SinaisESistCod & Modelos Mat. Aplicados a Eng. Elétrica III                         & FEN 05-04923                 \\
        \Grafos                 & \GrafosCod      & Sem equivalência                                                   & --                           \\
        \parbox[t]{4cm}{\FisIV                                                                                                                        \\ \FisEIV} & \parbox[t]{2cm}{\FisIVCod~e \\ \FisEIVCod} & Física Teórica e Experimental IV & FIS 04-05212 \\
        \hline
        \ArqComp                & \ArqCompCod     & Arquitetura de Computadores I                                      & FEN 06-04119                 \\
        \CCA                    & \CCACod         & Circuitos Elétricos IV                                             & FEN 04-05222                 \\
        \EngSistA               & \EngSistACod    & Engenharia de Sistemas A                                           & FEN 06-04243                 \\
        \IC                     & \ICCod          & Sem equivalência                                                   & --                           \\
        \ICII                   & \ICIICod        & Sem equivalência                                                   & --                           \\
        \MatEle                 & \MatEleCod      & Materiais Elétricos e Magnéticos I                                 & FEN 04-05197                 \\
        \hline
        \MacroEco               & \MacroEcoCod    & Introdução à Economia III                                          & FCE 02-04657                 \\
        \MineraDados            & \MineraDadosCod & Sem equivalência                                                   & --                           \\
        \ProjBD                 & \ProjBDCod      & Engenharia de Sistemas B                                           & FEN 06-04314                 \\
        \Telep                  & \TelepCod       & Teleproc. e Redes de Computadores                                  & FEN 06-04718                 \\
        \TeoComp                & \TeoCompCod     & Teoria de Compiladores                                             & FEN 06-04516                 \\
        \ProjSO                 & \ProjSOCod      & Arquitetura de Sistemas Operacionais                               & FEN 06-04664                 \\
        \hline
        \AnaProjSist            & \AnaProjSistCod & Engenharia de Sistemas C                                           & FEN 06-04386                 \\
        \CompParal              & \CompParalCod   & Microcomp. e Microprocessadores                                    & FEN 06-04192                 \\
        \Control                & \ControlCod     & Controle de Processos por Comp.                                    & FEN 06-05080                 \\
        \Empre                  & \EmpreCod       & Sem equivalência                                                   & --                           \\
        \Sredes                 & \SredesCod      & Sem equivalência                                                   & --                           \\
        \SistEmb                & \SistEmbCod     & Sem equivalência                                                   & --                           \\
        \hline
        \EstSup                 & \EstSupCod      & Estágio Supervisionado XI-A                                        & FEN 06-05038                 \\
        \Instala                & \InstalaCod     & Instalações de Ambiente Computacional                              & FEN 06-03860                 \\
        \ProjA                  & \ProjACod       & Projeto de Graduação XI-A                                          & FEN 06-04578                 \\
        \hline
        \Adm                    & \AdmCod         & Administração Aplicada à Engenharia III                            & FAF 03-04439                 \\
        \ProjB                  & \ProjBCod       & Projeto de Graduação XI-B                                          & FEN 06-04635                 \\
        \hline
        Eletivas Restritas      & FEN 06-xxxx     & Tópicos Especiais em Engenharia de Sistemas e Computação A, B ou C & \parbox[t]{2cm}{FEN 06-04889 \\ FEN 06-04939 \\ FEN 06-04990} \\
    \end{longtable}
\end{small}


\section{Ementário das Disciplinas}

As ementas das disciplinas obrigatórias e eletivas são apresentadas no anexo \ref{ementas}.

\section{Articulaçao com a Pós-Graduação }

A estrutura curricular do curso de Engenharia de Computação integra atividades teóricas e práticas, com foco na aplicação dos conhecimentos adquiridos à solução de problemas em Engenharia de Computação e áreas correlatas, como a Ciência da Computação. As atividades laboratoriais distribuídas ao longo da graduação são enriquecidas pela atuação dos docentes em projetos de pesquisa, desenvolvidos como parte de sua carga horária regular, o que favorece a inserção dos estudantes no ambiente acadêmico-científico.

Essa inserção ocorre principalmente por meio da participação em projetos de Iniciação Científica (IC), com apoio do Programa Institucional de Bolsa de Iniciação Científica da UERJ (PIBIC/UERJ) ou por meio de estágio voluntário. O Trabalho de Conclusão de Curso (TCC) também representa uma via relevante de articulação entre ensino e pesquisa. O envolvimento dos alunos em atividades de pesquisa desde a graduação estimula o pensamento crítico, a autonomia intelectual e o interesse pela investigação científica, criando condições favoráveis à continuidade da formação em programas de pós-graduação.

O corpo docente do curso participa ativamente dos Programas de Pós-Graduação da UERJ, contribuindo com a oferta de disciplinas, orientação de alunos, produção científica e organização de eventos acadêmicos. Esse cenário promove um ambiente de excelência e formação continuada, incentivando os estudantes da graduação a prosseguir seus estudos em nível avançado, ampliando suas competências técnicas e científicas de acordo com as demandas da sociedade e do mercado de trabalho.

\section{Atividades de Extensão}

\subsection{Filosofia}
A extensão universitária é concebida como um processo transformador que integra a universidade à sociedade, promovendo a democratização do conhecimento e contribuindo para a formação acadêmica, cidadã e ética dos estudantes. No curso de Engenharia de Computação da UERJ, as atividades de extensão reforçam o compromisso social da universidade, possibilitando a aplicação dos saberes técnicos em benefício da comunidade, especialmente em contextos de exclusão digital.

\subsection{Regulamentação}
Conforme a Resolução CNE/CES \ordm{n} 7/2018 (Anexo \ref{rcne2018}), todos os cursos de graduação devem destinar, no mínimo, 10\% da carga horária total às atividades de extensão. Na UERJ, a Deliberação \ordm{n} 04/2023 (Anexo \ref{del4}) regulamenta a curricularização da extensão, permitindo sua realização por meio de:

\begin{itemize}
    \item Disciplinas com carga horária parcialmente dedicada à extensão;
    \item Disciplinas integralmente dedicadas à extensão;
    \item Atividade Curricular de Extensão (ACE).
\end{itemize}

\subsection{Inserção no Curso}

No curso de Engenharia de Computação, a extensão é integrada à formação acadêmica de maneira estruturada e interdisciplinar. O objetivo é aproximar os estudantes das demandas da sociedade e estimular o desenvolvimento de competências técnicas, analíticas e críticas.

As ações extensionistas são planejadas para gerar impacto social real, promovendo, além da aplicação prática dos conhecimentos, a reflexão sobre o papel da tecnologia no desenvolvimento social. A extensão é, portanto, concebida como uma oportunidade de formação cidadã, contribuindo para a inserção dos estudantes em projetos que ampliem sua visão acadêmica e social.

\subsection{Estrutura das Atividades Extensionistas}

A carga horária de extensão no curso totaliza \hExtensao horas, representando \num{\fpeval{ceil(100 * \thehExtensaoCounter / \thethorasCursoCounter,2)}}\% da carga horária total de \tHorasCurso\ horas, conforme a legislação vigente.

As atividades estão organizadas em duas modalidades:

\textbf{1. Disciplinas Integralmente Dedicada à Extensão:}
a disciplina obrigatória \textbf{\Ext}, oferecida no terceiro período, com carga horária de \ExtCH{ } horas, tem como objetivo introduzir os estudantes às práticas de extensão universitária, por meio de projetos que envolvem inovação social, divulgação científica e uso da tecnologia para benefício comunitário.

Além disso, a disciplina \textbf{\EstSup}, com carga horária de \EstSupCH~horas, amplia essa formação ao proporcionar experiências práticas em ambientes reais de trabalho, onde o aluno consolida competências técnicas e extensionistas. Assim, o estágio supervisionado se configura como uma oportunidade de impactar positivamente a sociedade enquanto reforça a formação acadêmica (ver Seção~\ref{sec:estagio-supervisionado-extensionista}).


\textbf{2. Atividades Curriculares de Extensão (ACE):}
para integralizar o curso de graduação em Engenharia de Computação da UERJ, o aluno deve cumprir, no mínimo, \hACE horas de Atividade Curricular de Extensão (ACE).

Conforme estabelecido na Deliberação n\textordmasculine~04/2023 (Anexo \ref{del4}), as ACEs devem ser realizadas ao longo do curso e concluídas antes do término da graduação. Nessas atividades, é imprescindível que o aluno atue de forma ativa e protagonista.

Em consonância com a referida deliberação e alinhado aos objetivos pedagógicos da formação em Engenharia de Computação, as ACEs válidas para a integralização do curso incluem:


\begin{itemize}
    \item \textbf{Participação em programas e projetos de extensão} (como bolsista ou voluntário), coordenados por professores ou técnicos da carreira de nível superior na Universidade do Estado do Rio de Janeiro, com validação da carga horária por meio de Declaração ou Certificado emitido pelo coordenador do projeto;
    \item \textbf{Ações extensionistas realizadas em programas institucionais das Pró-Reitorias}, com validação da carga horária por meio de Declaração ou Certificado do coordenador do programa;
    \item \textbf{Estágios não obrigatórios caracterizados como ação extensionista}, com validação da carga horária por meio de declaração ou Certificado da instituição responsável;
    \item \textbf{Desenvolvimento conjunto de soluções tecnológicas} visando atender as demandas dos atores da sociedade civil, com validação da carga horária por meio de Declaração ou Certificado da instituição responsável;
    \item \textbf{Participação em eventos}, tanto na organização quanto na realização, com validação da carga horária por meio de Declaração ou Certificado da instituição responsável;
    \item \textbf{Publicações relacionadas à extensão}, nas seguintes categorias:
          \begin{itemize}
              \item Artigo em revista indexada ou capítulo de livro com ISBN: 15 horas por publicação, validadas com informações catalográficas, ISSN ou ficha catalográfica e primeira página do artigo ou capítulo;
              \item Livro com ISBN: 30 horas por publicação, validadas com capa, contracapa e ficha catalográfica;
              \item Resumos e resumos expandidos publicados em anais de eventos: 5 horas por publicação, validadas com informações catalográficas e primeira página do resumo.
          \end{itemize}

\end{itemize}

Em todas as situações citadas acima, as declarações ou certificados emitidos pelas entidades responsáveis deverão conter a carga horária e a descrição das atividades realizadas pelo aluno. Compete à Coordenação das Atividades Curriculares de Extensão (CACE) do curso de Engenharia de Computação receber, analisar e validar a documentação comprobatória da realização das ACE.

Para que possam ser computadas para a integralização curricular do curso, as atividades devem ser preferencialmente realizadas no âmbito das unidades pertencentes ao Centro de Tecnologia e Ciências (CTC/UERJ). No caso de alunos que tenham cursado Atividades Curriculares de Extensão (ACEs) em outras instituições, a carga horária poderá ser aproveitada mediante análise realizada pela CACE do curso de Engenharia de Computação. Essa análise avaliará a pertinência e a compatibilidade da atividade com os objetivos do curso.

Após a conclusão da ACE, o estudante deverá encaminhar à CACE do curso de Engenharia de Computação a documentação comprobatória contendo a carga horária total, o tipo e o título da atividade. Com base nesta documentação, o responsável pela coordenação deliberará sobre a aprovação ou não da atividade realizada. Detalhes adicionais sobre o processo de homologação podem ser consultados na Deliberação n\textordmasculine~04/2023 (Anexo \ref{del4}).

\textbf{Observação:} Em caso de dúvidas, os alunos são orientados a consultar antecipadamente a CACE para verificar a pertinência das atividades planejadas, evitando problemas de interpretação.


\section{Estágio Supervisionado}
\label{sec:estagio-supervisionado-extensionista}
As Diretrizes Curriculares Nacionais para os cursos de graduação da área de Computação (Anexo \ref{cne2016}), no Art.~\ordm{7}, \S~\ordm{1}, conferem às Instituições de Ensino Superior autonomia para definir a obrigatoriedade ou não do Estágio Supervisionado, bem como sua regulamentação:

\begin{itquotation}
    \noindent Art.~\ordm{7}\\
    \ldots\\
    \S~\ordm{1}   As Instituições de Educação Superior deverão estabelecer a
    obrigatoriedade ou não do Estágio Supervisionado para os cursos de bacharelado, bem como a
    sua regulamentação, especificando formas de operacionalização e de avaliação.
\end{itquotation}

Neste contexto, o presente Projeto Pedagógico propõe que a disciplina \textbf{\EstSup}, com carga horária de \EstSupCH~ horas (\EstSupCred~  créditos), seja integralmente reconhecida como \textbf{Disciplina de Extensão}.

Esta proposta está alinhada às recentes diretrizes nacionais de inserção da extensão na educação superior, encontrando respaldo em modelos já implementados com sucesso, como no Projeto Pedagógico do curso de Engenharia de Computação da Universidade Federal do Rio Grande do Norte (UFRN) \cite{ufrn2024}. No referido curso, o Estágio Curricular Obrigatório (página 42), com carga horária de 160 horas, é integralmente contabilizado como atividade extensionista, justificando-se pelo atendimento aos princípios estabelecidos nos artigos \ordm{5} e \ordm{7}  da Resolução CNE/CES \ordm{n}  7/2018 (Anexo \ref{rcne2018}), que define as Diretrizes para a Extensão na Educação Superior Brasileira. A resoluçao determina que (grifo nosso):


\begin{itquotation}
    \noindent% Zera o recuo da primeira linha
    Art. \ordm{5} Estruturam a concepção e a prática das Diretrizes da Extensão na Educação
    Superior:\\
    I - a \underline{interação dialógica da comunidade acadêmica com a sociedade} por meio da
    troca de conhecimentos, da participação e do contato com as questões complexas
    contemporâneas presentes no contexto social;\\
    II - a formação cidadã dos estudantes, marcada e constituída pela \underline{vivência dos} \underline{seus
        conhecimentos}, que, de modo \underline{interprofissional e interdisciplinar}, seja
    valorizada e integrada à matriz curricular;\\
    III - a produção de mudanças na própria instituição superior e nos \underline{demais setores} \underline{da
        sociedade}, a partir da construção e aplicação de conhecimentos, bem como por outras
    atividades acadêmicas e sociais;\\
    IV - a \underline{articulação entre ensino/extensão/pesquisa}, ancorada em processo pedagógico
    único, interdisciplinar, político educacional, cultural, científico e \underline{tecnológico}.\\
    \ldots\\
    Art. \ordm{7} \underline{São consideradas atividades de extensão as intervenções que envolvam} diretamente \underline{as comunidades externas às instituições de ensino superior} e que estejam vinculadas \underline{à formação do estudante},
    nos termos desta Resolução, e conforme normas institucionais próprias.
\end{itquotation}

A UFRN ressalta que o estágio supervisionado, enquanto atividade prática supervisionada, contribui para a prática extensionista ao inserir o estudante em contextos reais de atuação no setor produtivo da sociedade, promovendo a interação transformadora entre universidade e sociedade.

Na UERJ, a Deliberação \ordm{n} 04/2023 (Anexo \ref{del4}) oferece respaldo adicional a esta proposta. Em especial, seu Art. \ordm{2} destaca que a Inserção Curricular da Extensão tem como finalidade reforçar a interação com a sociedade, \textit{visando a impactos positivos nos âmbitos culturais, científicos, artísticos, educacionais, sociais, ambientais e esportivos}, além da \textit{geração de emprego e renda, da inovação, do empreendedorismo} e do atendimento a \textit{demandas coletivas}. Este princípio é plenamente atendido pela inserção do estudante em experiências reais proporcionadas pelo estágio supervisionado, consolidando a extensão como instrumento de transformação e desenvolvimento social.

A disciplina \EstSup, pelas suas características e objetivos, \textbf{atende plenamente aos requisitos para caracterização como disciplina de extensão}, conforme demonstrado a seguir:

\begin{itemize} \item \textbf{Interação Transformadora com a Sociedade:} O estágio proporciona ao estudante a oportunidade de aplicar conhecimentos acadêmicos em ambientes profissionais reais, estabelecendo um canal efetivo de diálogo entre a universidade e a sociedade, com impactos no desenvolvimento tecnológico e social. \item \textbf{Impacto Social e Inovação:} As atividades realizadas no estágio frequentemente resultam em soluções inovadoras, otimização de processos e desenvolvimento de novos produtos ou serviços, gerando benefícios concretos para a sociedade. \item \textbf{Protagonismo Estudantil:} O estudante desempenha papel ativo e protagonista na execução das atividades de estágio, sob supervisão acadêmica e profissional, fortalecendo sua autonomia, responsabilidade e formação cidadã. \item \textbf{Articulação entre Ensino e Pesquisa:} O estágio favorece a consolidação de conhecimentos adquiridos e pode estimular novas linhas de pesquisa, integrando práticas de ensino, pesquisa e extensão. \item \textbf{Adequação às Modalidades de Extensão:} As ações desenvolvidas nos estágios enquadram-se nas modalidades previstas no Art. \ordm{8}   da Resolução CNE/CES \ordm{n} 7/2018 (Anexo \ref{rcne2018}), como \textbf{projetos}, \textbf{prestação de serviços} e, eventualmente, \textbf{cursos ou oficinas}. \end{itemize}

Assim, a caracterização da disciplina \textbf{\EstSup} como Disciplina de Extensão neste PPC atende não apenas às exigências legais e normativas vigentes, mas também valoriza a formação prática dos estudantes, amplia o impacto social do curso e aproxima a universidade das demandas reais da sociedade. Esta prática, já adotada por instituições de referência, fortalece a formação acadêmica e profissional dos futuros engenheiros de computação da UERJ.


\section{Administração Acadêmica do Curso}

Caberá ao DESC exercer a administração do curso de Engenharia da Computação.
Como a reformulação do curso propõe sua reestruturação como uma habilitação específica em Engenharia de Computação, a administração do curso deverá ser reestruturada também.

\subsection{Chefia e Subchefia do Departamento}

A administração do Departamento de Engenharia de Sistemas e Computação (DESC) é conduzida por um Chefe de Departamento e um Subchefe. Esses cargos são ocupados por docentes efetivos do departamento. A eleição do chefe e subchefe se dá por meio do voto dos membros do corpo deliberativo do DESC com um mandato de até dois anos. O corpo deliberativo do DESC é formado por todos os professores efetivos do departamento e é convocado pela chefia, a fim de deliberar sobre as questões pertinentes ao curso, tanto acadêmicas quanto administrativas.

A chefia do departamento tem a responsabilidade de coordenar as atividades acadêmicas e administrativas do departamento, incluindo:

\begin{itemize}
    \item Gestão do corpo docente, distribuição de disciplinas e planejamento acadêmico;
    \item Alocação de salas e os planos individuais dos docentes (PLANINDs) a cada semestre;
    \item Representação do departamento junto à direção da Faculdade de Engenharia e demais instâncias da UERJ;
    \item Organizações de reuniões com o corpo deliberativo do DESC, registrando em ata o que foi decidido e os professores que compareceram;
\end{itemize}

\subsection{Conselho Departamental}

O Conselho Departamental da Faculdade de Engenharia é o órgão deliberativo da unidade, presidido pela Direção da Faculdade de Engenharia e composto pelas chefias dos departamentos, representante do técnicos-administrativos e representantes dos alunos. O DESC tem um representante no Conselho Departamental da Faculdade de Engenharia, o representante do DESC no Conselho Departamental é o chefe de departamento ou o seu subchefe.


\subsection{Coordenação de Graduação}

A Coordenação de Graduação é responsável pelo gerenciamento acadêmico do curso, garantindo o cumprimento das diretrizes curriculares e auxiliando os estudantes em sua trajetória acadêmica. Suas funções incluem:

\begin{itemize}
    \item Gestão Acadêmica: supervisiona a execução do projeto pedagógico do curso, garantindo o cumprimento das diretrizes curriculares e regulamentos institucionais;
    \item Apoio aos Estudantes: acompanha o desempenho dos alunos, orientando-os em questões acadêmicas, como matrícula, integralização curricular e aproveitamento de disciplinas;
    \item Articulação com Professores: coordena e apoia o corpo docente, promovendo a atualização e melhoria das práticas de ensino.

\end{itemize}

\subsection{Coordenação das Atividades Curriculares de Extensão (CACE)}

A Coordenação das Atividades Curriculares de Extensão (CACE) orienta e verifica o cumprimento das ações de extensão realizadas pelos estudantes e valida a carga horária a ser contabilizada semestralmente. A CACE deve garantir que cada aluno cumpra as horas obrigatórias de extensão.

De acordo com a Deliberação n\textordmasculine~04/2023 (Anexo \ref{del4}), as atribuições da CACE incluem:

\begin{itemize}
    \item Propor ao Conselho Departamental a indicação de atividades de extensão adicionais;
    \item Receber, analisar e validar a documentação comprobatória da realização das atividades de extensão;
    \item Fixar e divulgar a data-limite para o recebimento da documentação mencionada no item anterior;
    \item Consultar o Departamento de Extensão da PR-3 (DEPEXT/PR-3) a respeito das atividades de extensão ativas e vinculadas a ações devidamente cadastradas e/ou validadas na PR-3, assim como as demais Pró-reitorias;
    \item Encaminhar à Secretaria do curso de graduação ou ao setor devidamente designado pela Unidade Acadêmica a relação de alunos que desenvolveram atividades de extensão no semestre, bem como a carga horária total da atividade, tipo e título da atividade em período consoante o calendário acadêmico da UERJ, para fins de registro acadêmico;
    \item Computar a carga horária atribuída pela instituição de origem;
    \item Validar o aproveitamento das cargas horárias das atividades de extensão certificadas/declaradas por outras IES no Brasil ou no exterior;
    \item Analisar e emitir parecer sobre o aproveitamento integral da carga horária nas atividades curriculares de extensão executadas anteriormente, nos casos de transferência externa, transferência interna, ingressos por aproveitamento de estudos ou por Vestibular;
    \item Apreciar pedidos de recurso formulados pelos alunos em relação ao indeferimento do cômputo de carga horária relativa às atividades de extensão.

\end{itemize}

Compete à Direção da Faculdade de Engenharia a indicação de docente responsável pela CACE para o curso de Engenharia de Computação.

\subsection{Coordenação de Áreas}

O curso também conta com Coordenações de Área, que são responsáveis por organizar e orientar o ensino dentro de diferentes campos do conhecimento da Engenharia de Computação.
A integração das disciplinas em áreas de conhecimento permite o compartilhamento de informações sobre interesses e objetivos comuns. Além disso, favorece a atuação conjunta de alunos e professores em temas globais e impulsiona a criação de linhas de pesquisa.


As coordenações de área têm como função:

\begin{itemize}
    \item Orientar os alunos em questões referentes às disciplinas;
    \item Analisar os requerimentos de quebras de pré-requisitos e conflitos de horário;
    \item Tratar questões relativas aos conteúdos programáticos das disciplinas;
    \item Promover a integração dos professores da mesma área, e incentivar a pesquisa na área.
\end{itemize}

As disciplinas do curso de Engenharia de Computação sob resposabilidade do DESC estão divididas em quatro grandes áreas de conhecimento: Algoritmos e Linguagens de Programação; Arquitetura de Sistemas de Computação; Lógica e Semântica de Programas; Sistemas de Informação. Cada área de conhecimento deverá possuir um Professor Coordenador. A Tabela \ref{tab:areas} mostra a distribuição das disciplinas por área de conhecimento.

\rowcolors{1}{gray!10}{white}
\begin{table}[ht]
    \centering
    \caption{Tabela de divisão de disciplinas por área de conhecimento}
    \label{tab:areas}
    \begin{tabularx}{0.9\textwidth}{ X l }
        \hiderowcolors
        \toprule
        {\bf Área de Conhecimento}                              & {\bf Disciplinas} \\
        \hline
        \multirow{5}{*}{Algoritmos e Linguagens de Programação} & \AlgComp          \\ % Algoritmos Computacionais
                                                                & \AnAlg            \\ % Análise de Algoritmos
                                                                & \EstrInf          \\ % Estruturas de Informação
                                                                & \LabProgA         \\ % Laboratório de Programação A
                                                                & \LabProgPOO       \\ % Laboratório de POO
                                                                & \TeoComp          \\ % Teoria da Compiladores
                                                                & \Grafos           \\ \hline
        \multirow{8}{*}{Arquitetura de Sistemas de Computação}  & \ArqComp          \\ % Arqutetura de Computadores
                                                                & \CompParal        \\ % Computação Paralela
                                                                & \Control          \\ % Controle de Processos
                                                                & \FundComp         \\ % FUndamentos de Computadores II
                                                                & \Instala          \\ % Inst. de Ambientes Computacionais
                                                                & \ProjSO           \\ % Projeto de Sistemas
                                                                & \Telep            \\ % Redes de Computadores
                                                                & \Sredes           \\ % Segurança em Redes
                                                                & \SistEmb          \\ % Sistema Embutidos
        \hline
        \multirow{3}{*}{Lógica e Semântica de Programas}        & \IC               \\ % Inteligência Computacional
                                                                & \ICII             \\ % Inteligência Computacional II
                                                                & \LogProg          \\ % Lógica de Programação
                                                                & \MineraDados      \\ % Mineração de Dados
                                                                & \ProcImag         \\ % Processamento de Imagens
        \hline
        \multirow{4}{*}{Sistemas de Informação}                 & \AnaProjSist      \\ % Análise de Projeto de Sistemas
                                                                & \EngSistA         \\ % Engenharia de Sistemas
                                                                & \ProjBD           \\
                                                                & \EngCompSoc       \\
        \toprule
    \end{tabularx}

\end{table}


