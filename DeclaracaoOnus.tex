\documentclass[oneside,envcountsame,envcountchap,openany]{svmono}
\usepackage[utf8]{inputenc}
\usepackage{tikz}			% para gráficos e desenhos
\usetikzlibrary{calc}		% para cálculos em gráficos
\usepackage{fancyhdr}
\usepackage{makeidx}         % permite geracao de indice
% Definição de comandos personalizados
% Definição de ordinal masculino e feminino
\usepackage{etoolbox} % para garantir compatibilidade em várias situações
\usepackage{pdfpages}		 % para incluir pdfs
\usepackage{tocloft} 		 % para formatar o sumário
% Pacotes de hiperlinks
\usepackage{hyperref}
\hypersetup{
	colorlinks,
	citecolor=blue,
	filecolor=magenta,
	linkcolor=blue,
	urlcolor=cyan
}

\DeclareRobustCommand{\subsup}[1]{%
  \textsuperscript{%
    \tikz[baseline=(char.base)]{
      \node[inner sep=0.5pt, anchor=base] (char) {\textnormal{#1}};
      \draw[line width=0.3pt, yshift=-0.2ex]
        ($(char.south west)!0.15!(char.south east)$) --
        ($(char.south west)!0.85!(char.south east)$);
    }%
  }%
}
\DeclareRobustCommand{\ordm}[1]{#1\subsup{o}} % ordinal masculino
% Pacotes de formatação e layout
\usepackage[top=1in,left=2.5cm,bottom=1in,right=2.5cm]{geometry}
\title{Documentos Oficiais}
\begin{document}
\pagestyle{empty}
\thispagestyle{empty}
\begin{center}
      \textbf{DECLARAÇÃO DE NÃO EXISTÊNCIA DE ÔNUS PARA A UNIVERSIDADE}
\end{center}
A presente proposta de criação do curso de Engenharia de Computação foi desenvolvida considerando apenas o corpo
docente, quadro administrativo, e infraestrutura física atual do Departamento de Engenharia de Sistemas e Computação.

Desta forma a sua implantação não implicará em nenhum ônus adicional para a universidade,
não demandando ampliação do quadro docente, quadro administrativo, expansão/reforma de
espaço físico ou aquisição de equipamentos.



\end{document}