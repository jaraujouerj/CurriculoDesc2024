\documentclass[a4paper, landscape]{article}
\usepackage{prerex}
\usepackage{multicol}
%\usepackage{showframe}
\usepackage[utf8]{inputenc} % codificacao de caracteres
\usepackage{geometry}
\usepackage[portuguese]{babel}
\usepackage{gensymb}
\usepackage{lmodern}
\usepackage{tikz}
\usepackage{xcolor}
\usetikzlibrary{calc}
\usepackage{ifthen}
\usepackage{disciplinasDB}
\tikzset{
    %Define standard arrow tip
    >=stealth,
    %Define style for boxes
    box/.style={
           rectangle,
           draw=black,
           text width=7em,
           minimum height=5em,
           text centered,
           inner sep=1mm]},
    % Define arrow style
    pil/.style={
           ->,
           thick,
           shorten <=3pt,
           shorten >=3pt,},
}
\usetikzlibrary{positioning,shapes,shadows,arrows}

\colorlet{cbasico}{yellow}
\colorlet{cdesc}{orange}
\colorlet{celetronica}{green}
\colorlet{celetrica}{blue!20!white}
\colorlet{cmecanica}{gray!20!white}
\colorlet{cambiental}{olive!20!white}
\colorlet{cindustrial}{violet!40!white}
\colorlet{celetiva}{orange!50!white}
% Define estilos de caixas de acordo com a origem da disciplina
\tikzstyle{basico}=[box, fill=cbasico]
\tikzstyle{desc}=[box, fill=cdesc]
\tikzstyle{eletronica}=[box, fill=celetronica]
\tikzstyle{eletrica}=[box, fill=celetrica]
\tikzstyle{mecanica}=[box, fill=cmecanica]
\tikzstyle{ambiental}=[box, fill=cambiental]
\tikzstyle{industrial}=[box, fill=cindustrial]
\tikzstyle{eletiva}=[box, fill=celetiva, dashed]
\tikzstyle{info}=[box, fill=yellow]

%contador de creditos totais do curso
\newcounter{cred} 
\setcounter{cred}{0}

%contador de horas totais do curso
\newcounter{thoras}
\setcounter{thoras}{0}

%parâmetros de cada disciplina
%{basico || eletronica || desc || eletiva} 	#1
%{Período} de 1 a 10 							#2
%{Linha no fluxograma} de 1 a 6. 				#3
%{Código da disciplina} 						#4
%{Nome da Disciplina} 							#5
%{Carga Horária} 								#6
%{Número de Créditos} 							#7
%{nome abreviado para referência} 				#8
\newcommand{\disciplina}[8] {
  \addtocounter{cred}{#7}
  \addtocounter{thoras}{#6}
  \node at ($ (#2*2.8cm-7cm,14.6-#3*2.3) $) (Item)  [#1] (#8) 
        {
            \textbf{#4}\\
            \textbf{#5}\\
            \textbf{#6} \hfill #7
        };}

\geometry{margin=0.5cm,bottom=0.5cm} 

\thispagestyle{empty}
\begin{document}
\noindent
{\Large \textbf{Curso: Engenharia}}\\
{\Large \textbf{Habilitação: Computação}}  (Versão de \today)

\begin{tikzpicture}
  \foreach \x in {1,...,10} {\node at (\x*2.8-7,13.5) {\x\degree Período};}

  \footnotesize
  \fontfamily{lms}\fontseries{l}\selectfont
  %parâmetros de cada disciplina, pode ser posto numa única linha
  %\disciplina
  %{basico || eletronica || desc || eletiva}
  %{Período} de 1 a 10
  %{Linha no fluxograma} de 1 a 8.
  %{Código da disciplina}
  %{Nome da Disciplina}
  %{Carga Horária}
  %{Número de Créditos}
  %{nome abreviado para referência} 

  %Básico
  \disciplina{basico}{2}{5}{\AlgLinCod}	{\AlgLinSName}	{\AlgLinCH}	{\AlgLinCred}	{AlgLin} % Álgebra Linear
  \disciplina{basico}{3}{2}{\AnaVetCod}	{\AnaVetSName}	{\AnaVetCH}	{\AnaVetCred}	{AnaVet} % Análise Vetorial
  \disciplina{basico}{1}{3}{\CalcICod}	{\CalcISName}	{\CalcICH}	{\CalcICred}	{CalcI} % Cálculo I
  \disciplina{basico}{2}{3}{\CalcIICod}	{\CalcIISName}	{\CalcIICH}	{\CalcIICred}	{CalcII} % Cálculo II
  \disciplina{basico}{3}{3}{\CalcIIICod}	{\CalcIIISName}	{\CalcIIICH}{\CalcIIICred}	{CalcIII} % Cálculo III
  \disciplina{basico}{1}{6}{\DesBasCod}	{\DesBasSName}	{\DesBasCH}	{\DesBasCred}	{DesBas} % Desenho Básico
  \disciplina{basico}{2}{6}{\FisICod}		{\FisISName}	{\FisICH}	{\FisICred}		{FisI} % Física I
  \disciplina{basico}{2}{7}{\FisEICod}		{\FisEISName}	{\FisEICH}	{\FisEICred}		{FisEI} % Física Experimental I
  \disciplina{basico}{3}{6}{\FisIICod}	{\FisIISName}	{\FisIICH}	{\FisIICred}	{FisII} % Física II
  \disciplina{basico}{3}{7}{\FisEIICod}	{\FisEIISName}	{\FisEIICH}	{\FisEIICred}	{FisEII} % Física Experimental II
  \disciplina{basico}{4}{6}{\FisIIICod}	{\FisIIISName}	{\FisIIICH}	{\FisIIICred}	{FisIII} % Física III
  \disciplina{basico}{4}{7}{\FisEIIICod}	{\FisEIIISName}	{\FisEIIICH}	{\FisEIIICred}	{FisEIII} % Física Experimental III
  \disciplina{basico}{5}{6}{\FisIVCod}	{\FisIVSName}	{\FisIVCH}	{\FisIVCred}	{FisIV} % Física IV
  \disciplina{basico}{5}{7}{\FisEIVCod}	{\FisEIVSName}	{\FisEIVCH}	{\FisEIVCred}	{FisEIV} % Física Experimental IV
  \disciplina{basico}{1}{2}{\GeoAnaCod}	{\GeoAnaSName}	{\GeoAnaCH}	{\GeoAnaCred}	{GeoAna} % Geometria Analítica
  \disciplina{mecanica}{3}{5}{\MecTecCod}	{\MecTecSName}	{\MecTecCH}	{\MecTecCred}	{MecTec} % Mecânica Técnica
  \disciplina{basico}{3}{4}{\ProbEstCod}	{\ProbEstSName}	{\ProbEstCH}{\ProbEstCred}	{ProbEst} % Probabilidade e Estatística
  \disciplina{basico}{1}{4}{\QuiTCod}		{\QuiTSName}	{\QuiTCH}	{\QuiTCred}		{QuiT} % Química Teorica
  \disciplina{basico}{1}{5}{\QuiECod}		{\QuiESName}	{\QuiECH}	{\QuiECred}		{QuiE} % Química Experimental

  %disciplinas de outros departamentos da engenharia
  \disciplina{eletronica}{10}{5}{\AdmCod}	{\AdmSName}		{\AdmCH}	{\AdmCred}		{Adm} % Administração
  %\disciplina{eletrica}{7}{6}	{\AnaFisCod}{\AnaFisSName}	{\AnaFisCH}	{\AnaFisCred}	{AnaFis} % Análise de Sistemas Físicos
  \disciplina{eletrica}{5}{4}	{\CEVCod}	{\CEVSName}		{\CEVCH}	{\CEVCred}		{CEV} % Circuitos Elétricos
  \disciplina{eletrica}{6}{5}{\CEVICod}	{\CEVISName}	{\CEVICH}	{\CEVICred}		{CEVI} % Circuitos Elétricos VI
  %\disciplina{eletronica}{8}{6}{\CServMecCod}{\CServMecSName}{\CServMecCH}{\CServMecCred}{ServoMec} % Controle de Servomecanismos
  \disciplina{eletronica}{5}{5}{\EletICod}	{\EletISName}{\EletICH}{\EletICred}{EletI} % Eletrônica I
  \disciplina{eletronica}{6}{6}{\EletIIACod}	{\EletIIASName}{\EletIIACH}{\EletIIACred}{EletII} % Eletrônica II
  \disciplina{mecanica}{4}{2}{\FenTranCod}{\FenTranSName}{\FenTranCH}{\FenTranCred}{FenTran} % Fenômenos de Transporte
  \disciplina{eletronica}{9}{5}{\IntEcoCod}{\IntEcoSName}{\IntEcoCH}{\IntEcoCred}{IntEco} % Introdução à Economia
  \disciplina{ambiental}{9}{2}	{\IntAmbCod}	{\IntAmbSName}{\IntAmbCH}{\IntAmbCred}{IntAmb} % Introdução à Engenharia Ambiental
  \disciplina{eletrica}{5}{3}	{\MatEleCod}	{\MatEleSName}{\MatEleCH}{\MatEleCred}{MatEle} % Materiais Elétricos
  \disciplina{eletronica}{4}{3} {\ModMatCod}	{\ModMatSName}{\ModMatCH}{\ModMatCred}{ModMat} % Modelos Matemáticos
  \disciplina{industrial}{4}{4} {\MetQuantCod}	{\MetQuantSName}{\MetQuantCH}{\MetQuantCred}{PO} % Métodos Quantitativos
  \disciplina{eletronica}{7}{5}	{\PrincTelecCod}	{\PrincTelecSName}{\PrincTelecCH}{\PrincTelecCred}{PrincTelec} % Princípios de Telecomunicações
  \disciplina{mecanica}{4}{5}	{\ResMatCod}	{\ResMatSName}{\ResMatCH}{\ResMatCred}{ResMat} % Resistência dos Materiais
  \disciplina{industrial}{8}{4}	{\SegHigCod}	{\SegHigSName}{\SegHigCH}{\SegHigCred}{SegHig} % Engenharia do Trabalho


  %Disciplinas Desc
  \disciplina{desc}{1}{1} {\AlgCompCod}	{\AlgCompSName}{\AlgCompCH}{\AlgCompCred}{AlgComp} % Algoritmos Computacionais
  \disciplina{desc}{6}{2} {\AnAlgCod}	{\AnAlgSName}{\AnAlgCH}{\AnAlgCred}{AnAlg} % Análise de Algoritmos
  \disciplina{desc}{6}{3} {\ArqCompCod}	{\ArqCompSName}{\ArqCompCH}{\ArqCompCred}{ArqComp} % Arquitetura de Computadores
  \disciplina{desc}{10}{2}{\ControlCod}	{\ControlSName}{\ControlCH}{\ControlCred}{Control} % Controle
  \disciplina{desc}{2}{2} {\EngComputCod}    {\EngComputSName}{\EngComputCH}{\EngComputCred}{CalcNum} % Engenharia de Computação
  \disciplina{desc}{7}{3} {\EngCompSocCod}	{\EngCompSocSName}{\EngCompSocCH}{\EngCompSocCred}{ETS} % Engenharia e Sociedade
  \disciplina{desc}{6}{1} {\EngSistACod}	{\EngSistASName}{\EngSistACH}{\EngSistACred}{EngSistA} % Engenharia de Sistemas A
  \disciplina{desc}{7}{1} {\ProjBDCod}	{\ProjBDSName}{\ProjBDCH}{\ProjBDCred}{EngSistB} % Engenharia de Sistemas B
  \disciplina{desc}{9}{3} {\EstSupCod}	{\EstSupSName}{\EstSupCH}{\EstSupCred}{EstSup} % Estágio Supervisionado
  \disciplina{desc}{2}{1} {\EstrInfCod}	{\EstrInfSName}{\EstrInfCH}{\EstrInfCred}{EstrInf} % Estruturas de Informação
  \disciplina{desc}{5}{2} {\FundCompCod}	{\FundCompSName}{\FundCompCH}{\FundCompCred}{FundComp} % Fundamentos de Computadores
  \disciplina{desc}{6}{4} {\ICCod}	{\ICSName}{\ICCH}{\ICCred}{IC} % Inteligência Computacional
  \disciplina{desc}{3}{1} {\LabProgACod}	{\LabProgASName}{\LabProgACH}{\LabProgACred}{LabProgI} % Laboratório de Programação I
  \disciplina{desc}{4}{1} {\LabProgBCod}	{\LabProgBSName}{\LabProgBCH}{\LabProgBCred}{LabProgII} % Laboratório de Programação II
  \disciplina{desc}{5}{1} {\LogProgCod}	{\LogProgSName}{\LogProgCH}{\LogProgCred}{LogProg} % Lógica de Programação
  \disciplina{desc}{7}{4}	{\MineraDadosCod}	{\MineraDadosSName}{\MineraDadosCH}{\MineraDadosCred}{MineraDados} % Mineração de Dados
  \disciplina{desc}{8}{5} {\ProcImagCod}	{\ProcImagSName}{\ProcImagCH}{\ProcImagCred}{ProcImag} % Processamento de Imagens
  \disciplina{desc}{8}{1} {\CompParalCod}	{\CompParalSName}{\CompParalCH}{\CompParalCred}{ProgPara} % Programação Paralela
  \disciplina{desc}{9}{1} {\ProjACod}	{\ProjASName}{\ProjACH}{\ProjACred}{ProjA} % Projeto A
  \disciplina{desc}{10}{1}{\ProjBCod}	{\ProjBSName}{\ProjBCH}{\ProjBCred}{ProjB} % Projeto B
  \disciplina{desc}{7}{6} {\ProjSOCod}	{\ProjSOSName}{\ProjSOCH}{\ProjSOCred}{SO} % Sistemas Operacionais
  \disciplina{desc}{8}{2} {\SistEmbCod}	{\SistEmbSName}{\SistEmbCH}{\SistEmbCred}{SistEmb} % Sistemas Embarcados
  \disciplina{desc}{8}{3} {\TelepCod}	{\TelepSName}{\TelepCH}{\TelepCred}{Telep} % Teleprocessamento
  \disciplina{desc}{7}{2} {\TeoCompCod}	{\TeoCompSName}{\TeoCompCH}{\TeoCompCred}{Comp} % Teoria da Compiladores

  %eletivas
  \disciplina{eletiva}{9}{4} {\EletACod}	{Eletiva Restrita}{\EletACH}{\EletACred}{EA}
  \disciplina{eletiva}{10}{3}{\EletACod}	{Eletiva Restrita}{\EletACH}{\EletACred}{EB}
  \disciplina{eletiva}{10}{4}{\EletACod}	{Eletiva Restrita}{\EletACH}{\EletACred}{EC}

  %ligação com pre-requititos
  \newcommand{\prereq}[3][black] { % \prereq[cor]{disciplina}{pre}
    \draw [->, thick, #1](#2.east)--(#3.west);} % Uma linha direta para o próximo bloco

  \newcommand{\pre}[8][black] {
    \draw [->, thick, #1]
    ($(#2.east) + (0 mm,#4mm) $) -| % #4 distância do centro de saída do bloco prérequisito
    \ifnum#6=0
      ($ (#3.west)+(#5mm,-#7mm)$) --
      ($(#3.west)-(0,#7mm)$);
      \else
      ($(#2.east) + (#5mm,#6mm) $) -| % #5 até onde vai a linha horizontal #6 até onde vai a linha vertical
      ($(#3.west) - (#7mm,#8mm)$) -- ($(#3.west) - (0,#8mm)$);
    \fi
  }

  % %Período 1
  % \prereq			  {AlgComp}	{EstrInf}
  % \pre	[purple]{AlgComp}	{CalcNum} {-3}{-5}{0}{-3}{0}
  % \pre	[red]   {AlgComp}	{AnAlg}   {-1}{3}{-13}{4}{0}
  % \pre	[blue]  {AlgComp}	{FundComp}{-2}{2}{-14}{2}{0}
  % \pre	[blue]	{CalcI}		{CalcNum}{1}{-2}{0}{-1}{0}
  % \prereq				{CalcI}		{CalcII}

  % % Período 2
  % \prereq				{EstrInf}	{LabProgI}
  % \pre	[teal]	{EstrInf}	{EngSistB}  {-1}{3}{-10}{3}{2}
  % \prereq				{CalcII}	{CalcIII}
  % \prereq				{FisI}		{FisII}
  % \prereq				{FisEI}		{FisEII}
  % \pre	[blue]  {FisI}		{MecTec}    {1}{-3}{0}{-1}{0}
  % \pre	[blue]	{CalcII}	{ProbEst}   {-1}{-2}{0}{-3}{0}
  % \pre	[red]		{CalcII}	{AnaVet}    {1}{-2}{0}{-1}{0}
  % \pre	[purple]{CalcII}	{MecTec}    {-2}{-3}{0}{-3}{0}
  % \pre  [blue]	{FisII}		{FenTran}   {1}{-2}{0}{2}{0}
  % \pre			    {AlgLin}	{PO}        {2}{1}{14}{3}{0}
  % \pre	[red]	  {AlgLin}	{ProcImag}  {-1}{1}{-10}{3}{2}


  % % %Período 3
  % \prereq				{LabProgI}	{LabProgII}
  % \prereq				{FisII}		{FisIII}
  % \prereq				{FisEII}		{FisEIII}
  % \prereq				{AnaVet}	{FenTran}
  % \prereq				{MecTec}	{ResMat}
  % \pre [red]		{AnaVet}	{ModMat}    {-3}{-3}{0}{-3}{0}
  % \prereq			  {CalcIII}	{ModMat}
  % \pre [red]		{ProbEst}	{ProcImag}  {-1}{1}{-13}{4}{-1}
  % \pre  [blue]	{LabProgI}{IC}        {-1}{3}{-12}{3}{0}

  % % Período 4
  % \prereq				{FisIII}	{FisIV}
  % \pre [red]	  {FisIII}	{MatEle}  {3}{-4}{0}{-1}{0}
  % \pre [blue]	  {FisIII}	{EletI}   {1}{-2}{0}{-1}{0}
  % \pre [teal] 	{FisIII}	{CEV}     {2}{-3}{0}{-1}{0}
  % \pre [black]	{LabProgII}{AnAlg}  {-2}{2}{-8}{2}{-2}
  % \pre			    {ModMat}	{CEVI}    {-2}{3}{-35}{3}{-2}

  % % Período 5
  % \pre			    {EletI}		{EletII}  {-1}{-2}{0}{-4}{0}
  % \pre[purple]	{FundComp}{ArqComp} {-3}{-4}{0}{-3}{0}
  % \pre[blue]	  {CEV}		  {CEVI}    {-1}{-2}{0}{-4}{0}
  % \pre[purple]	{CEV}		  {EletII}  {-3}{-4}{0}{0}{0}
  % \prereq				{FisEIII}		{FisEIV}

  % % Período 6
  % \prereq[red]  {AnAlg}		{Comp}
  % \prereq		    {EngSistA}{EngSistB}
  % \prereq				{CEVI}		{PrincTelec}
  % \pre[blue]		{ArqComp}	{SO}{2}{2}{14}{4}{0}
  % \pre[red]		  {ArqComp}	{Telep}{-2}{2}{-12}{4}{0}
  % \pre				  {ArqComp}	{SistEmb}{0}{4}{12}{2}{0}
  % \prereq[blue] {IC}		  {MineraDados}
  % %\pre [red]		{AnaFis}	{Control}{2}{2}{10}{3}{2}
  % %\pre			    {CEVI}	  {AnaFis} {-3}{-4}{0}{-3}{0}

  % % Período 7
  % %\prereq				{AnaFis}	{ServoMec}
  % \pre			    {PrincTelec}{Telep} {3}{-3}{0}{4}{0}

  % % Período 8
  % \pre				{SO}		{ProgPara}{-3}{-4}{0}{-3}{0}
  % \pre	[blue]		{SO}		{Control}{-1}{3}{-13}{4}{0}

  % % Período 9
  % \prereq				{ProjA}		{ProjB}

  % pre de créditos
  \newcommand{\trava}[2] {
    \node [below left=0mm and -20mm of #1,text width=40pt, font=\scriptsize] (#2) {#2};
    \draw [->, thick]
    ($(#2.west)  $)  -| ($ (#1.west) -(2mm,4mm)$) -- ($(#1.west) -(0,4mm)  $);}

  \trava {SegHig}{90 cred}
  \trava {ETS}{94 cred}
  \trava {EA}{170 cred}
  \trava {EB}{170 cred}
  \trava {EC}{170 cred}
  \trava {EstSup}{170 cred}
  \trava {ProjA}{170 cred}
  \trava {ProcImag}{94 cred}

  \node at (20cm,14mm) (legenda)
  {\textbf{Código para as Disciplinas}};
  \node  [below right= 1.5mm and -36mm of legenda,fill=cbasico, draw=black,
    label=right:Básico.](basico) {};
  \node [right  of=basico,node distance=22mm,fill=cmecanica, draw=black, label=right:Mecânica] (mecan) {};
  \node [below  of=basico,node distance=5mm,fill=cambiental, draw=black, label=right:Ambiental] (ambient) {};
  \node [below  of=mecan,node distance=5mm,fill=cindustrial, draw=black, label=right:Industrial] (indust) {};
  \node [below  of=ambient,node distance=5mm,fill=celetrica, draw=black, label=right:Elétrica] (eletri) {};
  \node [below  of=indust,node distance=5mm,fill=celetronica, draw=black, label=right:Eletrônica] (prof) {};
  \node [below of=eletri,node distance=5mm,fill=cdesc, draw=black, label=right:Computacão] (especifico){};
  \node [below of=prof,node distance=5mm,fill=celetiva, draw=gray, draw=black, label=right:Eletivas] {};

\end{tikzpicture}
\multiply\value{thoras} by 5
\divide\value{thoras} by 6
\newcounter{hextensao} 
\setcounter{hextensao}{\the\value{thoras}}
\divide\value{hextensao} by 9
\addtocounter{thoras}{\the\value{hextensao}}

\vspace{-4mm}
\setlength\parindent{0pt}

\begin{multicols}{4}
  \vfill
  Os números situados na parte inferior direita e esquerda representam os créditos e carga horária, respectivamente, conferidos à disciplina.

  
  \columnbreak
  O Curso de Engenharia será integralizado em um mínimo de 10 e um máximo de 18 períodos.

  \columnbreak
  \textbf{Total de Créditos:} \the\value{cred}\\
  \addtocounter{cred}{-9}
  (\the\value{cred} em disciplinas obrigatórias e\\ 9 em disciplinas eletivas restritas)

  \columnbreak
  Total de horas do curso: \the\value{thoras}\\
  Total de horas de extensão: \the\value{hextensao}

  
\end{multicols}


\end{document}
