\documentclass[a4paper, landscape]{article}
\usepackage{prerex}
\usepackage{multicol}
%\usepackage{showframe}
\usepackage[utf8]{inputenc} % codificacao de caracteres
\usepackage{geometry}
\usepackage[portuguese]{babel}
\usepackage{gensymb}
%\usepackage{lmodern}
\usepackage{tikz}
\usepackage{xcolor}
\usetikzlibrary{calc}
\usepackage{ifthen}
\usepackage{disciplinasDB}
\tikzset{
    %Define standard arrow tip
    >=stealth,
    %Define style for boxes
    box/.style={
           rectangle,
           draw=black,
           text width=7em,
           minimum height=5em,
           text centered,
           inner sep=1mm]},
    % Define arrow style
    pil/.style={
           ->,
           thick,
           shorten <=3pt,
           shorten >=3pt,},
}
\usetikzlibrary{positioning,shapes,shadows,arrows}

\colorlet{cbasico}{yellow}
\colorlet{cdesc}{orange}
\colorlet{celetronica}{green}
\colorlet{celetrica}{blue!20!white}
\colorlet{cmecanica}{gray!20!white}
\colorlet{cambiental}{olive!20!white}
\colorlet{cindustrial}{violet!40!white}
\colorlet{celetiva}{orange!50!white}
% Define estilos de caixas de acordo com a origem da disciplina
\tikzstyle{basico}=[box, fill=cbasico]
\tikzstyle{desc}=[box, fill=cdesc]
\tikzstyle{eletronica}=[box, fill=celetronica]
\tikzstyle{eletrica}=[box, fill=celetrica]
\tikzstyle{mecanica}=[box, fill=cmecanica]
\tikzstyle{ambiental}=[box, fill=cambiental]
\tikzstyle{industrial}=[box, fill=cindustrial]
\tikzstyle{eletiva}=[box, fill=celetiva, dashed]
\tikzstyle{info}=[box, fill=yellow]

%contador de creditos totais do curso
\newcounter{cred} 
\setcounter{cred}{0}

%contador de horas totais do curso
\newcounter{thoras}
\setcounter{thoras}{0} 
\newcounter{tdisciplinas}
\setcounter{tdisciplinas}{0}

%parâmetros de cada disciplina
%{basico || eletronica || desc || eletiva} 	#1
%{Período} de 1 a 10 							#2
%{Linha no fluxograma} de 1 a 6. 				#3
%{Código da disciplina} 						#4
%{Nome da Disciplina} 							#5
%{Carga Horária} 								#6
%{Número de Créditos} 							#7
%{nome abreviado para referência} 				#8
\newcommand{\disciplina}[8] {
  \addtocounter{cred}{#7}
  \addtocounter{thoras}{#6}
  \addtocounter{tdisciplinas}{1}
  \node at ($ (#2*2.8cm-7cm,14.6-#3*2.3) $) (Item)  [#1] (#8) 
        {
            \textbf{#4}\\
            \textbf{#5}\\
            \textbf{#6} \hfill #7
        };}

\geometry{margin=0.5cm,bottom=0.5cm} 

\thispagestyle{empty}
\usepackage{fontspec}
\usepackage{helvet}


\setmainfont{Arial}
\begin{document}
\noindent
{\Large \textbf{Curso: Engenharia}}\\
{\Large \textbf{Habilitação: Computação}}  (Versão de \today)

\begin{tikzpicture}
  \foreach \x in {1,...,10} {\node at (\x*2.8-7,13.5) {\x\degree Período};}

  \footnotesize
  %\fontfamily{lms}\fontseries{l}\selectfont
  %parâmetros de cada disciplina, pode ser posto numa única linha
  %\disciplina
  %{basico || eletronica || desc || eletiva}
  %{Período} de 1 a 10
  %{Linha no fluxograma} de 1 a 8.
  %{Código da disciplina}
  %{Nome da Disciplina}
  %{Carga Horária}
  %{Número de Créditos}
  %{nome abreviado para referência} 

  %Básico
  \disciplina{basico}{1}{3}{\AlgLinCod}	{\AlgLinSName}	{\AlgLinCH}	{\AlgLinCred}	{AlgLin} % Álgebra Linear
  \disciplina{basico}{1}{4}{\CalcICod}	{\CalcISName}	{\CalcICH}	{\CalcICred}	{CalcI} % Cálculo I
  \disciplina{basico}{2}{4}{\CalcIICod}	{\CalcIISName}	{\CalcIICH}	{\CalcIICred}	{CalcII} % Cálculo II
  \disciplina{basico}{3}{4}{\CalcIIICod}	{\CalcIIISName}	{\CalcIIICH}{\CalcIIICred}	{CalcIII} % Cálculo III
  \disciplina{basico}{2}{3} {\EngComputCod}    {\EngComputSName}{\EngComputCH}{\EngComputCred}{CalcNum} % Cálculo Numérico
  \disciplina{basico}{2}{5}{\FisICod}		{\FisISName}	{\FisICH}	{\FisICred}		{FisI} % Física I
  \disciplina{basico}{2}{6}{\FisEICod}		{\FisEISName}	{\FisEICH}	{\FisEICred}		{FisEI} % Física Experimental I
  \disciplina{basico}{3}{5}{\FisIICod}	{\FisIISName}	{\FisIICH}	{\FisIICred}	{FisII} % Física II
  \disciplina{basico}{3}{6}{\FisEIICod}	{\FisEIISName}	{\FisEIICH}	{\FisEIICred}	{FisEII} % Física Experimental II
  \disciplina{basico}{4}{5}{\FisIIICod}	{\FisIIISName}	{\FisIIICH}	{\FisIIICred}	{FisIII} % Física III
  \disciplina{basico}{4}{6}{\FisEIIICod}	{\FisEIIISName}	{\FisEIIICH}	{\FisEIIICred}	{FisEIII} % Física Experimental III
  \disciplina{basico}{5}{6}{\FisIVCod}	{\FisIVSName}	{\FisIVCH}	{\FisIVCred}	{FisIV} % Física IV
  \disciplina{basico}{5}{7}{\FisEIVCod}	{\FisEIVSName}	{\FisEIVCH}	{\FisEIVCred}	{FisEIV} % Física Experimental IV
  \disciplina{basico}{3}{3}{\ProbEstCod}	{\ProbEstSName}	{\ProbEstCH}{\ProbEstCred}	{ProbEst} % Probabilidade e Estatística

  %disciplinas de outros departamentos da engenharia
  \disciplina{industrial}{10}{4}{\AdmCod}	{\AdmSName}		{\AdmCH}	{\AdmCred}		{Adm} % Administração
  \disciplina{eletrica}{5}{5}	{\CEVCod}	{\CEVSName}		{\CEVCH}	{\CEVCred}		{CEV} % Circuitos em corrente Conínua
  \disciplina{eletrica}{6}{5}{\CEVICod}	{\CEVISName}	{\CEVICH}	{\CEVICred}		{CEVI} % Circuitos em Corrente Alternada
  \disciplina{eletronica}{6}{6}{\EletICod}	{\EletISName}{\EletICH}{\EletICred}{EletI} % Eletrônica I
  \disciplina{industrial}{8}{1}{\EmpreCod}{\EmpreSName}{\EmpreCH}{\EmpreCred}{Empre} % Empreendedorismo
  \disciplina{industrial}{7}{6}{\IntEcoCod}{\IntEcoSName}{\IntEcoCH}{\IntEcoCred}{IntEco} % Introdução à Economia
  \disciplina{ambiental}{1}{5}	{\IntAmbCod}	{\IntAmbSName}{\IntAmbCH}{\IntAmbCred}{IntAmb} % Introdução à Engenharia Ambiental
  \disciplina{eletrica}{5}{4}	{\MatEleCod}	{\MatEleSName}{\MatEleCH}{\MatEleCred}{MatEle} % Materiais Elétricos
  \disciplina{eletrica}{5}{3} {\ModMatCod}	{\ModMatSName}{\ModMatCH}{\ModMatCred}{ModMat} % Modelos Matemáticos


  %Disciplinas Desc
  \disciplina{desc}{1}{1} {\AlgCompCod}	{\AlgCompSName}{\AlgCompCH}{\AlgCompCred}{AlgComp} % Algoritmos Computacionais
  \disciplina{desc}{3}{1} {\AnAlgCod}	{\AnAlgSName}{\AnAlgCH}{\AnAlgCred}{AnAlg} % Análise de Algoritmos
  \disciplina{desc}{5}{1} {\GrafosCod}	{\GrafosSName}{\GrafosCH}{\GrafosCred}{Grafos} % Teoria de Grafos e Aplicações
  \disciplina{desc}{6}{4} {\ArqCompCod}	{\ArqCompSName}{\ArqCompCH}{\ArqCompCred}{ArqComp} % Arquitetura de Computadores
  \disciplina{desc}{8}{6}{\ControlCod}	{\ControlSName}{\ControlCH}{\ControlCred}{Control} % Controle
  \disciplina{desc}{1}{2} {\EngCompSocCod}	{\EngCompSocSName}{\EngCompSocCH}{\EngCompSocCred}{ETS} % Engenharia e Sociedade
  \disciplina{desc}{6}{3} {\EngSistACod}	{\EngSistASName}{\EngSistACH}{\EngSistACred}{EngSistA} % Engenharia de Sistemas A
  \disciplina{desc}{8}{2} {\EngSistCCod}	{\EngSistCSName}{\EngSistCCH}{\EngSistCCred}{EngSistC} % Sistemas de Informação
  \disciplina{desc}{7}{3} {\ProjBDCod}	{\ProjBDSName}{\ProjBDCH}{\ProjBDCred}{EngSistB} % Engenharia de Sistemas B
  \disciplina{desc}{9}{3} {\EstSupCod}	{\EstSupSName}{\EstSupCH}{\EstSupCred}{EstSup} % Estágio Supervisionado
  \disciplina{desc}{2}{1} {\EstrInfCod}	{\EstrInfSName}{\EstrInfCH}{\EstrInfCred}{EstrInf} % Estruturas de Informação
  \disciplina{eletrica}{4}{4} {\FundICompCod}	{\FundICompSName}{\FundICompCH}{\FundICompCred}{FundIComp} % Técnicas Digitais
  \disciplina{desc}{5}{2} {\FundCompCod}	{\FundCompSName}{\FundCompCH}{\FundCompCred}{FundComp} % Fundamentos de Computadores
  \disciplina{desc}{6}{1} {\ICCod}	{\ICSName}{\ICCH}{\ICCred}{IC} % Inteligência Computacional I
  \disciplina{desc}{6}{2} {\ICIICod}	{\ICIISName}{\ICIICH}{\ICIICred}{ICII} % Inteligência Computacional II
  \disciplina{desc}{9}{4} {\InstalaCod}	{\InstalaSName}{\InstalaCH}{\InstalaCred}{Instala} % Instalações Ambiente Computacional I
  \disciplina{desc}{4}{1} {\LabProgACod}	{\LabProgASName}{\LabProgACH}{\LabProgACred}{LabProgI} % Laboratório de Programação I
  \disciplina{desc}{4}{2} {\LabProgBCod}	{\LabProgBSName}{\LabProgBCH}{\LabProgBCred}{LabProgII} % Laboratório de Programação II
  \disciplina{desc}{2}{2} {\LogProgCod}	{\LogProgSName}{\LogProgCH}{\LogProgCred}{LogProg} % Lógica de Programação
  \disciplina{desc}{7}{1}	{\MineraDadosCod}	{\MineraDadosSName}{\MineraDadosCH}{\MineraDadosCred}{MineraDados} % Mineração de Dados
  \disciplina{desc}{4}{3} {\ProcImagCod}	{\ProcImagSName}{\ProcImagCH}{\ProcImagCred}{ProcImag} % Processamento de Imagens
  \disciplina{desc}{8}{5} {\CompParalCod}	{\CompParalSName}{\CompParalCH}{\CompParalCred}{ProgPara} % Programação Paralela
  \disciplina{desc}{9}{1} {\ProjACod}	{\ProjASName}{\ProjACH}{\ProjACred}{ProjA} % Projeto A
  \disciplina{desc}{10}{1}{\ProjBCod}	{\ProjBSName}{\ProjBCH}{\ProjBCred}{ProjB} % Projeto B
  \disciplina{desc}{8}{4} {\SredesCod}	{\SredesSName}{\SredesCH}{\SredesCred}{Sredes} % Segurança em Redes de Computadores
  \disciplina{desc}{7}{5} {\ProjSOCod}	{\ProjSOSName}{\ProjSOCH}{\ProjSOCred}{SO} % Sistemas Operacionais
  \disciplina{desc}{8}{3} {\SistEmbCod}	{\SistEmbSName}{\SistEmbCH}{\SistEmbCred}{SistEmb} % Sistemas Embutidos
  \disciplina{desc}{7}{4} {\TelepCod}	{\TelepSName}{\TelepCH}{\TelepCred}{Telep} % Redes
  \disciplina{desc}{7}{2} {\TeoCompCod}	{\TeoCompSName}{\TeoCompCH}{\TeoCompCred}{Comp} % Teoria da Compiladores
  \disciplina{desc}{3}{2} {\ExtCod}	{\ExtSName}{\ExtCH}{\ExtCred}{Ext} % Projetos de Extensão

  %eletivas
  \disciplina{eletiva}{9}{2} {\EletACod}	{Eletiva Restrita}{\EletACH}{\EletACred}{EA}
  \disciplina{eletiva}{10}{2}{\EletACod}	{Eletiva Restrita}{\EletACH}{\EletACred}{EB}
  \disciplina{eletiva}{10}{3}{\EletACod}	{Eletiva Restrita}{\EletACH}{\EletACred}{EC}

  %ligação com pre-requititos
  \newcommand{\prereq}[3][black] { % \prereq[cor]{disciplina}{pre}
    \draw [->, thick, #1](#2.east)--(#3.west);} % Uma linha direta para o próximo bloco

  \newcommand{\pre}[8][black] {
    \draw [->, thick, #1]
    ($(#2.east) + (0 mm,#4mm) $) -| % #4 distância do centro de saída do bloco prérequisito
    \ifnum#6=0
      ($ (#3.west)+(#5mm,-#7mm)$) --
      ($(#3.west)-(0,#7mm)$);
      \else
      ($(#2.east) + (#5mm,#6mm) $) -| % #5 até onde vai a linha horizontal #6 até onde vai a linha vertical
      ($(#3.west) - (#7mm,#8mm)$) -- ($(#3.west) - (0,#8mm)$);
    \fi
  }

  %Período 1
  \prereq			  {AlgComp}	{EstrInf}
  \pre  [olive] {AlgComp}	{CalcNum}   {-4}{-4}{0}{-3}{0}
  \pre	[red]   {AlgComp}	{AnAlg}     {-1}{5}{-8}{4}{0}
  \pre	[blue]  {AlgComp}	{FundComp}  {-2}{4}{-15}{2}{-4}
  \pre          {LogProg} {AnAlg}     {2}{-3}{0}{2}{0}
  \pre	[blue]	{CalcI}		{CalcNum}   {5}{-4}{0}{3}{0}
  \prereq				{CalcI}		{CalcII}
  \pre	[red]	  {CalcI}	  {ProbEst}   {2}{4}{14}{2}{0}
  \pre	[blue]	{AlgLin}  {ProcImag}  {5}{1}{10}{2}{-2}

  % Período 2
  \pre          {EstrInf}	{LabProgI}  {2}{3}{8}{2}{-3}
  \pre  [blue]  {EstrInf}	{Grafos}    {4}{1}{9}{2}{-3}
  \pre  [teal] 	{EstrInf}	{LabProgII} {-3}{1}{-13}{4}{-4}
  \prereq				{CalcII}	{CalcIII}
  \prereq				{FisI}		{FisII}
  \prereq				{FisEI}		{FisEII}

  %\pre	[blue]  {FisI}		{MecTec}    {1}{-3}{0}{-1}{0}
  %\pre  [blue]	{FisII}		{FenTran}   {1}{-2}{0}{2}{0}
  \pre  [blue]   {CalcIII}	{ModMat}   {-1}{1}{-13}{3}{-1}
  %\pre			    {AlgLin}	{PO}        {-2}{2}{-12}{2}{0}
  \pre  [olive]  {LogProg} {LabProgI}  {0}{5}{15}{4}{0}
  \pre          {LogProg} {LabProgII} {-2}{3}{-10}{3}{0}


  % %Período 3
  \pre[red]     {AnAlg}		{Comp}       {-4}{4}{-13}{2}{0}
  \pre	[blue]  {AnAlg}	  {Grafos}     {-2}{5}{-8}{2}{3}
  \prereq				{FisII}		{FisIII}
  \prereq				{FisEII}  {FisEIII}
  %\prereq				{MecTec}	{ResMat}
  \prereq [red]	{ProbEst}	{ProcImag}
  \pre  	      {LabProgI}{IC}        {-1}{3}{-10}{3}{0}
  \pre[teal]   	{LabProgI}{ICII}        {-3}{2}{-12}{5}{-2}

  % % Período 4
  \pre[blue]    {LabProgII}{EngSistA} {3}{1}{-9}{5}{-2}
  \pre				{FisIII}	{FisIV} {-3}{-4}{0}{0}{-4}
  \pre [red]	  {FisIII}	{MatEle}  {3}{-2}{0}{4}{0}
  \pre	      {FundIComp} {FundComp}{5}{-4}{0}{3}{0}
  \prereq [teal] 	{FisIII}	{CEV}
  \pre [red]		{ModMat}	{CEVI}    {0}{-5}{0}{-4}{0}
  \pre	  {CEV}	{EletI}    {-3}{-4}{0}{0}{-4}
  \prereq[blue]	  {CEV}		  {CEVI}


  % Período 5
  \pre[purple]	{FundComp}{ArqComp} {-3}{-3}{0}{-3}{2}
  \pre			{FisEIII}		{FisEIV} {-3}{-4}{0}{0}{-4}

  % Período 6
  \prereq		    {EngSistA}{EngSistB}
  \pre[blue]		{ArqComp}	{SO}{-3}{-3}{0}{-3}{1}
  \prereq[red]		  {ArqComp}	{Telep}
  \pre				  {ArqComp}	{SistEmb}{3}{3}{10}{4}{0}
  \prereq[teal]  {IC}		  {MineraDados}
  \pre[black]    {CEVI}{Instala} {3}{1}{10}{4}{-2}

  % % Período 7
  \prereq				{SO}		{ProgPara}
  \prereq[red] {Telep} {Sredes}
  \pre   {EngSistB}{EngSistC}{3}{-2}{0}{-1}{0}

  % % Período 8

  \pre	[blue]		{SO}		{Control}{-1}{-4}{0}{-1}{0}

  % % Período 9
  \prereq				{ProjA}		{ProjB}

  % pre de créditos
  \newcommand{\trava}[2] {
    \node [below left=0mm and -20mm of #1,text width=40pt, font=\scriptsize] (#2) {#2};
    \draw [->, thick]
    ($(#2.west)  $)  -| ($ (#1.west) -(2mm,4mm)$) -- ($(#1.west) -(0,4mm)  $);}

  %\trava {SegHig}{50 cred}
  %\trava {ETS}{94 cred}
  \trava {EA}{150 cred}
  \trava {EB}{150 cred}
  \trava {EC}{150 cred}
  \trava {EstSup}{150 cred}
  \trava {ProjA}{150 cred}


  \node at (20cm,14mm) (legenda)
  {\textbf{Código para as Disciplinas}};
  \node  [below right= 1.5mm and -36mm of legenda,fill=cbasico, draw=black,
    label=right:Básico.](basico) {};
  \node [right  of=basico,node distance=22mm,fill=cmecanica, draw=black, label=right:Mecânica] (mecan) {};
  \node [below  of=basico,node distance=5mm,fill=cambiental, draw=black, label=right:Ambiental] (ambient) {};
  \node [below  of=mecan,node distance=5mm,fill=cindustrial, draw=black, label=right:Industrial] (indust) {};
  \node [below  of=ambient,node distance=5mm,fill=celetrica, draw=black, label=right:Elétrica] (eletri) {};
  \node [below  of=indust,node distance=5mm,fill=celetronica, draw=black, label=right:Eletrônica] (prof) {};
  \node [below of=eletri,node distance=5mm,fill=cdesc, draw=black, label=right:Computacão] (especifico){};
  \node [below of=prof,node distance=5mm,fill=celetiva, draw=gray, draw=black, label=right:Eletivas] {};

\end{tikzpicture}
\newcounter{haula}
\addtocounter{thoras}{-30}
\setcounter{haula}{\the\value{thoras}}
%\multiply\value{thoras} by 5
%\divide\value{thoras} by 6
\newcounter{hextensao}
\setcounter{hextensao}{\the\value{thoras}}
\divide\value{hextensao} by 9
\addtocounter{thoras}{\the\value{hextensao}}
\addtocounter{hextensao}{1} % Arredonda para cima

\vspace{-4mm}
\setlength\parindent{0pt}

\begin{multicols}{4}
  \vfill
  Os números situados na parte inferior direita e esquerda representam os créditos e carga horária, respectivamente, conferidos à disciplina.


  \columnbreak
  O Curso de Engenharia será integralizado em um mínimo de 10 e um máximo de 18 períodos.

  \columnbreak
  \textbf{Total de Créditos:} \the\value{cred}\\
  \the\value{tdisciplinas} disciplinas\\
  \addtocounter{cred}{-14}
  (\the\value{cred} créditos em disciplinas obrigatórias\\ 2 créditos em disciplina de extensão e\\ 12 créditos em disciplinas eletivas restritas)

  \columnbreak
  %Total de horas-aula do curso: \the\value{haula}\\
  \addtocounter{thoras}{1} % Arredonda para cima
  Total de horas do curso: \the\value{thoras}\\
  Total de horas do curso (sem ext.): \the\value{haula}\\
  Total de horas de extensão: \the\value{hextensao}


\end{multicols}


\end{document}
