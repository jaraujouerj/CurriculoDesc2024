\section{Equivalência com o Curso Anterior}
O curso de Engenharia de Computação ora proposto substituirá o curso de Engenharia Elétrica com ênfase em Sistemas e Computação e, na hipótese de algum aluno desejar migrar do curso antigo para este novo, será possível dispensar disciplinas do novo currículo iguais ou equivalentes às disciplinas do curso antigo.

A tabela \ref{DiscIguais} mostra as disciplinas que são equivalentes entre o novo curso de Engenharia de Computação e o antigo curso de Engenharia Elétrica com ênfase em Sistemas e Computação. Por outro lado, a tabela \ref{DiscSemEqui} lista as disciplinas do novo currículo que não possuem equivalência direta com as disciplinas do curso anterior.

\rowcolors{1}{gray!5}{white}
\begin{table}[ht]
	\caption{Disciplinas Equivalentes}
	\label{DiscIguais}
	\centering
	\renewcommand{\arraystretch}{1.5}
	\begin{tabularx}{\textwidth}{|X|l|}
		\showrowcolors
		\hline
		{\textbf{Disciplina}} & \textbf{Código} \\
		\hline
		\Adm                  & \AdmCod         \\ % Administração
		\AlgLin               & \AlgLinCod      \\ % Álgebra Linear
		\CEV                  & \CEVCod         \\ % Circuitos em Corrente Contínua
		\EletI                & \EletICod       \\ % Eletrônica I
		\FisI                 & \FisICod        \\ % Física I
		\FisII                & \FisIICod       \\ % Física II
		\FisIII               & \FisIIICod      \\ % Física III
		\FisIV                & \FisIVCod       \\ % Física IV
		\IntEco               & \IntEcoCod      \\ % Macroeconomia
		\IntAmb               & \IntAmbCod      \\ % Introdução à Engenharia Ambiental
		\MatEle               & \MatEleCod      \\ % Materiais Elétricos e Magnéticos
		\ModMat               & \ModMatCod      \\ % Sinais e Sistemas
		\ProbEst              & \ProbEstCod     \\ % Probabilidade e Estatística
		\ProjA                & \ProjACod       \\ % Metodologia Científica
		\ProjB                & \ProjBCod       \\ % Projeto de Graduação XI
		\hline
	\end{tabularx}
\end{table}

\rowcolors{1}{gray!5}{white}
\begin{table}
	\centering
	\renewcommand{\arraystretch}{1.5}
	\caption{Equivalências no novo currículo}
	\label{equivalencias}
	\begin{tabularx}{\textwidth}{|X||X|l|}
		\hline
		{\textbf{Currículo Novo}}	& \textbf{Equivalente no Currículo Antigo} 	& \textbf{Código}\\
		\hline
		\AlgComp	& Algoritmos Computacionais				    	& FEN06-03559       \\
		\AnAlg      & Análise de Algoritmos                       & FEN06-03713       \\
		\ArqComp    & Arquitetura de Computadores I               & FEN06-04119       \\
		\ProjSO     & Arquitetura de Sistemas Operacionais        & FEN06-04664       \\
		\CalcI      & Cálculo Diferencial e Integral I            & IME01-00508       \\
		\CalcII     & Cálculo Diferencial e Integral II           & IME01-00854       \\
		\CalcIII    & Cálculo Diferencial e Integral III          & IME01-03646       \\
		\EngComput  & Cálculo Numérico IV                         & IME04-04541       \\
		\LabProgB   & Carac. das Linguagens de Prog. I            & FEN06-03980       \\
		\CEVI       & Circuitos Elétricos IV                      & FEN04-05222       \\
		\Control    & Controle de Processos por Comp.             & FEN06-05080       \\
		\EngSistA   & Engenharia de Sistemas A					& FEN06-04243       \\
		\ProjBD     & Engenharia de Sistemas B                    & FEN06-04314       \\
		\EstrInf    & Estruturas de Informação I                  & FEN06-03648       \\
		\FundComp   & Fundamentos de Comp. Digitais I    & FEN06-03787  				\\
		\LabProgA   & Laboratório de Programação I        & FEN06-04049                 \\
		\Telep      & Teleproc. e Redes de Computadores   & FEN06-04718                 \\
		\TeoComp                               & Teoria de Compiladores            & FEN06-04516                 \\
		Eletivas Restritas & Tóp. Especiais em Eng. de Sistemas e Computação A, B ou C & \parbox[t]{2cm}{FEN06-04889                                          \\FEN06-04939\\FEN06-04990}  \\
		\hline
	\end{tabularx}
\end{table}

\begin{table}
	\centering
	\renewcommand{\arraystretch}{1.5}
	\caption{Disciplinas sem Equivalências}
	\label{DiscSemEqui}
	\begin{tabularx}{\textwidth}{|X|}
		\hline
		{\textbf{Disciplinas do Novo Currículo sem Equivalência}} \\
		\hline
		\LogProg                                                  \\
		\IC                                                       \\
		\EngCompSoc                                               \\
		\MineraDados                                              \\
		\SistEmb                                                  \\
		\ProcImag                                                 \\
		\CompParal                                                \\
		\EstSup                                                   \\
		\hline
	\end{tabularx}
\end{table}
