%----------------------------------------------------------------------------------------
%	Latex para ementa tipo dep da uerj
%----------------------------------------------------------------------------------------
%apagar .aux e executar biber ementalatex para obter bibliografia atualizada
\documentclass{ementa} 

\begin{document}

\ndisciplina 	{\EngCompSoc}
\nomeprofessor	{\araujo}
\matriculaprofessor{\mataraujo}
\chta 		{\EngCompSocCH}%carga horária total aluno
\codigo		{FEN06-xxxxx}%Código da disciplina 

\objetivos	{Apresentar as bases histórica, teórica e conceitual relativas à engenharia e a tecnologia, bem como sua inter-relação com a ciência, de modo a proporcionar aos estudantes elementos de compreensão e de reflexão sobre a importância da engenharia para a sociedade, no que tange ao desenvolvimento econômico e social, especialmente, na atuação profissional do engenheiro de computação.
}

\ementa	{Bases histórica, teórica e conceitual relativas à engenharia e a tecnologia. A engenharia e a metodologia científica. Comunicação e Expressão. A ciência e a tecnologia e suas inter-relações com a engenharia. Impactos políticos, estratégicos e econômicos da tecnologia na indústria. Tecnologias industriais básicas e propriedade intelectual. Prospecção tecnológica, transferência de tecnologia e cerceamento tecnológico. Inovação e empreendedorismo na formação do engenheiro. Pesquisa e o desenvolvimento em engenharia. O projeto de engenharia. A função social do engenheiro. A ética em engenharia. O engenheiro de computação e sua inserção na indústria e na área de serviços.
}

\credteorica		{2} %

\formementa{EngCompSoc}
\end{document}
