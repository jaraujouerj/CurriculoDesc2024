%----------------------------------------------------------------------------------------
%	Latex para ementa tipo dep da uerj
%----------------------------------------------------------------------------------------
%apagar .aux e executar biber ementalatex para obter bibliografia atualizada
\documentclass{ementa} 

\begin{document}

\ndisciplina 	{\EngCompSoc}
\nomeprofessor	{\araujo}
\objetivos	{Apresentar as bases histórica, teórica e conceitual relativas à engenharia de computação e à tecnologia, bem como sua inter-relação com a ciência, de modo a proporcionar aos estudantes elementos de compreensão e de reflexão sobre a importância da engenharia de computação para a sociedade, no que tange ao desenvolvimento econômico/social, e seu impacto ambiental, especialmente, na atuação profissional do engenheiro de computação.
}

\ementa	{Bases histórica, teórica e conceitual da engenharia de computação. A engenharia e a metodologia científica. A ciência e a tecnologia e suas inter-relações com a engenharia. Impactos políticos, estratégicos, econômicos e ambientais da tecnologia na indústria. Sustentabilidade na engenharia de computação: desafios e soluções. Impactos ambientais do ciclo de vida de hardware/software. Computação verde: eficiência energética e a redução de emissões de carbono. Casos de inovação sustentável. Tecnologias industriais básicas e propriedade intelectual. Prospecção tecnológica, transferência de tecnologia e cerceamento tecnológico. Inovação e empreendedorismo na formação do engenheiro. Pesquisa e o desenvolvimento em engenharia. O projeto de engenharia. A função social do engenheiro. A ética em engenharia. O engenheiro de computação e sua inserção na indústria e na área de serviços.
}

\credteorica		{4} %

\formementa{EngCompSoc}
\end{document}
