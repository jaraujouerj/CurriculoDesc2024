%----------------------------------------------------------------------------------------
%	Latex para ementa tipo dep da uerj
%----------------------------------------------------------------------------------------
%apagar .aux e executar biber ementalatex para obter bibliografia atualizada
\documentclass{ementa} % Use A4 paper with a 12pt font size 

\begin{document}

\ndisciplina 	{\TecProgOtim}
\nomeprofessor	{\simone}
\matriculaprofessor{\matsimone}
\chta 		{\TecProgOtimCH}%carga horária total aluno
\codigo		{\TecProgOtim}%Código da disciplina 


\objetivos	{Ao final do período, o aluno deverá assimilar e compreender as técnicas algorítmicas de Otimização Combinatória com o objetivo de encontrar soluções ótimas ou quase ótimas de maneira eficiente e eficaz. 
}
\ementa	{Introdução aos Princípios de Otimização Combinatória; Principais problemas em P, NP, NP-completo e NP-Difíceis. Programação Linear e Não Linear; Programação Dinâmica; Algoritmos Gulosos; Algoritmos de Backtracking; Algoritmos de Branch and Bound. Algoritmos de Branch and Cut; Algoritmos de Branch and Price; Heurísticas e Metaheurísticas.}

\preum		{\LabProgA} %
\predois	{\AnAlg} %
\codpreum		{\LabProgACod} %
\codpredois		{\AnAlgCod} %
% \semipresencial
\credteorica		{4} %

\naoobrigatoria{}
\eletivarestrita{}
\formementa{TecnicasOC}
\end{document}
