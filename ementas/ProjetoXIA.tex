%----------------------------------------------------------------------------------------
%	Latex para ementa tipo dep da uerj
%----------------------------------------------------------------------------------------
%apagar .aux e executar biber ementalatex para obter bibliografia atualizada
\documentclass{ementa} % Use A4 paper with a 12pt font size 

\begin{document}
\ndisciplina 		{\ProjA}
\nomeprofessor		{\araujo}
\ementa	{
	\begin{itemize}
		\item Introdução à Ciência e ao Método Científico: Conceito de ciência, conhecimento e pesquisa. História e evolução do método científico. Tipos de conhecimento e seus fundamentos. Definição do projeto, de acordo com as pretensões profissionais do aluno.
		\item Tipos de Pesquisa: Pesquisa básica e aplicada. Pesquisa qualitativa e quantitativa. Estudos exploratórios, descritivos e explicativos.
		\item Planejamento da Pesquisa: Delimitação do problema e formulação de hipóteses. Objetivos gerais e específicos.
		      Justificativa e relevância da pesquisa.
		\item Revisão de Literatura: Identificação e seleção de fontes. Leitura crítica e elaboração de fichamentos. Uso de  ferramentas de busca e gerenciadores de referências.
		\item Técnicas de Coleta de Dados: Observação, entrevistas e questionários. Experimentos e estudos de caso. Fontes documentais e análise bibliográfica.
		\item Análise e Interpretação de Dados: Métodos estatísticos básicos. Análise de conteúdo e análise temática. Apresentação de resultados (gráficos, tabelas e diagramas).
		\item Comunicação Científica: Estrutura de artigos, monografias e relatórios. Normas técnicas (ex.: ABNT, APA).
		      Ética na pesquisa: plágio, confidencialidade e responsabilidade científica.
		\item Elaboração de Projetos de Pesquisa: Estrutura do projeto de pesquisa. Cronograma e viabilidade. Avaliação de propostas de pesquisa. Elaboração e qualificação do anteprojeto, em conjunto com o cronograma proposto.
		\item Seminários sobre o andamento do projeto.
	\end{itemize}
}
\objetivos			{
	Desenvolver nos estudantes as habilidades necessárias para compreender, planejar e executar pesquisas científicas, bem como comunicar os resultados de maneira clara e objetiva, considerando aspectos éticos e metodológicos da ciência. Induzir o aluno a iniciar e desenvolver um projeto de Engenharia de Computação, correlacionando e consolidando os conhecimentos adquiridos no curso, bem como estimulando sua capacidade de autocrítica e autoaprendizado.
}
\travadecreditos   {150} % trava de créditos
\credteorica		{2} %
\formementa{MetodologiaCien}
\end{document}
