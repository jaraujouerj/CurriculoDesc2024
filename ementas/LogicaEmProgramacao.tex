%----------------------------------------------------------------------------------------
%	Latex para ementa tipo dep da uerj
%----------------------------------------------------------------------------------------
%apagar .aux e executar biber ementalatex para obter bibliografia atualizada
\documentclass{ementa} % Use A4 paper with a 12pt font size 

\begin{document}

\ndisciplina 	{\LogProg}
\nomeprofessor	{\felipe}
\matriculaprofessor{\matfelipe}
\chta 		{\LogProgCH} % carga horária total aluno
\codigo		{\LogProgCod} % Código da disciplina 
\objetivos	{Ao final do período, o aluno deverá assimilar noções de lógica de primeira ordem e ser expostos a alguns formalismos usados para representação e raciocínio do conhecimento em computação.
}
\ementa	{Lógica proposicional: sintaxe, semântica, complexidade, sistemas dedutivos (método de tableaux e de resolução). Lógica de primeira ordem: sintaxe, sistemas dedutivos (método de tableaux e de resolução), cláusulas de Horn, semântica (estruturas de primeira  ordem).
    Lógicas descritivas: Introdução à representação e raciocínio do conhecimento; introdução a ontologias e lógicas descritivas; introdução a modelagem e raciocínio com a lógica descritiva ALC; aplicação de lógica descritiva usando PROTEGÉ. Introdução ao PROLOG: linguagem, árvore de prova, recursão;  estruturas de dados em PROLOG (listas , árvores, grafos);  aplicação de PROLOG a problemas clássicos de Inteligência Artificial (busca automática, programação não determinística, geração e teste).
}
\credteorica	{2} %
\credlab	{2} %

\formementa{LogProg}
\end{document}
