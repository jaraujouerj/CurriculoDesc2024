%----------------------------------------------------------------------------------------
%	Latex para ementa tipo dep da uerj
%----------------------------------------------------------------------------------------
%apagar .aux e executar biber ementalatex para obter bibliografia atualizada
\documentclass{ementa} % Use A4 paper with a 12pt font size 

\begin{document}

\nomeprofessor{\simone}
\matriculaprofessor{\matsimone}

\ndisciplina {\Grafos}

\chta	{\GrafosCH} % carga horária total aluno
\codigo	{\GrafosCod} % Código da disciplina 

\disciplinaisolada
\objetivos	{
    Ao final do período, o aluno deverá ter assimilado conhecimentos teóricos e práticos envolvendo a teoria dos grafos e suas aplicações, tornando o aluno apto a aplicar e resolver problemas reais modelados por meio dos grafos.
}

\ementa{
    Noções e definições básicas em Teoria dos grafos. Características em grafos: subgrafos, isomorfismo, grafos bipartidos, grafos completos e grafos complementar. Dígrafos ou grafos orientados: Principais definições; arcos, grau de entrada e de saída, poço e sumidouro, grafos direcionados acíclicos. Representação de grafos: estruturas de dados para representar os grafos; matriz e lista de adjacência e matriz de incidência. Busca em grafos: busca em largura e busca em profundidade. Aplicações da busca em grafos como a ordenação topológica. Conexividade em grafos: Principais teoremas de componentes conexos, $k$-conexidade por vértices e arestas; pontes, articulação e cortes, grafos fortemente conexos; Teorema de Menger. Caminhos em Grafos: Principais Definições; Principais teoremas em caminhos; grafos eulerianos e hamiltonianos; Problema do caminhos mínimos;Algoritmos de Camínhos mínimos: Dikjstra, Bellman Ford e Floyd-Warshall. Árvores em Grafos: Definição de árvores e florestas; Árvores geradora mínima (AGM); Propriedades dos Cortes em AGM; Algoritmos de AGM; Algoritmo de Kruskal e Algoritmo de Prim.    Grafos Planares: Teorema de Euler; Teorema de Kuratowski; Teorema das Quatro Cores;  Coloração de Vértices; Número cromático. Problemas difíceis em grafos: coloração de vértices, circuitos Hamiltonianos e eulerianos; Problema do caixeiro viajante; Problema da Clique em grafos e do conjunto independente.

}
\preum	  {\EstrInf} %
\predois  {\AnAlg} %
\codpreum	{\EstrInfCod} %
\codpredois	{\AnAlgCod} %
\credteorica	{2} %
\credlab	    {2} %


\formementa{TeoriaGrafos}


\end{document}
