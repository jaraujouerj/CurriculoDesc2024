%----------------------------------------------------------------------------------------
%	Latex para ementa tipo dep da uerj
%----------------------------------------------------------------------------------------
%apagar .aux e executar biber ementalatex para obter bibliografia atualizada
\documentclass{ementa} % Use A4 paper with a 12pt font size 

\begin{document}

\ndisciplina 	{\ProjSO}
\nomeprofessor	{\rafaela}
\objetivos	{Os principais objetivos da disciplina são fornecer aos alunos os fundamentos e os detalhes do projeto de um sistema operacional. Ao final da disciplina, espera-se que os alunos sejam capazes de: entender a arquitetura e o funcionamento geral dos principais componentes de um sistema operacional; descrever os problemas e as respectivas soluções teóricas que são normalmente encontrados no projeto de um sistema operacional; projetar e implementar soluções para problemas de programação concorrente/paralela utilizando threads e processos.
}
\ementa	{Introdução: histórico; funções e estrutura de um sistema operacional; estruturas de hardware do computador no qual é executado um sistema operacional; multiprogramação. Processos: conceitos básicos de processos e \textit{threads}; mecanismos de comunicação e sincronização; escalonamento. Gerência de memória: partições fixas e variáveis; \textit{swapping}; memória virtual. Sistemas de Arquivo: organização; métodos de acesso. Sistemas de E/S.
}

\preum		{\ArqComp} %
\credteorica	{2} %
\credlab		{2} %

\formementa{ProjSO}
\end{document}
