%----------------------------------------------------------------------------------------
%	Latex para ementa tipo dep da uerj
%----------------------------------------------------------------------------------------
%apagar .aux e executar biber ementalatex para obter bibliografia atualizada
\documentclass{ementa} % Use A4 paper with a 12pt font size 

\begin{document}

\ndisciplina 	{\ProjSO}
\nomeprofessor	{\rafaela}
\matriculaprofessor{\matrafaela}
\chta 		{\ProjSOCH}%carga horária total aluno
\codigo		{\ProjSOCod}%Código da disciplina 
\objetivos	{Os principais objetivos da disciplina são fornecer aos alunos os fundamentos e os detalhes do projeto de um sistema operacional. Ao final da disciplina, espera-se que os alunos sejam capazes de: entender a arquitetura e o funcionamento geral dos principais componentes de um sistema operacional; descrever os problemas e as respectivas soluções teóricas que são normalmente encontrados no projeto de um sistema operacional; projetar e implementar soluções para problemas de programação concorrente/paralela utilizando threads e processos.
}
\ementa	{Funções e estrutura de um sistema operacional; estruturas de hardware do computador no qual executa um sistema operacional; multiprogramação. Processos: conceitos básicos de processos e threads; mecanismos de comunicação e sincronização; escalonamento. Gerenciamento de memória: partições fixas e variáveis; swapping; memória virtual. Sistemas de Arquivo: organização; métodos de acesso. Sistemas de E/S.
}
\semipresencial
\disciplinaisolada
\preum		{\ArqComp} %
\codpreum		{\ArqCompCod} %
%\predois		{Laboratório de Programação B} %
%\codpredois		{FEN06-XXXXX} %

\credteorica		{4} %

\formementa{ArqSisOp}
\end{document}
