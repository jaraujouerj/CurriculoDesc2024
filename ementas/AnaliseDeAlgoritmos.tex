%----------------------------------------------------------------------------------------
%	Latex para ementa tipo dep da uerj
%----------------------------------------------------------------------------------------
%apagar .aux e executar biber ementalatex para obter bibliografia atualizada
\documentclass{ementa} % Use A4 paper with a 12pt font size 

\begin{document}

\ndisciplina {\AnAlg}
\nomeprofessor	{\giomar}

\ementa{
    Definições e conceitos básicos: notação assintótica e comparação assintótica de funções, complexidades de melhor, médio e pior caso.
    Análises de complexidade de tempo em algoritmos de ordenação por comparações.
    Divisão e conquista: definição e aplicabilidade, recursão e recorrência, algoritmos de ordenação, multiplicação de matrizes e teorema mestre.
    Programação dinâmica: Problema da mochila e Subcadeia comum máxima.
    Algoritmos gulosos: definição e aplicabilidade, problema da árvore geradora mínima, problema da mochila fracionária e códigos de Huffman.
    Teoria da complexidade: problemas de decisão, transformações polinomiais, classe P, algoritmos não determinísticos, classes NP, NP-completo e NP-difíceis.

}
\objetivos	{
    Ao final do período, o aluno deverá ser capaz de analisar, avaliar e comparar a eficiência computacional de algoritmos em termos de tempo e recursos computacionais.
    O aluno estará apto a projetar algoritmos eficientes, quando possível, e será capaz de identificar a estratégia mais indicada em cada caso.

}

\preum		    {\AlgComp} % Algoritmos Computacionais
\predois		{\LogProg} % Lógica em Programação
\credteorica	{2} %
\credlab		{2} %

\formementa{AnaAlg}
\end{document}
