%----------------------------------------------------------------------------------------
%	Latex para ementa tipo dep da uerj
%----------------------------------------------------------------------------------------
%apagar .aux e executar biber ementalatex para obter bibliografia atualizada
\documentclass{ementa} % Use A4 paper with a 12pt font size 

\begin{document}

\ndisciplina 	{\ProcImag}
\nomeprofessor	{\giomar}
\matriculaprofessor{\matgiomar}
\chta 		{\ProcImagCH}%carga horária total aluno
\codigo		{\ProcImagCod}%Código da disciplina 

\objetivos{Ao final do período, o aluno deverá ter compreendido os fundamentos do processamento de sinais e imagens. Habilitar o aluno na identificação do tipo de processamento mais adequado à cada situação, pela compreensão dos algoritmos, tanto no aspecto teórico como no prático
}

\ementa{Introdução a sistemas e processamento de sinais; Classificação dos sinais; Análise de Sinais; Fundamentos de imagens; Transformação de Imagens; Realce no Domínio Espacial; Realce no Domínio da Frequência; Restauração de Imagens; Processamento Morfológico; Segmentação;
Representação e Descrição; Reconhecimento de Padrões em Imagens.
}

\preum		    {\AlgLin} % Álgebra Linear
\codpreum		{\AlgLinCod} % Código do pré-requisito 1
\predois		{\ProbEst} % Pré-requisito 2
\codpredois	    {\ProbEstCod} % Código do pré-requisito 2
\credteorica	{2} %
\credlab 		{2}
\semipresencial

\formementa{ProcSinImag}
\end{document}