%----------------------------------------------------------------------------------------
%	Latex para ementa tipo dep da uerj
%----------------------------------------------------------------------------------------
%apagar .aux e executar biber ementalatex para obter bibliografia atualizada
\documentclass{ementa} % Use A4 paper with a 12pt font size 

\begin{document}

\ndisciplina 	{\EletArq}
\nomeprofessor	{Cristiana Barbosa Bentes}
\matriculaprofessor{30729-8}
\chta 		{\EletArqCH}%carga horária total aluno
\naoobrigatoria
\eletivarestrita {}
\objetivos	{Ao final do período, o aluno deverá estar apto a compreender os conceitos e técnicas avançadas de arquiteturas paralelas e programação de alto desempenho.
}
\ementa	{Níveis de paralelismo em uma arquitetura. Nivel de instrução (arquiteturas pipelined, super escalares e VLIW). Nível de threads (SMT, multicore). Nível de processos (Computadores paralelos). Arquiteturas heterogêneas. Programação em arquiteturas heterogêneas.
}
\preum		{Computação Paralela e Distribuída} %
\codpreum		{FEN 06-XXXXX} %

\travadecreditos   {170} % trava de créditos
\credteorica		{2} %
\credlab		{2} %

\formementa{Eletiva4_CompAltoDesemp}
\end{document}
