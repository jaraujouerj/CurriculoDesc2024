%apagar .aux e executar biber ementalatex para obter bibliografia atualizada
\documentclass{ementa} % Use A4 paper with a 12pt font size 
\begin{document}

\ndisciplina 		{\CompParal}
\nomeprofessor	    {\rafaela}
\matriculaprofessor {\matrafaela}
\chta 		        {\CompParalCH} %carga horária total aluno
\objetivos	        {Os objetivos da disciplina são apresentar conceitos e técnicas de programação paralela e distribuída voltada ao alto desempenho. Ao final da disciplina, espera-se que os alunos sejam capazes de entender as diversas arquiteturas paralelas modernas, conhecer os modelos de programação paralela, desenvolver programas paralelos nos modelos de memória compartilhada e troca de mensagens.
}
\ementa	            {Computação de alto desempenho: CPUs multinucleadas (multi-core); programação de propósito geral em unidades de processamento gráfico (GPGPU); computadores paralelos; multiprocessadores; multicomputadores; aglomerados computacionais (clusters) e grades computacionais (grids); computação em nuvem. Conceitos de Sistemas Distribuídos. Programação Paralela: desenvolvimento de programas paralelos com threads e memória compartilhada; desenvolvimento de programas distribuídos com troca de mensagens. Ambientes bibliotecas para programação paralela e distribuída.
}

\preum		        {Projeto de Sistemas Operacionais} %
\codpreum		    {FEN06-XXXXX} %
\credteorica		    {2} %
\credlab		        {2} %

\formementa{CompParal}

\end{document}
