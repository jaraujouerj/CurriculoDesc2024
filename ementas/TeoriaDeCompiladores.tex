%----------------------------------------------------------------------------------------
%	Latex para ementa tipo dep da uerj
%----------------------------------------------------------------------------------------
%apagar .aux e executar biber ementalatex para obter bibliografia atualizada
\documentclass{ementa} % Use A4 paper with a 12pt font size 

\begin{document}
\ndisciplina 	{\TeoComp}
\nomeprofessor	{\gabriel}
\matriculaprofessor{\matgabriel}
\chta 		{\TeoCompCH}%carga horária total aluno
\codigo		{\TeoCompCod}%Código da disciplina 
\objetivos	{Ao final do período, o aluno deverá ter assimilado as bases formais da Ciência da Computação, através do estudo dos elementos da Teoria da Computação.  O aluno também deverá ter assimilado as técnicas de construção de compiladores e ser capaz de transcender o seu uso para problemas extra compilação.
}
\ementa	{Introdução à teoria de linguagens formais, gramáticas, linguagens e expressões regulares.  Reconhecedores, autômatos finitos, autômatos de pilha e Máquinas de Turing. A tese de Church-Turing, Máquinas de Turing universais, Problemas indecidíveis sobre máquinas de Turing, Complexidade computacional. Análise léxica, sintática e semântica. Código intermediário e objeto, geração e otimização de código. Ferramentas para a implementação de compiladores. Aplicação de ferramentas de compilação em problemas de caráter geral.
}

\preum		    {\AnAlg} %
\codpreum		{\AnAlgCod} %
\credteorica	{5} %
\formementa{TeoCompil}
\end{document}
