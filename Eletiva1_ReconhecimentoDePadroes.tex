%----------------------------------------------------------------------------------------
%	Latex para ementa tipo dep da uerj
%----------------------------------------------------------------------------------------
%apagar .aux e executar biber ementalatex para obter bibliografia atualizada
\documentclass{ementa} % Use A4 paper with a 12pt font size 
\begin{document}

\ndisciplina 		{\EletRec}
\nomeprofessor		{Orlando Bernardo Filho}
\matriculaprofessor	{30802-3}
\chta 				{\EletRecCH h}%carga horária total aluno
\preum				{\AlgLin} %
\codpreum			{\AlgLinCod} %
\predois			{\ProbEst} %
\codpredois			{\ProbEstCod} %
\travadecreditos   	{170} % trava de créditos
\credteorica		{2} %
\credlab			{2}
\objetivos			{Ao final do período o aluno deverá ter assimilado os métodos de reconhecimento de 
					padrões e aprendizado de máquina. O aluno deverá também ter se familiarizado com os 
					conceitos fundamentais, teorias e algoritmos para reconhecimento de padrões e aprendizado 
					de máquina, devendo ter conhecimento para aplicá-los em visão computacional, processamento 
					de imagens, reconhecimento de voz, mineração de dados, estatística e aplicações médicas.}
\ementa				{Distribuições de Probabilidade; Modelos Lineares para Regressão; Modelos Lineares para Classificação; 
					Redes Neurais; Métodos de Núcleo; Máquinas de Agrupamento Esparso; Modelos Gráficos; Modelos de Mistura e EM (Estimação-Maximização); Inferência Aproximada; Métodos de Amostragem; Variáveis Latentes Contínuas; Dados Sequenciais; 
					Métodos não paramétricos: k-vizinhos mais próximos (kNN); Teoria de decisão de Bayes, Combinação de Modelos.}
\naoobrigatoria		 %
\eletivarestrita	{Engenharia de Computação}				
\formementa{Eletiva1_ReconhecDePad}
\end{document}
