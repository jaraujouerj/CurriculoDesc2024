\section{Concepção}

A estrutura curricular do curso de Engenharia de Computação do \desc da Faculdade de Engenharia da UERJ orientar-se-á pelas \textit{Diretrizes Curriculares Nacionais para os cursos de graduação na área da Computação}, do Ministério da Educação (anexo \ref{cne}), pelos \textit{Referenciais de Formação para os cursos de Graduação em Computação}, da Sociedade Brasileira de Computação (anexo \ref{sbc}), e pela regulamentação do exercício da profissão de Engenheiro, estabelecida pelo Sistema CREA/CONFEA (Resolução 1.010 CONFEA, anexo \ref{res1010}), em vigor atualmente. A inserção da extensão no curriculo terá como base a Deliberação n\textordmasculine{} 4/2023 do Conselho Superior de Ensino, Pesquisa e Extensão (CSEPE/UERJ, anexo \ref{del4}).

A grade curricular totaliza \totalhoras horas, sendo \hobrigatorias horas em disciplinas e \hextensao horas em atividades de extensão. As \hobrigatorias horas de disciplinas estão distribuídas em \ndisciplinas disciplinas, sendo \nobrigatorias  obrigatórias e \neletivas eletivas restritas. Além das disciplinas teóricas, o curso inclui práticas laboratoriais para complementar a base teórica. O currículo também contempla Estágio Supervisionado e Projeto de Graduação (trabalho de conclusão de curso) como atividades de síntese e integração do conhecimento científico, tecnológico e instrumental. Como atividades acadêmicas complementares facultativas, os alunos podem optar por Estágio Interno, Monitoria, Iniciação Científica, Cursos, Eventos, Palestras e Visitas Técnicas, que visam proporcionar uma melhor compreensão da Engenharia, do setor no Brasil e das áreas de atuação e atividades dos Engenheiros de Computação.
