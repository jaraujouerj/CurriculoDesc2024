\documentclass[12pt,a4paper]{article}
\usepackage{fancyhdr} % Required to customize headers
\pagestyle{fancy}
\usepackage[portuguese]{babel}
\usepackage[utf8]{inputenc}
\usepackage[T1]{fontenc}
\usepackage{textcomp}
\usepackage{xcolor} % Pacote para cores
\usepackage{xspace} % Adiciona suporte para espaçamento inteligente
\usepackage{graphicx} % Para incluir imagens
\usepackage{pdfpages} % Para incluir arquivos PDF
\usepackage[left=2.48cm,right=1.7cm,top=2.35cm,marginparwidth=3.4cm]{geometry}
%\usepackage{lipsum} % Para gerar texto de exemplo
%\renewcommand{\headrulewidth}{0pt}
% Defina o caminho para o logotipo da universidade
\newcommand{\universitylogo}{imagens/logo_uerj_cor.jpg}
\fancyheadoffset[L]{2mm}
\renewcommand{\headrulewidth}{0pt} % Remove a linha horizontal no cabeçalho
% Pacotes de hiperlinks
\usepackage{hyperref}
\hypersetup{
	colorlinks,
	citecolor=blue,
	filecolor=magenta,
	linkcolor=blue,
	urlcolor=cyan
}

% Configurar o cabeçalho apenas para a primeira página
\fancypagestyle{firstpage}{
  \fancyhf{} % Limpar os estilos padrão

  % logo e Nome da universidade
  \lhead{
    \begin{minipage}{0.15\textwidth}
      \includegraphics[width=2.5cm]{\universitylogo}
    \end{minipage}
    \hfill
    \begin{minipage}{0.85\textwidth} % Ajuste o tamanho conforme necessário
      \raggedright % Alinha o texto à esquerda
      UNIVERSIDADE DO ESTADO DO RIO DE JANEIRO \\
      CENTRO DE TECNOLOGIA E CIÊNCIAS \\
      FACULDADE DE ENGENHARIA \\
      DEPARTAMENTO DE ENGENHARIA DE SISTEMAS E COMPUTAÇÃO \\
    \end{minipage}
  }
}
% reduzir tamanho das letras nas seções
\makeatletter
\renewcommand{\section}{\@startsection{section}{1}{\z@}%
      {-3.5ex \@plus -1ex \@minus -.2ex}%
      {2.3ex \@plus.2ex}%
      {\normalfont\normalsize\bfseries}}
% Remover cabeçalho das outras páginas
%\pagestyle{plain} % Define o estilo das páginas subsequentes como básico (sem cabeçalho)
% Nomes das Disciplinas
\usepackage{./disciplinasDB} % Ensure the file disciplinasDB.sty is in the same directory as this .tex file
% NAO EDITE MANUALMENTE
% Este arquivo foi gerado automaticamente em 2025-5-12
% MACROS CRIADAS
% 1. tHorasCurso: Total de horas do curso: 
% 2. hTotaisDisc: Total de horas de disciplinas (regulares+eletivas+extensao): 
% 3. hTotaisDiscObrigComDiscExt: Total de horas de disciplinas obrigatórias (regulares+extensao)
% 4. hDiscObrigSemExtensao: Total de horas de disciplinas obrigatórias sem extensão
% 5. tHorasSemExtensao: Total de horas de Disc obrig sem extensao
% 6. hDiscExtensao: Total de horas de disciplinas de extensão
% 7. hExtensao: Total de horas de extensão
% 8. hACE Total de horas de atividades de extensão (sem disciplinas)
% 9. nEletivas: Número de disciplinas eletivas
% 10. credEletivas: Total de créditos de disciplinas eletivas
% 11. tCredCurso: Total de créditos do curso
% 12. credObrigSemExtensao: Total de créditos de disciplinas obrigatórias sem extensão
% 13. hEletivas: Total de horas de disciplinas eletivas
% 14. credDiscExtensao: Total de créditos disciplinas de extensao
% 15. nDisciplinas: Número total de disciplinas
% 16. nDiscObrigatorias: Número de disciplinas obrigatorias
\RequirePackage{siunitx}
\sisetup{ group-separator = {.}, group-minimum-digits = 4, output-decimal-marker={,}, }
\newcounter {thorasCursoCounter}
\setcounter {thorasCursoCounter}{3501}
\newcounter {hExtensaoCounter}
\setcounter {hExtensaoCounter}{351}
\newcommand {\tHorasCurso}{3501\xspace }
\newcommand {\hTotaisDisc}{\num{3345}\xspace }
\newcommand {\hTotaisDiscObrigComDiscExt}{\num{3225}\xspace }
\newcommand {\tHorasSemExtensao}{3150\xspace }
\newcommand {\hDiscObrigSemExtensao}{3030\xspace }
\newcommand {\hDiscExtensao}{195\xspace }
\newcommand {\hExtensao}{351\xspace }
\newcommand {\hACE}{156\xspace }
\newcommand {\nEletivas}{2\xspace }
\newcommand {\credEletivas}{8\xspace }
\newcommand {\hEletivas}{120\xspace }
\newcommand {\tCredCurso}{223\xspace }
\newcommand {\credObrigSemExtensao}{202\xspace }
\newcommand {\credDiscExtensao}{13\xspace }
\newcommand {\nDisciplinas}{55\xspace }
\newcommand {\nDiscObrigatorias}{53\xspace }

 % Arquivo com os dados do curso

\begin{document}
\thispagestyle{firstpage} % Aplica o cabeçalho na primeira página
\headsep = 20pt
%\setlength{\parindent}{0cm} % Remove paragraph indentation
\setlength{\tabcolsep}{5pt} % Espaço horizontal
\vspace*{2.0cm}

% Retira cabeçalho das páginas seguintes
\pagestyle{plain} % Define o estilo das páginas subsequentes como básico (sem cabeçalho)

\begin{center}
  \textbf{\LARGE Estágio Supervisionado}
\end{center}
\vspace*{0.5cm}

A disciplina de \textbf{\EstSup}, com \EstSupCred~créditos e \EstSupCH~horas, representa a primeira vivência profissional do aluno como futuro Engenheiro de Computação. Por isso, espera-se que as atividades sejam desenvolvidas em um ambiente corporativo ou outro contexto de atuação profissional. O estágio curricular, supervisionado, contribui para a prática extensionista no setor produtivo da sociedade e permite ao estudante enfrentar situações reais do exercício da profissão.

Durante o estágio, o aluno será acompanhado por um supervisor da empresa e contará com o apoio de um professor da disciplina, que poderá orientar o desenvolvimento das atividades e sugerir aprimoramentos.

A inscrição nesta disciplina está condicionada à conclusão de um número mínimo de 140 créditos, conforme especificado na ementa, garantindo que o aluno possua a maturidade necessária para desempenhar bem essa função. Independentemente do local ou do tipo de trabalho, é fundamental que o estágio envolva a aplicação integrada de conhecimentos, habilidades e competências adquiridas ao longo do curso.

O estágio deve permitir que o aluno reflita sobre problemas reais, proponha soluções e aprofunde a articulação entre teoria e prática. As atividades podem ser desenvolvidas em qualquer área da Engenharia de Computação, desde que envolvam a identificação e análise de um problema com base científica e tecnológica, a avaliação de soluções possíveis e a proposta de um projeto de engenharia. Também é necessário considerar os aspectos econômicos envolvidos e os impactos sociais e ambientais. Espera-se que o aluno demonstre domínio de avanços científicos e tecnológicos e sensibilidade aos desafios da inovação.

Ao final da disciplina, o aluno deverá elaborar um Relatório Final, no qual descreva detalhadamente as atividades realizadas e a aplicação dos conhecimentos adquiridos no curso. Esse relatório será avaliado em conjunto com uma apresentação oral feita ao professor da disciplina. A avaliação do relatório considerará a clareza da redação, o uso correto da língua portuguesa, a relevância do trabalho e sua contribuição para a formação do aluno. Na apresentação, serão avaliadas a organização das ideias e a capacidade de responder às perguntas do professor.

\section*{Exercício Profissional}
Considerando a importância de reconhecer a experiência profissional como parte integrante da formação acadêmica, o curso de Engenharia de Computação da UERJ permitirá que alunos com vínculo empregatício formal possam solicitar o reconhecimento de seu exercício profissional como equivalente ao estágio supervisionado obrigatório, desde que a atividade se relacione diretamente com as áreas de formação ou de exercício profissional do Engenheiro de Computação.

Para o reconhecimento, será seguido procedimento análogo ao da realização regular do estágio supervisionado, incluindo:

\begin{itemize}
  \item Cumprimento dos 140 créditos exigidos para inscrição no estágio supervisionado;
  \item Definição de um período de supervisão da atividade profissional, com carga horária igual ou superior a \EstSupCH~ horas;
  \item Apresentação e aprovação de um relatório técnico ao final do período supervisionado, nos mesmos moldes exigidos para o estágio supervisionado tradicional. \end{itemize}

O aluno deverá apresentar, previamente ao início da contagem de horas, documentação comprobatória do vínculo empregatício, como Carteira de Trabalho e Previdência Social (CTPS) assinada, contrato de trabalho ou outro documento oficial equivalente.




\end{document}

