\chapter{Infraestrutura e Corpo Docente \textcolor{red}{Felipe}}

%\section{Edificações e Instalações}
A Faculdade de Engenharia está situada no quinto andar do pavilhão João Lyra Filho. O Departamento de Engenharia de Sistemas e Computação está localizado no bloco D, do mesmo pavilhão.

\section{Biblioteca}
O acervo bibliográfico à disposição dos estudantes do curso de Engenharia de Computação está distribuído entre a Biblioteca Central e as bibliotecas setoriais da Universidade. O sistema de bibliotecas da UERJ conta com mais de 20.000 títulos, totalizando aproximadamente 30.000 exemplares, além de cerca de 1.200 títulos de periódicos, abrangendo diversas áreas do conhecimento.

A biblioteca setorial que atende diretamente ao curso está localizada no quinto andar do Pavilhão João Lyra Filho, reunindo o acervo básico recomendado nas ementas das disciplinas. O espaço conta ainda com área destinada ao estudo individual e coletivo, atendendo a discentes e docentes do curso.

O acesso ao acervo é facilitado por meio de terminais de computador com conexão à internet e integração a um sistema informatizado de busca bibliográfica, que permite a consulta eficiente dos materiais disponíveis.

Além do acervo físico, os estudantes contam com acesso ao \textit{Portal de Periódicos da CAPES}, por meio da autenticação via rede da UERJ ou por acesso remoto com identificação institucional. O portal disponibiliza milhares de periódicos científicos, livros digitais, teses e outras publicações relevantes para o ensino, a pesquisa e a extensão.

O processo de atualização do acervo é realizado por meio de aquisições com recursos institucionais e de doações recebidas regularmente, garantindo a renovação e a ampliação contínua dos materiais disponíveis.


\section{Laboratórios de Ensino e Pesquisa}

As atividades práticas do curso de Engenharia de Computação contam com laboratórios adequados às diferentes fases da formação dos estudantes. A utilização dos espaços está organizada da seguinte forma:

\begin{itemize}
    \item \textbf{Laboratórios das Ciências Básicas}: utilizados nas disciplinas do núcleo de conteúdos básicos, especialmente nas áreas de Física, Química e Matemática. Esses laboratórios oferecem infraestrutura adequada para experimentação e desenvolvimento de competências fundamentais à formação em Engenharia.

    \item \textbf{Laboratórios de Engenharia Elétrica}: utilizados nas disciplinas do núcleo de formação profissional, oferecem suporte às atividades práticas de Eletricidade, Eletrônica e Sistemas Digitais. Apoiam tanto o ensino quanto a pesquisa e extensão, proporcionando aos estudantes experiências em ambientes técnicos reais.

    \item \textbf{Laboratórios de Engenharia Eletrônica e Telecomunicações}: voltados às disciplinas que envolvem circuitos eletrônicos, instrumentação e sistemas de comunicação. Contribuem para a consolidação do conhecimento por meio de atividades experimentais.

    \item \textbf{Laboratórios de Computação}: mantidos pelo Departamento de Engenharia de Sistemas e Computação, são utilizados em disciplinas relacionadas a Linguagens de Programação, Arquitetura de Computadores, Sistemas Operacionais, Estruturas de Dados, Redes de Computadores, Inteligência Artificial, entre outras. O departamento conta com os seguintes laboratórios:
          \begin{itemize}
              \item \textbf{LABCOMP01} – sala 5023-D: equipado com 14 computadores.
              \item \textbf{LABCOMP02} – sala 5029-D: equipado com 12 computadores.
              \item \textbf{LABCOMP03} – sala 5032-D: equipado com 16 computadores com GPUs, voltados ao ensino de computação de alto desempenho.
          \end{itemize}
\end{itemize}

Todos os laboratórios de computação possuem computadores com sistemas operacionais Linux e Windows, interligados em rede, com acesso à internet e aos softwares necessários às atividades de ensino, pesquisa e extensão. Dentre os softwares disponíveis, destacam-se: GCC, Clang, Geany, VS Code, Git, mcedit, LibreOffice, Apache NetBeans, Java SDK, MPI, Python, Yasm e DDD.

Além dos espaços destinados ao ensino, o Departamento de Engenharia de Sistemas e Computação disponibiliza dois laboratórios voltados à pesquisa: o \textbf{Laboratório de Síntese Automática de Circuitos (LSAC)}, dedicado ao desenvolvimento de técnicas e ferramentas para projeto automatizado de circuitos digitais; e o \textbf{Laboratório de Inteligência de Enxame}, voltado à investigação de algoritmos inspirados em sistemas naturais distribuídos, com aplicações em otimização e inteligência computacional.

\section{Corpo Docente}

O corpo docente do Departamento de Engenharia de Sistemas e Computação (DESC), responsável pela oferta das disciplinas do curso de Engenharia de Computação, é composto por 10 docentes com título de doutorado e 3 com título de mestre, sendo que 2 destes últimos encontram-se em fase final de conclusão de seus respectivos doutorados. A Tabela~\ref{CorpoDocente} apresenta a relação completa dos docentes, com suas respectivas titulações e cargos. O regime de trabalho de todos os professores é de 40h.

Ressalta-se que o corpo docente atualmente vinculado ao DESC possui qualificação e disponibilidade suficientes para atender integralmente à matriz curricular do curso, não sendo necessária, portanto, a contratação de novos professores para a implementação deste Projeto Pedagógico.

\rowcolors{1}{gray!10}{white}
\begin{table}
    \centering
    \caption{Corpo Docente}
    \label{CorpoDocente}
    \begin{tabular}{lll}
        \hline
        \rowcolor{gray!20}
        {\textbf{Docente}}              & \textbf{Titulação}                         & \textbf{Cargo}   \\
        \hline
        Cristiana Barbosa Bentes        & Doutorado em Eng. de Sistemas e Computação & Prof. Titular    \\
        Felipe Cassemiro Ulrichsen      & Mestrado em Ciências Computacionais        & Prof. Assistente \\
        Gabriel Cardoso de Carvalho     & Doutorado em Computação                    & Prof. Assistente \\
        Giomar Oliver Sequeiros Olivera & Doutorado em Computação                    & Prof. Adjunto    \\
        João Araujo Ribeiro             & Doutorado em Computação                    & Prof. Associado  \\
        Luigi Maciel Ribeiro            & Mestrado em Engenharia Eletrônica          & Prof. Assistente \\
        Luiza de Macedo Mourelle        & Doutorado em Computação                    & Prof. Titular    \\
        Margareth Gonçalves Simões      & Doutorado em Geografia                     & Prof. Associado  \\
        Rafaela Correia Brum            & Doutorado em Computação                    & Prof. Adjunto    \\
        Robert Mota Oliveira            & Doutorado em Engenharia Elétrica           & Prof. Adjunto    \\
        Silas Pereira Lima Filho        & Doutorado em Informática                   & Prof. Adjunto    \\
        Simone Ingrid Monteiro Gama     & Mestrado em Informática                    & Prof. Assistente \\
        Thiago Medeiros Carvalho        & Doutorado em Engenharia Elétrica           & Prof. Assistente \\
        \hline
    \end{tabular}
\end{table}


