\chapter{Caracterização das Instalações Físicas - \textcolor{red}{Felipe}}

%\section{Edificações e Instalações}
A Faculdade de Engenharia está situada no quinto andar do pavilhão João Lyra Filho. O Departamento de Engenharia de Sistemas e Computação está localizado no bloco D, do mesmo pavilhão.

\section{Biblioteca}
Os recursos bibliográficos postos à disposição dos alunos estão sob a guarda da biblioteca central e das bibliotecas setoriais. São mais de vinte mil (20.000) títulos com cerca de trinta mil exemplares (30.000), cerca de mil e duzentos títulos de periódicos sobre os mais diversos assuntos de todas as áreas.

A Biblioteca setorial do curso está situada no quinto andar do pavilhão João Lyra Filho e reúne o acervo básico, oferecendo área de estudos específica para os discentes e docentes.

Associado a esses recursos, os alunos, por meio do uso de computadores e da Internet, têm acesso ao sistema automático de busca bibliográfica.

Em relação aos mecanismos de atualização, a biblioteca conta com doações e verbas próprias da UERJ.

\section{Laboratórios}

A Faculdade de Engenharia da UERJ disponibiliza uma infraestrutura de laboratórios que atende tanto aos cursos de graduação quanto aos programas de pós-graduação. O curso de Engenharia de Computação fará uso dessa estrutura para o desenvolvimento das atividades práticas previstas em seu currículo.

Nas disciplinas pertencentes ao núcleo de conteúdos básicos, os estudantes utilizarão laboratórios vinculados às Ciências Básicas, destinados ao suporte das atividades práticas nas áreas de Física, Matemática e afins.

Para o desenvolvimento das atividades práticas das disciplinas do núcleo de formação profissional, serão utilizados os Laboratórios de Engenharia Elétrica, os Laboratórios de Engenharia Eletrônica e de Telecomunicações, bem como os Laboratórios de Computação, estes últimos vinculados ao Departamento de Engenharia de Sistemas e Computação.

Os Laboratórios de Engenharia Elétrica e os Laboratórios de Engenharia Eletrônica e de Telecomunicações oferecem infraestrutura adequada para o apoio às atividades de ensino e pesquisa nas áreas de Eletricidade, Eletrônica e Sistemas Digitais, assegurando as condições necessárias para o cumprimento das atividades experimentais previstas nas disciplinas de formação profissional.

O Departamento de Engenharia de Sistemas e Computação disponibiliza três Laboratórios de Computação: \textbf{LABCOMP01}, \textbf{LABCOMP02} e \textbf{LABCOMP03}, localizados nas salas 5023-D, 5029-D e 5032-D, respectivamente. Estes laboratórios são utilizados para o desenvolvimento de atividades práticas de ensino, pesquisa e extensão nas áreas de Linguagens de Programação, Lógica e Semântica de Programas, Arquitetura de Sistemas de Computação e Sistemas de Informação.

Os Laboratórios de Computação são prioritariamente destinados aos alunos e professores vinculados ao Departamento de Engenharia de Sistemas e Computação, podendo também atender, sob demanda, outras unidades acadêmicas da Faculdade de Engenharia. Nesses ambientes são realizadas aulas práticas, seminários internos, estudos de caso e palestras com especialistas convidados.

Cada laboratório é equipado com computadores de alto desempenho. O \textbf{LABCOMP01} dispõe de 14 computadores; o \textbf{LABCOMP02}, de 12 computadores; e o \textbf{LABCOMP03}, de 16 computadores, todos com placas gráficas (GPUs) que permitem o desenvolvimento de atividades relacionadas à computação de alto desempenho.

Todos os computadores possuem ambientes operacionais Linux e Windows, estão interligados em rede e oferecem acesso pleno à Internet. Contam ainda com software essencial para as atividades acadêmicas de ensino, pesquisa e extensão, incluindo: \textit{GCC}, \textit{Clang}, \textit{Geany}, \textit{Visual Studio Code}, \textit{Git}, \textit{Mcedit}, \textit{LibreOffice}, \textit{Apache NetBeans}, \textit{Java SDK}, \textit{MPI}, \textit{Python}, \textit{Yasm} e \textit{DDD}.

O Departamento de Engenharia de Sistemas e Computação também disponibiliza dois laboratórios ligados à pesquisa: Laboratório de Síntese Automática de Circuitos (LSAC) e Laboratório de Inteligência de Enxame.
