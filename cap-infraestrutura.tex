\chapter{Caracterização das Instalações Físicas - \textcolor{red}{Felipe}}

\section{Edificações e Instalações}
A Faculdade de Engenharia está situada no quinto andar do pavilhão João Lyra Filho e possui, no bloco F, 22 salas de aula com capacidade média para 40 alunos. O Departamento de Engenharia de Sistemas e Computação se situa na sala 5028 do bloco D.

\section{Biblioteca}
Os recursos bibliográficos postos à disposição dos alunos estão sob a guarda da biblioteca central e das bibliotecas setoriais. São mais de vinte mil (20.000) títulos com cerca de trinta mil exemplares (30.000), cerca de mil e duzentos títulos de periódicos sobre os mais diversos assuntos de todas as áreas.

A Biblioteca setorial do curso está situada no quinto andar do pavilhão João Lyra Filho e reúne o acervo básico, oferecendo área de estudos específica para os discentes e docentes.

Associado a esses recursos, os alunos, por meio do uso de computadores e da Internet, têm acesso ao sistema automático de busca bibliográfica.

Em relação aos mecanismos de atualização, a biblioteca conta com doações e verbas próprias da UERJ.

\section{Laboratórios}

A Faculdade de Engenharia possui laboratórios que atendem tanto aos cursos de graduação como também aos cursos de pós-graduação. O Curso de Engenharia de Computação utilizará, para as aulas práticas das disciplinas do núcleo de conteúdos básicos, os laboratórios vinculados às Ciências Básicas: Física e Química. Para as disciplinas do núcleo profissional, serão utilizados o Laboratório de Engenharia Elétrica e os Laboratórios de Computação.

O Laboratório de Engenharia Elétrica (LEE) apoia as atividades de ensino e pesquisa em Eletricidade, Eletrônica, Máquinas Elétricas, Sistemas de Controle, Acionamentos Elétricos, Eletrônica Industrial, Conversão Eletromecânica de Energia, Sistemas Digitais e Telecomunicações.

Os Laboratórios de Computação apoiam as atividades de ensino e pesquisa e extensão em Linguagem de Progamação, Lógica e Semântica de Programas, Arquitetura de Sistemas e Computação e Sistemas de Informação.

O Departamento de Engenharia de Sistemas e Computação possui três Laboratórios de Computação: \textbf{LABOGEO2},  \textbf{LABCOMP01} e \textbf{LABCOMP02}, localizados respectivamente nas salas 5032-D, 5023-D e 5029-D.

Os laboratórios são destinados aos alunos e professores ligados ao Departamento de Engenharia de Sistemas e Computação, porém outras unidades e departamentos podem utilizá-los sob demanda. Nestes laboratórios, são realizadas as aulas práticas, são apresentados seminários internos, apresentação de estudos de casos e palestras de convidados. Todos os laboratórios são equipados com computadores: o LABOGEO2 possui 16 computadores, alguns com GPUs para ensino de computação de alto desempenho; o LABCOMP1 possui 14 computadores; e o LABCOMP2 possui 12 computadores.


Todos os computadores dos laboratórios possuem ambiente Linux, Windows 10 ou Windows 11 e estão interligados em rede, propiciando acesso completo à informação disponível na Internet. Além disso, possuem softwares necessários para o ensino pesquisa e extensão, como por exemplo GCC,
Clang,
Geany,
VS Code,
GIT,
Mcedit,
LibreOffice,
Apache Netbeans,
Java SDK,
MPI, 
Python, 
Yasm, e
DDD.

