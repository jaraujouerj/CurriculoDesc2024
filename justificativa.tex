\documentclass[12pt,a4paper]{article}
\usepackage{fancyhdr} % Required to customize headers
\pagestyle{fancy}
\usepackage[portuguese]{babel}
\usepackage[utf8]{inputenc}
\usepackage[T1]{fontenc}
\usepackage{textcomp}
\usepackage{xcolor} % Pacote para cores
\usepackage{xspace} % Adiciona suporte para espaçamento inteligente
\usepackage{graphicx} % Para incluir imagens
\usepackage{pdfpages} % Para incluir arquivos PDF
\usepackage[left=2.48cm,right=1.7cm,top=2.35cm,marginparwidth=3.4cm]{geometry}
%\usepackage{lipsum} % Para gerar texto de exemplo
%\renewcommand{\headrulewidth}{0pt}
% Defina o caminho para o logotipo da universidade
\newcommand{\universitylogo}{imagens/logo_uerj_cor.jpg}
\fancyheadoffset[L]{2mm}
\renewcommand{\headrulewidth}{0pt} % Remove a linha horizontal no cabeçalho
% Pacotes de hiperlinks
\usepackage{hyperref}
\hypersetup{
	colorlinks,
	citecolor=blue,
	filecolor=magenta,
	linkcolor=blue,
	urlcolor=cyan
}

% Configurar o cabeçalho apenas para a primeira página
\fancypagestyle{firstpage}{
  \fancyhf{} % Limpar os estilos padrão

  % logo e Nome da universidade
  \lhead{
    \begin{minipage}{0.15\textwidth}
      \includegraphics[width=2.5cm]{\universitylogo}
    \end{minipage}
    \hfill
    \begin{minipage}{0.85\textwidth} % Ajuste o tamanho conforme necessário
      \raggedright % Alinha o texto à esquerda
      UNIVERSIDADE DO ESTADO DO RIO DE JANEIRO \\
      CENTRO DE TECNOLOGIA E CIÊNCIAS \\
      FACULDADE DE ENGENHARIA \\
      DEPARTAMENTO DE ENGENHARIA DE SISTEMAS E COMPUTAÇÃO \\
    \end{minipage}
  }
}
% reduzir tamanho das letras nas seções
\makeatletter
\renewcommand{\section}{\@startsection{section}{1}{\z@}%
      {-3.5ex \@plus -1ex \@minus -.2ex}%
      {2.3ex \@plus.2ex}%
      {\normalfont\normalsize\bfseries}}
% Remover cabeçalho das outras páginas
%\pagestyle{plain} % Define o estilo das páginas subsequentes como básico (sem cabeçalho)

% NAO EDITE MANUALMENTE
% Este arquivo foi gerado automaticamente em 2025-5-12
% MACROS CRIADAS
% 1. tHorasCurso: Total de horas do curso: 
% 2. hTotaisDisc: Total de horas de disciplinas (regulares+eletivas+extensao): 
% 3. hTotaisDiscObrigComDiscExt: Total de horas de disciplinas obrigatórias (regulares+extensao)
% 4. hDiscObrigSemExtensao: Total de horas de disciplinas obrigatórias sem extensão
% 5. tHorasSemExtensao: Total de horas de Disc obrig sem extensao
% 6. hDiscExtensao: Total de horas de disciplinas de extensão
% 7. hExtensao: Total de horas de extensão
% 8. hACE Total de horas de atividades de extensão (sem disciplinas)
% 9. nEletivas: Número de disciplinas eletivas
% 10. credEletivas: Total de créditos de disciplinas eletivas
% 11. tCredCurso: Total de créditos do curso
% 12. credObrigSemExtensao: Total de créditos de disciplinas obrigatórias sem extensão
% 13. hEletivas: Total de horas de disciplinas eletivas
% 14. credDiscExtensao: Total de créditos disciplinas de extensao
% 15. nDisciplinas: Número total de disciplinas
% 16. nDiscObrigatorias: Número de disciplinas obrigatorias
\RequirePackage{siunitx}
\sisetup{ group-separator = {.}, group-minimum-digits = 4, output-decimal-marker={,}, }
\newcounter {thorasCursoCounter}
\setcounter {thorasCursoCounter}{3501}
\newcounter {hExtensaoCounter}
\setcounter {hExtensaoCounter}{351}
\newcommand {\tHorasCurso}{3501\xspace }
\newcommand {\hTotaisDisc}{\num{3345}\xspace }
\newcommand {\hTotaisDiscObrigComDiscExt}{\num{3225}\xspace }
\newcommand {\tHorasSemExtensao}{3150\xspace }
\newcommand {\hDiscObrigSemExtensao}{3030\xspace }
\newcommand {\hDiscExtensao}{195\xspace }
\newcommand {\hExtensao}{351\xspace }
\newcommand {\hACE}{156\xspace }
\newcommand {\nEletivas}{2\xspace }
\newcommand {\credEletivas}{8\xspace }
\newcommand {\hEletivas}{120\xspace }
\newcommand {\tCredCurso}{223\xspace }
\newcommand {\credObrigSemExtensao}{202\xspace }
\newcommand {\credDiscExtensao}{13\xspace }
\newcommand {\nDisciplinas}{55\xspace }
\newcommand {\nDiscObrigatorias}{53\xspace }

 % Arquivo com os dados do curso

\begin{document}
\thispagestyle{firstpage} % Aplica o cabeçalho na primeira página
\headsep = 20pt
%\setlength{\parindent}{0cm} % Remove paragraph indentation
\setlength{\tabcolsep}{5pt} % Espaço horizontal
\vspace*{2.0cm}

% Retira cabeçalho das páginas seguintes
\pagestyle{plain} % Define o estilo das páginas subsequentes como básico (sem cabeçalho)

\begin{center}
  \textbf{\large Justificativa para a criação do curso de Engenharia de Computação} \\
\end{center}
\small
\section{Modernização do curso de Engenharia Elétrica com ênfase em Sistemas e Computação}

O Departamento de Engenharia de Sistemas e Computação (DESC), desde 1977, desempenha um papel pioneiro na consolidação do ensino de Engenharia de Computação no Brasil, ao instituir, naquele ano, a primeira graduação do país com foco específico na área, por meio da ênfase em Sistemas e Computação no curso de Engenharia Elétrica. Tal ênfase foi formalmente estabelecida pela Resolução n\textordmasculine{} 466/76, de 29 de dezembro de 1976, que conferiu respaldo legal à sua implementação. Em 1994, com a Deliberação n\textordmasculine{} 035/94, o curso passou por uma significativa reformulação, resultando na consolidação de sua proposta pedagógica. Desde então, sua estrutura curricular manteve-se essencialmente estável, mesmo diante das profundas transformações ocorridas no campo da computação e da engenharia. Considerando os avanços científicos, tecnológicos e sociais das últimas décadas, torna-se indispensável a atualização do projeto pedagógico, a fim de garantir a formação de profissionais qualificados, inovadores e alinhados às demandas contemporâneas da sociedade e do mundo do trabalho.

Atualmente, os candidatos aos cursos da Faculdade de Engenharia da UERJ enfrentam uma limitação importante: ao optarem pela Engenharia Elétrica com Ênfase em Sistemas e Computação, recebem o título de Engenheiro com habilitação em Elétrica, ainda que sua formação esteja majoritariamente voltada para o desenvolvimento de software e hardware. Essa designação nem sempre reflete com precisão as competências adquiridas, o que impacta sua adequação às diretrizes do Exame Nacional de Desempenho dos Estudantes (ENADE).

Diante desse contexto, propõe-se a criação do curso de Engenharia de Computação (versão 1) com identidade própria, em substituição ao atual curso de Engenharia Elétrica com Ênfase em Sistemas e Computação e independente da habilitação em Engenharia Elétrica. Essa mudança visa, por um lado, conferir maior coerência entre a estrutura curricular, o perfil do egresso e sua titulação, e, por outro, atualizar o curso frente às necessidades emergentes da área de computação e às expectativas do setor produtivo e da academia. Trata-se, portanto, de uma iniciativa estratégica para consolidar a Engenharia de Computação como uma área de formação distinta dentro da UERJ, reforçando o compromisso da instituição com a excelência acadêmica, a inovação tecnológica e a inserção social de seus egressos.

Destaca-se a redução substancial da carga horária total do curso, que passou de 4.260 horas na versão atual do curso de Engenharia Elétrica com Ênfase em Sistemas e Computação, para \tHorasCurso horas no novo curso de Engenharia de Computação, representando uma diminuição de \inteval{4260 - \thethorasCursoCounter} horas. Essa adequação foi viabilizada pelas novas Diretrizes Curriculares Nacionais específicas para os cursos da área de Computação (Resolução CNE/CES n\textordmasculine{}~5, de 16 de novembro de 2016), que estabelecem um mínimo de 3.200 horas, em contraste com as 3.600 horas exigidas pelas diretrizes gerais das Engenharias. A redução permitiu a eliminação de conteúdos mais relevantes para outras engenharias, porém menos pertinentes à formação em Computação, mantendo a qualidade e a abrangência da formação profissional exigida para o engenheiro de computação.

\section{Demanda Social}
As demandas sociais e tecnológicas da sociedade brasileira têm se intensificado, especialmente em um cenário global caracterizado pela transformação digital, pela constante inovação tecnológica e pela crescente aplicação de inteligência artificial e aprendizado de máquina na resolução de desafios complexos. Nesse contexto, a formação de profissionais com competências técnicas e humanas no campo da computação torna-se imprescindível para enfrentar os desafios impostos pela modernização e pela globalização.

No Brasil, observa-se uma necessidade crescente de profissionais qualificados para atuar com sistemas informatizados em distintos setores da sociedade. A computação desempenha um papel estratégico em áreas como saúde, educação, administração pública, indústria, segurança, inovação e empreendedorismo. Assim, torna-se fundamental a formação de engenheiros capazes de conceber, desenvolver, implementar e gerenciar soluções computacionais inovadoras, que contribuam de maneira efetiva para o desenvolvimento econômico e social do país.

A iniciativa de criação do curso de Engenharia de Computação visa responder à crescente necessidade por profissionais altamente qualificados, capazes de enfrentar os desafios tecnológicos contemporâneos e de atuar de maneira crítica e criativa diante das demandas do mercado e da sociedade, promovendo soluções que aliem excelência técnica, responsabilidade social e impacto positivo.

\section{Inserção Curricular da Extensão}
A proposta de criação do curso de Engenharia de Computação já incorpora, desde sua concepção, uma estrutura curricular compatível com as diretrizes estabelecidas pela Resolução CNE/CES n\textordmasculine{} 7, de 18 de dezembro de 2018, que determina que, no mínimo, 10\% da carga horária total dos cursos de graduação sejam destinadas a atividades de extensão universitária. No âmbito da UERJ, essa diretriz foi regulamentada pela Deliberação UERJ n\textordmasculine{} 04/2023, que orienta a inclusão e a organização da extensão nos projetos pedagógicos de curso. Assim, o novo curso de Engenharia de Computação já incorpora uma estrutura curricular que promove a indissociabilidade entre ensino, pesquisa e extensão, reforçando a função social da universidade.

\end{document}
