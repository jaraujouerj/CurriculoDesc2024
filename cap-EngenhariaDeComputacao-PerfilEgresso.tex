\section{Perfil do Egresso (competência, habilidades e atitudes pretendidas)}
O curso de Engenharia de Computação tem como perfil do egresso o engenheiro, com formação técnico-científica sólida, generalista, humanista, crítica e reflexiva, capacitado a absorver e desenvolver novas tecnologias, estimulando a sua atuação crítica e criativa na identificação e resolução de problemas, considerando seus aspectos políticos, econômicos, sociais, ambientais e culturais, com visão ética e humanística, em atendimento às demandas da sociedade. Faz parte do perfil do egresso a postura de permanente busca da atualização profissional, além das seguintes habilidades:
\begin{enumerate} [I -]
	\item possuir conhecimento das questões humanísticas, sociais, ambientais, éticas, profissionais, legais e políticas;
	\item possuir compreensão do impacto da Engenharia de Computação e suas tecnologias no que concerne ao atendimento e à antecipação estratégica das necessidades da sociedade;
	\item possuir atitude crítica, interdisciplinar e criativa na identificação e resolução de problemas;
	\item possuir compreensão das necessidades de contínua atualização e aprimoramento de suas competências e habilidades;
	\item possuir uma sólida formação em Computação, Física, Matemática, Eletrônica, Automação e Telecomunicações.
	\item conhecer a estrutura dos sistemas de computação e os processos envolvidos na sua análise e construção;
	\item considerar os aspectos ambientais, econômicos, financeiros, de gestão e de qualidade, associados a novos produtos e organizações;
	\item considerar fundamental a inovação, a criatividade, a atitude empreendedora e a inserção internacional.
\end{enumerate}

O egresso da Engenharia de Computação, no processo de sua formação, deverá desenvolver as seguintes competências:
\begin{enumerate} [I -]
	\item antever as implicações humanísticas, sociais, ambientais, éticas, profissionais, legais (inclusive relacionadas à propriedade intelectual) e políticas dos sistemas computacionais;
	\item identificar demandas socioeconômicas e ambientais relevantes, planejar, especificar e projetar sistemas de computação, seguindo teorias, princípios, métodos e procedimentos interdisciplinares;
	\item construir, testar, verificar e validar sistemas de computação, seguindo métodos, técnicas e procedimentos interdisciplinares;
	\item perceber as necessidades de atualização decorrentes da evolução tecnológica e social;
	\item relacionar problemas do mundo real com suas soluções, considerando aspectos de computabilidade e de escalabilidade;
	\item analisar, desenvolver, avaliar e aperfeiçoar software e hardware em arquiteturas de computadores;
	\item analisar, desenvolver, avaliar e aperfeiçoar sistemas de automação e sistemas inteligentes;
	\item analisar, desenvolver, avaliar e aperfeiçoar sistemas de informação computacionais;
	\item analisar, desenvolver, avaliar e aperfeiçoar circuitos eletroeletrônicos;
	\item gerenciar pessoas e infraestrutura de Sistemas de Computação;
	\item perceber as necessidades de inovação e inserção internacional com atitudes criativas e empreendedoras.
\end{enumerate}

O curso de Engenharia de Computação tem, predominantemente, o ensino da computação como atividade fim, visando à formação de recursos humanos para o desenvolvimento científico e tecnológico da computação. Assim sendo, o curso deve capacitar indivíduos para desenvolver software e hardware, com uma forte base matemática e física.

Os egressos do curso de Engenharia de Computação estarão situados no estado da arte da ciência e da tecnologia da computação, de tal forma que possam continuar suas atividades na pesquisa, promovendo o desenvolvimento científico, ou aplicando os conhecimentos científicos, propiciando o desenvolvimento tecnológico. Para tal, é dada uma forte ênfase no uso de laboratórios para capacitar os egressos no projeto e construção tanto de software quanto de hardware.
