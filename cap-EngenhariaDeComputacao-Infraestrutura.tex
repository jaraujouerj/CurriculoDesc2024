\section{Identificação das Condições Técnico-Ambientais}

\subsection{Edificações e Instalações}
A Faculdade de Engenharia está situada no quinto andar do pavilhão João Lyra Filho e possui, no bloco F, 22 salas de aula com capacidade média para 40 alunos.

\subsection{Biblioteca}
Os recursos bibliográficos postos à disposição dos alunos estão sob a guarda da biblioteca central e das bibliotecas setoriais. São mais de vinte mil (20.000) títulos com cerca de trinta mil exemplares (30.000), cerca de mil e duzentos títulos de periódicos sobre os mais diversos assuntos de todas as áreas.

A Biblioteca setorial do curso está situada no quinto andar do pavilhão João Lyra Filho e reúne o acervo básico, oferecendo área de estudos específica para os discentes e docentes.

Associado a esses recursos, os alunos, por meio do uso de computadores e da Internet, têm acesso ao sistema automático de busca bibliográfica.

Em relação aos mecanismos de atualização, a biblioteca conta com doações e verbas próprias da UERJ.

\subsection{Laboratórios}
A Faculdade de Engenharia possui laboratórios que atendem tanto os cursos de graduação como também à pós-graduação. O Curso de Engenharia de Computação utilizará, para as aulas práticas das disciplinas do núcleo de conteúdos básicos, os laboratórios vinculados às Ciências Básicas: Física, Química e de Informática. Para as disciplinas do núcleo profissional, serão utilizados o Laboratório de Engenharia Elétrica e o Laboratório de Computação.

O Laboratório de Engenharia Elétrica (LEE) apoia as atividades de ensino e pesquisa em Eletricidade, Eletrônica, Máquinas Elétricas, Sistemas de Controle, Acionamentos Elétricos, Eletrônica Industrial, Conversão Eletromecânica de Energia, Sistemas Digitais e Telecomunicações.

O Laboratório de Computação (LabComp) apoia as atividades de ensino e pesquisa em Arquitetura de Computadores, Desenvolvimento de Sistemas, Linguagens de Programação, Análise de Algoritmos e Sistemas Embutidos.
