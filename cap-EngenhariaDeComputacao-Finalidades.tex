\section{Finalidades Gerais}
O curso visa formar engenheiros capacitados para atuar em todos os setores da economia, incluindo comércio, indústria e serviços. Dá-se especial atenção às empresas que fabricam equipamentos computacionais e às que oferecem serviços de assistência técnica, preparando os alunos para contribuir significativamente nesses campos.

\section{Finalidades Específicas}
O curso tem o objetivo de formar profissionais qualificados para identificar e diagnosticar problemas, avaliar alternativas de solução, propor e implementar soluções inovadoras, e gerenciar projetos em diversas áreas da computação. Isso abrange Inteligência Computacional, Banco de Dados, Engenharia de Software, Arquitetura de Sistemas Computacionais, Redes de Computadores, Processamento Distribuído e de Alto Desempenho, Automação, Processamento Gráfico, Linguagens Formais, Compiladores, e Análise de Algoritmos Computacionais. O intuito é preparar os alunos para serem líderes na inovação tecnológica e no desenvolvimento de novas soluções computacionais.