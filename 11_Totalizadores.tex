\documentclass[oneside,envcountsame,envcountchap,openany]{svmono}
\usepackage{fancyhdr} % Required to customize headers
\pagestyle{fancy}
\usepackage[portuguese]{babel}
\usepackage[utf8]{inputenc}
\usepackage[T1]{fontenc}
\usepackage{textcomp}
\usepackage{xcolor} % Pacote para cores
\usepackage{xspace} % Adiciona suporte para espaçamento inteligente
\usepackage{graphicx} % Para incluir imagens
\usepackage{pdfpages} % Para incluir arquivos PDF
\usepackage[top=1in,left=2cm,bottom=1in,right=2.5cm]{geometry}
%\usepackage{lipsum} % Para gerar texto de exemplo
%\renewcommand{\headrulewidth}{0pt}
% Defina o caminho para o logotipo da universidade
\newcommand{\universitylogo}{imagens/logo_uerj_cor.jpg}
\fancyheadoffset[L]{2mm}
\renewcommand{\headrulewidth}{0pt} % Remove a linha horizontal no cabeçalho
% Pacotes de hiperlinks
\usepackage{hyperref}
\hypersetup{
	colorlinks,
	citecolor=blue,
	filecolor=magenta,
	linkcolor=blue,
	urlcolor=cyan
}
\newenvironment{itquotation}
{\begin{quotation}
  \itshape % itálico
  }
{\end{quotation}}

% Pacotes de tabelas e cores
\usepackage[table]{xcolor}  % para tabelas coloridas
\usepackage{tabularx}		% para tabelas com largura automática
\usepackage{longtable}		% para tabelas longas
\usepackage{makecell}     % para tabelas com células mescladas
\usepackage{spreadtab}	    % para tabelas com cálculos
\usepackage{array}  % para tabelas com colunas de largura variável
\usepackage{booktabs}		% para tabelas com linhas horizontais
\usepackage{multirow}		 % para tabelas com multiplas linhas
\usepackage{caption}
\usepackage{chngcntr}
\counterwithout{table}{section}  % ou {chapter}, dependendo da classe


% Define uma nova coluna para alinhamento superior e quebra automática
\newcolumntype{L}[1]{>{\raggedright\arraybackslash}p{#1}}
\captionsetup{labelfont=bf, font=small, textfont=small}  % Força o negrito somente no "Tabela x:"
\renewcommand{\arraystretch}{1.5}

% Configurar o cabeçalho apenas para a primeira página
\fancypagestyle{firstpage}{
  \fancyhf{} % Limpar os estilos padrão

  % logo e Nome da universidade
  \lhead{
    \begin{minipage}{0.15\textwidth}
      \includegraphics[width=2.5cm]{\universitylogo}
    \end{minipage}
    \hfill
    \begin{minipage}{0.85\textwidth} % Ajuste o tamanho conforme necessário
      \raggedright % Alinha o texto à esquerda
      UNIVERSIDADE DO ESTADO DO RIO DE JANEIRO \\
      CENTRO DE TECNOLOGIA E CIÊNCIAS \\
      FACULDADE DE ENGENHARIA \\
      DEPARTAMENTO DE ENGENHARIA DE SISTEMAS E COMPUTAÇÃO \\
    \end{minipage}
  }
}
% reduzir tamanho das letras nas seções
\makeatletter
\renewcommand{\section}{\@startsection{section}{1}{\z@}%
      {-3.5ex \@plus -1ex \@minus -.2ex}%
      {2.3ex \@plus.2ex}%
      {\normalfont\normalsize\bfseries}}
% Remover cabeçalho das outras páginas
%\pagestyle{plain} % Define o estilo das páginas subsequentes como básico (sem cabeçalho)
% Nomes das Disciplinas
\usepackage{./disciplinasDB} % Ensure the file disciplinasDB.sty is in the same directory as this .tex file
% NAO EDITE MANUALMENTE
% Este arquivo foi gerado automaticamente em 2025-5-6
% MACROS CRIADAS
% 1. tHorasCurso: Total de horas do curso: 
% 2. hTotaisDisc: Total de horas de disciplinas (regulares+eletivas+extensao): 
% 3. hTotaisDiscObrigComDiscExt: Total de horas de disciplinas obrigatórias (regulares+extensao)
% 4. hDiscObrigSemExtensao: Total de horas de disciplinas obrigatórias sem extensão
% 5. tHorasSemExtensao: Total de horas de Disc obrig sem extensao
% 6. hDiscExtensao: Total de horas de disciplinas de extensão
% 7. hExtensao: Total de horas de extensão
% 8. hACE Total de horas de atividades de extensão (sem disciplinas)
% 9. nEletivas: Número de disciplinas eletivas
% 10. credEletivas: Total de créditos de disciplinas eletivas
% 11. tCredCurso: Total de créditos do curso
% 12. credObrigSemExtensao: Total de créditos de disciplinas obrigatórias sem extensão
% 13. hEletivas: Total de horas de disciplinas eletivas
% 14. credDiscExtensao: Total de créditos disciplinas de extensao
% 15. nDisciplinas: Número total de disciplinas
% 16. nDiscObrigatorias: Número de disciplinas obrigatorias
\newcommand {\tHorasCurso}{3567\xspace }
\newcommand {\hTotaisDisc}{3405\xspace }
\newcommand {\hTotaisDiscObrigComDiscExt}{3225\xspace }
\newcommand {\tHorasSemExtensao}{3210\xspace }
\newcommand {\hDiscObrigSemExtensao}{3030\xspace }
\newcommand {\hDiscExtensao}{195\xspace }
\newcommand {\hExtensao}{357\xspace }
\newcommand {\hACE}{162\xspace }
\newcommand {\nEletivas}{3\xspace }
\newcommand {\credEletivas}{12\xspace }
\newcommand {\hEletivas}{180\xspace }
\newcommand {\tCredCurso}{227\xspace }
\newcommand {\credObrigSemExtensao}{202\xspace }
\newcommand {\credDiscExtensao}{13\xspace }
\newcommand {\nDisciplinas}{56\xspace }
\newcommand {\nDiscObrigatorias}{53\xspace }

 % Arquivo com os dados do curso

\begin{document}
\thispagestyle{firstpage} % Aplica o cabeçalho na primeira página
\headsep = 20pt
%\setlength{\parindent}{0cm} % Remove paragraph indentation
\setlength{\tabcolsep}{5pt} % Espaço horizontal
\vspace*{2.0cm}

% Retira cabeçalho das páginas seguintes
\pagestyle{plain} % Define o estilo das páginas subsequentes como básico (sem cabeçalho)

\begin{center}
  \textbf{\LARGE Estrutura, Titulação e Totalizadores do Curso}
\end{center}
\vspace*{0.5cm}

\begin{description}
  \item [Curso:] Graduação plena em \textbf{Engenharia de Computação}.
  \item [Modalidade:] presencial.
  \item [Número de vagas no Vestibular:] \vagas vagas anuais divididas em 2 semestres. 25 vagas para o primeiro semestre e 25 vagas para o segundo semestre. Os ingressantes no primeiro semestre estão associados ao turno Manhã/Tarde e os ingressantes no segundo semestre ao turno Tarde/Noite.
  \item [Regime acadêmico:] semestral por créditos.
  \item [Tempo mínimo e máximo de integralização:] 10 semestres (5 anos) e 18 semestres (9 anos).
  \item [Carga horária total do curso:] \tHorasCurso horas.
  \item [Total de créditos:] \tCredCurso créditos, sendo \credEletivas créditos correspondentes a disciplinas eletivas restritas.
  \item [Carga horária de extensão:] \hExtensao horas.
  \item [Carga horária de atividades complementares:] \hACC horas.
  \item [Título conferido:] Engenheiro de Computação.

\end{description}

\rowcolors{1}{gray!10}{white}
\renewcommand{\arraystretch}{1.5}
\begin{table}[h!]
  \centering
  \caption{Distribuição da carga horária do curso de Engenharia de Computação}
  \label{tab:cargahoraria}
  \begin{tabularx}{0.8\textwidth}{Xr@{\hspace{0.5em}}r@{\hspace{0.5em}}}
    \hline
    \rowcolor{gray!20} \textbf{Componente}         & \textbf{Créditos}     & \textbf{Carga Horária (h)} \\
    \hline
    Disciplinas obrigatórias  (exceto de extensão) & \credObrigSemExtensao & \hDiscObrigSemExtensao     \\
    Disciplinas obrigatórias de extensão           & \credDiscExtensao     & \hDiscExtensao             \\
    Disciplinas eletivas restritas                 & \credEletivas         & \hEletivas                 \\
    Atividades Curriculares de Extensão            & --                    & \hACE                      \\
    Atividades Curriculares Complementares         & --                    & \hACC                      \\
    \hline \rowcolor{gray!20}
    \textbf{Total do curso}                        & \textbf{\tCredCurso } & \textbf{\tHorasCurso}      \\
    \hline
  \end{tabularx}
\end{table}


\end{document}