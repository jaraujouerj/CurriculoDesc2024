%----------------------------------------------------------------------------------------
%	Latex para ementa tipo dep da uerj
%----------------------------------------------------------------------------------------
%apagar .aux e executar biber ementalatex para obter bibliografia atualizada
\documentclass{ementa} % Use A4 paper with a 12pt font size 

\begin{document}
\ndisciplina 	{\IC}
\nomeprofessor	{Marley M. Bernardes Rebuzzi Vellasco}
\matriculaprofessor{31250-4}
\chta 		{\ICCH}%carga horária total aluno
\disciplinaisolada %Permite
\objetivos	{Introduzir conceitos básicos de redes neurais, algoritmos genéticos e lógica fuzzy; exemplificar a modelagem e  aplicações em problemas reais; utilizar softwares e demos para a implementação de sistemas de redes neurais, algoritmos genéticos e lógica fuzzy em problemas de classificação, previsão, otimização e controle. 
}

\ementa	{Redes Neurais: Definição e Características; Histórico, Conceitos Básicos e Aplicações; Neurônio Artificial; Estruturas de Interconexão; Processamento Neural - Aprendizado e Recuperação dos Dados; Tipos de Aprendizado - Supervisionado e Não-Supervisionado; Redes Multi-Layer Perceptron, Algoritmo de aprendizado Back Propagation; Aplicações em Classificação de Padrões e Previsão de Séries Temporais. 
 
 Computação Evolucionária: Componentes de um Algoritmo Genético (AG); Desenvolvimento de AGs; Reprodução e Seleção; Técnicas e Operadores; Problemas de Otimização Combinatorial. 
 
 Lógica Fuzzy: Introdução; Conjuntos Fuzzy; Operações com conjuntos fuzzy: interseção, união e negação de conjuntos; Sistemas de Inferência Fuzzy; Extração automática de regras fuzzy.
}

\preum		{Laboratório de Programação A} %
\codpreum		{FEN 06-xxxxx} %
\credteorica		{4} %

\formementa{IntComp}
\end{document}
