\documentclass[12pt,a4paper]{article}
\usepackage{fancyhdr} % Required to customize headers
\pagestyle{fancy}
\usepackage[portuguese]{babel}
\usepackage[utf8]{inputenc}
\usepackage[T1]{fontenc}
\usepackage{textcomp}
\usepackage{xcolor} % Pacote para cores
\usepackage{xspace} % Adiciona suporte para espaçamento inteligente
\usepackage{graphicx} % Para incluir imagens
\usepackage{pdfpages} % Para incluir arquivos PDF
\usepackage[left=2.48cm,right=1.7cm,top=2.35cm,marginparwidth=3.4cm]{geometry}
%\usepackage{lipsum} % Para gerar texto de exemplo
%\renewcommand{\headrulewidth}{0pt}
% Defina o caminho para o logotipo da universidade
\newcommand{\universitylogo}{imagens/logo_uerj_cor.jpg}
\fancyheadoffset[L]{2mm}
\renewcommand{\headrulewidth}{0pt} % Remove a linha horizontal no cabeçalho
% Pacotes de hiperlinks
\usepackage{hyperref}
\hypersetup{
	colorlinks,
	citecolor=blue,
	filecolor=magenta,
	linkcolor=blue,
	urlcolor=cyan
}

% Configurar o cabeçalho apenas para a primeira página
\fancypagestyle{firstpage}{
  \fancyhf{} % Limpar os estilos padrão

  % logo e Nome da universidade
  \lhead{
    \begin{minipage}{0.15\textwidth}
      \includegraphics[width=2.5cm]{\universitylogo}
    \end{minipage}
    \hfill
    \begin{minipage}{0.85\textwidth} % Ajuste o tamanho conforme necessário
      \raggedright % Alinha o texto à esquerda
      UNIVERSIDADE DO ESTADO DO RIO DE JANEIRO \\
      CENTRO DE TECNOLOGIA E CIÊNCIAS \\
      FACULDADE DE ENGENHARIA \\
      DEPARTAMENTO DE ENGENHARIA DE SISTEMAS E COMPUTAÇÃO \\
    \end{minipage}
  }
}
% reduzir tamanho das letras nas seções
\makeatletter
\renewcommand{\section}{\@startsection{section}{1}{\z@}%
      {-3.5ex \@plus -1ex \@minus -.2ex}%
      {2.3ex \@plus.2ex}%
      {\normalfont\normalsize\bfseries}}
% Remover cabeçalho das outras páginas
%\pagestyle{plain} % Define o estilo das páginas subsequentes como básico (sem cabeçalho)
% Nomes das Disciplinas
\usepackage{./disciplinasDB} % Ensure the file disciplinasDB.sty is in the same directory as this .tex file
% NAO EDITE MANUALMENTE
% Este arquivo foi gerado automaticamente em 2025-5-12
% MACROS CRIADAS
% 1. tHorasCurso: Total de horas do curso: 
% 2. hTotaisDisc: Total de horas de disciplinas (regulares+eletivas+extensao): 
% 3. hTotaisDiscObrigComDiscExt: Total de horas de disciplinas obrigatórias (regulares+extensao)
% 4. hDiscObrigSemExtensao: Total de horas de disciplinas obrigatórias sem extensão
% 5. tHorasSemExtensao: Total de horas de Disc obrig sem extensao
% 6. hDiscExtensao: Total de horas de disciplinas de extensão
% 7. hExtensao: Total de horas de extensão
% 8. hACE Total de horas de atividades de extensão (sem disciplinas)
% 9. nEletivas: Número de disciplinas eletivas
% 10. credEletivas: Total de créditos de disciplinas eletivas
% 11. tCredCurso: Total de créditos do curso
% 12. credObrigSemExtensao: Total de créditos de disciplinas obrigatórias sem extensão
% 13. hEletivas: Total de horas de disciplinas eletivas
% 14. credDiscExtensao: Total de créditos disciplinas de extensao
% 15. nDisciplinas: Número total de disciplinas
% 16. nDiscObrigatorias: Número de disciplinas obrigatorias
\RequirePackage{siunitx}
\sisetup{ group-separator = {.}, group-minimum-digits = 4, output-decimal-marker={,}, }
\newcounter {thorasCursoCounter}
\setcounter {thorasCursoCounter}{3501}
\newcounter {hExtensaoCounter}
\setcounter {hExtensaoCounter}{351}
\newcommand {\tHorasCurso}{3501\xspace }
\newcommand {\hTotaisDisc}{\num{3345}\xspace }
\newcommand {\hTotaisDiscObrigComDiscExt}{\num{3225}\xspace }
\newcommand {\tHorasSemExtensao}{3150\xspace }
\newcommand {\hDiscObrigSemExtensao}{3030\xspace }
\newcommand {\hDiscExtensao}{195\xspace }
\newcommand {\hExtensao}{351\xspace }
\newcommand {\hACE}{156\xspace }
\newcommand {\nEletivas}{2\xspace }
\newcommand {\credEletivas}{8\xspace }
\newcommand {\hEletivas}{120\xspace }
\newcommand {\tCredCurso}{223\xspace }
\newcommand {\credObrigSemExtensao}{202\xspace }
\newcommand {\credDiscExtensao}{13\xspace }
\newcommand {\nDisciplinas}{55\xspace }
\newcommand {\nDiscObrigatorias}{53\xspace }

 % Arquivo com os dados do curso

\begin{document}
\thispagestyle{firstpage} % Aplica o cabeçalho na primeira página
\headsep = 20pt
%\setlength{\parindent}{0cm} % Remove paragraph indentation
\setlength{\tabcolsep}{5pt} % Espaço horizontal
\vspace*{2.0cm}

% Retira cabeçalho das páginas seguintes
\pagestyle{plain} % Define o estilo das páginas subsequentes como básico (sem cabeçalho)

\begin{center}
  \textbf{\LARGE Possibilidade de Opção}
\end{center}
\vspace*{0.5cm}

Com o objetivo de efetuar uma rápida substituição das disciplinas antigas pelas novas, o
aluno de Engenharia Elétrica –- ênfase em Sistemas e Computação, versão 2  terá, em qualquer período, a possibilidade de
optar pelo novo curso de Engenharia de Computação.

Os coordenadores do Curso divulgarão uma tabela de equivalência de disciplinas para que
o aluno possa avaliar o impacto de sua escolha na conclusão do curso.

Para viabilizar a opção pelo novo Curso, as disciplinas do currículo anterior continuarão a
ser oferecidas em seus quadros de horários completos da seguinte forma:

\begin{enumerate}
  \item Disciplinas equivalentes dos currículos de Engenharia Elétrica –- ênfase em Sistemas
        e Computação (Versão 2)  e do novo curso de Engenharia de Computação (Versão 1), deverão ser preferencialmente
        oferecidas no mesmo horário, sempre que possível. As disciplinas equivalentes estão
        identificadas no Projeto Pedagógico do Curso de Engenharia de Computação (Versão 1);
  \item Disciplinas sem equivalência continuarão, durante a implantação do novo Curso, a ser
        oferecidas, de acordo com a disponibilidade do Departamento;
  \item Os novos horários poderão ser implementados em sua totalidade, caso os alunos
        regularmente matriculados em seus períodos corretos tenham optado pelo novo Curso;
  \item As disciplinas do Curso anterior continuarão a ser oferecidas em conformidade com a
        demanda de alunos que decidirem nele permanecer, respeitando o prazo máximo de
        integralização destes alunos.
\end{enumerate}

\end{document}
