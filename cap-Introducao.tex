\chapter{Introdução}
\label{intro} % Always give a unique label
% use \chaptermark{}
% to alter or adjust the chapter heading in the running head

Este documento apresenta o projeto da estrutura curricular do novo curso de Engenharia de Computação da Universidade do Estado do Rio de Janeiro. Este curso será oferecido pelo Departamento de Engenharia de Sistemas e Computação (DESC) da Faculdade de Engenharia (FEN). Desde 1977, o DESC oferece o curso de Engenharia Elétrica com Ênfase em Sistemas e Computação, sendo o primeiro no Brasil a oferecer uma graduação na área de Engenharia de Computação. O curso passou por uma reformulação significativa no início da década de 1990 e, desde então, manteve o mesmo currículo, sem alterações. O objetivo agora é promover uma reforma deste curso e oferecê-lo como uma nova habilitação em engenharia.

A Engenharia, especialmente a Engenharia de Computação, tem evoluído a passos largos no cenário contemporâneo. Com mais de três décadas desde a última atualização curricular, torna-se imperativo adaptar o curso às exigências atuais, alinhando-o com as demandas da sociedade por avanços científicos e tecnológicos nos setores industrial e de serviços. Atualmente, em 2024, os candidatos aos cursos da Faculdade de Engenharia da UERJ, incluindo a Engenharia Elétrica, enfrentam um dilema. Ao optarem pela Engenharia Elétrica com Ênfase em Sistemas e Computação, recebem um diploma de Engenheiro Eletricista, uma designação que não reflete adequadamente a especialização adquirida, especialmente considerando a necessidade crescente de conhecimentos aprofundados em desenvolvimento de sistemas e dispositivos computacionais, conforme orientado pelas diretrizes do ENADE.

A proposta de introduzir o curso de \textbf{Engenharia de Computação} como uma nova habilitação visa não apenas modernizar o currículo em resposta às transformações tecnológicas, mas também proporcionar uma formação mais alinhada com as competências exigidas no campo da computação. O atual curso de Engenharia Elétrica com Ênfase em Sistemas e Computação, embora prepare os alunos para desenvolver software e projetar hardware, não os capacita adequadamente para projetar sistemas de potência, uma expectativa comum para engenheiros eletricistas. Essa discrepância tem levado a dificuldades para os alunos no ENADE, evidenciando a lacuna entre a formação oferecida e as competências específicas requeridas na área elétrica.

Portanto, a criação de um curso dedicado à Engenharia de Computação, como uma habilitação distinta, é uma resposta estratégica às necessidades educacionais emergentes, garantindo que os egressos estejam melhor preparados para enfrentar os desafios tecnológicos do futuro. Este projeto pedagógico não apenas reflete uma atualização necessária diante das mudanças tecnológicas desde os anos 1990, mas também reafirma o compromisso da UERJ com a excelência educacional e a inovação no ensino de engenharia.