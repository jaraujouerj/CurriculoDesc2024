\chapter{Introdução}
\label{intro} % 

Este documento apresenta a proposta de reformulação da estrutura curricular do curso de Engenharia de Computação da Universidade do Estado do Rio de Janeiro (UERJ), atualmente oferecido sob a denominação de Engenharia Elétrica com Ênfase em Sistemas e Computação. A revisão, conduzida pelo Departamento de Engenharia de Sistemas e Computação (DESC) da Faculdade de Engenharia (FEN), tem como objetivo modernizar a formação oferecida, alinhando-a às transformações tecnológicas e às exigências do mercado.

O DESC, desde 1977, desempenha um papel pioneiro na consolidação do ensino de Engenharia de Computação no Brasil, ao instituir, naquele ano, a primeira graduação do país com foco específico na área, por meio da ênfase em Engenharia de Sistemas e Computação no curso de Engenharia Elétrica. Tal ênfase foi formalmente estabelecida pela Resolução \ordm{n} 466/76, de 29 de dezembro de 1976, que conferiu respaldo legal à sua implementação. Em 1994, com a Deliberação \ordm{n} 035/94, o curso passou por uma significativa reformulação, resultando na consolidação de sua proposta pedagógica. Desde então, sua estrutura curricular manteve-se essencialmente estável, mesmo diante das profundas transformações ocorridas no campo da computação e da engenharia. Considerando os avanços científicos e tecnológicos das últimas décadas, torna-se premente a atualização do projeto pedagógico, de modo a garantir a formação de profissionais qualificados e alinhados às demandas contemporâneas da sociedade e do mercado.

Atualmente, os candidatos aos cursos da Faculdade de Engenharia da UERJ enfrentam um desafio: ao optarem pela Engenharia Elétrica com Ênfase em Sistemas e Computação, recebem o título de Engenheiro Eletricista, ainda que sua formação esteja majoritariamente voltada para o desenvolvimento de software e hardware. Essa designação nem sempre reflete com precisão as competências adquiridas, o que impacta tanto o reconhecimento profissional quanto a adequação às diretrizes do Exame Nacional de Desempenho dos Estudantes (ENADE).

Diante desse contexto, a reformulação do curso propõe sua reestruturação como uma habilitação específica em Engenharia de Computação. Essa mudança visa não apenas atualizar o currículo para atender às novas demandas do setor, mas também garantir que a formação esteja plenamente alinhada às expectativas do mercado e da academia. Além disso, busca corrigir a discrepância existente no perfil do egresso, que atualmente não recebe uma formação aprofundada em sistemas de potência, embora o título de Engenheiro Eletricista gere essa expectativa.

Portanto, esta revisão curricular representa uma iniciativa estratégica para consolidar a Engenharia de Computação como uma área de formação distinta dentro da UERJ. Ao modernizar o curso e garantir uma identidade mais clara para seus egressos, reafirma-se o compromisso da universidade com a excelência acadêmica e a inovação tecnológica, preparando os futuros engenheiros para os desafios da era digital.

Por fim, é fundamental destacar que o corpo docente existente possui a capacidade necessária para ministrar todas as disciplinas do curso de Engenharia de Computação. Assim, a adoção do novo currículo não exigirá a contratação de novos professores.
