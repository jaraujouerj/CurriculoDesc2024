\chapter{Introdução}
\thispagestyle{plain}
\label{intro} % 

Este documento apresenta a proposta de criação do curso de Engenharia de Computação da Universidade do Estado do Rio de Janeiro (UERJ), em substituição ao curso atualmente oferecido sob a denominação de Engenharia Elétrica com Ênfase em Sistemas e Computação. A iniciativa, conduzida pelo Departamento de Engenharia de Sistemas e Computação (DESC), da Faculdade de Engenharia (FEN), tem como objetivo modernizar a formação acadêmica, alinhando-a às transformações tecnológicas, às exigências do mercado e às diretrizes nacionais de educação superior.

O DESC, desde 1977, desempenha um papel pioneiro na consolidação do ensino de Engenharia de Computação no Brasil, ao instituir, naquele ano, a primeira graduação do país com foco específico na área, por meio da ênfase em Engenharia de Sistemas e Computação no curso de Engenharia Elétrica. Tal ênfase foi formalmente estabelecida pela Resolução \ordm{n} 466/76, de 29 de dezembro de 1976 \cite{uerj1976}, que conferiu respaldo legal à sua implementação. Em 1994, com a Deliberação \ordm{n} 035/94 \cite{uerj1994}, o curso passou por uma significativa reformulação, resultando na consolidação de sua proposta pedagógica. Desde então, sua estrutura curricular manteve-se essencialmente estável, mesmo diante das profundas transformações ocorridas no campo da computação e da engenharia. Considerando os avanços científicos, tecnológicos e sociais das últimas décadas, torna-se indispensável a atualização do projeto pedagógico, a fim de garantir a formação de profissionais qualificados, inovadores e alinhados às demandas contemporâneas da sociedade e do mundo do trabalho.

Atualmente, os candidatos aos cursos da Faculdade de Engenharia da UERJ enfrentam uma limitação importante: ao optarem pela Engenharia Elétrica com Ênfase em Sistemas e Computação, recebem o título de Engenheiro com habilitação em Elétrica, ainda que sua formação esteja majoritariamente voltada para o desenvolvimento de software e hardware. Essa designação nem sempre reflete com precisão as competências adquiridas, o que impacta tanto o reconhecimento profissional quanto a adequação às diretrizes do Exame Nacional de Desempenho dos Estudantes (ENADE). Além disso, o título de Engenheiro com habilitação em Elétrica pode sugerir uma formação em sistemas de potência que o curso, em sua ênfase atual, não contempla de maneira aprofundada.

Diante desse contexto, propõe-se a criação do curso de Engenharia de Computação (versão 1) com identidade própria e independente da habilitação em Engenharia Elétrica. Essa mudança visa, por um lado, conferir maior coerência entre a estrutura curricular, o perfil do egresso e sua titulação, e, por outro, atualizar o curso frente às necessidades emergentes da área de computação e às expectativas do setor produtivo e da academia. Trata-se, portanto, de uma iniciativa estratégica para consolidar a Engenharia de Computação como uma área de formação distinta dentro da UERJ, reforçando o compromisso da instituição com a excelência acadêmica, a inovação tecnológica e a inserção social de seus egressos.

A proposta também observa o disposto na Resolução CNE/CES \ordm{n} 7, de 18 de dezembro de 2018 \cite{cne2018}, que estabelece as Diretrizes para a Extensão na Educação Superior Brasileira, determinando que, no mínimo, 10\% da carga horária total dos cursos de graduação sejam destinadas a atividades de extensão universitária. No âmbito da UERJ, essa diretriz foi regulamentada pela Deliberação UERJ \ordm{n} 04/2023 \cite{uerj2023}, que orienta a inclusão e a organização da extensão nos projetos pedagógicos de curso. Assim, o novo curso de Engenharia de Computação já incorpora, desde sua concepção, uma estrutura curricular compatível com essas exigências, promovendo a indissociabilidade entre ensino, pesquisa e extensão e reforçando a função social da universidade.

Por fim, destaca-se que o corpo docente atualmente vinculado ao DESC possui a qualificação e a disponibilidade necessárias para ministrar integralmente as disciplinas previstas na nova matriz curricular. Composto por professores com ampla experiência acadêmica e atuação em diversas áreas da computação, o quadro docente é suficiente para viabilizar a implementação do novo curso sem a necessidade de contratações adicionais.
