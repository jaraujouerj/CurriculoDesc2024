%----------------------------------------------------------------------------------------
%	Latex para ementa tipo dep da uerj
%----------------------------------------------------------------------------------------
%apagar .aux e executar biber ementalatex para obter bibliografia atualizada
\documentclass{ementa} 

\begin{document}

\ndisciplina 		{\AnAlg} 
\nomeprofessor	    {Sheila Regina Murgel Veloso}
\matriculaprofessor {34558-7}
\objetivos	        {O aluno deverá ter assimilado o tratamento matemático de um algoritmo, verificando sua correção e determinando sua eficiência; ser capaz de distinguir a melhor técnica para elaborar um algoritmo e ter assimilado conceitos básicos de Teoria dos Grafos.
}

\ementa	            {Princípio de indução matemática (forte e fraca): Aplicação em Verificação de Corretude de  Algoritmos, Soluções de Recorrências, Estruturas   Algébricas, Combinatória, Ordens parciais e totais.

Complexidade de algoritmos: Complexidade Assintótica, Complexidade de Algoritmos Recursivos, Algoritmos Polinomiais.

Método da Divisão e Conquista: Princípios e aplicações algoritmos de pesquisa e de Ordenação, Busca Binária e Complexidade. Máximo e Mínimo de uma lista. Complexidade. 

Método Guloso: Princípios  e Aplicações: árvore geradora mínima, armazenamento. 
Programação Dinâmica: Princípios e Aplicações: escalonamento, caminhos mínimos, mochila 0/1. 

Classes de problemas: Problemas de decisão; Algoritmos não determinísticos; Classes P e NP; Problemas árduos e problemas NP-completos; Redução entre problemas de decisão. 

Teoria dos grafos. Conceitos básicos: (grafos,e subgrafos; isomorfismo, matrizes de adjacência e incidência, caminhos e ciclos. Árvores, caracterização de árvores, cortes de arestas, cortes de vértices. Conectividade de vértices e arestas; ciclos eulerianos e hamiltonianos; emparelhamentos; coloração de vértices e de arestas; planaridade).
}

\preum		    {Algoritmos Computacionais I} %
\codpreum		{FEN 06-XXXXX} %
\predois		{Laboratório de Programação B} %
\codpredois	    {FEN 06-XXXXX} %
\credteorica	{4} %
\chta 			{\AnAlgCH h}%carga horária total aluno

\formementa{AnAlg}
\end{document}
