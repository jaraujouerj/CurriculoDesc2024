\section{Normas Gerais de Ensino de Graduação da UERJ}
O curso de Engenharia de Computação obedecerá ao regime de créditos e as aulas serão oferecidas nos turnos manhã e tarde, com aulas predominantemente pela manhã, para os aprovados classificados no primeiro semestre; e tarde e noite, com aulas predominantemente pela tarde, para os aprovados classificados no segundo semestre. O turno da manhã transcorre no horário das 07:00h às 12:20h; o da tarde das 12:30h às 17:50h e o da noite das 18:00h às 22:40h. As aulas têm duração de 50 minutos nos turnos da manhã e tarde e de 45 minutos no turno da noite.

As Normas Gerais de Ensino de Graduação da UERJ são definidas pela deliberação n\textordmasculine~33/95 da UERJ (anexo \ref{delib3395}), sendo seus aspectos principais apresentados a seguir:

\subsection{Relação entre crédito e carga horária}
A deliberação n\textordmasculine~59/2019 (anexo \ref{delib592019}) da UERJ alterou a antiga deliberação, passando o artigo 57 a contar com a redaçao a seguir:

\textit{
	\textbf{Art. 57} -- O número mínimo de créditos necessários para integralizar o
	currículo será estabelecido com base na carga horária total do curso.}

\textit{
	\textbf{§ 1º} - Nos cursos de regime de crédito, a unidade padrão de crédito
	corresponde a 15 (quinze) horas, e as atividades de que trata o caput do
	presente artigo são:
	\begin{enumerate}[a)]
		\item Aula teórica;
		\item Trabalho de campo;
		\item Laboratório/aula prática;
		\item Estágio curricular;
		\item Prática como componente curricular.
	\end{enumerate}}

\subsection{Aproveitamento escolar}
\begin{itquotation}
	\setcounter{artigo}{94}
	\artigo A aprovação do aluno em disciplinas do Curso de Graduação desta Universidade terá por base notas e frequência. São condições para aprovação: obtenção de nota final mínima 5,0 (cinco vírgula zero), constituída pela média aritmética da média semestral e nota da prova final, frequência mínima de 75\% (setenta e cinco por cento) do total de horas/aula determinado para a disciplina.

	\begin{paragrafos}
		\paragrafo Para cada disciplina haverá, pelo menos, duas avaliações por turma, por período letivo, sendo uma delas necessariamente individual e escrita. A média dos resultados dessas avaliações constitui a média semestral do aluno na disciplina.\\
		\paragrafo O aluno que obtiver média semestral igual ou superior a 4,0 (quatro vírgula zero) terá direito à prova final.\\
		\paragrafo O aluno que obtiver média semestral igual ou superior a 7,0 (sete vírgula zero) estará dispensado de prestar prova final.\\
		\ldots

		\setcounter{paragrafo}{6}
		\paragrafo  O aluno que obtiver nota final menor que 5,0 (cinco vírgula zero) ou média semestral inferior a 4,0 (quatro vírgula zero) será reprovado.\\
		\paragrafo O aluno que não obtiver frequência mínima de 75\% (setenta e cinco por cento) do total de horas/aula determinadas pela disciplina será reprovado, sem direito à prova final e independente de alcançar nota final superior a 7,0 (sete vírgula zero).\\
	\end{paragrafos}

\end{itquotation}
\subsection{Período de integralização do curso}
\setcounter{artigo}{98}
\begin{itquotation}
	\artigo Somente receberá o diploma o aluno que cumprir a Integralização Curricular.
\end{itquotation}

O período mínimo de integralização curricular dos cursos de engenharia é de 10 (dez) semestres, exceto para os casos de isenção de disciplinas, em que é possível um tempo mínimo menor. Já o prazo máximo para essa integralização é de 18 (dezoito) semestres.
