\section{Estrutura Curricular}
O currículo do curso de Engenharia de Computação é constituído por disciplinas obrigatórias e eletivas, estágio supervisionado, trabalho de conclusão de curso e atividades de extensão. O curso é organizado em 10 semestres, podendo o aluno cumpri-lo em um máximo de 18 semestres.

Para uma eficaz orientação pedagógica, é proposto o aconselhamento curricular apresentado nas tabelas \ref{tab1p} a \ref{tab10p}. Os pré-requisitos das disciplinas podem ser observados no fluxograma do curso (anexo \ref{fluxograma}).

O aluno deverá cursar no mínimo três das disciplinas eletivas restritas oferecidas (ver tabela \ref{tabeletivas}). Deve ser
ressaltado que estas disciplinas são oferecidas de acordo com o interesse dos corpos
docente e discente, não sendo necessariamente disponibilizadas todos os semestres.

\rowcolors{1}{gray!5}{white}
\setlength{\tabcolsep}{5pt}
\renewcommand{\arraystretch}{1.5}
\begin{table}[ht]
	\centering
	\caption{1\textordmasculine~Período}
	\label{tab1p}
	\begin{spreadtab}{{tabularx}{\textwidth}{ | X|c|c| }}
		\hline
		@ {\textbf{Disciplina}} & @ {\textbf{CH}} & @ {\textbf{Créditos}} \\
		\hline
		@ \AlgComp	& \AlgCompCH	& \AlgCompCred	\\ % Algoritmos Computacionais
		@ \EngCompSoc 	& \EngCompSocCH & \EngCompSocCred	\\ % Engenharia e Computação Sociedade
		@\AlgLin	& \AlgLinCH		& \AlgLinCred	\\ % Álgebra Linear
		@ \CalcI	& \CalcICH		& \CalcICred	\\ % Cálculo I
		@ \IntAmb	& \IntAmbCH		& \IntAmbCred	\\ % Introdução à Engenharia Ambiental
		\hline
		@ Total 	& sum(b2:b6) 	& sum(c2:c6)	\\
		\hline
	\end{spreadtab}
\end{table}

\rowcolors{1}{gray!5}{white}
\begin{table}
	\centering
	\caption{2\textordmasculine~Período}
	\label{tab2p}
	\begin{spreadtab}{{tabularx}{\textwidth}{|X|c|c|}}
		\hline
		@ {\textbf{Disciplina}} & @ {\textbf{CH}} & @ {\textbf{Créditos}} \\
		\hline
		@ \EstrInf	& \EstrInfCH	& \EstrInfCred 	\\ % Estruturas de Informação
		@ \LogProg	& \LogProgCH	& \LogProgCred	\\ % Lógica de Programação
		@ \CalcII	& \CalcIICH		& \CalcIICred	\\ % Cálculo II
		@ \EngComput& \EngComputCH	& \EngComputCred\\ % Calculo Numérico
		@ \FisI		& \FisICH		& \FisICred		\\ % Física I
		@ \FisEI	& \FisEICH		& \FisEICred	\\ % Física Experimental I
		\hline
		@ Total 	& sum(b2:b7) 	& sum(c2:c7)	\\
		\hline
	\end{spreadtab}
\end{table}
	
\rowcolors{1}{gray!5}{white}
\begin{table}
	\centering
	\caption{3\textordmasculine~Período}
	\label{tab3p}
	\begin{spreadtab}{{tabularx}{\textwidth}{|X|c|c|}}
		\hline
		@ {\textbf{Disciplina}} & @ {\textbf{CH}} & @ {\textbf{Créditos}} \\
		\hline
		@ \AnAlg	& \AnAlgCH		& \AnAlgCred	\\ % Análise de Algoritmos
		@ \CalcIII	& \CalcIIICH 	& \CalcIIICred	\\ % Cálculo III
		@ \FisII	& \FisIICH		& \FisIICred	\\ % Física II
		@ \FisEII	& \FisEICH		& \FisEICred	\\ % Física Experimental II
		@ \ProbEst	& \ProbEstCH	& \ProbEstCred	\\ % Probabilidade e Estatística
		\hline
		@ Total 	& sum(b2:b6) 	& sum(c2:c6)	\\
		\hline
	\end{spreadtab}
\end{table}

\rowcolors{1}{gray!5}{white}
\begin{table}
	\centering
	\caption{4\textordmasculine~Período}
	\label{tab4p}
	\begin{spreadtab}{{tabularx}{\textwidth}{|X|c|c|}}
		\hline
		@ {\textbf{Disciplina}} & @ {\textbf{CH}} & @ {\textbf{Créditos}} \\
		\hline
		@ \LabProgA	& \LabProgACH	& \LabProgACred		\\ % Laboratório de Programação A
		@ \LabProgB	& \LabProgBCH	& \LabProgBCred		\\ % Laboratório de Programação B
		@ \FisIII	& \FisIIICH		& \FisIIICred		\\ % Física III
		@ \FisEIII	& \FisEIIICH	& \FisEIIICred		\\ % Física Experimental III
		@ \ProcImag 	& \ProcImagCH	& \ProcImagCred	\\ % Processamento de Sinais e Imagens
		@ \FundIComp	& \FundICompCH	& \FundICompCred\\ %Técnicas Digitais I
		\hline
		@ Total 	& sum(b2:b7) 	& sum(c2:c7)	\\
		\hline
	\end{spreadtab}
\end{table}

\rowcolors{1}{gray!5}{white}
\begin{table}
	\centering
	\caption{5\textordmasculine~Período}
	\label{tab5p}
	\begin{spreadtab}{{tabularx}{\textwidth}{|X|c|c|}}
		\hline
		@ {\textbf{Disciplina}} & @ {\textbf{CH}} & @ {\textbf{Créditos}} \\
		\hline
		@ \Grafos	& \GrafosCH		& \GrafosCred	\\ % Teoria dos Grafos e Aplicações
		@ \FundComp	& \FundCompCH	& \FundCompCred	\\ % Fundamentos de Computadores I
		@\CEV		& \CEVCH		& \CEVCred		\\ % Circuitos em Corrente 
		@ \FisIV	& \FisIVCH		& \FisIVCred	\\ % Física IV
		@ \FisEIV	& \FisEIVCH		& \FisEIVCred	\\ % Física Experimental IV
		@ \MatEle 	& \MatEleCH		& \MatEleCred	\\ % Materiais Elétricos e Magnéticos 
		@ \ModMat	& \ModMatCH		& \ModMatCred	\\ % Sinais e Sistemas
		\hline
		@ Total 	& sum(b2:b8) 	& sum(c2:c8)	\\
		\hline
	\end{spreadtab}
\end{table}

\rowcolors{1}{gray!5}{white}
\begin{table}
	\centering
	\caption{6\textordmasculine~Período}
	\label{tab6p}
	\begin{spreadtab}{{tabularx}{\textwidth}{|X|c|c|}}
		\hline
		@ {\textbf{Disciplina}} & @ {\textbf{CH}} & @ {\textbf{Créditos}} \\
		\hline
		@ \ArqComp	& \ArqCompCH	& \ArqCompCred	\\ % Arquitetura de Computadores A
		@ \EngSistA & \EngSistACH	& \EngSistACred	\\ % Engenharia de Sistemas
		@ \IC		& \ICCH			& \ICCred		\\ % Inteligência Computacional I
		@ \ICII 	& \ICIICH		& \ICIICred		\\ % Inteligência Computacional II
		@ \CEVI		& \CEVICH 		& \CEVICred		\\ % Circuitos em Corrente Alternada
		@ \EletI	& \EletICH		& \EletICred	\\ % Eletrônica I
		\hline
		@ Total 	& sum(b2:b7) 	& sum(c2:c7)	\\
		\hline
	\end{spreadtab}
\end{table}

\rowcolors{1}{gray!5}{white}
\begin{table}
	\centering
	\caption{7\textordmasculine~Período}
	\label{tab7p}
	\begin{spreadtab}{{tabularx}{\textwidth}{|X|c|c|}}
		\hline
		@ {\textbf{Disciplina}} & @ {\textbf{CH}} & @ {\textbf{Créditos}} \\
		\hline
		@ \MineraDados	& \MineraDadosCH	& \MineraDadosCred	\\ % Mineração de Dados
		@ \ProjBD		& \ProjBDCH		& \ProjBDCred		\\ % Projeto de Banco de 
		@ \ProjSO		& \ProjSOCH		& \ProjSOCred		\\ % Projeto de Sistemas Operacionais
		@ \Telep 		& \TelepCH		& \TelepCred		\\ % Redes de Computadores
		@ \TeoComp		& \TeoCompCH	& \TeoCompCred		\\ % Teoria da Compiladores
		@ \IntEco		& \IntEcoCH		& \IntEcoCred	\\ % Macroeconomia 
		\hline
		@ Total			& sum(b2:b7)	& sum(c2:c7)		\\
		\hline
	\end{spreadtab}
\end{table}

\rowcolors{1}{gray!5}{white}
\begin{table}
	\centering
	\caption{8\textordmasculine~Período}
	\label{tab8p}
	\begin{spreadtab}{{tabularx}{\textwidth}{|X|c|c|}}
		\hline
		@ {\textbf{Disciplina}} & @ {\textbf{CH}} & @ {\textbf{Créditos}} \\
		\hline
		@ \EngSistC 	& \EngSistCCH		& \EngSistCCred		\\ % Análise e Projeto de Sistemas
		@ \Control		& \ControlCH		& \ControlCred		\\ % Controle de Processos
		@ \CompParal	& \CompParalCH		& \CompParalCred	\\ % Computação Paralela
		@ \Sredes 		& \SredesCH			& \SredesCred		\\ % Segurança em Redes
		@ \SistEmb		& \SistEmbCH		& \SistEmbCred		\\ % Sistemas Embutidos
		@ \Empre 		& \EmpreCH			& \EmpreCred		\\ % Empreendedorismo
		\hline
		@ Total				& sum(b2:b7)			& sum(c2:c7)			\\
		\hline
	\end{spreadtab}
\end{table}

\rowcolors{1}{gray!5}{white}
\begin{table}
	\centering
	\caption{9\textordmasculine~Período}
	\label{tab9p}
	\begin{spreadtab}{{tabularx}{\textwidth}{|X|c|c|}}
		\hline
		@ {\textbf{Disciplina}} & @ {\textbf{CH}} & @ {\textbf{Créditos}} \\
		\hline
		@ \EletA		& \EletACH		& \EletACred	\\ % Disciplina Eletiva A
		@ \EstSup		& \EstSupCH		& \EstSupCred	\\ % Estágio Supervisionado
		@ \ProjA		& \ProjACH		& \ProjACred	\\ % Metodologia Ciêntífica
		@ \Instala 		& \InstalaCH	& \InstalaCred	\\ % Instalações de Ambientes Computacionais
		\hline
		@ Total			& sum(b2:b5)	& sum(c2:c5)	\\
		\hline
	\end{spreadtab}
\end{table}

\begin{table}
	\centering
	\caption{10\textordmasculine~Período}
	\label{tab10p}
	\begin{spreadtab}{{tabularx}{\textwidth}{|X|c|c|}}
		\hline
		@ {\textbf{Disciplina}} & @ {\textbf{CH}} & @ {\textbf{Créditos}} \\
		\hline
		@ \EletB	& \EletBCH	& \EletBCred	\\ % Disciplina Eletiva B
		@ \EletC	& \EletCCH	& \EletCCred	\\ % Disciplina Eletiva C
		@ \ProjB	& \ProjBCH	& \ProjBCred	\\ % Projeto de Graduação XI
		@ \Adm		& \AdmCH	& \AdmCred		\\ % Administração
		\hline
		@ Total		& sum(b2:b5)& sum(c2:c5)	\\
		\hline
	\end{spreadtab}
\end{table}

\begin{table}
	\centering
	\caption{Disciplinas Eletivas Restritas}
	\label{tabeletivas}
	\begin{spreadtab}{{tabularx}{\textwidth}{|X|c|c|}}
		\hline
		@ {\textbf{Disciplina}} & @ {\textbf{CH}} & @ {\textbf{Créditos}} \\
		\hline
		@ \EletArq	& \EletArqCH	& \EletArqCred	\\
		@ \EletGeo	& \EletGeoCH	& \EletGeoCred	\\
		@ \EletPadroes	& \EletPadroesCH	& \EletPadroesCred	\\
		@ \EletRec	& \EletRecCH	& \EletRecCred	\\
		@ \EletRedes	& \EletRedesCH& \EletRedesCred	\\
		@ \EletMov	& \EletMovCH	& \EletMovCred	\\
		\hline
	\end{spreadtab}
\end{table}

