\chapter{Dados Gerais da Unidade Acadêmica - \textcolor{red}{Rafaela}}


\section{Identificação Geral}

A instituição de educação superior mantenedora do curso de Engenharia de Computação é a Universidade do Estado do Rio de Janeiro que exercerá, por meio da Pró-reitoria de Graduação, a administração acadêmica do curso.

A Universidade do Estado do Rio de Janeiro é organizada em: Administração Central, Centros Setoriais, Unidades Acadêmicas e Departamentos. Todos estão diretamente ligados à Reitoria, por meio de órgãos de assessoria que são: Vice-reitoria, Pró-reitoria de Graduação, Pró-reitoria de Pós-graduação e Pesquisa, Pró-reitoria de Extensão e Cultura, Pró-reitoria de Políticas e Assistência Estudantis, Pró-reitoria de Saúde e a Superintendência de Gestão de Pessoas.


\begin{description}
	\item[Mantenedor:] Governo do Estado do Rio de Janeiro.
	\item [Mantida:] Faculdade de Engenharia -- Universidade do Estado do Rio de Janeiro.
	\item [Local de Funcionamento:] Rua São Francisco Xavier 524, 5$^{o}$ andar, Maracanã, Rio de Janeiro -- RJ.
	\item [Reitora:] Gulnar Azevedo e Silva.
	\item [Vice-Reitor:] Bruno Rêgo Deusdará Rodrigues.
	\item [Pró-Reitor de Graduação (PR1):] Antonio Soares da Silva.
	\item [Pró-Reitora de Pós-graduação e Pesquisa (PR2):] Elizabeth Fernandes de Macedo.
	\item [Pró-Reitora de Extensão e Cultura (PR3):] Ana Maria de Almeida Santiago.
	\item [Pró-Reitora de Políticas e Assistência Estudantis (PR4):] Daniel Pinha Silva.
	\item [Pró-Reitora de Saúde (PR5):] Ronaldo Damião.
	\item [Diretora do Centro de Tecnologia e Ciências:] Nádia Pimenta Lima.
	\item [Diretora da Faculdade de Engenharia:]  Maria Eugênia Gouvêa.
\end{description}

\subsection{Centro de Vinculação}

Na Universidade do Estado do Rio de Janeiro existem quatro Centros Setoriais: Centro Biomédico, Centro de Ciências Sociais, Centro de Educação e Humanidades e Centro de Tecnologia e Ciências.

O curso de Engenharia de Computação está vinculado ao Centro de Tecnologia e Ciências (CTC).
O CTC é um órgão com função deliberativa e executiva destinado a coordenar e integrar as atividades afins de ensino, pesquisa e extensão nas suas áreas de atuação. Coordena 11 unidades acadêmicas e atualmente possui 732 docentes efetivos o que corresponde a 26\% dos docentes da UERJ. Destes, 83\% possuem título de doutor. Conta com 258 técnico-administrativos e 8506 alunos ativos de graduação e pós-graduação. Congrega 20 cursos de graduação e 27 cursos de pós-graduação. As unidades acadêmicas ligadas ao CTC são as seguintes:

\begin{itemize}

	\item Escola Superior de Desenho Industrial -- ESDI
	\item Faculdade de Ciências Exatas e Engenharias -- FCEE
	\item Faculdade de Engenharia -- FEN
	\item Faculdade de Geologia -- FGEL
	\item Faculdade de Oceanografia -- FAOC
	\item Faculdade de Tecnologia -- FAT
	\item Instituto de Física Armando Dias Tavares -- IFADT
	\item Instituto de Geografia -- IGEOG
	\item Instituto de Matemática e Estatística -- IME
	\item Instituto Politécnico -- IPRJ
	\item Instituto de Química -- QUI

\end{itemize}

\section{Identificação da Unidade Acadêmica}

A Faculdade de Engenharia (FEN) é uma unidade do Centro de Tecnologia e Ciências que oferece na graduação as habilitações de Engenharia Elétrica, Civil, Sanitária e Ambiental, Mecânica, de Produção, Cartográfica, além dos cursos de pós-graduação em níveis lato sensu e stricto sensu, não listados.

A Faculdade de Engenharia é constituída por nove departamentos, com representação no Conselho Departamental:

\begin{itemize}
	\item Departamento de Engenharia Cartográfica -- CARTO
	\item Departamento de Construção Civil e Transportes -- DCCT
	\item Departamento de Engenharia Elétrica -- DEE
	\item Departamento de Eletrônica e Telecomunicações -- DETEL
	\item Departamento de Estruturas e Fundações -- ESTR
	\item Departamento de Engenharia Mecânica -- MECAN
	\item Departamento de Engenharia Industrial -- DEIN
	\item Departamento de Engenharia Sanitária e Meio Ambiente -- DESMA
	\item Departamento de Engenharia de Sistemas e Computação -- DESC
\end{itemize}

A Tabela~\ref{tabvagas} apresenta as vagas oferecidas para o vestibular pela Faculdade de Engenharia da Universidade do Estado do Rio de Janeiro, campus Maracanã, para o primeiro e para o segundo semestre.
\begin{table}
	\centering
	\caption{Vagas oferecidas no 1\textordmasculine{} e no 2\textordmasculine{} semestre}
	\label{tabvagas}
	\begin{tabularx}{\textwidth}{|X|c|c|c|c|}
		\hline
		\multirow{2}{*}{\textbf{Habilitação}}                 & \multirow{2}{*}{\textbf{Turno}} & \multicolumn{2}{c|}{\textbf{Vagas}} & \multirow{2}{*}{\textbf{Total}}          \\
		\cline{3-4}                                           &                                 & \textbf{1\textordmasculine{} Sem.}  & \textbf{2\textordmasculine{} Sem.} &     \\
		\hline
		Engenharia Ambiental e Sanitária                      & Manhã/Tarde                     & 40                                  & --                                 &     \\
		                                                      & Tarde/Noite                     & --                                  & 40                                 & 80  \\
		\hline
		Engenharia Cartográfica                               & Manhã/Tarde                     & 20                                  & --                                 &     \\
		                                                      & Tarde/Noite                     & --                                  & 20                                 & 40  \\
		\hline
		Engenharia Civil                                      & Manhã/Tarde                     & 60                                  & --                                 &     \\
		(Construção Civil/Transportes/Estruturas)             & Tarde/Noite                     & --                                  & 60                                 & 120 \\
		\hline
		Engenharia de Produção                                & Manhã/Tarde                     & 40                                  & --                                 &     \\
		                                                      & Tarde/Noite                     & --                                  & 40                                 & 80  \\
		\hline
		Engenharia Elétrica (Sistemas e Computação/           & Manhã/Tarde                     & 100                                 & --                                 &     \\
		Sist. de Potência/Sist. Eletrônicos/Telecomunicações) & Tarde/Noite                     & --                                  & 100                                & 200 \\
		\hline
		Engenharia Mecânica                                   & Manhã/Tarde                     & 40                                  & --                                 &     \\
		                                                      & Tarde/Noite                     & --                                  & 40                                 & 80  \\
		\hline
	\end{tabularx}
\end{table}
