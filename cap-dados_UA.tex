\chapter{Dados Gerais da Instituição e do Curso}
\thispagestyle{plain}


\section{Identificação Geral}

A instituição de educação superior mantenedora do curso de Engenharia de Computação é a Universidade do Estado do Rio de Janeiro que exercerá, por meio da Pró-reitoria de Graduação, a administração acadêmica do curso.

A Universidade do Estado do Rio de Janeiro é organizada em: Administração Central, Centros Setoriais, Unidades Acadêmicas e Departamentos. Todos estão diretamente ligados à Reitoria, por meio de órgãos de assessoria que são: Vice-reitoria, Pró-reitoria de Graduação, Pró-reitoria de Pós-graduação e Pesquisa, Pró-reitoria de Extensão e Cultura, Pró-reitoria de Políticas e Assistência Estudantis, Pró-reitoria de Saúde e a Superintendência de Gestão de Pessoas.

\subsection{Centro de Vinculação}

Na Universidade do Estado do Rio de Janeiro existem quatro Centros Setoriais: Centro Biomédico, Centro de Ciências Sociais, Centro de Educação e Humanidades e Centro de Tecnologia e Ciências.

O curso de Engenharia de Computação está vinculado ao Centro de Tecnologia e Ciências (CTC).
O CTC é um órgão com função deliberativa e executiva destinado a coordenar e integrar as atividades afins de ensino, pesquisa e extensão nas suas áreas de atuação. Coordena 11 unidades acadêmicas e atualmente possui 732 docentes efetivos o que corresponde a 26\% dos docentes da UERJ. Destes, 83\% possuem título de doutor. Conta com 258 técnico-administrativos e 8506 alunos ativos de graduação e pós-graduação. Congrega 20 cursos de graduação e 27 cursos de pós-graduação. As unidades acadêmicas ligadas ao CTC são as seguintes:

\begin{itemize}

	\item Escola Superior de Desenho Industrial -- ESDI
	\item Faculdade de Ciências Exatas e Engenharias -- FCEE
	\item Faculdade de Engenharia -- FEN
	\item Faculdade de Geologia -- FGEL
	\item Faculdade de Oceanografia -- FAOC
	\item Faculdade de Tecnologia -- FAT
	\item Instituto de Física Armando Dias Tavares -- IFADT
	\item Instituto de Geografia -- IGEOG
	\item Instituto de Matemática e Estatística -- IME
	\item Instituto Politécnico -- IPRJ
	\item Instituto de Química -- QUI

\end{itemize}

\section{Identificação da Unidade Acadêmica}

A Faculdade de Engenharia (FEN) é uma unidade do Centro de Tecnologia e Ciências que oferece na graduação as habilitações de Engenharia Elétrica, Civil, Sanitária e Ambiental, Mecânica, de Produção, Cartográfica, além dos cursos de pós-graduação em níveis lato sensu e stricto sensu, não listados.

A Faculdade de Engenharia é constituída por nove departamentos, com representação no Conselho Departamental:

\begin{itemize}
	\item Departamento de Engenharia Cartográfica -- CARTO
	\item Departamento de Construção Civil e Transportes -- DCCT
	\item Departamento de Engenharia Elétrica -- DEE
	\item Departamento de Eletrônica e Telecomunicações -- DETEL
	\item Departamento de Estruturas e Fundações -- ESTR
	\item Departamento de Engenharia Mecânica -- MECAN
	\item Departamento de Engenharia Industrial -- DEIN
	\item Departamento de Engenharia Sanitária e Meio Ambiente -- DESMA
	\item Departamento de Engenharia de Sistemas e Computação -- DESC
\end{itemize}

\section{Cursos oferecidos pela Faculdade de Engenharia}

A Tabela~\ref{tabvagas} apresenta a distribuição atual de vagas oferecidas no vestibular da Faculdade de Engenharia, campus Maracanã, para o primeiro e o segundo semestres letivos. No caso específico da habilitação em Engenharia Elétrica, são disponibilizadas 200 vagas anuais, sendo 100 para ingresso no primeiro semestre e 100 para o segundo.

A habilitação de Engenharia Elétrica está estruturado em cinco ênfases: Eletricidade Industrial, Sistemas de Potência, Sistemas Eletrônicos, Telecomunicações e Sistemas e Computação. Dessa forma, a média de vagas disponíveis por ênfase é de aproximadamente 20 por semestre, embora a escolha da ênfase pelos estudantes ocorra em momento posterior à admissão.


\begin{table}[!ht]
	\centering
	\caption{Vagas oferecidas no 1\textordmasculine{} e no 2\textordmasculine{} semestre}
	\label{tabvagas}
	\begin{tabularx}{\textwidth}{>{\raggedright\arraybackslash}Xcccc}
		\hline
		\rowcolor{gray!20}
		\textbf{Habilitação}                                                                                                                                    & \textbf{Turno} & \textbf{1\textordmasculine{} Sem.} & \textbf{2\textordmasculine{} Sem.} & \textbf{Total}       \\
		\hline
		\multirow{2}{=}{Engenharia Ambiental e Sanitária}                                                                                                       & Manhã/Tarde    & 40                                 & --                                 & \multirow{2}{*}{80}  \\
		                                                                                                                                                        & Tarde/Noite    & --                                 & 40                                 &                      \\
		\hline
		\multirow{2}{=}{Engenharia Cartográfica}                                                                                                                & Manhã/Tarde    & 20                                 & --                                 & \multirow{2}{*}{40}  \\
		                                                                                                                                                        & Tarde/Noite    & --                                 & 20                                 &                      \\
		\hline
		\multirow{2}{=}{Engenharia Civil (Construção Civil / Transportes / Estruturas)}                                                                         & Manhã/Tarde    & 60                                 & --                                 & \multirow{2}{*}{120} \\
		                                                                                                                                                        & Tarde/Noite    & --                                 & 60                                 &                      \\
		\hline
		\multirow{2}{=}{Engenharia de Produção}                                                                                                                 & Manhã/Tarde    & 40                                 & --                                 & \multirow{2}{*}{80}  \\
		                                                                                                                                                        & Tarde/Noite    & --                                 & 40                                 &                      \\
		\hline
		\multirow{2}{=}{Engenharia Elétrica (Eletricidade Industrial / Sistemas de Potência / Sistemas Eletrônicos / Telecomunicações / Sistemas e Computação)} & Manhã/Tarde    & 100                                & --                                 & \multirow{2}{*}{200} \\
		                                                                                                                                                        & Tarde/Noite    & --                                 & 100                                &                      \\
		\hline
		\multirow{2}{=}{Engenharia Mecânica}                                                                                                                    & Manhã/Tarde    & 40                                 & --                                 & \multirow{2}{*}{80}  \\
		                                                                                                                                                        & Tarde/Noite    & --                                 & 40                                 &                      \\
		\hline
	\end{tabularx}
\end{table}

