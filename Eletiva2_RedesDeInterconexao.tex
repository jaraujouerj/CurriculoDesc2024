%----------------------------------------------------------------------------------------
%	Latex para ementa tipo dep da uerj
%----------------------------------------------------------------------------------------
%apagar .aux e executar biber ementalatex para obter bibliografia atualizada
\documentclass{ementa} % Use A4 paper with a 12pt font size 

\begin{document}
\ndisciplina 		{\EletRedes} 
\chta 				{\EletRedesCH h}%carga horária total aluno
\nomeprofessor		{\luiza}
\matriculaprofessor	{\matluiza}
\naoobrigatoria		 %
\eletivarestrita	{Engenharia de Computação}
\eletivadefinida 	{Engenharia Elétrica/Eletrônica e Elétrica/Telecomunicações}%disciplina é eletiva definida
\preum				{\ArqComp} %
\codpreum			{\ArqCompCod}
\travadecreditos   	{170}
\credteorica		{2} %
\credlab			{2} %
\semipresencial
\objetivos			{Ao final do período, o aluno deverá ter assimilado as características lógicas e físicas das 
					redes de interconexão em sistemas computacionais paralelos.}
\ementa				{Computação paralela e redes. Arquiteturas de computadores paralelos. Considerações sobre 
					o projeto de redes de interconexão. Classificação das redes de interconexão. Redes de meio 
					compartilhado. Redes diretas. Redes indiretas. Redes híbridas. Chaveamento de mensagem. 
					Algoritmos de roteamento.}
\formementa{Eletiva2_RedesDeInter}
\end{document}
