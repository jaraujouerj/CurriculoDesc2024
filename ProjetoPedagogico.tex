 %!TEX program = lualatex
\documentclass[envcountsame,envcountchap,openany]{svmono}

% Pacotes de formatação e layout
\usepackage[top=1in,left=1.5in,bottom=1in,right=2.5cm]{geometry}
\usepackage{fancyhdr}
\usepackage{makeidx}         % permite geracao de indice
\usepackage{graphicx}        % ferramenta padrao latex
\usepackage{multicol}        % usada para fazer múltiplas colunas
\usepackage[bottom]{footmisc}% coloca rodapés no final da página
\usepackage{placeins}		 % para forçar o float a ficar na seção
\usepackage{pdfpages}		 % para incluir pdfs
\usepackage{tocloft} 		 % para formatar o sumário
\setlength\cftparskip{-2pt}
\setlength\cftbeforechapskip{0pt}
\usepackage[titletoc,toc,page]{appendix} % para anexos
\usepackage{scrextend}		 % para adicionar margens

% Pacotes de tabelas e cores
\usepackage[table]{xcolor}  % para tabelas coloridas
\usepackage{tabularx}		% para tabelas com largura automática
\usepackage{longtable}		% para tabelas longas
\usepackage{spreadtab}	    % para tabelas com cálculos
\usepackage{booktabs}		% para tabelas com linhas horizontais
\usepackage{multirow}		 % para tabelas com multiplas linhas

% Pacotes de texto e fontes
\usepackage[T1]{fontenc} 	% para acentuação correta
\usepackage[portuguese]{babel} % para português
\usepackage{enumerate}		% para listas enumeradas
\usepackage{textcomp}		% para símbolos de texto
\usepackage{xspace}			% para espaçamento automático

% Pacotes de gráficos e desenhos
\usepackage{tikz}			% para gráficos e desenhos
\usetikzlibrary{calc}		% para cálculos em gráficos

% Pacote para posicionamento de texto
\usepackage[absolute]{textpos}

% Pacotes de hiperlinks
\usepackage{hyperref}
\hypersetup{
	colorlinks,
	citecolor=blue,
	filecolor=magenta,
	linkcolor=black, %era red
	urlcolor=cyan
}

% Configuração de cabeçalhos e rodapés
\pagestyle{fancy}
\lhead{}
\chead{}
\renewcommand{\headrule}{}
\rhead{{\footnotesize \thepage}}
\fancyfoot{}
\addtolength{\headwidth}{0.3\marginparwidth}

% Configuração de índices e listas
\makeindex             % para gerar índice

% Configuração de anexos
\renewcommand{\appendixtocname}{Anexos}
\renewcommand{\appendixpagename}{Anexos}

% Definição de comandos personalizados

% Nomes das Disciplinas
\usepackage{disciplinasDB}

% Definição de ordinal masculino e feminino
\usepackage{etoolbox} % para garantir compatibilidade em várias situações

\DeclareRobustCommand{\subsup}[1]{%
  \textsuperscript{%
    \tikz[baseline=(char.base)]{
      \node[inner sep=0.5pt, anchor=base] (char) {\textnormal{#1}};
      \draw[line width=0.3pt, yshift=-0.2ex]
        ($(char.south west)!0.15!(char.south east)$) --
        ($(char.south west)!0.85!(char.south east)$);
    }%
  }%
}

\DeclareRobustCommand{\ordm}[1]{#1\subsup{o}} % ordinal masculino
\DeclareRobustCommand{\ordf}[1]{#1\subsup{a}} % ordinal feminino

\newcounter{artigo}
\newcommand{\artigo}{\refstepcounter{artigo} % 
	\ifnum\theartigo<10 %
	{\bfseries Art.~\arabic{artigo}º~--~}%
	\else
	{\bfseries Art. \arabic{artigo}~--~}%
	\fi
	}
\newcounter{paragrafo}
\newcommand{\paragrafo}{\refstepcounter{paragrafo} % 
	\S~\arabic{paragrafo}º~%
}

\newenvironment{paragrafos}{\setcounter{paragrafo}{0}
	\setlength{\parindent}{0pt}
	\begin{addmargin}[4em]{0pt} 
	}
	{\end{addmargin}}

\newenvironment{itquotation}
{\begin{quotation}\itshape}
{\end{quotation}}	
% No paragraph indentation
\parindent0pt
\setlength{\parskip}{0.8\baselineskip}

\normalsize 
\setlength{\headheight}{15pt}
\newcounter{thobrigatorias}  
\setcounter{thobrigatorias}{3210}
\newcommand{\hobrigatorias}{\the\value{thobrigatorias}\xspace}
\newcounter{thextensao}
\setcounter{thextensao}{357}
\newcounter{thtotal}
\setcounter{thtotal}{\the\value{thobrigatorias}}
\addtocounter{thtotal}{\the\value{thextensao}}
\newcommand{\vagas}{50\xspace}
\newcommand{\desc}{Departamento de Engenharia de Sistemas e Computação\xspace}
\newcommand{\totalhoras}{\the\value{thtotal}\xspace}
\newcommand{\ndisciplinas}{55\xspace}
\newcommand{\nobrigatorias}{52\xspace}
\newcommand{\neletivas}{3\xspace}
\newcommand{\hextensao}{\the\value{thextensao}\xspace}
\begin{document}
%\author {Departamento de Engenharia de Sistemas e Computação}

\begin{textblock*}{3cm}[0.5,0.5](4cm,4cm)
	\includegraphics[width=3cm]{imagens/logo_uerj_cor.jpg}
\end{textblock*}
\begin{textblock*}{12cm}(6cm,3cm)   % 6.375=8.5 - 1.5 - 0.625
	Universidade do Estado do Rio de Janeiro\\
	Sub-Reitoria de Graduação \\
	Faculdade de Engenharia\\
	Departamento de Engenharia de Sistemas e Computação
\end{textblock*}

\title{Projeto Pedagógico do Curso de Engenharia de Computação}
\subtitle{2025}
\date{}
\maketitle
\pagestyle{empty}
\vfill
\pagestyle{fancy}
\pagenumbering{roman}
\tableofcontents
\listoftables

\frontmatter%%%%%%%%%%%%%%%%%%%%%%%%%%%%%%%%%%%%%%%%%%%%%%%%%%%%%%

\mainmatter%%%%%%%%%%%%%%%%%%%%%%%%%%%%%%%%%%%%%%%%%%%%%%%%%%%%%%%
\pagestyle{fancy}
\chapter{Introdução}
\label{intro} % Always give a unique label
% use \chaptermark{}
% to alter or adjust the chapter heading in the running head

Este documento apresenta o projeto da estrutura curricular do novo curso de Engenharia de Computação da Universidade do Estado do Rio de Janeiro. Este curso será oferecido pelo Departamento de Engenharia de Sistemas e Computação (DESC) da Faculdade de Engenharia (FEN). Desde 1977, o DESC oferece o curso de Engenharia Elétrica com Ênfase em Sistemas e Computação, sendo o primeiro no Brasil a oferecer uma graduação na área de Engenharia de Computação. O curso passou por uma reformulação significativa no início da década de 1990 e, desde então, manteve o mesmo currículo, sem alterações. O objetivo agora é promover uma reforma deste curso e oferecê-lo como uma nova habilitação em engenharia.

A Engenharia, especialmente a Engenharia de Computação, tem evoluído a passos largos no cenário contemporâneo. Com mais de três décadas desde a última atualização curricular, torna-se imperativo adaptar o curso às exigências atuais, alinhando-o com as demandas da sociedade por avanços científicos e tecnológicos nos setores industrial e de serviços. Atualmente, em 2025, os candidatos aos cursos da Faculdade de Engenharia da UERJ, incluindo a Engenharia Elétrica, enfrentam um dilema: ao optarem pela Engenharia Elétrica com Ênfase em Sistemas e Computação, recebem um diploma de Engenheiro Eletricista, uma designação que não reflete adequadamente a especialização adquirida, especialmente considerando a necessidade crescente de conhecimentos aprofundados em desenvolvimento de sistemas e dispositivos computacionais, conforme orientado pelas diretrizes do Exame Nacional de Desempenho dos Estudantes (ENADE).

A proposta de introduzir o curso de \textbf{Engenharia de Computação} como uma nova habilitação visa não apenas modernizar o currículo em resposta às transformações tecnológicas, mas também proporcionar uma formação mais alinhada com as competências exigidas no campo da computação. O atual curso de Engenharia Elétrica com Ênfase em Sistemas e Computação, embora prepare os alunos para desenvolver software e projetar hardware, não os capacita adequadamente para projetar sistemas de potência, uma expectativa comum para engenheiros eletricistas. Essa discrepância tem levado a dificuldades para os alunos no ENADE, evidenciando a lacuna entre a formação oferecida e as competências específicas requeridas na área elétrica.

Portanto, a criação de um curso dedicado à Engenharia de Computação, como uma habilitação distinta, é uma resposta estratégica às necessidades educacionais emergentes, garantindo que os egressos estejam melhor preparados para enfrentar os desafios tecnológicos do futuro. Este projeto pedagógico não apenas reflete uma atualização necessária diante das mudanças tecnológicas desde os anos 1990, mas também reafirma o compromisso da UERJ com a excelência educacional e a inovação no ensino de engenharia.
\chapter{Dados Gerais da Unidade Acadêmica - \textcolor{red}{Rafaela}}


\section{Identificação Geral}

A instituição de educação superior mantenedora do curso de Engenharia de Computação é a Universidade do Estado do Rio de Janeiro que exercerá, por meio da Pró-reitoria de Graduação, a administração acadêmica do curso.

A Universidade do Estado do Rio de Janeiro é organizada em: Administração Central, Centros Setoriais, Unidades Acadêmicas e Departamentos. Todos estão diretamente ligados à Reitoria, por meio de órgãos de assessoria que são: Vice-reitoria, Pró-reitoria de Graduação, Pró-reitoria de Pós-graduação e Pesquisa, Pró-reitoria de Extensão e Cultura, Pró-reitoria de Políticas e Assistência Estudantis, Pró-reitoria de Saúde e a Superintendência de Gestão de Pessoas.


\begin{description}
	\item[Mantenedor:] Universidade do Estado do Rio de Janeiro.
	\item [Curso:] Engenharia de Computação.
	\item [Local de Funcionamento:] Rua São Francisco Xavier 524, 5$^{o}$ andar, Maracanã, Rio de Janeiro -- RJ.
\end{description}

\subsection{Centro de Vinculação}

Na Universidade do Estado do Rio de Janeiro existem quatro Centros Setoriais: Centro Biomédico, Centro de Ciências Sociais, Centro de Educação e Humanidades e Centro de Tecnologia e Ciências.

O curso de Engenharia de Computação está vinculado ao Centro de Tecnologia e Ciências (CTC).
O CTC é um órgão com função deliberativa e executiva destinado a coordenar e integrar as atividades afins de ensino, pesquisa e extensão nas suas áreas de atuação. Coordena 11 unidades acadêmicas e atualmente possui 732 docentes efetivos o que corresponde a 26\% dos docentes da UERJ. Destes, 83\% possuem título de doutor. Conta com 258 técnico-administrativos e 8506 alunos ativos de graduação e pós-graduação. Congrega 20 cursos de graduação e 27 cursos de pós-graduação. As unidades acadêmicas ligadas ao CTC são as seguintes:

\begin{itemize}

	\item Escola Superior de Desenho Industrial -- ESDI
	\item Faculdade de Ciências Exatas e Engenharias -- FCEE
	\item Faculdade de Engenharia -- FEN
	\item Faculdade de Geologia -- FGEL
	\item Faculdade de Oceanografia -- FAOC
	\item Faculdade de Tecnologia -- FAT
	\item Instituto de Física Armando Dias Tavares -- IFADT
	\item Instituto de Geografia -- IGEOG
	\item Instituto de Matemática e Estatística -- IME
	\item Instituto Politécnico -- IPRJ
	\item Instituto de Química -- QUI

\end{itemize}

\section{Identificação da Unidade Acadêmica}

A Faculdade de Engenharia (FEN) é uma unidade do Centro de Tecnologia e Ciências que oferece na graduação as habilitações de Engenharia Elétrica, Civil, Sanitária e Ambiental, Mecânica, de Produção, Cartográfica, além dos cursos de pós-graduação em níveis lato sensu e stricto sensu, não listados.

A Faculdade de Engenharia é constituída por nove departamentos, com representação no Conselho Departamental:

\begin{itemize}
	\item Departamento de Engenharia Cartográfica -- CARTO
	\item Departamento de Construção Civil e Transportes -- DCCT
	\item Departamento de Engenharia Elétrica -- DEE
	\item Departamento de Eletrônica e Telecomunicações -- DETEL
	\item Departamento de Estruturas e Fundações -- ESTR
	\item Departamento de Engenharia Mecânica -- MECAN
	\item Departamento de Engenharia Industrial -- DEIN
	\item Departamento de Engenharia Sanitária e Meio Ambiente -- DESMA
	\item Departamento de Engenharia de Sistemas e Computação -- DESC
\end{itemize}

A Tabela~\ref{tabvagas} apresenta as vagas atualmente oferecidas para o vestibular pela Faculdade de Engenharia da Universidade do Estado do Rio de Janeiro, campus Maracanã, para o primeiro e para o segundo semestre.
\begin{table}
	\centering
	\caption{Vagas oferecidas no 1\textordmasculine{} e no 2\textordmasculine{} semestre}
	\label{tabvagas}
	\begin{tabularx}{\textwidth}{|X|c|c|c|c|}
		\hline
		\multirow{2}{*}{\textbf{Habilitação}}                 & \multirow{2}{*}{\textbf{Turno}} & \multicolumn{2}{c|}{\textbf{Vagas}} & \multirow{2}{*}{\textbf{Total}}          \\
		\cline{3-4}                                           &                                 & \textbf{1\textordmasculine{} Sem.}  & \textbf{2\textordmasculine{} Sem.} &     \\
		\hline
		Engenharia Ambiental e Sanitária                      & Manhã/Tarde                     & 40                                  & --                                 &     \\
		                                                      & Tarde/Noite                     & --                                  & 40                                 & 80  \\
		\hline
		Engenharia Cartográfica                               & Manhã/Tarde                     & 20                                  & --                                 &     \\
		                                                      & Tarde/Noite                     & --                                  & 20                                 & 40  \\
		\hline
		Engenharia Civil                                      & Manhã/Tarde                     & 60                                  & --                                 &     \\
		(Construção Civil/Transportes/Estruturas)             & Tarde/Noite                     & --                                  & 60                                 & 120 \\
		\hline
		Engenharia de Produção                                & Manhã/Tarde                     & 40                                  & --                                 &     \\
		                                                      & Tarde/Noite                     & --                                  & 40                                 & 80  \\
		\hline
		Engenharia Elétrica (Sistemas e Computação/           & Manhã/Tarde                     & 100                                 & --                                 &     \\
		Sist. de Potência/Sist. Eletrônicos/Telecomunicações) & Tarde/Noite                     & --                                  & 100                                & 200 \\
		\hline
		Engenharia Mecânica                                   & Manhã/Tarde                     & 40                                  & --                                 &     \\
		                                                      & Tarde/Noite                     & --                                  & 40                                 & 80  \\
		\hline
	\end{tabularx}
\end{table}

\chapter{Identificação do curso - \textcolor{red}{Luigi}}

\section{Denominação}

O curso de Engenharia de Computação, ora proposto, está inserido no Departamento de Engenharia de Sistemas e Computação (DESC) e se vincula à Faculdade de Engenharia – FEN, pelo organograma geral da UERJ. Funcionará à Rua São Francisco Xavier 524, Pavilhão João Lyra Filho, quinto andar, bloco D, Maracanã, Rio de Janeiro – RJ.

O DESC (Departamento de Engenharia de Sistemas e Computação) da Faculdade de Engenharia da UERJ forma graduados em nível superior pleno da engenharia, com conhecimento técnico-científico abrangente e forte para atuação no desenvolvimento de software e hardware, tendo predominantemente a computação como atividade fim. Destacam-se as seguintes áreas de atuação:

\begin{enumerate}
	\item Concepção, projeto e análise de sistemas, produtos e processos computacionais;
	\item Planejamento, supervisão, elaboração e coordenação de projetos e serviços de engenharia de computação;
	\item Identificação, formulação e resolução de problemas de engenharia de computação.
\end{enumerate}

A criação do curso de Engenharia de Computação a ser oferecido pelo Departamento de Engenharia de Sistemas e Computação vem substituir o curso de Engenharia Elétrica com ênfase em Sistemas e Computação.

\section{Bases Legais e Normas}

A seguinte documentação pertinente a este processo de reforma curricular está apensada nas páginas a seguir:

\section{Duração, Regime e Tempo de Integralização Curricular}

Olhar PP Elétrica pagina 365

\section{Local, Turno e Horário de Funcionamento}

Olhar PP Elétrica pagina 374

\section{Número de Turmas e Vagas}

Olhar PP Elétrica pagina 374

\section{Número de Alunos e Número de Docentes}

Olhar PP Elétrica pagina 374

\section{Formas de Ingresso}

Olhar PP Elétrica pagina 375

Olhar PP Engenharia Quimica pagina 10




% !TEX root = ProjetoPedagogico.tex

\chapter{Organização Didático Pedagógica}

O curso ora proposto de Engenharia de Computação obedecerá ao regime de créditos, oferecendo \vagas vagas anuais, repartidas igualmente em dois semestres letivos. O aluno interessado em cursar a graduação em Engenharia de Computação fará tal opção diretamente a partir da sua inscrição no vestibular.




\section{Justificativa das Necessidades Sociais do Curso -\textcolor{red}{Gabriel}}

Olhar PP da Engenharia Eletrica pagina 376

\section{Finalidades e Objetivos do Curso - \textcolor{red}{Simone}}

\subsection{Histórico}

\subsection{Concepção}

A estrutura curricular do curso de Engenharia de Computação do \desc da Faculdade de Engenharia da UERJ orientar-se-á pelas \textit{Diretrizes Curriculares Nacionais para os cursos de graduação na área da Computação}, do Ministério da Educação (anexo \ref{cne}), pelos \textit{Referenciais de Formação para os cursos de Graduação em Computação}, da Sociedade Brasileira de Computação (anexo \ref{sbc}), e pela regulamentação do exercício da profissão de Engenheiro, estabelecida pelo Sistema CREA/CONFEA (Resolução 1.010 CONFEA, anexo \ref{res1010}), em vigor atualmente. A inserção da extensão no curriculo terá como base a Deliberação n\textordmasculine{} 4/2023 do Conselho Superior de Ensino, Pesquisa e Extensão (CSEPE/UERJ, anexo \ref{del4}).

A grade curricular totaliza \totalhoras horas, sendo \hobrigatorias horas em disciplinas e \hextensao horas em atividades de extensão. As \hobrigatorias horas de disciplinas estão distribuídas em \ndisciplinas disciplinas, sendo \nobrigatorias  obrigatórias e \neletivas eletivas restritas. Além das disciplinas teóricas, o curso inclui práticas laboratoriais para complementar a base teórica. O currículo também contempla Estágio Supervisionado e Projeto de Graduação (trabalho de conclusão de curso) como atividades de síntese e integração do conhecimento científico, tecnológico e instrumental. Como atividades acadêmicas complementares facultativas, os alunos podem optar por Estágio Interno, Monitoria, Iniciação Científica, Cursos, Eventos, Palestras e Visitas Técnicas, que visam proporcionar uma melhor compreensão da Engenharia, do setor no Brasil e das áreas de atuação e atividades dos Engenheiros de Computação.

\subsection{Finalidades}

\subsection{Objetivos}

\section{Nível de Formação e Título Acadêmico}

O curso é de graduação plena e a titulação concedida e habilitação são:

\begin{itemize}
\item{Título: Engenheiro}
\item{Habilitação: Engenharia de Computação}
\end{itemize}

\section{Perfil do Egresso (competência, habilidades e atitudes pretendidas) - \textcolor{red}{Gabriel}}

O curso de Engenharia de Computação tem como perfil do egresso o engenheiro, com formação técnico-científica sólida, generalista, humanista, crítica e reflexiva, capacitado a absorver e desenvolver novas tecnologias, estimulando a sua atuação crítica e criativa na identificação e resolução de problemas, considerando seus aspectos políticos, econômicos, sociais, ambientais e culturais, com visão ética e humanística, em atendimento às demandas da sociedade. Faz parte do perfil do egresso a postura de permanente busca da atualização profissional, além das seguintes habilidades:
\begin{enumerate} [I -]
	\item possuir conhecimento das questões humanísticas, sociais, ambientais, éticas, profissionais, legais e políticas;
	\item possuir compreensão do impacto da Engenharia de Computação e suas tecnologias no que concerne ao atendimento e à antecipação estratégica das necessidades da sociedade;
	\item possuir atitude crítica, interdisciplinar e criativa na identificação e resolução de problemas;
	\item possuir compreensão das necessidades de contínua atualização e aprimoramento de suas competências e habilidades;
	\item possuir uma sólida formação em Computação, Física, Matemática, Eletrônica, Automação e Telecomunicações.
	\item conhecer a estrutura dos sistemas de computação e os processos envolvidos na sua análise e construção;
	\item considerar os aspectos ambientais, econômicos, financeiros, de gestão e de qualidade, associados a novos produtos e organizações;
	\item considerar fundamental a inovação, a criatividade, a atitude empreendedora e a inserção internacional.
\end{enumerate}

O egresso da Engenharia de Computação, no processo de sua formação, deverá desenvolver as seguintes competências:
\begin{enumerate} [I -]
	\item antever as implicações humanísticas, sociais, ambientais, éticas, profissionais, legais (inclusive relacionadas à propriedade intelectual) e políticas dos sistemas computacionais;
	\item identificar demandas socioeconômicas e ambientais relevantes, planejar, especificar e projetar sistemas de computação, seguindo teorias, princípios, métodos e procedimentos interdisciplinares;
	\item construir, testar, verificar e validar sistemas de computação, seguindo métodos, técnicas e procedimentos interdisciplinares;
	\item perceber as necessidades de atualização decorrentes da evolução tecnológica e social;
	\item relacionar problemas do mundo real com suas soluções, considerando aspectos de computabilidade e de escalabilidade;
	\item analisar, desenvolver, avaliar e aperfeiçoar software e hardware em arquiteturas de computadores;
	\item analisar, desenvolver, avaliar e aperfeiçoar sistemas de automação e sistemas inteligentes;
	\item analisar, desenvolver, avaliar e aperfeiçoar sistemas de informação computacionais;
	\item analisar, desenvolver, avaliar e aperfeiçoar circuitos eletroeletrônicos;
	\item gerenciar pessoas e infraestrutura de Sistemas de Computação;
	\item perceber as necessidades de inovação e inserção internacional com atitudes criativas e empreendedoras.
\end{enumerate}

O curso de Engenharia de Computação tem, predominantemente, o ensino da computação como atividade fim, visando à formação de recursos humanos para o desenvolvimento científico e tecnológico da computação. Assim sendo, o curso deve capacitar indivíduos para desenvolver software e hardware, com uma forte base matemática e física.

Os egressos do curso de Engenharia de Computação estarão situados no estado da arte da ciência e da tecnologia da computação, de tal forma que possam continuar suas atividades na pesquisa, promovendo o desenvolvimento científico, ou aplicando os conhecimentos científicos, propiciando o desenvolvimento tecnológico. Para tal, é dada uma forte ênfase no uso de laboratórios para capacitar os egressos no projeto e construção tanto de software quanto de hardware.

\section{Administração Acadêmica do Curso - \textcolor{red}{Rafaela}}

\subsection{Coordenação de Áreas}

As disciplinas do curso de Engenharia de Computação estão divididas em quatro grandes áreas de conhecimento: (1) Sistemas de Informação; (2) Arquitetura de Sistemas de Computação; (3) Algoritmos e Linguagens de Programação; (4) Lógica e Inteligência Computacional.

A integração das disciplinas em áreas de conhecimento permite o compartilhamento de informações sobre interesses e objetivos comuns. Favorece a atuação conjunta de alunos e professores em temas globais e impulsiona a criação de linhas de pesquisa.

Cada área de conhecimento deverá possuir um Professor Coordenador. O Coordenador de área será responsável pelas disciplinas de sua área, cabendo a ele(a): orientar os alunos em questões referentes às disciplinas, analisar os requerimentos de quebras de pré-requisitos e conflitos de horário, tratar questões relativas aos conteúdos programáticos das disciplinas, promover a integração dos professores da mesma área, e  incentivar a pesquisa na área.

A tabela \ref{tab:areas} mostra a distribuição das disciplinas por área de conhecimento.

\begin{table}[ht]
	\centering
	\caption{Tabela de divisão de disciplinas por área de conhecimento}
	\label{tab:areas}
	\begin{tabularx}{\textwidth}{ X  l }
		\hiderowcolors
		\hline
		{\bf Área de Conhecimento}                              & {\bf Disciplinas} \\
		\hline
		\multirow{5}{*}{Algoritmos e Linguagens de Programação} & \AlgComp          \\ % Algoritmos Computacionais
																& \AnAlg            \\ % Análise de Algoritmos
		                                                        & \EstrInf          \\ % Estruturas de Informação
		                                                        & \LabProgA         \\ % Laboratório de Programação A
		                                                        & \LabProgB         \\ % Laboratório de POO
		                                                        & \TeoComp          \\ % Teoria da Compiladores
																& \Grafos			\\ \hline
		\multirow{8}{*}{Arquitetura de Sistemas de Computação}  & \ArqComp          \\ % Arqutetura de Computadores
																& \CompParal        \\ % Computação Paralela
		                                                        & \Control          \\ % Controle de Processos
		                                                        & \FundIComp         \\ % FUndamentos de Computadores I
																& \FundComp         \\ % FUndamentos de Computadores II
																& \Instala			\\ % Instalações de Ambientes Computacionais
		                                                        & \ProjSO           \\ % Projeto de Sistemas 
																& \Telep            \\ % Redes de Computadores
		                                                        & \Sredes         	\\ % Segurança em Redes
		                                                        & \SistEmb          \\ % Sistema Embutidos
		                                                        \hline	
		\multirow{3}{*}{Lógica e Inteligência Computacional}    & \IC               \\ % Inteligência Computacional
		                                                        & \ICII              \\ % Inteligência Computacional II
		                                                        & \LogProg          \\ % Lógica de Programação
																& \MineraDados      \\ % Mineração de Dados
		                                                        & \ProcImag         \\ % Processamento de Imagens
		                                                        \hline													
		\multirow{4}{*}{Sistemas de Informação}                 & \EngSistC         \\ % Análise de Projeto de Sistemas
																& \EngSistA		 \\  % Engenharia de Sistemas
		                                                        & \ProjBD           \\
		                                                        & \EngCompSoc       \\
		                                                        \hline
	\end{tabularx}
\end{table}

\subsection{Conselho Departamental}

\subsection{Chefia de Departamento}

\section{Currículo Pleno e Estrutura Curricular - \textcolor{red}{Robert}}

\subsection{Organização do Currículo}

O currículo do curso de Engenharia de Computação é constituído por disciplinas obrigatórias e eletivas, estágio supervisionado, trabalho de conclusão de curso e atividades de extensão. O curso é organizado em 10 semestres, podendo o aluno cumpri-lo em um máximo de 18 semestres.

Para uma eficaz orientação pedagógica, é proposto o aconselhamento curricular apresentado nas tabelas \ref{tab1p} a \ref{tab10p}. Os pré-requisitos das disciplinas podem ser observados no fluxograma do curso (anexo \ref{fluxograma}).

O aluno deverá cursar no mínimo três das disciplinas eletivas restritas oferecidas (ver tabela \ref{tabeletivas}). Deve ser
ressaltado que estas disciplinas são oferecidas de acordo com o interesse dos corpos
docente e discente, não sendo necessariamente disponibilizadas todos os semestres.

\rowcolors{1}{gray!5}{white}
\setlength{\tabcolsep}{5pt}
\renewcommand{\arraystretch}{1.5}
\begin{table}[ht]
	\centering
	\caption{1\textordmasculine~Período}
	\label{tab1p}
	\begin{spreadtab}{{tabularx}{\textwidth}{ | X|c|c| }}
		\hline
		@ {\textbf{Disciplina}} & @ {\textbf{CH}} & @ {\textbf{Créditos}} \\
		\hline
		@ \AlgComp	& \AlgCompCH	& \AlgCompCred	\\ % Algoritmos Computacionais
		@ \EngCompSoc 	& \EngCompSocCH & \EngCompSocCred	\\ % Engenharia e Computação Sociedade
		@\AlgLin	& \AlgLinCH		& \AlgLinCred	\\ % Álgebra Linear
		@ \CalcI	& \CalcICH		& \CalcICred	\\ % Cálculo I
		@ \IntAmb	& \IntAmbCH		& \IntAmbCred	\\ % Introdução à Engenharia Ambiental
		\hline
		@ Total 	& sum(b2:b6) 	& sum(c2:c6)	\\
		\hline
	\end{spreadtab}
\end{table}

\rowcolors{1}{gray!5}{white}
\begin{table}
	\centering
	\caption{2\textordmasculine~Período}
	\label{tab2p}
	\begin{spreadtab}{{tabularx}{\textwidth}{|X|c|c|}}
		\hline
		@ {\textbf{Disciplina}} & @ {\textbf{CH}} & @ {\textbf{Créditos}} \\
		\hline
		@ \EstrInf	& \EstrInfCH	& \EstrInfCred 	\\ % Estruturas de Informação
		@ \LogProg	& \LogProgCH	& \LogProgCred	\\ % Lógica de Programação
		@ \CalcII	& \CalcIICH		& \CalcIICred	\\ % Cálculo II
		@ \EngComput& \EngComputCH	& \EngComputCred\\ % Calculo Numérico
		@ \FisI		& \FisICH		& \FisICred		\\ % Física I
		@ \FisEI	& \FisEICH		& \FisEICred	\\ % Física Experimental I
		\hline
		@ Total 	& sum(b2:b7) 	& sum(c2:c7)	\\
		\hline
	\end{spreadtab}
\end{table}
	
\rowcolors{1}{gray!5}{white}
\begin{table}
	\centering
	\caption{3\textordmasculine~Período}
	\label{tab3p}
	\begin{spreadtab}{{tabularx}{\textwidth}{|X|c|c|}}
		\hline
		@ {\textbf{Disciplina}} & @ {\textbf{CH}} & @ {\textbf{Créditos}} \\
		\hline
		@ \AnAlg	& \AnAlgCH		& \AnAlgCred	\\ % Análise de Algoritmos
		@ \CalcIII	& \CalcIIICH 	& \CalcIIICred	\\ % Cálculo III
		@ \FisII	& \FisIICH		& \FisIICred	\\ % Física II
		@ \FisEII	& \FisEICH		& \FisEICred	\\ % Física Experimental II
		@ \ProbEst	& \ProbEstCH	& \ProbEstCred	\\ % Probabilidade e Estatística
		\hline
		@ Total 	& sum(b2:b6) 	& sum(c2:c6)	\\
		\hline
	\end{spreadtab}
\end{table}

\rowcolors{1}{gray!5}{white}
\begin{table}
	\centering
	\caption{4\textordmasculine~Período}
	\label{tab4p}
	\begin{spreadtab}{{tabularx}{\textwidth}{|X|c|c|}}
		\hline
		@ {\textbf{Disciplina}} & @ {\textbf{CH}} & @ {\textbf{Créditos}} \\
		\hline
		@ \LabProgA	& \LabProgACH	& \LabProgACred		\\ % Laboratório de Programação A
		@ \LabProgB	& \LabProgBCH	& \LabProgBCred		\\ % Laboratório de Programação B
		@ \FisIII	& \FisIIICH		& \FisIIICred		\\ % Física III
		@ \FisEIII	& \FisEIIICH	& \FisEIIICred		\\ % Física Experimental III
		@ \ProcImag 	& \ProcImagCH	& \ProcImagCred	\\ % Processamento de Sinais e Imagens
		@ \FundIComp	& \FundICompCH	& \FundICompCred\\ %Técnicas Digitais I
		\hline
		@ Total 	& sum(b2:b7) 	& sum(c2:c7)	\\
		\hline
	\end{spreadtab}
\end{table}

\rowcolors{1}{gray!5}{white}
\begin{table}
	\centering
	\caption{5\textordmasculine~Período}
	\label{tab5p}
	\begin{spreadtab}{{tabularx}{\textwidth}{|X|c|c|}}
		\hline
		@ {\textbf{Disciplina}} & @ {\textbf{CH}} & @ {\textbf{Créditos}} \\
		\hline
		@ \Grafos	& \GrafosCH		& \GrafosCred	\\ % Teoria dos Grafos e Aplicações
		@ \FundComp	& \FundCompCH	& \FundCompCred	\\ % Fundamentos de Computadores I
		@\CEV		& \CEVCH		& \CEVCred		\\ % Circuitos em Corrente 
		@ \FisIV	& \FisIVCH		& \FisIVCred	\\ % Física IV
		@ \FisEIV	& \FisEIVCH		& \FisEIVCred	\\ % Física Experimental IV
		@ \MatEle 	& \MatEleCH		& \MatEleCred	\\ % Materiais Elétricos e Magnéticos 
		@ \ModMat	& \ModMatCH		& \ModMatCred	\\ % Sinais e Sistemas
		\hline
		@ Total 	& sum(b2:b8) 	& sum(c2:c8)	\\
		\hline
	\end{spreadtab}
\end{table}

\rowcolors{1}{gray!5}{white}
\begin{table}
	\centering
	\caption{6\textordmasculine~Período}
	\label{tab6p}
	\begin{spreadtab}{{tabularx}{\textwidth}{|X|c|c|}}
		\hline
		@ {\textbf{Disciplina}} & @ {\textbf{CH}} & @ {\textbf{Créditos}} \\
		\hline
		@ \ArqComp	& \ArqCompCH	& \ArqCompCred	\\ % Arquitetura de Computadores A
		@ \EngSistA & \EngSistACH	& \EngSistACred	\\ % Engenharia de Sistemas
		@ \IC		& \ICCH			& \ICCred		\\ % Inteligência Computacional I
		@ \ICII 	& \ICIICH		& \ICIICred		\\ % Inteligência Computacional II
		@ \CEVI		& \CEVICH 		& \CEVICred		\\ % Circuitos em Corrente Alternada
		@ \EletI	& \EletICH		& \EletICred	\\ % Eletrônica I
		\hline
		@ Total 	& sum(b2:b7) 	& sum(c2:c7)	\\
		\hline
	\end{spreadtab}
\end{table}

\rowcolors{1}{gray!5}{white}
\begin{table}
	\centering
	\caption{7\textordmasculine~Período}
	\label{tab7p}
	\begin{spreadtab}{{tabularx}{\textwidth}{|X|c|c|}}
		\hline
		@ {\textbf{Disciplina}} & @ {\textbf{CH}} & @ {\textbf{Créditos}} \\
		\hline
		@ \MineraDados	& \MineraDadosCH	& \MineraDadosCred	\\ % Mineração de Dados
		@ \ProjBD		& \ProjBDCH		& \ProjBDCred		\\ % Projeto de Banco de 
		@ \ProjSO		& \ProjSOCH		& \ProjSOCred		\\ % Projeto de Sistemas Operacionais
		@ \Telep 		& \TelepCH		& \TelepCred		\\ % Redes de Computadores
		@ \TeoComp		& \TeoCompCH	& \TeoCompCred		\\ % Teoria da Compiladores
		@ \IntEco		& \IntEcoCH		& \IntEcoCred	\\ % Macroeconomia 
		\hline
		@ Total			& sum(b2:b7)	& sum(c2:c7)		\\
		\hline
	\end{spreadtab}
\end{table}

\rowcolors{1}{gray!5}{white}
\begin{table}
	\centering
	\caption{8\textordmasculine~Período}
	\label{tab8p}
	\begin{spreadtab}{{tabularx}{\textwidth}{|X|c|c|}}
		\hline
		@ {\textbf{Disciplina}} & @ {\textbf{CH}} & @ {\textbf{Créditos}} \\
		\hline
		@ \EngSistC 	& \EngSistCCH		& \EngSistCCred		\\ % Análise e Projeto de Sistemas
		@ \Control		& \ControlCH		& \ControlCred		\\ % Controle de Processos
		@ \CompParal	& \CompParalCH		& \CompParalCred	\\ % Computação Paralela
		@ \Sredes 		& \SredesCH			& \SredesCred		\\ % Segurança em Redes
		@ \SistEmb		& \SistEmbCH		& \SistEmbCred		\\ % Sistemas Embutidos
		@ \Empre 		& \EmpreCH			& \EmpreCred		\\ % Empreendedorismo
		\hline
		@ Total				& sum(b2:b7)			& sum(c2:c7)			\\
		\hline
	\end{spreadtab}
\end{table}

\rowcolors{1}{gray!5}{white}
\begin{table}
	\centering
	\caption{9\textordmasculine~Período}
	\label{tab9p}
	\begin{spreadtab}{{tabularx}{\textwidth}{|X|c|c|}}
		\hline
		@ {\textbf{Disciplina}} & @ {\textbf{CH}} & @ {\textbf{Créditos}} \\
		\hline
		@ \EletA		& \EletACH		& \EletACred	\\ % Disciplina Eletiva A
		@ \EstSup		& \EstSupCH		& \EstSupCred	\\ % Estágio Supervisionado
		@ \ProjA		& \ProjACH		& \ProjACred	\\ % Metodologia Ciêntífica
		@ \Instala 		& \InstalaCH	& \InstalaCred	\\ % Instalações de Ambientes Computacionais
		\hline
		@ Total			& sum(b2:b5)	& sum(c2:c5)	\\
		\hline
	\end{spreadtab}
\end{table}

\begin{table}
	\centering
	\caption{10\textordmasculine~Período}
	\label{tab10p}
	\begin{spreadtab}{{tabularx}{\textwidth}{|X|c|c|}}
		\hline
		@ {\textbf{Disciplina}} & @ {\textbf{CH}} & @ {\textbf{Créditos}} \\
		\hline
		@ \EletB	& \EletBCH	& \EletBCred	\\ % Disciplina Eletiva B
		@ \EletC	& \EletCCH	& \EletCCred	\\ % Disciplina Eletiva C
		@ \ProjB	& \ProjBCH	& \ProjBCred	\\ % Projeto de Graduação XI
		@ \Adm		& \AdmCH	& \AdmCred		\\ % Administração
		\hline
		@ Total		& sum(b2:b5)& sum(c2:c5)	\\
		\hline
	\end{spreadtab}
\end{table}

\begin{table}
	\centering
	\caption{Disciplinas Eletivas Restritas}
	\label{tabeletivas}
	\begin{spreadtab}{{tabularx}{\textwidth}{|X|c|c|}}
		\hline
		@ {\textbf{Disciplina}} & @ {\textbf{CH}} & @ {\textbf{Créditos}} \\
		\hline
		@ \EletArq	& \EletArqCH	& \EletArqCred	\\
		@ \EletGeo	& \EletGeoCH	& \EletGeoCred	\\
		@ \EletPadroes	& \EletPadroesCH	& \EletPadroesCred	\\
		@ \EletRec	& \EletRecCH	& \EletRecCred	\\
		@ \EletRedes	& \EletRedesCH& \EletRedesCred	\\
		@ \EletMov	& \EletMovCH	& \EletMovCred	\\
		\hline
	\end{spreadtab}
\end{table}

\subsection{Normas Gerais de Ensino de Graduação da UERJ}

O curso de Engenharia de Computação obedecerá ao regime de créditos e as aulas serão oferecidas nos turnos manhã e tarde, com aulas predominantemente pela manhã, para os aprovados classificados no primeiro semestre; e tarde e noite, com aulas predominantemente pela tarde, para os aprovados classificados no segundo semestre. O turno da manhã transcorre no horário das 07:00h às 12:20h; o da tarde das 12:30h às 17:50h e o da noite das 18:00h às 22:40h. As aulas têm duração de 50 minutos nos turnos da manhã e tarde e de 45 minutos no turno da noite.

As Normas Gerais de Ensino de Graduação da UERJ são definidas pela deliberação n\textordmasculine~33/95 da UERJ (anexo \ref{delib3395}), sendo seus aspectos principais apresentados a seguir:

\subsection{Relação entre crédito e carga horária}
\textit{
	\textbf{Art. 57} -- O número mínimo de créditos necessários para integralizar o currículo será estabelecido com base na carga horária total do curso.}

\textit{
	\textbf{Parágrafo Único} - A unidade de crédito corresponde a:
	\begin{enumerate}[a)]
		\item 15 (quinze) horas de aula teórica, ou
		\item 30 (trinta) horas de aula prática, laboratório ou estágio curricular.
	\end{enumerate}}

\subsection{Aproveitamento escolar}
\begin{itquotation}
	\setcounter{artigo}{94}
	\artigo A aprovação do aluno em disciplinas do Curso de Graduação desta Universidade terá por base notas e frequência. São condições para aprovação: obtenção de nota final mínima 5,0 (cinco vírgula zero), constituída pela média aritmética da média semestral e nota da prova final, frequência mínima de 75\% (setenta e cinco por cento) do total de horas/aula determinado para a disciplina.

	\begin{paragrafos}
		\paragrafo Para cada disciplina haverá, pelo menos, duas avaliações por turma, por período letivo, sendo uma delas necessariamente individual e escrita. A média dos resultados dessas avaliações constitui a média semestral do aluno na disciplina.\\
		\paragrafo O aluno que obtiver média semestral igual ou superior a 4,0 (quatro vírgula zero) terá direito à prova final.\\
		\paragrafo O aluno que obtiver média semestral igual ou superior a 7,0 (sete vírgula zero) estará dispensado de prestar prova final.\\
		\ldots

		\setcounter{paragrafo}{6}
		\paragrafo  O aluno que obtiver nota final menor que 5,0 (cinco vírgula zero) ou média semestral inferior a 4,0 (quatro vírgula zero) será reprovado.\\
		\paragrafo O aluno que não obtiver frequência mínima de 75\% (setenta e cinco por cento) do total de horas/aula determinadas pela disciplina será reprovado, sem direito à prova final e independente de alcançar nota final superior a 7,0 (sete vírgula zero).\\
	\end{paragrafos}

\end{itquotation}
\subsection{Período de integralização do curso}
\setcounter{artigo}{98}
\begin{itquotation}
	\artigo Somente receberá o diploma o aluno que cumprir a Integralização Curricular.
\end{itquotation}

O período mínimo de integralização curricular dos cursos de engenharia é de 10 (dez) semestres, exceto para os casos de isenção de disciplinas, em que é possível um tempo mínimo menor. Já o prazo máximo para essa integralização é de 18 (dezoito) semestres.


\subsection{Estágio Supervisionado}

A atividade de Estágio Supervisionado é um elemento curricular obrigatório. De forma a possibilitar que docentes de diferentes departamentos possam participar das atividades de orientação, esta atividade é oferecida como uma disciplina eletiva restrita, na qual o discente deve selecionar uma das alternativas oferecidas de acordo com sua ênfase.
Disciplinas obrigatórias do curso de Engenharia Elétrica da UERJ relativas à atividade de Estágio Supervisionado
Todas as disciplinas possuem a mesma carga horária e número de créditos, respectivamente, 165 horas e 11 créditos, atendendo assim às Diretrizes Curriculares Nacionais dos cursos de Engenharia. O conjunto das ementas destas disciplinas estão contidos no anexo do ementário. A cada período o Departamento de Engenharia Elétrica irá abrir uma turma de forma a atender a demanda discente.
Como a disciplina de Estágio Supervisionado é encarada como sendo o primeiro trabalho profissional do aluno no papel de futuro Engenheiro Eletricista, espera-se que o trabalho a ser desenvolvido seja realizado em um ambiente corporativo. Além da supervisão de um profissional da empresa, o aluno contará com o apoio do professor da disciplina que poderá dar sugestões e contribuir com o desenvolvimento do trabalho. O aluno só poderá se inscrever nesta disciplina tendo cursado um número mínimo de 180 créditos, valor compatível com o desejado amadurecimento do estudante para a realização desta atividade (trava de créditos).
Independentemente da sua natureza, esse trabalho deve unir conhecimentos, competências e habilidades que foram adquiridos durante o curso. O Estágio deverá dar aos estudantes a oportunidade de refletir, analisar e propor soluções para problemas reais em desenvolvimento, através da articulação da teoria e da prática. A realização desse trabalho deverá abordar qualquer área do conhecimento da Engenharia Elétrica e contemplar a identificação e abordagem de um problema com foco
científico/tecnológico, a análise da viabilidade de possíveis soluções, a proposição de desenvolvimento de um projeto específico de engenharia, além de considerar a questão econômica e os impactos ambientais e sociais que possam estar envolvidos. Também se espera uma atualização em relação aos avanços da ciência e da tecnologia, bem como aos desafios da inovação. Ao término da disciplina, o aluno deverá redigir um Relatório que descreva as atividades realizadas, explicitando claramente os conteúdos do curso de Engenharia Elétrica que foram aplicados durante o período de Estágio. Este Relatório será avaliado concomitantemente a uma apresentação oral perante ao Professor da Disciplina. No texto do Relatório Final, serão avaliadas a redação, o uso correto da língua portuguesa, a qualidade do trabalho e as contribuições conferidas à formação do estudante. Na apresentação oral será computada a exposição do trabalho e a arguição realizada pelo professor da Disciplina.

\subsection{Projeto de Graduação}

A conclusão do curso se dá com as disciplinas de Projeto de Graduação A-II e Projeto de Graduação B-II para Engenharia Elétrica com ênfase em Sistemas de Potência, e com as disciplinas de Projeto de Graduação A-I e Projeto de Graduação B-I para Engenharia Elétrica com ênfase em Sistemas Elétricos e de Automação Industrial.
Estas disciplinas correspondem à elaboração do Projeto Final de Curso em suas respectivas ênfases, e envolvem o desenvolvimento de um projeto aplicado de engenharia elétrica, sob a orientação de docentes. Para garantir que o aluno só inicie o Projeto Final de Curso tendo já realizado um número de disciplinas compatível com o processo de finalização do seu curso, há um limite mínimo de 180 créditos (trava de créditos) a serem cursados para inscrição na primeira das duas disciplinas (Projeto de Graduação A).
Nas disciplinas, os alunos se organizarão individualmente ou em duplas de trabalho e, sob efetiva orientação docente, conceberão e/ou projetarão soluções criativas e viáveis no contexto do desenvolvimento do tema selecionado.
Trata-se de um trabalho que irá congregar todo o arcabouço técnico abordado ao longo do curso, com ênfase na integração de conceitos e conhecimentos adquiridos. Desta forma, estas duas disciplinas expõem cada aluno a uma tarefa de grande porte a ser realizada em dois períodos, ao longo dos quais será necessário aplicar todos os conhecimentos adquiridos ao longo do curso de forma integrada.
As etapas do desenvolvimento desta atividade ao longo das duas disciplinas contemplam:

\begin{itemize}
\item Projeto de Graduação A: o levantamento das informações, elaboração do tema, pesquisa bibliográfica e desenvolvimento da parte teórica de embasamento do projeto final de curso conforme as normas ABNT.
\item Projeto de Graduação B: Ensaios em bancada, prototipagem e/ou simulação computacional, finalização do documento e defesa com banca composta, no mínimo, por dois membros convidados pelo professor da disciplina, mais o orientador.
Neste contexto, os documentos gerados irão seguir a formatação sugerida pela Biblioteca.
\end{itemize}

Na defesa serão avaliados o texto, a qualidade do trabalho, a coerência dos resultados e o domínio do assunto durante a arguição.

\section{Equivalência com o Curso Anterior - \textcolor{red}{Thiago}}

O curso de Engenharia de Computação ora proposto substituirá o curso de Engenharia Elétrica com ênfase em Sistemas e Computação e, na hipótese de algum aluno desejar migrar do curso antigo para este novo, será possível dispensar disciplinas do novo currículo iguais ou equivalentes às disciplinas do curso antigo.

A tabela \ref{DiscIguais} mostra as disciplinas que são equivalentes entre o novo curso de Engenharia de Computação e o antigo curso de Engenharia Elétrica com ênfase em Sistemas e Computação. Por outro lado, a tabela \ref{DiscSemEqui} lista as disciplinas do novo currículo que não possuem equivalência direta com as disciplinas do curso anterior.

\rowcolors{1}{gray!5}{white}
\begin{table}[ht]
	\caption{Disciplinas Equivalentes}
	\label{DiscIguais}
	\centering
	\renewcommand{\arraystretch}{1.5}
	\begin{tabularx}{\textwidth}{|X|l|}
		\showrowcolors
		\hline
		{\textbf{Disciplina}} & \textbf{Código} \\
		\hline
		\Adm                  & \AdmCod         \\ % Administração
		\AlgLin               & \AlgLinCod      \\ % Álgebra Linear
		\CEV                  & \CEVCod         \\ % Circuitos em Corrente Contínua
		\EletI                & \EletICod       \\ % Eletrônica I
		\FisI                 & \FisICod        \\ % Física I
		\FisII                & \FisIICod       \\ % Física II
		\FisIII               & \FisIIICod      \\ % Física III
		\FisIV                & \FisIVCod       \\ % Física IV
		\IntEco               & \IntEcoCod      \\ % Macroeconomia
		\IntAmb               & \IntAmbCod      \\ % Introdução à Engenharia Ambiental
		\MatEle               & \MatEleCod      \\ % Materiais Elétricos e Magnéticos
		\ModMat               & \ModMatCod      \\ % Sinais e Sistemas
		\ProbEst              & \ProbEstCod     \\ % Probabilidade e Estatística
		\ProjA                & \ProjACod       \\ % Metodologia Científica
		\ProjB                & \ProjBCod       \\ % Projeto de Graduação XI
		\hline
	\end{tabularx}
\end{table}

\rowcolors{1}{gray!5}{white}
\begin{table}
	\centering
	\renewcommand{\arraystretch}{1.5}
	\caption{Equivalências no novo currículo}
	\label{equivalencias}
	\begin{tabularx}{\textwidth}{|X||X|l|}
		\hline
		{\textbf{Currículo Novo}}	& \textbf{Equivalente no Currículo Antigo} 	& \textbf{Código}\\
		\hline
		\AlgComp	& Algoritmos Computacionais				    	& FEN06-03559       \\
		\AnAlg      & Análise de Algoritmos                       & FEN06-03713       \\
		\ArqComp    & Arquitetura de Computadores I               & FEN06-04119       \\
		\ProjSO     & Arquitetura de Sistemas Operacionais        & FEN06-04664       \\
		\CalcI      & Cálculo Diferencial e Integral I            & IME01-00508       \\
		\CalcII     & Cálculo Diferencial e Integral II           & IME01-00854       \\
		\CalcIII    & Cálculo Diferencial e Integral III          & IME01-03646       \\
		\EngComput  & Cálculo Numérico IV                         & IME04-04541       \\
		\LabProgB   & Carac. das Linguagens de Prog. I            & FEN06-03980       \\
		\CEVI       & Circuitos Elétricos IV                      & FEN04-05222       \\
		\Control    & Controle de Processos por Comp.             & FEN06-05080       \\
		\EngSistA   & Engenharia de Sistemas A					& FEN06-04243       \\
		\ProjBD     & Engenharia de Sistemas B                    & FEN06-04314       \\
		\EstrInf    & Estruturas de Informação I                  & FEN06-03648       \\
		\FundComp   & Fundamentos de Comp. Digitais I    & FEN06-03787  				\\
		\LabProgA   & Laboratório de Programação I        & FEN06-04049                 \\
		\Telep      & Teleproc. e Redes de Computadores   & FEN06-04718                 \\
		\TeoComp                               & Teoria de Compiladores            & FEN06-04516                 \\
		Eletivas Restritas & Tóp. Especiais em Eng. de Sistemas e Computação A, B ou C & \parbox[t]{2cm}{FEN06-04889                                          \\FEN06-04939\\FEN06-04990}  \\
		\hline
	\end{tabularx}
\end{table}

\begin{table}
	\centering
	\renewcommand{\arraystretch}{1.5}
	\caption{Disciplinas sem Equivalências}
	\label{DiscSemEqui}
	\begin{tabularx}{\textwidth}{|X|}
		\hline
		{\textbf{Disciplinas do Novo Currículo sem Equivalência}} \\
		\hline
		\LogProg                                                  \\
		\IC                                                       \\
		\EngCompSoc                                               \\
		\MineraDados                                              \\
		\SistEmb                                                  \\
		\ProcImag                                                 \\
		\CompParal                                                \\
		\EstSup                                                   \\
		\hline
	\end{tabularx}
\end{table}

\section{Ementário das Disciplinas - \textcolor{red}{Thiago}}

As ementas das disciplinas obrigatórias e eletivas são apresentadas no anexo \ref{ementas}. As ementas das disciplinas já existentes foram obtidas no site do próprio DEP, Departamento de Orientação e Supervisão Pedagógica. Essas disciplinas são apresentadas no formulário antigo e não foram feitas correções ou alterações no texto original.

\section{Atividades Acadêmicas da Graduação articuladas ao ensino de Pós-Graduação (Aperfeiçoamento, Mestrado, Doutorado) \textcolor{red}{Luiza} }

\section{Atividades de Extensão \textcolor{red}{Giomar}}

O currículo do curso de Engenharia de Sistemas e Computação da UERJ tem como uma de suas vertentes combater a evasão, propiciando maior motivação e engajamento dos alunos oferecendo-lhes disciplinas profissionalizantes desde o primeiro período do curso. 

Nesta mesma direção, o curso busca proporcionar aos alunos uma visão realística e analítica sobre o papel da tecnologia na sociedade através da oferta de novas disciplinas e de atividades de Extensão Universitária, suscitando o desenvolvimento de uma perspectiva disruptiva e contextualizada através da qual os alunos se sintam encorajados a se envolver com iniciativas voltadas para o desenvolvimento social. Desta forma, as atividades de Extensão Universitária são vistas como um processo de aprofundamento educativo e cultural, propiciando o exercício da interdisciplinaridade e estimulando o pensamento reflexivo, analítico e crítico nos estudantes. A partir do conjunto coordenado de projetos, programas, eventos e disciplinas, promovem interações transformadoras entre a universidade e os diversos setores da sociedade.

Para alcançar esses objetivos, as seguintes mudanças foram implementadas: cada aluno do curso deve completar pelo menos \hextensao horas em atividades de extensão complementares, conforme exigido pela Lei 10.172 que aprova o Plano Nacional de Educação. Isso corresponde a pelo menos 10\% do total da carga horária do curso.

A participação discente e o cumprimento das horas pode ser obtido através das seguintes formas de atividades de extensão:
\begin{enumerate}[I -]
    \item Participação em programas e projetos de extensão coordenados por professores ou técnicos da carreira de nível superior na Universidade do Estado do Rio de Janeiro, com ou sem o recebimento de bolsa;
    \item Promoção de cursos de extensão, incluindo a organização, preparação e apresentação de aulas, videoaulas e reuniões com a comunidade;
    \item Participação em disciplinas e atividades relacionadas à extensão fornecidas pela universidade, visando a compreensão, aprimoramento e melhor desempenho do discente na realização de tarefas de extensão;
    \item Desenvolvimento conjunto de soluções tecnológicas visando atender as demandas dos atores da sociedade civil;
    \item Participação em eventos, tanto na organização quanto na realização.
\end{enumerate}

A Universidade do Estado do Rio de Janeiro é uma instituição compromissada com a formação da cidadania e a inclusão social. Neste contexto, o Curso de Engenharia de Sistemas e Computação da UERJ, pretende colaborar com a inclusão digital dos cidadãos fluminenses.

A extensão universitária permite o estreitamento dos laços entre a academia e a sociedade, inseridas em diversas realidades socioeconômicas no âmbito de comunidades rurais, periurbanas e urbanas. Ao aproximar os conhecimentos obtidos em sala de aula à realidade, contribui para a sustentabilidade de setores socioeconômicos que desempenham um papel essencial para o futuro das gerações, mas, que enfrentam no dia a dia constantes desafios, notadamente a exclusão digital. Além disso, oferece aos alunos a chance de se envolverem em atividades nas quais poderão exercitar a cidadania ao prestar um serviço relevante a segmentos da sociedade frequentemente esquecidos e carentes de assistência.


\chapter{Conformidade com as Diretrizes Nacionais}
\thispagestyle{plain}

Este capítulo tem como objetivo apresentar uma análise da relação entre as \textit{Diretrizes Curriculares Nacionais para Cursos de Graduação em Engenharia de Computação} e a estrutura curricular do curso proposto.

A Engenharia de Computação é uma área dinâmica e em constante evolução, exigindo uma formação sólida que contemple tanto os fundamentos teóricos quanto as habilidades práticas necessárias para o desenvolvimento de soluções inovadoras que integrem hardware e software. Os referenciais da SBC (Anexo \ref{sbc2017}) surgem como um guia para a elaboração de Projetos Pedagógicos de Curso  alinhados com as D\textit{iretrizes Curriculares Nacionais} (DCNs) e com as demandas da sociedade e do mercado de trabalho.

Primeiramente, é importante notar que os \textit{Referenciais de Formação da SBC} foram elaborados com base nas DCNs, especificamente a \textit{Resolução \ordm{n} 5 de 16 de novembro de 2016} (Anexo \ref{cne2016}), para servir como referência na elaboração de Projetos Pedagógicos para cursos de Bacharelado em Engenharia de Computação. Portanto, há uma forte correlação entre esses dois documentos.

As DCNs permitem que a formação em Engenharia de Computação siga tanto as próprias \textbf{Diretrizes da área da Computação} quanto as Diretrizes gerais para os cursos de Engenharia. Independentemente da linha adotada, espera-se que o projeto pedagógico defina o perfil do egresso, competências e habilidades, conteúdos curriculares, organização curricular, estágio, trabalho de curso (TCC), e atividades complementares. A carga horária mínima para os cursos de bacharelado na área de Computação, incluindo Engenharia de Computação, é de 3.200 horas.

Buscaremos, neste texto, demonstrar como as disciplinas ofertadas no curso dialogam com os eixos de formação, competências e conteúdos preconizados pela SBC, visando a formação de engenheiros de computação qualificados, críticos, criativos e éticos.

\section{Eixos de Formação}
Os Referenciais de \textbf{Formação da SBC para Engenharia de Computação }estão estruturados em cinco eixos principais, cada um com suas competências gerais e derivadas, além de um conjunto de conteúdos essenciais. A seguir, analisaremos cada um desses eixos e como as disciplinas do curso, conforme suas ementas (Anexo \ref{ementas}), contribuem para o desenvolvimento das competências propostas.

\section{Eixo 1: Fundamentos de Sistemas de Computação}
Este eixo visa desenvolver a competência geral de ``\textit{Lembrar e entender teorias e princípios da computação, matemática e ciências; aplicando estas teorias e princípios para resolver problemas técnicos de sistemas computacionais, incluindo sistemas de aplicação específica}''. Ele abrange uma base sólida em programação, teoria da computação, arquitetura de computadores, sistemas operacionais, matemática e física.

\rowcolors{2}{gray!10}{white}
\begin{small}
    \begin{longtable}{p{10cm} L{4.8cm}}
        \caption{Relação entre as competências do Eixo 1 da SBC e as disciplinas do curso} \label{eixo1}                                                                                                                                                                                                                                                            \\
        \toprule
        \textbf{Competência - Conteúdo (SBC)}                                                                                                                                                                                                                                                    & \textbf{Disciplinas}                                             \\
        \midrule
        \endfirsthead

        \multicolumn{2}{c}%
        {{\bfseries \tablename\ \thetable{} -- Continuação da página anterior}}                                                                                                                                                                                                                                                                                     \\
        \toprule
        \textbf{Competência - Conteúdo (SBC)}                                                                                                                                                                                                                                                    & \textbf{Disciplinas}                                             \\
        \midrule
        \endhead

        \midrule \multicolumn{2}{r}{{Continua na próxima página}}                                                                                                                                                                                                                                                                                                   \\
        \endfoot

        \bottomrule
        \endlastfoot

        C.1.1 - Aplicar os conceitos de programação imperativa e dominar o uso de abstrações de controle e dados, analisando o problema em questão para determinar trade\textit{off}s de memória e processamento ao aplicar diferentes estruturas de controle e de dados.                        & \AlgComp                                                         \\
        \addlinespace
        C.1.2 - Dominar noções básicas de teoria da computação, como lógica básica, complexidade de algoritmos, e linguagens formais e autômatos.                                                                                                                                                & \AnAlg, \LogProg, \TeoComp                                       \\
        \addlinespace
        C.1.3 - Elaborar sistemas considerando o mapeamento de programas para arquiteturas de computadores convencionais: representação de código e de dados, entrada e saída, geração de programas e sua execução. Analisar programas e avaliar o custo de aplicação de diferentes construções. & \FundComp, \ArqComp                                              \\
        \addlinespace
        C.1.4 - Criticar e escolher sistemas operacionais para contextos específicos, considerando como funcionam os principais componentes de cada sistema e os requisitos do contexto de aplicação.                                                                                            & \ProjSO                                                          \\
        \addlinespace
        C.1.5 - Avaliar o desempenho de sistemas computacionais usando técnicas teóricas e práticas de forma complementar.                                                                                                                                                                       & \AnAlg, \ArqComp                                                 \\
        \addlinespace
        C.1.6 - Dominar o ferramental matemático básico, da Engenharia compreendendo noções de cálculo e mapeá-lo para técnicas de cálculo numérico e métodos de matemática aplicada.                                                                                                            & \CalcI, II, III, \CalcNum                                        \\
        \addlinespace
        C.1.7 - Dominar conceitos de probabilidade e estatística e aplicá-los em diferentes contextos, como análise de desempenho ou estudo de capacidade.                                                                                                                                       & \ProbEst                                                         \\
        \addlinespace
        C.1.8 - Aplicar conceitos de matemática, como indução, combinatória e teoria de grafos, em diferentes situações e problemas.                                                                                                                                                             & \Grafos                                                          \\
        \addlinespace
        C.1.9 - Dominar conceitos básicos da física relacionados a eletricidade e magnetismo e transmissão de ondas.                                                                                                                                                                             & \FisI, II e IV, \FisEI, II e IV, \FisIII~e Experimental, \MatEle \\
        \addlinespace
        C.1.10 - Analisar e projetar circuitos eletrônicos simples, entendendo requisitos e \textit{tradeoffs}. Avaliar circuitos digitais usados em sistemas computacionais. Analisar os efeitos de características e estilos de projeto sobre temporização, desempenho e energia.              & \CCC, \CCA, \CircEletI, \TecDig                                  \\
        \addlinespace
        C.1.11 - Aplicar e integrar os conhecimentos teóricos aprendidos nas diferentes disciplinas na resolução de problemas práticos. Criar soluções para novos problemas e analisar os tradeoffs associados a soluções alternativas.                                                          & \LabProgA. \LabProgPOO, \Ext, \EstSup, \ProjB                    \\
    \end{longtable}
\end{small}
\section{Eixo 2: Desenvolvimento de Sistemas Computacionais}
A competência geral deste eixo é ``\textit{Criar, implementar e manter soluções computacionais eficientes para diversos tipos de problemas, envolvendo hardware, software e processos, analisando o espaço de projeto considerando restrições e custo-benefício; e criar e integrar componentes de hardware, de software e sua interface}''. Este eixo é o cerne da atividade prática do engenheiro de computação.

\rowcolors{2}{gray!10}{white}
\begin{small}
    \begin{longtable}{p{10cm} L{4.8cm}}
        \caption{Relação entre as competências do Eixo 2 da SBC e as disciplinas do curso} \label{eixo1}    \\
        \toprule
        \textbf{Competência - Conteúdo (SBC)}           & \textbf{Disciplinas}                              \\
        \midrule
        \endfirsthead

        \multicolumn{2}{c}%
        {{\bfseries \tablename\ \thetable{} -- Continuação da página anterior}}                             \\
        \toprule
        \textbf{Competência - Conteúdo (SBC)}           & \textbf{Disciplinas}                              \\
        \midrule
        \endhead

        \midrule \multicolumn{2}{r}{{Continua na próxima página}}                                           \\
        \endfoot

        \bottomrule
        \endlastfoot

        C.2.1 - Determinar os requisitos de desempenho e
        confiabilidade, projeto, implementação e teste de
        componentes eletrônicos e sistemas em hardware. & \ArqComp, \TecDig, \CCC, \CCA                     \\
        \addlinespace
        C.2.2 - Especificar e validar os requisitos, projetar,
        Especificar e validar os requisitos, projetar,
        soluções de software baseadas no conhecimento
        apropriado de teorias, modelos e técnicas.      & \EstrInf, \EngSistA, \AnaProjSist, \ProjBD        \\
        \addlinespace
        C.2.3 - Conhecer técnicas, arquiteturas e
        ferramentas para a seleção e integração otimizada
        de recursos de hardware e software. Para
        construção desta capacidade, são necessários
        conhecimentos em: sistemas operacionais,
        sistemas paralelos e distribuídos, programação de
        periféricos, sistemas em tempo real e sistemas
        embarcados.                                     & \ArqComp, \ProjSO, \CompParal, \SistEmb, \Control \\
        \addlinespace
        C.2.4 - Realizar o projeto de sistemas integrados de
        hardware e software para diversas áreas da
        indústria eletro-eletrônica. Esta capacitação
        envolve o conhecimento de áreas relacionadas a
        telecomunicações, redes de computadores,
        tratamento digital de sinais (para aplicações de
        tratamento de imagens, vídeo e áudio), e projeto de
        Controle e Automação de processos.              & \AnaProjSist, \SistEmb, \Control, \Telep          \\
    \end{longtable}
\end{small}

\section*{Eixo 3: Gerenciamento de Sistemas Computacionais}

Este eixo tem como objetivo desenvolver no estudante a capacidade de compreender, planejar e gerenciar projetos, serviços e experimentos de engenharia no contexto da Computação. As competências associadas incluem a elaboração e acompanhamento de projetos de software e hardware, o estudo de viabilidade técnica e econômica, o gerenciamento de recursos e equipes, bem como o entendimento das estruturas organizacionais e da governança corporativa.
\rowcolors{2}{gray!10}{white}
\begin{small}
    \begin{longtable}{p{10cm} L{4.8cm}}
        \caption{Relação entre as competências do Eixo 3 da SBC e as disciplinas do curso} \label{eixo1} \\
        \toprule
        \textbf{Competência - Conteúdo (SBC)}                        & \textbf{Disciplinas}              \\
        \midrule
        \endfirsthead

        \multicolumn{2}{c}%
        {{\bfseries \tablename\ \thetable{} -- Continuação da página anterior}}                          \\
        \toprule
        \textbf{Competência - Conteúdo (SBC)}                        & \textbf{Disciplinas}              \\
        \midrule
        \endhead

        \midrule \multicolumn{2}{r}{{Continua na próxima página}}                                        \\
        \endfoot

        \bottomrule
        \endlastfoot
        C.3.1 - Compreender conceitos relevantes sobre
        projetos, serviços e experimentos de engenharia na
        área de computação.                                          & \EngSistA, \AnaProjSist           \\
        \addlinespace
        C.3.2 - Compreender as estruturas organizacionais
        e os papéis relacionados ao desenvolvimento de
        projetos, serviços e experimentos de Engenharia de
        Computação.                                                  & \Adm, \MacroEco                   \\
        \addlinespace
        C.3.3 - Identificar normas e documentações
        técnicas necessárias em projetos, serviços e
        experimentos de Engenharia de Computação.                    & \EstSup, \ProjA, \ProjB           \\
        \addlinespace
        C.3.4 - Aplicar metodologias de gestão de projetos,
        serviços e experimentos de engenharia na área de computação. & \Adm, \Empre                      \\
    \end{longtable}
\end{small}

\section*{Eixo  4: Inovação e Empreendedorismo}
Com a competência geral de ``\textit{Criar ferramentas, técnicas e conhecimentos científicos e/ou tecnológicos inovadores na área, empreendendo na área de engenharia de computação, reconhecendo oportunidades e resolvendo problemas de forma a agregar valor à sociedade}'', este eixo é vital para a formação de um profissional transformador.

\rowcolors{2}{gray!10}{white}
\begin{small}
    \begin{longtable}{p{10cm} L{4.8cm}}
        \caption{Relação entre as competências do Eixo 4 da SBC e as disciplinas do curso} \label{eixo1} \\
        \toprule
        \textbf{Competência - Conteúdo (SBC)}              & \textbf{Disciplinas}                        \\
        \midrule
        \endfirsthead

        \multicolumn{2}{c}%
        {{\bfseries \tablename\ \thetable{} -- Continuação da página anterior}}                          \\
        \toprule
        \textbf{Competência - Conteúdo (SBC)}              & \textbf{Disciplinas}                        \\
        \midrule
        \endhead

        \midrule \multicolumn{2}{r}{{Continua na próxima página}}                                        \\
        \endfoot

        \bottomrule
        \endlastfoot
        C.4.1 - Entender a relação entre teoria e prática. & \Ext, \ProjA                                \\
        \addlinespace
        C.4.2 - Entender processos e questões
        relativos ao desenvolvimento de produto
        e sua manufatura.                                  & \Empre, \EstSup                             \\
        \addlinespace
        C.4.3 - Aplicar os fundamentos da economia na análise e no
        desenvolvimento de projetos de Engenharia de Computação, realizando
        estudos de viabilidade técnico-econômica, considerando o contexto
        social.                                            & \Adm, \Empre, \MacroEco                     \\
        \addlinespace
        C.4.4 - Integrar conceitos de áreas
        diferentes em um sistema completo para
        prover uma solução.                                & \ProjB, \Ext                                \\
        \addlinespace
        C.4.5 - Aplicar
        fundamentos
        administração na análise e
        desenvolvimento de projetos
        Engenharia de Computação.                          & \Adm                                        \\
        \addlinespace
        C.4.6  - Empreender e exercer liderança
        na sua área de atuação profissional.               & \Empre                                      \\
    \end{longtable}
\end{small}

\section*{Eixo  5: Desenvolvimento Pessoal e Profissional}
Este eixo busca ``\textit{Compreender a importância e responsabilidade da prática profissional, agindo de forma ética, sustentável e socialmente responsável, respeitando aspectos legais e normas envolvidas e observando direitos e propriedades intelectuais inerentes à produção e à utilização de sistemas de computação}".

\rowcolors{2}{gray!10}{white}
\begin{small}
    \begin{longtable}{p{10cm} L{4.8cm}}
        \caption{Relação entre as competências do Eixo 5 da SBC e as disciplinas do curso} \label{eixo1} \\
        \toprule
        \textbf{Competência - Conteúdo (SBC)} & \textbf{Disciplinas}                                     \\
        \midrule
        \endfirsthead

        \multicolumn{2}{c}%
        {{\bfseries \tablename\ \thetable{} -- Continuação da página anterior}}                          \\
        \toprule
        \textbf{Competência - Conteúdo (SBC)} & \textbf{Disciplinas}                                     \\
        \midrule
        \endhead

        \midrule \multicolumn{2}{r}{{Continua na próxima página}}                                        \\
        \endfoot

        \bottomrule
        \endlastfoot
        C.5.1 - Conhecer os direitos e propriedades
        intelectuais inerentes à produção e à
        utilização de sistemas de computação  & \EngCompSoc                                              \\
        \addlinespace
        C.5.2 - Compreender a importância da
        conduta ética e cidadã no exercício da
        Engenharia de Computação              & \EngCompSoc, Atividades Curriculares de Extensão         \\
        \addlinespace
        C.5.3 - Compreender o impacto que as
        soluções de sistemas de computação
        podem causar na sociedade e no meio
        ambiente.                             & \IntAmb                                                  \\
    \end{longtable}
\end{small}



\include{cap-perfil_docente}
% (Removed the commented-out \include command for clarity)
\chapter{Caracterização das Instalações Físicas - \textcolor{red}{Felipe}}

Olhar tabelas no Roteiro Sugerido (usar quando necessário)
Olhar PP Elétrica

\section{Edificações e Instalações}
A Faculdade de Engenharia está situada no quinto andar do pavilhão João Lyra Filho e possui, no bloco F, 22 salas de aula com capacidade média para 40 alunos.

\section{Biblioteca}
Os recursos bibliográficos postos à disposição dos alunos estão sob a guarda da biblioteca central e das bibliotecas setoriais. São mais de vinte mil (20.000) títulos com cerca de trinta mil exemplares (30.000), cerca de mil e duzentos títulos de periódicos sobre os mais diversos assuntos de todas as áreas.

A Biblioteca setorial do curso está situada no quinto andar do pavilhão João Lyra Filho e reúne o acervo básico, oferecendo área de estudos específica para os discentes e docentes.

Associado a esses recursos, os alunos, por meio do uso de computadores e da Internet, têm acesso ao sistema automático de busca bibliográfica.

Em relação aos mecanismos de atualização, a biblioteca conta com doações e verbas próprias da UERJ.

\section{Laboratórios}
A Faculdade de Engenharia possui laboratórios que atendem tanto os cursos de graduação como também à pós-graduação. O Curso de Engenharia de Computação utilizará, para as aulas práticas das disciplinas do núcleo de conteúdos básicos, os laboratórios vinculados às Ciências Básicas: Física, Química e de Informática. Para as disciplinas do núcleo profissional, serão utilizados o Laboratório de Engenharia Elétrica e o Laboratório de Computação.

O Laboratório de Engenharia Elétrica (LEE) apoia as atividades de ensino e pesquisa em Eletricidade, Eletrônica, Máquinas Elétricas, Sistemas de Controle, Acionamentos Elétricos, Eletrônica Industrial, Conversão Eletromecânica de Energia, Sistemas Digitais e Telecomunicações.

O Laboratório de Computação (LabComp) apoia as atividades de ensino e pesquisa em Arquitetura de Computadores, Desenvolvimento de Sistemas, Linguagens de Programação, Análise de Algoritmos e Sistemas Embutidos.



\backmatter%%%%%%%%%%%%%%%%%%%%%%%%%%%%%%%%%%%%%%%%%%%%%%%%%%%%%%%

\appendix
\appendixpage
\addappheadtotoc

% !TEX root = ProjetoPedagogico.tex
\chapter{Deliberação n\textordmasculine{} 33/95 da UERJ}
\label{delib3395}
%\includepdf[pages=-,pagecommand={\thispagestyle{plain}}]{pdf/Deliberacao33-95.pdf}
\includepdf[pages=-,pagecommand={\thispagestyle{fancy}}]{leis/Deliberacao33-95.pdf}

\chapter{Resolução CNE/CES n\textordmasculine{} 5, de 16 de novembro de 2016}
\label{cne}
\includepdf[pages=-,pagecommand={\thispagestyle{fancy}}]{leis/rces005_16.pdf}
%\includepdf[pages=-]{leis/CES112002.pdf}

\chapter{Referenciais de Formação para os Cursos de Graduação em Computação}
\label{sbc}
\includepdf[pages=-,pagecommand={\thispagestyle{fancy}}]{leis/sbc2017.pdf}

\chapter{Resolução n\textordmasculine{} 1.010 CREA/CONFEA}
\label{res1010}
\includepdf[pages=1,pagecommand={\thispagestyle{fancy}}]{leis/res1010.pdf}
\includepdf[pages=2, offset=1.8cm 0,pagecommand={\thispagestyle{fancy}}]{leis/res1010.pdf}
\includepdf[pages=3,pagecommand={\thispagestyle{fancy}}]{leis/res1010.pdf}
\includepdf[pages=4, offset=1.8cm 0,pagecommand={\thispagestyle{fancy}}]{leis/res1010.pdf}
\includepdf[pages=5,pagecommand={\thispagestyle{fancy}}]{leis/res1010.pdf}
\includepdf[pages=6, offset=1.8cm 0,pagecommand={\thispagestyle{fancy}}]{leis/res1010.pdf}
\includepdf[pages=7,pagecommand={\thispagestyle{fancy}}]{leis/res1010.pdf}

\chapter{Deliberação n\textordmasculine{} 4/2023 do CSEPE/UERJ}
\label{del4}
\includepdf[pages=-,pagecommand={\thispagestyle{fancy}}]{leis/del4.pdf}

\chapter{Fluxograma do Curso de Engenharia de Computação}
\label{fluxograma}
\includepdf[pages=-,angle=90]{fluxogramaEngenhariaComputacao.pdf}
\chapter{Ementas do Curso de Engenharia de Computação}
\label{ementas}
\includepdf[pages=-,addtotoc={1,section,1,{\Adm},},pagecommand={\thispagestyle{fancy}}]{ementasExternas/Basico/Administracao.pdf}
\includepdf[pages=-,addtotoc={1,section,1,{\AlgLin},},pagecommand={\thispagestyle{fancy}}]{ementasExternas/Basico/AlgebraLinearIII.pdf}
\includepdf[pages=-,addtotoc={1,section,1,{\AlgComp},},pagecommand={\thispagestyle{fancy}}]{AlgoritmosComputacionais.pdf}
\includepdf[pages=-,addtotoc={1,section,1,{\AnAlg},},pagecommand={\thispagestyle{fancy}}]{AnaliseDeAlgoritmos.pdf}
\includepdf[pages=-,addtotoc={1,section,1,{\AnaFis},},pagecommand={\thispagestyle{fancy}}]{AnaliseDeSistemasFisicos.pdf}

\includepdf[pages=-,addtotoc={1,section,1,{\AnaVet},},pagecommand={\thispagestyle{fancy}}]{ementasExternas/Basico/AnaliseVetorial.pdf}
\includepdf[pages=-,addtotoc={1,section,1,{\ArqComp},},pagecommand={\thispagestyle{fancy}}]{ArquiteturaDeComputadores.pdf}
\includepdf[pages=-,addtotoc={1,section,1,{\CalcI},},pagecommand={\thispagestyle{fancy}}]{ementasExternas/Basico/CalculoI.pdf}
\includepdf[pages=-,addtotoc={1,section,1,{\CalcII},},pagecommand={\thispagestyle{fancy}}]{ementasExternas/Basico/CalculoII.pdf}

\includepdf[pages=-,addtotoc={1,section,1,{\CEV},},pagecommand={\thispagestyle{fancy}}]{CircuitosEletricosI.pdf}
\includepdf[pages=-,addtotoc={1,section,1,{\CEVI},},pagecommand={\thispagestyle{fancy}}]{CircuitosEletricosII.pdf}
\includepdf[pages=-,addtotoc={1,section,1,{\CompParal},},pagecommand={\thispagestyle{fancy}}]{ComputacaoParalela.pdf}
\includepdf[pages=-,addtotoc={1,section,1,{\Control},},pagecommand={\thispagestyle{fancy}}]{ControleDeProcessosPorComputador.pdf}

\includepdf[pages=-,addtotoc={1,section,1,{\CServMec},},pagecommand={\thispagestyle{fancy}}]{ementasExternas/Eletronica/ControleEServomecanismosIII.pdf}
\includepdf[pages=-,addtotoc={1,section,1,{\DesBas},},pagecommand={\thispagestyle{fancy}}]{ementasExternas/Basico/DesenhoBasico.pdf}
\includepdf[pages=-,addtotoc={1,section,1,{\EletI},},pagecommand={\thispagestyle{fancy}}]{ementasExternas/Eletronica/EletronicaI.pdf}
\includepdf[pages=-,addtotoc={1,section,1,{\EletIIA},},pagecommand={\thispagestyle{fancy}}]{ementasExternas/Eletronica/EletronicaII.pdf}
\includepdf[pages=-,addtotoc={1,section,1,{\EngComput},},pagecommand={\thispagestyle{fancy}}]{EngenhariaComputacional.pdf}

\includepdf[pages=-,addtotoc={1,section,1,{\EngSistA},},pagecommand={\thispagestyle{fancy}}]{EngenhariaDeSistemas.pdf}
\includepdf[pages=-,addtotoc={1,section,1,{\EngCompSoc},},pagecommand={\thispagestyle{fancy}}]{EngenhariaDeComputacaoESociedade.pdf}
\includepdf[pages=-,addtotoc={1,section,1,{\SegHig},},pagecommand={\thispagestyle{fancy}}]{EngenhariaDoTrabalhoI.pdf}
\includepdf[pages=-,addtotoc={1,section,1,{\CalcIII},},pagecommand={\thispagestyle{fancy}}]{ementasExternas/Basico/EDO.pdf}
\includepdf[pages=-,addtotoc={1,section,1,{\EstSup},},pagecommand={\thispagestyle{fancy}}]{EstagioSupervisionadoXIA.pdf}
\includepdf[pages=-,addtotoc={1,section,1,{\EstrInf},},pagecommand={\thispagestyle{fancy}}]{EstruturasDeInformacao.pdf}

\includepdf[pages=-,addtotoc={1,section,1,{\FenTran},},pagecommand={\thispagestyle{fancy}}]{ementasExternas/FenomenosDeTransporte.pdf}
\includepdf[pages=-,addtotoc={1,section,1,{\FisI},},pagecommand={\thispagestyle{fancy}}]{ementasExternas/Basico/FisicaI.pdf}
\includepdf[pages=-,addtotoc={1,section,1,{\FisII},},pagecommand={\thispagestyle{fancy}}]{ementasExternas/Basico/FisicaII.pdf}
\includepdf[pages=-,addtotoc={1,section,1,{\FisIII},},pagecommand={\thispagestyle{fancy}}]{ementasExternas/Basico/FisicaIII.pdf}
\includepdf[pages=-,addtotoc={1,section,1,{\FisIV},},pagecommand={\thispagestyle{fancy}}]{ementasExternas/Basico/FisicaIV.pdf}

\includepdf[pages=-,addtotoc={1,section,1,{\FundComp},},pagecommand={\thispagestyle{fancy}}]{FundamentosDeComputadores.pdf}
\includepdf[pages=-,addtotoc={1,section,1,{\GeoAna},},pagecommand={\thispagestyle{fancy}}]{ementasExternas/Basico/GeometriaAnalitica.pdf}
\includepdf[pages=-,addtotoc={1,section,1,{\IC},},pagecommand={\thispagestyle{fancy}}]{InteligenciaComputacional.pdf}

\includepdf[pages=-,addtotoc={1,section,1,{\IntEco},},pagecommand={\thispagestyle{fancy}}]{ementasExternas/Basico/IntroducaoAEconomia.pdf}
\includepdf[pages=-,addtotoc={1,section,1,{\IntAmb},},pagecommand={\thispagestyle{fancy}}]{ementasExternas/IntroducaoAEngenhariaAmbiental.pdf}
\includepdf[pages=-,addtotoc={1,section,1,{\LabProgA},},pagecommand={\thispagestyle{fancy}}]{LaboratorioDeProgramacaoA.pdf}
\includepdf[pages=-,addtotoc={1,section,1,{\LabProgB},},pagecommand={\thispagestyle{fancy}}]{LaboratorioDeProgramacaoB.pdf}
\includepdf[pages=-,addtotoc={1,section,1,{\LogProg},},pagecommand={\thispagestyle{fancy}}]{LogicaEmProgramacao.pdf}

\includepdf[pages=-,addtotoc={1,section,1,{\MatEle},},pagecommand={\thispagestyle{fancy}}]{MateriaisEletricosEMagneticos.pdf}
\includepdf[pages=-,addtotoc={1,section,1,{\MecTec},},pagecommand={\thispagestyle{fancy}}]{ementasExternas/MecanicaTecnica.pdf}
\includepdf[pages=-,addtotoc={1,section,1,{\MetQuant},},pagecommand={\thispagestyle{fancy}}]{MetodosQuantitativos.pdf}
\includepdf[pages=-,addtotoc={1,section,1,{\MineraDados},},pagecommand={\thispagestyle{fancy}}]{MineracaoDeDados.pdf}
\includepdf[pages=-,addtotoc={1,section,1,{\ModMat},},pagecommand={\thispagestyle{fancy}}]{ementasExternas/Eletronica/ModelosMatematicos.pdf}

\includepdf[pages=-,addtotoc={1,section,1,{\PrincTelec},},pagecommand={\thispagestyle{fancy}}]{ementasExternas/Eletronica/PrincipiosDeTelecomunicacoesIII.pdf}
\includepdf[pages=-,addtotoc={1,section,1,{\ProbEst},},pagecommand={\thispagestyle{fancy}}]{ementasExternas/Basico/ProbEst.pdf}
\includepdf[pages=-,addtotoc={1,section,1,{\ProcImag},},pagecommand={\thispagestyle{fancy}}]{ProcessamentoDeImagens.pdf}


\includepdf[pages=-,addtotoc={1,section,1,{\ProjSO},},pagecommand={\thispagestyle{fancy}}]{ProjetoDeSistemasOperacionais.pdf}
\includepdf[pages=-,addtotoc={1,section,1,{\ProjA},},pagecommand={\thispagestyle{fancy}}]{ProjetoXIA.pdf}
\includepdf[pages=-,addtotoc={1,section,1,{\ProjA},},pagecommand={\thispagestyle{fancy}}]{ProjetoXIB.pdf}
\includepdf[pages=-,addtotoc={1,section,1,{\ProjBD},},pagecommand={\thispagestyle{fancy}}]{ProjetoEAdministracaoDeBancoDeDados.pdf}
\includepdf[pages=-,addtotoc={1,section,1,{\QuiT},},pagecommand={\thispagestyle{fancy}}]{ementasExternas/Basico/QuimicaTeorica.pdf}
\includepdf[pages=-,addtotoc={1,section,1,{\QuiE},},pagecommand={\thispagestyle{fancy}}]{ementasExternas/Basico/QuimicaExperimental.pdf}

\includepdf[pages=-,addtotoc={1,section,1,{\ResMat},},pagecommand={\thispagestyle{fancy}}]{ementasExternas/ResMat.pdf}
\includepdf[pages=-,addtotoc={1,section,1,{\SistEmb},},pagecommand={\thispagestyle{fancy}}]{SistemasEmbutidos.pdf}
\includepdf[pages=-,addtotoc={1,section,1,{\Telep},},pagecommand={\thispagestyle{fancy}}]{TeleprocessamentoERedes.pdf}

\includepdf[pages=-,addtotoc={1,section,1,{\TeoComp},},pagecommand={\thispagestyle{fancy}}]{TeoriaDeCompiladores.pdf}

\chapter{Ementas de Disciplinas Eletivas}
\includepdf[pages=-,addtotoc={1,section,1,{\EletRec},},pagecommand={\thispagestyle{fancy}}]{Eletiva1_ReconhecimentoDePadroes.pdf}
\includepdf[pages=-,addtotoc={1,section,1,{\EletRedes},},pagecommand={\thispagestyle{fancy}}]{Eletiva2_RedesDeInterconexao.pdf}
\includepdf[pages=-,addtotoc={1,section,1,{\EletGeo},},pagecommand={\thispagestyle{fancy}}]{Eletiva3_Geomatica.pdf}
\includepdf[pages=-,addtotoc={1,section,1,{\EletArq},},pagecommand={\thispagestyle{fancy}}]{Eletiva4_ComputacaoDeAltoDesempenho.pdf}
\includepdf[pages=-,addtotoc={1,section,1,{\EletMov},},pagecommand={\thispagestyle{fancy}}]{Eletiva5_ProgramacaoParaDispositivosMoveis.pdf}
\includepdf[pages=-,addtotoc={1,section,1,{\EletPadroes},},pagecommand={\thispagestyle{fancy}}]{Eletiva6_Padroes.pdf}

\end{document}
