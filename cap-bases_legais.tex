\chapter{Bases Legais}

A reforma curricular do curso de Engenharia de Computação da UERJ foi concebida em consonância com as principais normativas que regem a formação superior na área da Computação e o exercício da profissão de engenheiro no Brasil. Fundamenta-se, principalmente, na Resolução CNE/CES n\textsuperscript{o} 5/2016, que estabelece as \textit{Diretrizes Curriculares Nacionais} para os cursos de graduação da área, abrangendo os bacharelados em Ciência da Computação, Sistemas de Informação, \textbf{Engenharia de Computação}, Engenharia de Software e a licenciatura em Computação. Essa resolução define os princípios orientadores e os parâmetros estruturais para a organização curricular desses cursos.

Adicionalmente, a proposta considera os \textit{Referenciais de Formação para os Cursos de Graduação em Computação}, publicados em 2017 pela Sociedade Brasileira de Computação (SBC), os quais detalham diretrizes específicas para a formação nas diversas áreas da Computação. Esses referenciais foram elaborados com base na Resolução n\textsuperscript{o} 5/2016, citada anteriormente, oferecendo uma interpretação consolidada e atualizada das competências e conteúdos curriculares esperados, com ênfase nas transformações tecnológicas e nas demandas sociais contemporâneas.

A proposta incorpora ainda os preceitos da Resolução n\textsuperscript{o} 7/2018, do Conselho Nacional de Educação, que dispõe sobre as Diretrizes para a Extensão na Educação Superior Brasileira, em conformidade com a Meta 12.7 da Lei n\textsuperscript{o} 13.005/2014, que institui o Plano Nacional de Educação (PNE) para o período 2014--2024. Essa normativa orienta as instituições de ensino superior a integrarem, de forma indissociável, ensino, pesquisa e extensão, promovendo a formação integral dos estudantes e o fortalecimento dos vínculos com a sociedade.

Do ponto de vista da regulamentação profissional, considera-se a Resolução n\textsuperscript{o} 380 do Conselho Federal de Engenharia e Agronomia, de 17 de dezembro de 1993, que discrimina as atribuições profissionais dos Engenheiros de Computação.

A observância a essas diretrizes assegura a aderência da proposta curricular às normativas nacionais vigentes, bem como sua adequação às exigências contemporâneas da formação profissional e do mercado de trabalho. Busca-se, assim, consolidar uma formação ampla, sólida e atualizada, alinhada às competências requeridas para o exercício da Engenharia de Computação nos âmbitos acadêmico, científico e tecnológico.

A documentação relevante para este processo de reforma curricular está anexada ao final deste documento:

\begin{enumerate}
    \item \textbf{Câmara de Educação Superior do Conselho Nacional de Educação}  \\
          \textbf{RESOLUÇÃO CNE/CES N\textsuperscript{o} 5, DE 16 DE NOVEMBRO DE 2016}  \\
          Institui as Diretrizes Curriculares Nacionais para os cursos de graduação na área da Computação, abrangendo os cursos de bacharelado em Ciência da Computação, em Sistemas de Informação, em \textbf{Engenharia de Computação}, em Engenharia de Software e de licenciatura em Computação, e dá outras providências (Anexo \ref{cne2016}).

    \item \textbf{Sociedade Brasileira de Computação (SBC)}  \\
          \textbf{Referenciais de Formação para os Cursos de Graduação em Computação –-- 2017} \\
          Documento elaborado com base na Resolução CNE/CES n\textsuperscript{o} 5/2016, com o objetivo de orientar a estrutura curricular dos cursos da área de Computação, considerando competências, conteúdos e práticas pedagógicas atualizadas (Anexo \ref{sbc2017}).

    \item \textbf{Câmara de Educação Superior do Conselho Nacional de Educação}  \\
          \textbf{RESOLUÇÃO N\textsuperscript{o} 7, DE 18 DE DEZEMBRO DE 2018}  \\
          Estabelece as Diretrizes para a Extensão na Educação Superior Brasileira e regimenta o disposto na Meta 12.7 da Lei n\textsuperscript{o} 13.005/2014, que aprova o Plano Nacional de Educação –-- PNE –-- 2014-2024 e dá outras providências (Anexo \ref{rcne2018}).

    \item \textbf{CONFEA – Conselho Federal de Engenharia e Agronomia} \\
          \textbf{RESOLUÇÃO N\textsuperscript{o} 380, DE 17 DE DEZEMBRO DE 1993}  \\
          Discrimina as atribuições provisórias dos Engenheiros de Computação ou Engenheiros Eletricistas com ênfase em Computação e dá outras providências. Brasília: Diário Oficial da União, Seção I, página 193, 06 de janeiro de 1994 (Anexo \ref{confea1993}).
\end{enumerate}


\section{Normas Gerais de Ensino de Graduação da UERJ}

O funcionamento do curso obedecerá às Normas Gerais de Ensino de Graduação da UERJ definidas pela Deliberação N\textsuperscript{o} 33/95 da UERJ. Seus aspectos principais são apresentados a seguir:

\subsection{Período de integralização do curso}
\begin{quoting}
    Art. 99 –-- Somente receberá o diploma o aluno que cumprir a Integralização Curricular.
\end{quoting}

O período mínimo de integralização curricular dos cursos de Engenharia é de 10 (dez) semestres e o prazo máximo é de 18 (dezoito) semestres (Seção \ref{sec:integralizacao}).

\subsection{Relação entre crédito e carga horária}
Nova redação dada pela Deliberação \ordm{N} 59/2019
\begin{quoting}
    Art. 57 –- O número mínimo de créditos necessários para integralizar o
    currículo será estabelecido com base na carga horária total do curso.


    \S \ordm{1} - Nos cursos de regime de crédito, a unidade padrão de crédito
    corresponde a 15 (quinze) horas, e as atividades de que trata o caput do
    presente artigo são:
    \begin{itemize}
        \item[a)] Aula teórica;
        \item[b)] Trabalho de campo;
        \item[c)] Laboratório/aula prática;
        \item[d)] Estágio curricular;
        \item[e)] Prática como componente curricular.
    \end{itemize}

    \S \ordm{2} - Fica a critério de cada unidade acadêmica determinar a carga horária de  cada atividade/disciplina prevista no \S\ordm{1} deste artigo, de acordo com o Projeto Pedagógico de cada curso, respeitando-se o padrão de 01 (um) crédito para cada 15 (quinze) horas/aula e seus múltiplos.
\end{quoting}
\subsection{Aproveitamento escolar}
\begin{quoting}
    Art. 95 –-- A aprovação do aluno em disciplinas do Curso de Graduação desta Universidade terá por base notas e freqüência. São condições para aprovação: obtenção de nota final mínima 5,0 (cinco vírgula zero), constituída pela média aritmética da média semestral e nota da prova final; freqüência mínima de 75\% (setenta e cinco por cento) do total de horas/aula determinado para a disciplina.


    \S~1\textsuperscript{o} - Para cada disciplina haverá, pelo menos, duas avaliações por turma, por período letivo, sendo uma necessariamente individual e escrita. A média dos resultados dessas avaliações constitui a média semestral do aluno na disciplina.

    \S~2\textsuperscript{o} - O aluno que obtiver média semestral igual ou superior a 4,0 (quatro vírgula zero) terá direito à prova final.

    \S~3\textsuperscript{o} - O aluno que obtiver média semestral igual ou superior a 7,0 (sete vírgula zero) estará dispensado de prestar prova final.

    \S~7\textsuperscript{o} - O aluno que obtiver nota final menor que 5,0 (cinco vírgula zero) ou média semestral inferior a 4,0 (quatro vírgula zero) será reprovado.

    \S~8\textsuperscript{o} - O aluno que não obtiver freqüência mínima de 75\% (setenta e cinco por cento) do total de horas/aula determinadas pela disciplina será reprovado, sem direito à prova final e independente de alcançar nota final superior a 7,0 (sete vírgula zero).
\end{quoting}
