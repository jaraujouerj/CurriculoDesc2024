\chapter{Bases Legais}
\thispagestyle{plain}

A reforma curricular do curso de Engenharia de Computação da UERJ foi concebida em consonância com as principais normativas que regem a formação superior na área da Computação e o exercício da profissão de engenheiro no Brasil. Fundamenta-se, principalmente, na Resolução CNE/CES \ordm{n}~5/2016 \cite{cne2016}, que estabelece as \textit{Diretrizes Curriculares Nacionais} para os cursos de graduação da área de computação, abrangendo os bacharelados em Ciência da Computação, Sistemas de Informação, \textbf{Engenharia de Computação}, Engenharia de Software e a licenciatura em Computação. Essa resolução define os princípios orientadores e os parâmetros estruturais para a organização curricular desses cursos.

Complementarmente, foram considerados os \textit{Referenciais de Formação para os Cursos de Graduação em Computação} \cite{sbc2017}, publicados pela Sociedade Brasileira de Computação em 2017. Elaborados com base na referida Resolução \ordm{n}~5/2016, esses referenciais apresentam uma síntese atualizada das competências e conteúdos esperados, em sintonia com os avanços tecnológicos e as demandas da sociedade contemporânea.

A proposta também observa a Resolução \ordm{n}~7/2018 do Conselho Nacional de Educação \cite{cne2018}, que define as \textit{Diretrizes para a Extensão na Educação Superior}, em consonância com a Meta 12.7 da Lei \ordm{n}~13.005/2014, que institui o Plano Nacional de Educação (PNE) para o período de 2014 a 2024. Essa normativa orienta a articulação entre ensino, pesquisa e extensão, com vistas à formação integral do estudante e ao fortalecimento dos vínculos com a sociedade.

No âmbito institucional, o curso observa as Normas Gerais de Ensino de Graduação da UERJ, estabelecidas pela Deliberação \ordm{n}~33/95 \cite{uerj1995}, atualizadas pela Deliberação \ordm{n} 59/2019 \cite{uerj2019}, além da Deliberação \ordm{n} 04/2023 \cite{uerj2023}, que regulamenta o processo de curricularização das atividades extensionistas.

Quanto à regulamentação profissional, considera-se a Resolução \ordm{n}~380, de 17 de dezembro de 1993 \cite{confea1993}, do Conselho Federal de Engenharia e Agronomia (CONFEA), que especifica as atribuições dos Engenheiros de Computação.

A observância a esse conjunto de normativas garante a aderência do projeto pedagógico às exigências legais e à realidade da formação profissional em Engenharia de Computação, assegurando uma proposta atualizada, coerente e alinhada às demandas acadêmicas, científicas e tecnológicas do país.

A documentação complementar relativa ao processo de reforma curricular encontra-se anexada ao final deste documento.

\begin{enumerate}
      \item \textbf{Câmara de Educação Superior do Conselho Nacional de Educação}  \\
            \textbf{RESOLUÇÃO CNE/CES \ordm{n}~5, DE 16 DE NOVEMBRO DE 2016}  \\
            Institui as Diretrizes Curriculares Nacionais para os cursos de graduação na área da Computação, abrangendo os cursos de bacharelado em Ciência da Computação, em Sistemas de Informação, em \textbf{Engenharia de Computação}, em Engenharia de Software e de licenciatura em Computação, e dá outras providências (Anexo \ref{cne2016}).

      \item \textbf{Sociedade Brasileira de Computação (SBC)}  \\
            \textbf{Referenciais de Formação para os Cursos de Graduação em Computação –-- 2017} \\
            Documento elaborado com base na Resolução CNE/CES \ordm{n}~5/2016, com o objetivo de orientar a estrutura curricular dos cursos da área de Computação, considerando competências, conteúdos e práticas pedagógicas atualizadas (Anexo \ref{sbc2017}).

      \item \textbf{Câmara de Educação Superior do Conselho Nacional de Educação}  \\
            \textbf{RESOLUÇÃO \ordm{n}~7, DE 18 DE DEZEMBRO DE 2018}  \\
            Estabelece as Diretrizes para a Extensão na Educação Superior Brasileira e regimenta o disposto na Meta 12.7 da Lei \ordm{n}~13.005/2014, que aprova o Plano Nacional de Educação –-- PNE –-- 2014-2024 e dá outras providências (Anexo \ref{rcne2018}).

      \item \textbf{Universidade do Estado do Rio de Janeiro}  \\
            \textbf{Deliberação \ordm{n}~33, DE 28 DE DEZEMBRO DE 1995}  \\
            Dispõe sobre as Normas Gerais de Ensino de Graduação da UERJ (Anexo \ref{delib3395}).

      \item \textbf{Universidade do Estado do Rio de Janeiro}  \\
            \textbf{Deliberação \ordm{n}~59, DE 12 DE DEZEMBRO DE 2019}  \\
            Altera o Capítulo II, artigos 56, 57 e seu parágrafo único, da Deliberação 33/1995, que trata das normas gerais de Ensino de Graduação da UERJ. (Anexo \ref{delib592019}).

      \item \textbf{CONFEA – Conselho Federal de Engenharia e Agronomia} \\
            \textbf{RESOLUÇÃO \ordm{n}~380, DE 17 DE DEZEMBRO DE 1993}  \\
            Discrimina as atribuições provisórias dos Engenheiros de Computação ou Engenheiros Eletricistas com ênfase em Computação e dá outras providências. Brasília: Diário Oficial da União, Seção I, página 193, 06 de janeiro de 1994 (Anexo \ref{confea1993}).
\end{enumerate}
