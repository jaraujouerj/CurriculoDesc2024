%----------------------------------------------------------------------------------------
%	Latex para ementa tipo dep da uerj
%----------------------------------------------------------------------------------------
%apagar .aux e executar biber ementalatex para obter bibliografia atualizada
\documentclass{ementa} % Use A4 paper with a 12pt font size 

\begin{document}

\ndisciplina 	{\MineraDados}
\nomeprofessor	{Marley M. Bernardes Rebuzzi Vellasco}
\matriculaprofessor{31250-4}
\chta 		{60h}%carga horária total aluno
\codigo		{FEN06-xxxxx}%Código da disciplina 
\objetivos	{Introduzir conceitos básicos de mineração de dados; análise de dados, seleção de atributos; análise de problemas de classificação, agrupamento de dados e previsão. 
}
\ementa	{Introdução e Motivação ao Processo de Descoberta de Conhecimento em Bases de Dados, Etapas do Processo de Mineração de Dados, Conceitos de Dados, Pré-Processamento de Dados: Limpeza, Avaliação de \textit{Outlier}, Transformação, Redução e Discretização de Dados, Análise e Seleção de Variáveis; Construção de atributos; Representação do Conhecimento, Algoritmos e Técnicas para Classificação, Associação, Previsão e Agrupamento: Árvore de Decisão, Indução de regras da 1a ordem, Bayesiana, Algoritmos de Cobertura, Regressão, Aprendizagem Baseada em Instância, Agrupamento (\textit{Clustering}) por partição e por hierarquia, Softwares e Ferramentas de MD; Aplicações e Estudos de Casos.
}
\preum		{Inteligência Computacional} %
\codpreum		{FEN06-xxxxx} %
\credteorica		{4} %

\disciplinaisolada		{x} %Permite

\formementa{MinDados}
\end{document}
