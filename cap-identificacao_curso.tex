\chapter{Identificação do curso --- \textcolor{red}{Luigi}}

\begin{description}
    \item [Nome do curso:] Engenharia de Computação.
    \item [Departamento:] Departamento de Engenharia de Sistemas e Computação (DESC).
    \item [Modalidade:] presencial.
    \item [Carga horária total do curso:] \tHorasCurso horas.
    \item [Carga horária de extensão:] \hExtensao horas.
    \item [Número de vagas:] \vagas vagas anuais divididas em 2 semestres.
    \item [Tempo mínimo e máximo de integralização:] 10 semestres (5 anos) e 18 semestres (9 anos).
    \item [Grau conferido:] Bacharel em Engenharia de Computação.
    \item [Título conferido:] Engenheiro de Computação.
    \item [Regime acadêmico:] semestral por créditos.
    \item [Formas de ingresso:] vestibular, transferência externa, aproveitamento de estudos, entrada \textit{ex officio} e transferência interna.
    \item [Local de oferta:] rua São Francisco Xavier, n\textsuperscript{o} 524, Pavilhão João Lyra Filho, \ordm{5} andar, Maracanã, Rio de Janeiro – RJ.
\end{description}
\section{Denominação}

O curso de Engenharia de Computação, ora proposto, está vinculado ao Departamento de Engenharia de Sistemas e Computação (DESC), integrante da estrutura acadêmica da Faculdade de Engenharia (FEN) da Universidade do Estado do Rio de Janeiro (UERJ), conforme o organograma institucional vigente. O DESC localiza-se na Rua São Francisco Xavier, \ordm{n} 524, Pavilhão João Lyra Filho, \ordm{5} andar, Bloco D, Maracanã, Rio de Janeiro – RJ.

O DESC forma profissionais de nível superior pleno em Engenharia, com sólida base técnico-científica, capacitados para atuar no desenvolvimento de software e hardware, tendo como eixo central a computação. Entre as principais áreas de atuação profissional destacam-se:

\begin{enumerate}
    \item Concepção, projeto e análise de sistemas, produtos e processos computacionais;
    \item Planejamento, supervisão, elaboração e coordenação de projetos e serviços na área da Engenharia de Computação;
    \item Identificação, formulação e resolução de problemas relacionados à Engenharia de Computação.
\end{enumerate}

A presente reformulação do curso de Engenharia de Computação, ofertado pelo Departamento de Engenharia de Sistemas e Computação, substitui o curso anteriormente denominado Engenharia Elétrica com ênfase em Sistemas e Computação, acompanhando a evolução tecnológica e as novas demandas formativas da área.

\section{Duração, Regime e Tempo de Integralização Curricular}

O funcionamento do curso obedecerá às Normas Gerais de Ensino de Graduação da UERJ definidas pela Deliberação \ordm{n} 33/95 da UERJ. Seus aspectos principais são apresentados a seguir.

\subsection{Duração do curso}
\label{sec:integralizacao}
\begin{itquotation}
    \textbf{Art. 99} -- Somente receberá o diploma o aluno que cumprir a Integralização Curricular.
\end{itquotation}

O período mínimo de integralização curricular do curso de Engenharia de Computação  é de 10 (dez) semestres, exceto para os casos de isenção de disciplinas, em que é possível um tempo mínimo menor. Já o prazo máximo para essa integralização é de 18 (dezoito) semestres.

As atividades acadêmicas estão organizadas em dois turnos, de acordo com o semestre de ingresso: para as turmas ingressantes no \ordm{1} semestre letivo, as aulas serão ofertadas nos turnos da manhã e da tarde, preferencialmente pela manhã; para as turmas ingressantes no \ordm{2} semestre letivo, as aulas serão ofertadas nos turnos da tarde e da noite, preferencialmente à tarde.

O regime acadêmico do curso é semestral por créditos, com um total de \tCredCurso créditos, sendo \credEletivas créditos correspondentes a disciplinas eletivas.

\subsection{Relação entre crédito e carga horária}
A Deliberação n\textordmasculine~59/2019 (anexo \ref{delib592019}) da UERJ alterou a Deliberação n\textordmasculine~33/1995, passando o artigo 57 a contar com a redação a seguir:

\begin{itquotation}
    \textbf{Art. 57} -- O número mínimo de créditos necessários para integralizar o currículo será estabelecido com base na carga horária total do curso.

    \textbf{\S 1\textsuperscript{o}}- Nos cursos de regime de crédito, a unidade padrão de crédito
    corresponde a 15 (quinze) horas, e as atividades de que trata o caput do
    presente artigo são:
    \begin{enumerate}[a)]
        \item Aula teórica;
        \item Trabalho de campo;
        \item Laboratório/aula prática;
        \item Estágio curricular;
        \item Prática como componente curricular.
    \end{enumerate}
\end{itquotation}
\subsection{Aproveitamento escolar}
Os critérios para aproveitamento escolar seguem as normas estabelecidas pela Deliberação n\textordmasculine~35/95. O aproveitamento escolar é definido em função da nota e da frequência do aluno nas disciplinas do curso de graduação.
\begin{itquotation}
    \setcounter{artigo}{94}
    \artigo A aprovação do aluno em disciplinas do Curso de Graduação desta Universidade terá por base notas e frequência. São condições para aprovação: obtenção de nota final mínima 5,0 (cinco vírgula zero), constituída pela média aritmética da média semestral e nota da prova final, frequência mínima de 75\% (setenta e cinco por cento) do total de horas/aula determinado para a disciplina.

    \S~1\textsuperscript{o} -- Para cada disciplina haverá, pelo menos, duas avaliações por turma, por período letivo, sendo uma delas necessariamente individual e escrita. A média dos resultados dessas avaliações constitui a média semestral do aluno na disciplina.

    \S~2\textsuperscript{o} -- O aluno que obtiver média semestral igual ou superior a 4,0 (quatro vírgula zero) terá direito à prova final.

    \S~3\textsuperscript{o} -- O aluno que obtiver média semestral igual ou superior a 7,0 (sete vírgula zero) estará dispensado de prestar prova final.

    \S~4\textsuperscript{o} -- O aluno que, mesmo enquadrado no \S 3\textsuperscript{o}, o desejar, poderá prestar prova final. Deverá, neste caso, atender ao disposto no caput deste Artigo.

    \S~5\textsuperscript{o} -- A prova final terá seu conteúdo e data fixados pelo professor responsável pala turma-
    disciplina, respeitado o Calendário Escolar.

    \S~6\textsuperscript{o} -- Terá direito à segunda chamada o aluno que faltar a quaisquer avaliações, desde que
    comprove, através de documento, doença, viagem a serviço ou trabalho extraordinário, no prazo de,
    no máximo, 7 (sete) dias corridos após a data da avaliação.

    \S~7\textsuperscript{o} -- O aluno que obtiver nota final menor que 5,0 (cinco vírgula zero) ou média semestral inferior a 4,0 (quatro vírgula zero) será reprovado.

    \S~8\textsuperscript{o} -- O aluno que não obtiver frequência mínima de 75\% (setenta e cinco por cento) do total de horas/aula determinadas pela disciplina será reprovado, sem direito à prova final e independente de alcançar nota final superior a 7,0 (sete vírgula zero).
\end{itquotation}


\section{Local, Turno e Horário de Funcionamento}

O curso funcionará no 5\textsuperscript{o} andar do Pavilhão João Lyra Filho, à Rua São Francisco Xavier n\textsuperscript{o} 524, onde se situam as salas de aula e laboratórios específicos para suas atividades. A Faculdade de Engenharia possui, no bloco F, 22 salas de aula com capacidade média
para 40 alunos. O Departamento de Engenharia de Sistemas e Computação está localizado no Bloco D e conta com instalações adequadas para suas atividades.



As dependências do curso estão distribuídos conforme segue:
\begin{itemize}
    \item \textbf{Sala 5014} -- Sala de Redes;
    \item \textbf{Sala 5015} -- Laboratório de Síntese Automática de Circuitos (LSAC);
    \item \textbf{Sala 5018} -- Sala de Professores;
    \item \textbf{Sala 5022} -- Sala de Professores;
    \item \textbf{Sala 5023} -- Laboratório de Computação;
    \item \textbf{Sala 5028} -- Secretaria DESC e Sala de Professores;
    \item \textbf{Sala 5029} -- Laboratório de Computação;
    \item \textbf{Sala 5032} -- Laboratório de Computação;
    \item \textbf{Sala 5033} -- Laboratório de Inteligência de Enxame.
\end{itemize}


O horário de funcionamento do curso seguirá o escalonamento definido para os cursos da Faculdade de Engenharia, ou seja, o turno da manhã (M) se inicia às 07:00 horas e termina às 12:20 horas; o turno da tarde (T) com início às 12:30 horas e término às 17:50 horas. Todos os turnos têm aulas com 50 minutos de duração (Deliberação \ordm{n} 06/2018).

\section{Número de Turmas e Vagas}
O curso será ofertado em duas turmas por ano, com ingresso no primeiro e no segundo semestres letivos, sendo disponibilizadas 25 vagas por turma. Dessa forma, o total anual de vagas ofertadas será de 50. Esse quantitativo mantém a proporção de 25 estudantes por turma, em conformidade com a distribuição vigente nas quatro habilitações do curso de Engenharia Elétrica, conforme demonstrado na Tabela~\ref{tabvagas}.

\section{Formas de Ingresso}
\label{sec:forma-ingresso}

As formas de ingresso no curso de Engenharia de Computação da UERJ estão previstas na \textbf{Deliberação N\textsuperscript{o} 35/95}, que dispõe sobre as Normas Gerais de Ensino de Graduação da Universidade. O ingresso pode ocorrer por meio das seguintes modalidades: Vestibular, Transferência Externa, Aproveitamento de Estudos, Entrada \textit{Ex Officio} e Transferência Interna.

\subsection{Vestibular}
O ingresso por vestibular ocorre por meio do processo seletivo anual da UERJ para cursos presenciais. O exame é composto por duas provas de qualificação e uma avaliação discursiva, que inclui a elaboração de uma redação.

\subsection{Transferência Externa}
Permite a admissão de estudantes regularmente matriculados em cursos de Engenharia de Computação de outras instituições de ensino superior. A seleção ocorre mediante processo classificatório, que inclui prova específica, conforme edital anual. O número de vagas ofertadas é definido de acordo com a disponibilidade de vagas ociosas no curso.

\subsection{Aproveitamento de Estudos}
Modalidade destinada a portadores de diploma de curso superior em áreas afins à Engenharia de Computação, que desejem ingressar no curso. A seleção segue os mesmos critérios estabelecidos para a Transferência Externa, mediante edital específico.

\subsection{Entrada \textit{Ex Officio}}
A admissão por entrada \textit{Ex Officio} está prevista em legislação vigente e aplica-se a servidores públicos e seus dependentes transferidos de localidade durante o curso superior em outra instituição. Nessa modalidade, não há exigência de processo seletivo ou existência de vagas ociosas.

\subsection{Transferência Interna}
Discentes da UERJ que ingressaram em outros cursos de graduação da universidade por meio do vestibular podem solicitar transferência interna para o curso de Engenharia de Computação, conforme critérios e prazos estabelecidos em edital anual.





