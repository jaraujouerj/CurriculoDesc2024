\chapter{Identificação do curso - \textcolor{red}{Luigi}}

\section{Denominação}

O curso de Engenharia de Computação, ora proposto, está inserido no Departamento de Engenharia de Sistemas e Computação (DESC) e se vincula à Faculdade de Engenharia – FEN, pelo organograma geral da UERJ. Funcionará à Rua São Francisco Xavier 524, Pavilhão João Lyra Filho, quinto andar, bloco D, Maracanã, Rio de Janeiro – RJ.

O DESC (Departamento de Engenharia de Sistemas e Computação) da Faculdade de Engenharia da UERJ forma graduados em nível superior pleno da engenharia, com conhecimento técnico-científico abrangente e forte para atuação no desenvolvimento de software e hardware, tendo predominantemente a computação como atividade fim. Destacam-se as seguintes áreas de atuação:

\begin{enumerate}
	\item Concepção, projeto e análise de sistemas, produtos e processos computacionais;
	\item Planejamento, supervisão, elaboração e coordenação de projetos e serviços de engenharia de computação;
	\item Identificação, formulação e resolução de problemas de engenharia de computação.
\end{enumerate}

A criação do curso de Engenharia de Computação a ser oferecido pelo Departamento de Engenharia de Sistemas e Computação vem substituir o curso de Engenharia Elétrica com ênfase em Sistemas e Computação.

\section{Bases Legais}

A seguinte documentação pertinente a este processo de reforma curricular está apensada nas páginas a seguir:

\begin{enumerate}
    \item \textbf{Câmara de Educação Superior do Conselho Nacional de Educação}  \\
    \textbf{RESOLUÇÃO N\textsuperscript{o} 2, DE 18 DE JUNHO DE 2007}  \\
    Dispõe sobre carga horária mínima e procedimentos relativos à integralização e duração dos cursos de graduação, bacharelados, na modalidade presencial.
    
       \item \textbf{Câmara de Educação Superior do Conselho Nacional de Educação}  \\
    \textbf{RESOLUÇÃO CNE/CES N\textsuperscript{o} 5, DE 16 DE NOVEMBRO DE 2016}  \\
    Institui as Diretrizes Curriculares Nacionais para os cursos de graduação na área da Computação, abrangendo os cursos de bacharelado em Ciência da Computação, em Sistemas de Informação, em Engenharia de Computação, em Engenharia de Software e de licenciatura em Computação, e dá outras providências.

    \item \textbf{Câmara de Educação Superior do Conselho Nacional de Educação}  \\
    \textbf{RESOLUÇÃO N\textsuperscript{o} 7, DE 18 DE DEZEMBRO DE 2018}  \\
    Estabelece as Diretrizes para a Extensão na Educação Superior Brasileira e regimenta o disposto na Meta 12.7 da Lei n\textsuperscript{o} 13.005/2014, que aprova o Plano Nacional de Educação – PNE – 2014-2024 e dá outras providências.

    \item \textbf{Câmara de Educação Superior do Conselho Nacional de Educação}  \\
    \textbf{RESOLUÇÃO N\textsuperscript{o} 2, DE 24 DE ABRIL DE 2019}  \\
    Institui as Diretrizes Curriculares Nacionais do Curso de Graduação em Engenharia.

    \item \textbf{Câmara de Educação Superior do Conselho Nacional de Educação}  \\
    \textbf{RESOLUÇÃO N\textsuperscript{o} 1, DE 26 DE MARÇO DE 2021}  \\
    Altera o Art. 9\textsuperscript{o}, § 1\textsuperscript{o} da Resolução CNE/CES 2/2019 e o Art. 6\textsuperscript{o}, § 1\textsuperscript{o} da Resolução CNE/CES 2/2010, que institui as Diretrizes Curriculares Nacionais dos Cursos de Graduação de Engenharia, Arquitetura e Urbanismo.

    \item \textbf{CFE – Conselho Federal de Educação / Conselho Nacional de Educação}  \\
    \textbf{RESOLUÇÃO N\textsuperscript{o} 48, DE 27 DE ABRIL DE 1976}  \\
    Fixa os mínimos de conteúdo e duração do currículo do curso de graduação em Engenharia e define suas áreas de habilitações. Distrito Federal: Diário Oficial da União, 1976.

    \item \textbf{CONFEA - Conselho Federal de Engenharia, Arquitetura e Agronomia}  \\
    \textbf{RESOLUÇÃO N\textsuperscript{o} 218, DE 29 DE JUNHO DE 1973} \\
    Discrimina atividades das diferentes modalidades profissionais da Engenharia, Arquitetura e Agronomia. Brasília: Diário Oficial da União, 31 de julho de 1973.

    \item \textbf{CONFEA - Conselho Federal de Engenharia, Arquitetura e Agronomia}  \\
    \textbf{RESOLUÇÃO N\textsuperscript{o} 1.018, DE 8 DE DEZEMBRO DE 2006}  \\
    Dispõe sobre os procedimentos para registro das instituições de ensino superior e das entidades de classe de profissionais de nível superior ou de profissionais técnicos de nível médio nos Creas e dá outras providências. Brasília: Diário Oficial da União, Seção 1, páginas 73 e 74, 17 de janeiro de 2007.

    \item \textbf{CONFEA - Conselho Federal de Engenharia, Arquitetura e Agronomia}  \\
    \textbf{RESOLUÇÃO N\textsuperscript{o} 1.076, DE 5 DE JULHO DE 2016}  \\
    Discrimina as atividades e competências profissionais do engenheiro de energia e insere o título na Tabela de Títulos Profissionais do Sistema Confea/Crea, para efeito de fiscalização do exercício profissional.
\end{enumerate}

%\section{Normas Gerais de Ensino de Graduação da UERJ}
%
%O funcionamento do curso obedecerá às Normas Gerais de Ensino de Graduação da UERJ definidas pela Deliberação N\textsuperscript{o} 33/95 da UERJ. Seus aspectos principais são apresentados a seguir:
%
%\subsection{Período de integralização do curso}
%Art. 99 – Somente receberá o diploma o aluno que cumprir a Integralização Curricular.
%
%O período mínimo de integralização curricular dos cursos de Engenharia é de 10 (dez) semestres e o prazo máximo é de 18 (dezoito) semestres.
%
%\subsection{Relação entre crédito e carga horária}
%Art. 57 – O número mínimo de créditos necessários para integralizar o currículo será estabelecido com base na carga horária total do curso.
%
%Parágrafo Único - A unidade de crédito corresponde a:
%\begin{itemize}
%    \item 15 (quinze) horas de aula teórica, ou
%    \item 15 (quinze) horas de aula prática, laboratório ou estágio curricular.
%\end{itemize}

%\subsection{Aproveitamento escolar}
%Art. 95 – A aprovação do aluno em disciplinas do Curso de Graduação desta Universidade terá por base notas e freqüência. São condições para aprovação: obtenção de nota final mínima 5,0 (cinco vírgula zero), constituída pela média aritmética da média semestral e nota da prova final; freqüência mínima de 75\% (setenta e cinco por cento) do total de horas/aula determinado para a disciplina.
%
%§ 1\textsuperscript{o} - Para cada disciplina haverá, pelo menos, duas avaliações por turma, por período letivo, sendo uma necessariamente individual e escrita. A média dos resultados dessas avaliações constitui a média semestral do aluno na disciplina.
%
%§ 2\textsuperscript{o} - O aluno que obtiver média semestral igual ou superior a 4,0 (quatro vírgula zero) terá direito à prova final.
%
%§ 3\textsuperscript{o} - O aluno que obtiver média semestral igual ou superior a 7,0 (sete vírgula zero) estará dispensado de prestar prova final.
%
%§ 7\textsuperscript{o} - O aluno que obtiver nota final menor que 5,0 (cinco vírgula zero) ou média semestral inferior a 4,0 (quatro vírgula zero) será reprovado.
%
%§ 8\textsuperscript{o} - O aluno que não obtiver freqüência mínima de 75\% (setenta e cinco por cento) do total de horas/aula determinadas pela disciplina será reprovado, sem direito à prova final e independente de alcançar nota final superior a 7,0 (sete vírgula zero).


\section{Duração, Regime e Tempo de Integralização Curricular}

O curso de Engenharia de Computação terá duração de 10 semestres, compreendendo um período mínimo de 10 semestres (5 anos) e máximo de 18 semestres (9 anos). O regime será dividido em duas fases: a primeira em turno manhã/tarde para as turmas do 1\textsuperscript{o} semestre do ano de admissão, e a segunda em turno tarde/noite para as turmas do 2\textsuperscript{o} semestre do ano de admissão.

A estrutura curricular será orientada pelas Diretrizes Curriculares Nacionais para o Ensino de Graduação em Engenharia do MEC (anexo c) e pela regulamentação do exercício da profissão de Engenheiro, conforme estabelecido pelo Sistema CREA/CONFEA (Resolução 1.010 CONFEA, anexo E), atualmente em vigor.

A grade curricular totaliza 3210 aulas de 50 minutos cada (doravante chamadas de horas-aula), distribuídas em 55 disciplinas, sendo 52 obrigatórias e 3 eletivas restritas. Estão incluídas práticas laboratoriais como complementação à base teórica. Além disso, o curso prevê 320 horas obrigatórias de atividades de extensão.
  
%As atividades obrigatórias incluem o Estágio Supervisionado, com carga horária de 180 horas-aula, e o Projeto de Graduação, que consiste em uma atividade de síntese e integração do conhecimento científico, tecnológico e instrumental. Já as atividades acadêmicas complementares facultativas abrangem o Estágio Interno, Monitoria e Iniciação Científica, além da participação em cursos, eventos, palestras e visitas técnicas. Essas atividades têm como objetivo proporcionar uma visão mais ampla sobre a Engenharia, incluindo a compreensão do setor no Brasil, suas áreas de atuação e as atividades desenvolvidas pelos Engenheiros de Computação.


\section{Local, Turno e Horário de Funcionamento}

O curso funcionará no 5\textsuperscript{o} andar do Pavilhão João Lyra Filho, à Rua São Francisco Xavier n\textsuperscript{o} 524, onde se situam as salas de aula e laboratórios específicos para suas atividades. O Departamento de Engenharia de Sistemas e Computação se localiza na sala 5.014, bloco D, dispondo de instalações apropriadas para o exercício da coordenação do curso. A secretaria do departamento está situada na sala 5.028.

As dependências do curso estão distribuídos conforme segue:
\begin{itemize}
    \item \textbf{5.014} Departamento de Engenharia de Sistemas e Computação;
    \item \textbf{5.015} Laboratório de Computação;
    \item \textbf{5.016} Sala de Redes;
    \item \textbf{5.017} Sala de Professores;
    \item \textbf{5.018} Sala de Professores;
    \item \textbf{5.022} Sala de Professores;
    \item \textbf{5.023} Laboratório de Computação;
    \item \textbf{5.028} Secretaria DESC;
    \item \textbf{5.029} Laboratório de Computação;
    \item \textbf{5.032} Laboratório de Computação;
    \item \textbf{5.033} Laboratório de Computação.
\end{itemize}

O curso funcionará nos turnos manhã, tarde e noite, distribuídos da seguinte maneira: no turno manhã/tarde para as admissões no primeiro semestre e tarde/noite para as turmas com ingresso no segundo semestre.  

O horário de funcionamento do curso seguirá o escalonamento definido para os cursos de engenharia da Faculdade de Engenharia, ou seja, o turno da manhã (M) se inicia às 07:00 horas e termina às 12:20 horas; o turno da tarde (T) com início às 12:30 horas e término às 17:50 horas; e o turno da noite (N) com início às 18:00 horas e término às 22:40 horas. Os turnos da manhã e tarde têm aulas com 50 minutos de duração, enquanto os do turno da noite têm aulas com 45 minutos de duração.  

\section{Número de Turmas e Vagas}

O curso compreenderá duas turmas sendo uma no primeiro semestre e outra no segundo com 40 alunos cada. Desta forma o número total de vagas por ano letivo será de 80.

\section{Número de Alunos e Número de Docentes}

O número médio de alunos por docente no Departamento de Engenharia de Sistemas e Computação (DESC) é de aproximadamente 36 alunos, considerando o total de 400 alunos matriculados ao longo dos 5 anos do curso de Engenharia de Computação e o quadro de 11 professores. Este cálculo pressupõe uma entrada regular de 40 alunos por semestre, totalizando 80 alunos por ano, sem considerar abandonos ou atrasos no curso.  

\section{Formas de Ingresso}

Estão estabelecidas na \textbf{Deliberação N\textsuperscript{o} 35/95} que dispõe sobre as Normas Gerais de Ensino de Graduação da UERJ. O ingresso no curso de Engenharia de Computação da UERJ pode se dar das seguintes formas: Vestibular, Transferência Externa, Aproveitamento de Estudos, Entrada Ex Officio e Transferência Interna.

\subsection*{2.6.1. Vestibular}
O vestibular da UERJ para cursos presenciais ocorre uma vez ao ano. Ele é composto por dois exames de qualificação e uma prova discursiva acompanhada de uma redação.

\subsection*{2.6.2. Transferência Externa}
Permite que alunos que estejam cursando Engenharia de Computação em outras universidades possam ser transferidos para o curso de Engenharia de Computação da UERJ. Os candidatos devem ser aprovados e classificados por meio de uma prova. Para tal fim, há um edital anual que estabelece o número de vagas oferecidas (de acordo com as vagas ociosas do curso) e as datas das provas.

\subsection*{2.6.3. Aproveitamento de Estudos}
Alternativa de entrada voltada para potenciais interessados em cursar Engenharia de Computação na UERJ e que já tenham uma formação em nível superior em área afim (e.g. Engenharia de Computação). A admissão é semelhante àquela descrita acima para Transferência Externa.

\subsection*{2.6.4. Entrada Ex Officio}
A entrada \textit{Ex Officio} segue a legislação sobre o tema relativamente a servidores públicos e seus dependentes que estejam realizando o seu curso superior em outras instituições e que tenham sido transferidos de localidade. Esta transferência não envolve processo seletivo ou necessidade da existência de vagas ociosas.

\subsection*{2.6.5. Transferência Interna}
É facultado que discentes da UERJ que tenham sido aprovados em vestibular para outros cursos possam se candidatar a processos de Transferência Interna para cursar Engenharia de Computação na UERJ, conforme edital anual.




