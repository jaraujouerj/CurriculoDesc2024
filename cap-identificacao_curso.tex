\chapter{Identificação do curso - \textcolor{red}{Luigi}}

\section{Denominação}

O curso de Engenharia de Computação está inserido no Departamento de Engenharia de Sistemas e Computação (DESC) e se vincula à Faculdade de Engenharia (FEN), pelo organograma geral da UERJ. O DESC está localizado à Rua São Francisco Xavier 524, Pavilhão João Lyra Filho, \ordm{5} andar, bloco D, Maracanã, Rio de Janeiro –-- RJ.

O DESC forma graduados em nível superior pleno em Engenharia, com conhecimento técnico-científico abrangente e forte para atuação no desenvolvimento de software e hardware, tendo predominantemente a computação como atividade fim. Destacam-se as seguintes áreas de atuação:

\begin{enumerate}
    \item Concepção, projeto e análise de sistemas, produtos e processos computacionais;
    \item Planejamento, supervisão, elaboração e coordenação de projetos e serviços de engenharia de computação;
    \item Identificação, formulação e resolução de problemas de engenharia de computação.
\end{enumerate}

A reformulação do curso de Engenharia de Computação a ser oferecido pelo Departamento de Engenharia de Sistemas e Computação vem substituir o curso de Engenharia Elétrica com ênfase em Sistemas e Computação.

\section{Bases Legais}

A Reforma curricular do curso de Engenharia de Computação da UERJ foi concebida em consonância com as principais normativas que regem a formação superior na área da Computação e o exercício da profissão de engenheiro no Brasil. Fundamenta-se, principalmente, na Resolução CNE/CES n\textsuperscript{o} 5/2016, que estabelece as Diretrizes Curriculares Nacionais para os cursos de graduação da área, abrangendo os bacharelados em Ciência da Computação, Sistemas de Informação, \textbf{Engenharia de Computação}, Engenharia de Software e a licenciatura em Computação. Essa resolução define os princípios orientadores e os parâmetros estruturais para a organização curricular desses cursos.

Adicionalmente, a proposta considera os \textit{Referenciais de Formação para os Cursos de Graduação em Computação}, publicados em 2017 pela Sociedade Brasileira de Computação (SBC), os quais detalham diretrizes específicas para a formação nas diversas áreas da Computação. Esses referenciais foram elaborados com base na Resolução n\textsuperscript{o} 5/2016, citada anteriormente, oferecendo uma interpretação consolidada e atualizada das competências e conteúdos curriculares esperados, com ênfase nas transformações tecnológicas e nas demandas sociais contemporâneas.

A proposta incorpora ainda os preceitos da Resolução n\textsuperscript{o} 7/2018, do Conselho Nacional de Educação, que dispõe sobre as Diretrizes para a Extensão na Educação Superior Brasileira, em conformidade com a Meta 12.7 da Lei n\textsuperscript{o} 13.005/2014, que institui o Plano Nacional de Educação (PNE) para o período 2014--2024. Essa normativa orienta as instituições de ensino superior a integrarem, de forma indissociável, ensino, pesquisa e extensão, promovendo a formação integral dos estudantes e o fortalecimento dos vínculos com a sociedade.

Do ponto de vista da regulamentação profissional, considera-se a Resolução n\textsuperscript{o} 1.018/2006 do CONFEA, que estabelece os procedimentos para o registro de instituições de ensino e entidades profissionais junto aos Conselhos Regionais de Engenharia e Agronomia (Creas), além de regulamentar o exercício da profissão de engenheiro.

A observância a essas diretrizes assegura a aderência da proposta curricular às normativas nacionais vigentes, bem como sua adequação às exigências contemporâneas da formação profissional e do mercado de trabalho. Busca-se, assim, consolidar uma formação ampla, sólida e atualizada, alinhada às competências requeridas para o exercício da Engenharia de Computação nos âmbitos acadêmico, científico e tecnológico.

A documentação relevante para este processo de reforma curricular está anexada nas páginas seguintes:

\begin{enumerate}
    \item \textbf{Câmara de Educação Superior do Conselho Nacional de Educação}  \\
          \textbf{RESOLUÇÃO CNE/CES N\textsuperscript{o} 5, DE 16 DE NOVEMBRO DE 2016}  \\
          Institui as Diretrizes Curriculares Nacionais para os cursos de graduação na área da Computação, abrangendo os cursos de bacharelado em Ciência da Computação, em Sistemas de Informação, em Engenharia de Computação, em Engenharia de Software e de licenciatura em Computação, e dá outras providências (Anexo \ref{cne2016}).

    \item \textbf{Sociedade Brasileira de Computação (SBC)}  \\
          \textbf{Referenciais de Formação para os Cursos de Graduação em Computação –-- 2017} \\
          Documento elaborado com base na Resolução CNE/CES n\textsuperscript{o} 5/2016, com o objetivo de orientar a estrutura curricular dos cursos da área de Computação, considerando competências, conteúdos e práticas pedagógicas atualizadas (Anexo \ref{sbc2017}).

    \item \textbf{Câmara de Educação Superior do Conselho Nacional de Educação}  \\
          \textbf{RESOLUÇÃO N\textsuperscript{o} 7, DE 18 DE DEZEMBRO DE 2018}  \\
          Estabelece as Diretrizes para a Extensão na Educação Superior Brasileira e regimenta o disposto na Meta 12.7 da Lei n\textsuperscript{o} 13.005/2014, que aprova o Plano Nacional de Educação –-- PNE –-- 2014-2024 e dá outras providências (Anexo \ref{rcne2018}).

    \item \textbf{CONFEA – Conselho Federal de Engenharia e Agronomia} \\
          \textbf{RESOLUÇÃO N\textsuperscript{o} 1.018, DE 8 DE DEZEMBRO DE 2006}  \\
          Dispõe sobre os procedimentos para registro das instituições de ensino superior e das entidades de classe de profissionais de nível superior ou de profissionais técnicos de nível médio nos Creas e dá outras providências. Brasília: Diário Oficial da União, Seção 1, páginas 73 e 74, 17 de janeiro de 2007.
\end{enumerate}


\section{Normas Gerais de Ensino de Graduação da UERJ}

O funcionamento do curso obedecerá às Normas Gerais de Ensino de Graduação da UERJ definidas pela Deliberação N\textsuperscript{o} 33/95 da UERJ. Seus aspectos principais são apresentados a seguir:

\subsection{Período de integralização do curso}
\begin{quoting}
    Art. 99 –-- Somente receberá o diploma o aluno que cumprir a Integralização Curricular.
\end{quoting}

O período mínimo de integralização curricular dos cursos de Engenharia é de 10 (dez) semestres e o prazo máximo é de 18 (dezoito) semestres.

\subsection{Relação entre crédito e carga horária}
Nova redação dada pela Deliberação \ordm{N} 59/2019
\begin{quoting}
    Art. 57 –- O número mínimo de créditos necessários para integralizar o
    currículo será estabelecido com base na carga horária total do curso.


    \S \ordm{1} - Nos cursos de regime de crédito, a unidade padrão de crédito
    corresponde a 15 (quinze) horas, e as atividades de que trata o caput do
    presente artigo são:
    \begin{itemize}
        \item[a)] Aula teórica;
        \item[b)] Trabalho de campo;
        \item[c)] Laboratório/aula prática;
        \item[d)] Estágio curricular;
        \item[e)] Prática como componente curricular.
    \end{itemize}

    \S \ordm{2} - Fica a critério de cada unidade acadêmica determinar a carga horária de  cada atividade/disciplina prevista no \S\ordm{1} deste artigo, de acordo com o Projeto Pedagógico de cada curso, respeitando-se o padrão de 01 (um) crédito para cada 15 (quinze) horas/aula e seus múltiplos.
\end{quoting}
\subsection{Aproveitamento escolar}
\begin{quoting}
    Art. 95 –-- A aprovação do aluno em disciplinas do Curso de Graduação desta Universidade terá por base notas e freqüência. São condições para aprovação: obtenção de nota final mínima 5,0 (cinco vírgula zero), constituída pela média aritmética da média semestral e nota da prova final; freqüência mínima de 75\% (setenta e cinco por cento) do total de horas/aula determinado para a disciplina.


    \S~1\textsuperscript{o} - Para cada disciplina haverá, pelo menos, duas avaliações por turma, por período letivo, sendo uma necessariamente individual e escrita. A média dos resultados dessas avaliações constitui a média semestral do aluno na disciplina.

    \S~2\textsuperscript{o} - O aluno que obtiver média semestral igual ou superior a 4,0 (quatro vírgula zero) terá direito à prova final.

    \S~3\textsuperscript{o} - O aluno que obtiver média semestral igual ou superior a 7,0 (sete vírgula zero) estará dispensado de prestar prova final.

    \S~7\textsuperscript{o} - O aluno que obtiver nota final menor que 5,0 (cinco vírgula zero) ou média semestral inferior a 4,0 (quatro vírgula zero) será reprovado.

    \S~8\textsuperscript{o} - O aluno que não obtiver freqüência mínima de 75\% (setenta e cinco por cento) do total de horas/aula determinadas pela disciplina será reprovado, sem direito à prova final e independente de alcançar nota final superior a 7,0 (sete vírgula zero).
\end{quoting}

\section{Duração, Regime e Tempo de Integralização Curricular}

O curso de Engenharia de Computação terá duração de 10 semestres, compreendendo um período mínimo de 10 semestres (5 anos) e máximo de 18 semestres (9 anos). O regime será dividido em duas fases: a primeira em turno manhã/tarde, com aulas preferencialmente na manhã, para as turmas do \ordm{1} semestre do ano de admissão, e a segunda em turno tarde/noite, com aulas preferencialmente à tarde, para as turmas do \ordm{2} semestre do ano de admissão.

A estrutura curricular será orientada pelas Diretrizes Curriculares Nacionais para os
cursos de graduação na área da Computação do MEC (Anexo \ref{cne2016}) e pela regulamentação do exercício da profissão de Engenheiro, conforme estabelecido pelo Sistema CREA/CONFEA (Resolução 1.010 CONFEA, anexo \ref{res1010}), atualmente em vigor.

A grade curricular totaliza 3567 horas, distribuídas em 56 disciplinas, sendo 53 obrigatórias e 3 eletivas restritas. Estão incluídas práticas laboratoriais como complementação à base teórica. Além disso, o curso prevê 357 horas obrigatórias de atividades de extensão.

\rowcolors{1}{gray!5}{white}
\begin{table}[!ht]
    \centering
    \caption{Distribuição da carga horária.}
    \label{tab:distribuicao}
    \begin{spreadtab}{{tabularx}{0.8\textwidth}{Xcc}}
        \hline
        \toprule
        @ {\textbf{Componentes Curriculares}} & @ {\textbf{Créditos}} & @ {\textbf{Carga Horária}} \\
        \hline
        @ Disciplinas obrigatórias            & 202                   & 3030                       \\
        @ Disciplinas eletivas                & 12                    & 180                        \\
        @ Disciplina de extensão              & 2                     & 30                         \\
        @ Atividades de extensão              &                       & 327                        \\
        \midrule
        @ Total                               & sum(b2:b5)            & sum(c2:c5)                 \\
        \toprule
    \end{spreadtab}
\end{table}

\section{Local, Turno e Horário de Funcionamento}

O curso funcionará no 5\textsuperscript{o} andar do Pavilhão João Lyra Filho, à Rua São Francisco Xavier n\textsuperscript{o} 524, onde se situam as salas de aula e laboratórios específicos para suas atividades. O Departamento de Engenharia de Sistemas e Computação está localizado no Bloco D e conta com instalações adequadas para suas atividades. A secretaria do departamento encontra-se na sala 5028.

As dependências do curso estão distribuídos conforme segue:
\begin{itemize}
    \item \textbf{Sala 5014} Sala de Redes;
    \item \textbf{Sala 5015} Laboratório de Síntese Automática de Circuitos (LSAC);
    \item \textbf{Sala 5018} Sala de Professores;
    \item \textbf{Sala 5022} Sala de Professores;
    \item \textbf{Sala 5023} Laboratório de Computação;
    \item \textbf{Sala 5028} Secretaria DESC e Sala de Professores;
    \item \textbf{Sala 5029} Laboratório de Computação;
    \item \textbf{Sala 5032} Laboratório de Computação;
    \item \textbf{Sala 5033} Laboratório de Inteligência de Enxame.
\end{itemize}

O curso funcionará nos turnos manhã e tarde, distribuídos da seguinte maneira: no turno manhã para as admissões no primeiro semestre e tarde para as turmas com ingresso no segundo semestre.

O horário de funcionamento do curso seguirá o escalonamento definido para os cursos de engenharia da Faculdade de Engenharia, ou seja, o turno da manhã (M) se inicia às 07:00 horas e termina às 12:20 horas; o turno da tarde (T) com início às 12:30 horas e término às 17:50 horas. Os turnos da manhã e tarde têm aulas com 50 minutos de duração.

\section{Número de Turmas e Vagas}
O curso será ofertado em duas turmas por ano, com ingresso no primeiro e no segundo semestres letivos, sendo disponibilizadas 25 vagas por turma. Dessa forma, o total anual de vagas ofertadas será de 50. Esse quantitativo mantém a proporção de 25 estudantes por turma, em conformidade com a distribuição vigente nas quatro habilitações do curso de Engenharia Elétrica, conforme demonstrado na Tabela~\ref{tabvagas}.

\section{Formas de Ingresso}
\label{sec:forma-ingresso}

As formas de ingresso no curso de Engenharia de Computação da UERJ estão previstas na \textbf{Deliberação N\textsuperscript{o} 35/95}, que dispõe sobre as Normas Gerais de Ensino de Graduação da Universidade. O ingresso pode ocorrer por meio das seguintes modalidades: Vestibular, Transferência Externa, Aproveitamento de Estudos, Entrada \textit{Ex Officio} e Transferência Interna.

\subsection{Vestibular}
O ingresso por vestibular ocorre por meio do processo seletivo anual da UERJ para cursos presenciais. O exame é composto por duas provas de qualificação e uma avaliação discursiva, que inclui a elaboração de uma redação.

\subsection{Transferência Externa}
Permite a admissão de estudantes regularmente matriculados em cursos de Engenharia de Computação de outras instituições de ensino superior. A seleção ocorre mediante processo classificatório, que inclui prova específica, conforme edital anual. O número de vagas ofertadas é definido de acordo com a disponibilidade de vagas ociosas no curso.

\subsection{Aproveitamento de Estudos}
Modalidade destinada a portadores de diploma de curso superior em áreas afins à Engenharia de Computação, que desejem ingressar no curso. A seleção segue os mesmos critérios estabelecidos para a Transferência Externa, mediante edital específico.

\subsection{Entrada \textit{Ex Officio}}
A admissão por entrada \textit{Ex Officio} está prevista em legislação vigente e aplica-se a servidores públicos e seus dependentes transferidos de localidade durante o curso superior em outra instituição. Nessa modalidade, não há exigência de processo seletivo ou existência de vagas ociosas.

\subsection{Transferência Interna}
Discentes da UERJ que ingressaram em outros cursos de graduação da universidade por meio do vestibular podem solicitar transferência interna para o curso de Engenharia de Computação, conforme critérios e prazos estabelecidos em edital anual.





