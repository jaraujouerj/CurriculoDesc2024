\chapter{Identificação do curso - \textcolor{red}{Luigi}}
\section{Denominação}

O curso de Engenharia de Computação está vinculado ao Departamento de Engenharia de Sistemas e Computação (DESC), integrante da estrutura acadêmica da Faculdade de Engenharia (FEN) da Universidade do Estado do Rio de Janeiro (UERJ), conforme o organograma institucional vigente. O DESC localiza-se na Rua São Francisco Xavier, \ordm{n} 524, Pavilhão João Lyra Filho, \ordm{5} andar, Bloco D, Maracanã, Rio de Janeiro – RJ.

O curso forma profissionais de nível superior pleno em Engenharia, com sólida base técnico-científica, capacitados para atuar no desenvolvimento de software e hardware, tendo como eixo central a computação. Entre as principais áreas de atuação profissional destacam-se:

\begin{enumerate}
  \item Concepção, projeto e análise de sistemas, produtos e processos computacionais;
  \item Planejamento, supervisão, elaboração e coordenação de projetos e serviços na área da Engenharia de Computação;
  \item Identificação, formulação e resolução de problemas relacionados à Engenharia de Computação.
\end{enumerate}

A presente reformulação do curso de Engenharia de Computação, ofertado pelo Departamento de Engenharia de Sistemas e Computação, substitui o curso anteriormente denominado Engenharia Elétrica com ênfase em Sistemas e Computação, acompanhando a evolução tecnológica e as novas demandas formativas da área.

\section{Duração, Regime e Tempo de Integralização Curricular}

O curso de Engenharia de Computação terá duração de 10 semestres, compreendendo um período mínimo de 10 semestres (5 anos) e máximo de 18 semestres (9 anos). O regime será dividido em duas fases: a primeira em turno manhã/tarde, com aulas preferencialmente na manhã, para as turmas do \ordm{1} semestre do ano de admissão, e a segunda em turno tarde/noite, com aulas preferencialmente à tarde, para as turmas do \ordm{2} semestre do ano de admissão.

\rowcolors{1}{gray!5}{white}
\begin{table}[!ht]
  \centering
  \caption{Distribuição da carga horária.}
  \label{tab:distribuicao}
  \begin{spreadtab}{{tabularx}{0.8\textwidth}{Xcc}}
    \hline
    \toprule
    @ {\textbf{Componentes Curriculares}} & @ {\textbf{Créditos}} & @ {\textbf{Carga Horária}} \\
    \hline
    @ Disciplinas obrigatórias            & 202                   & 3030                       \\
    @ Disciplinas eletivas                & 12                    & 180                        \\
    @ Disciplina de extensão              & 2                     & 30                         \\
    @ Atividades de extensão              &                       & 327                        \\
    \midrule
    @ Total                               & sum(b2:b5)            & sum(c2:c5)                 \\
    \toprule
  \end{spreadtab}
\end{table}

\section{Local, Turno e Horário de Funcionamento}

O curso funcionará no 5\textsuperscript{o} andar do Pavilhão João Lyra Filho, à Rua São Francisco Xavier n\textsuperscript{o} 524, onde se situam as salas de aula e laboratórios específicos para suas atividades. O Departamento de Engenharia de Sistemas e Computação está localizado no Bloco D e conta com instalações adequadas para suas atividades. A secretaria do departamento encontra-se na sala 5028.

As dependências do curso estão distribuídos conforme segue:
\begin{itemize}
  \item \textbf{Sala 5014} Sala de Redes;
  \item \textbf{Sala 5015} Laboratório de Síntese Automática de Circuitos (LSAC);
  \item \textbf{Sala 5018} Sala de Professores;
  \item \textbf{Sala 5022} Sala de Professores;
  \item \textbf{Sala 5023} Laboratório de Computação;
  \item \textbf{Sala 5028} Secretaria DESC e Sala de Professores;
  \item \textbf{Sala 5029} Laboratório de Computação;
  \item \textbf{Sala 5032} Laboratório de Computação;
  \item \textbf{Sala 5033} Laboratório de Inteligência de Enxame.
\end{itemize}

O curso funcionará nos turnos manhã e tarde, distribuídos da seguinte maneira: no turno manhã para as admissões no primeiro semestre e tarde para as turmas com ingresso no segundo semestre.

O horário de funcionamento do curso seguirá o escalonamento definido para os cursos de engenharia da Faculdade de Engenharia, ou seja, o turno da manhã (M) se inicia às 07:00 horas e termina às 12:20 horas; o turno da tarde (T) com início às 12:30 horas e término às 17:50 horas. Todos os turnos têm aulas com 50 minutos de duração (Deliberação \ordm{n} 06/2018).

\section{Número de Turmas e Vagas}
O curso será ofertado em duas turmas por ano, com ingresso no primeiro e no segundo semestres letivos, sendo disponibilizadas 25 vagas por turma. Dessa forma, o total anual de vagas ofertadas será de 50. Esse quantitativo mantém a proporção de 25 estudantes por turma, em conformidade com a distribuição vigente nas quatro habilitações do curso de Engenharia Elétrica, conforme demonstrado na Tabela~\ref{tabvagas}.

\section{Formas de Ingresso}
\label{sec:forma-ingresso}

As formas de ingresso no curso de Engenharia de Computação da UERJ estão previstas na \textbf{Deliberação N\textsuperscript{o} 35/95}, que dispõe sobre as Normas Gerais de Ensino de Graduação da Universidade. O ingresso pode ocorrer por meio das seguintes modalidades: Vestibular, Transferência Externa, Aproveitamento de Estudos, Entrada \textit{Ex Officio} e Transferência Interna.

\subsection{Vestibular}
O ingresso por vestibular ocorre por meio do processo seletivo anual da UERJ para cursos presenciais. O exame é composto por duas provas de qualificação e uma avaliação discursiva, que inclui a elaboração de uma redação.

\subsection{Transferência Externa}
Permite a admissão de estudantes regularmente matriculados em cursos de Engenharia de Computação de outras instituições de ensino superior. A seleção ocorre mediante processo classificatório, que inclui prova específica, conforme edital anual. O número de vagas ofertadas é definido de acordo com a disponibilidade de vagas ociosas no curso.

\subsection{Aproveitamento de Estudos}
Modalidade destinada a portadores de diploma de curso superior em áreas afins à Engenharia de Computação, que desejem ingressar no curso. A seleção segue os mesmos critérios estabelecidos para a Transferência Externa, mediante edital específico.

\subsection{Entrada \textit{Ex Officio}}
A admissão por entrada \textit{Ex Officio} está prevista em legislação vigente e aplica-se a servidores públicos e seus dependentes transferidos de localidade durante o curso superior em outra instituição. Nessa modalidade, não há exigência de processo seletivo ou existência de vagas ociosas.

\subsection{Transferência Interna}
Discentes da UERJ que ingressaram em outros cursos de graduação da universidade por meio do vestibular podem solicitar transferência interna para o curso de Engenharia de Computação, conforme critérios e prazos estabelecidos em edital anual.





